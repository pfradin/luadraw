\section{Transformations}
In the following:
\begin{itemize}
    \item the argument \emph{L} is either a complex number, a list of complex numbers, or a list of lists of complex numbers,
    \item the line \emph{d} is a list of two complex numbers: a point on the line and a direction vector.
\end{itemize}

\subsection{affin}
The function \textbf{affin(L,d,v,k)} returns the image of \emph{L} by the affinity of base line \emph{d}, parallel to the vector \emph{v} and of ratio \emph{k}.

\subsection{ftransform}
The function \textbf{ftransform(L,f)} returns the image of \emph{L} by the function \emph{f}, which must be a function of the complex variable. If one of the elements of \emph{L} is the complex number \emph{cpx.Jump}, then it is returned as is in the result.

\subsection{hom}
The function \textbf{hom(L,factor,center)} returns the image of \emph{L} by the homothety with center \emph{center} and ratio \emph{factor}. By default, the argument \emph{center} is 0.

\subsection{inv}
The function \textbf{inv(L, radius, center)} returns the image of \emph{L} by the inversion with respect to the circle with center \emph{center} and radius \emph{radius}. By default, the argument \emph{center} is 0.

\subsection{proj}
The function \textbf{proj(L,d)} returns the image of \emph{L} by the orthogonal projection onto the line \emph{d}.

\subsection{projO}
The function \textbf{projO(L,d,v)} returns the image of \emph{L} by projection onto the line \emph{d} parallel to the vector \emph{v}.

\subsection{rotate}
The function \textbf{rotate(L,angle,center)} returns the image of \emph{L} by rotation with center \emph{center} and angle \emph{angle} (in degrees). By default, the argument \emph{center} is 0.

\subsection{shift}
The function \textbf{shift(L,u)} returns the image of \emph{L} by translation of vector \(u\).

\subsection{simil}
The function \textbf{simil(L,factor,angle,center)} returns the image of \emph{L} by the similarity of center \emph{center}, ratio \emph{factor}, and angle \emph{angle} (in degrees). By default, the argument \emph{center} is 0.

\subsection{sym}
The function \textbf{sym(L,d)} returns the image of \emph{L} by the orthogonal symmetry of axis \emph{d}.

\subsection{symG}
The function \textbf{symG(L,d,v)} returns the image of \emph{L} by the symmetry about the line \emph{d} followed by the translation of vector \emph{v} (sliding symmetry).

\subsection{symO}
The function \textbf{symO(L,d)} returns the image of \emph{L} by symmetry with respect to the line \emph{d} and parallel to the vector \emph{v} (oblique symmetry).

\begin{demo}{Using Transformations}
\begin{luadraw}{name=Sierpinski}
local g = graph:new{window={-5,5,-5,5},size={10,10}}
local i = cpx.I
local rand = math.random
local A, B, C = 5*i, -5-5*i, 5-5*i -- triangle initial
local T, niv = {{A,B,C}}, 5
for k = 1, niv do
    T = concat( hom(T,0.5,A), hom(T,0.5,B), hom(T,0.5,C) )
end
for _,cp in ipairs(T) do
    g:Filloptions("full", rgb(rand(),rand(),rand()))
    g:Dpolyline(cp,true)
end
g:Show()
\end{luadraw}
\end{demo}
