\section{Geometric Constructions}

This section groups together functions that construct geometric figures without dedicated graphics methods.

\subsection{Circumscribed circle, incircle: circumcircle3d(), incircle3d()}

\begin{itemize}
    \item The function \textbf{circumcircle3d(A,B,C)}, where $A$, $B$, and $C$ are three non-aligned 3D points, returns the circumcircle of the triangle formed by these three points, in the form of a sequence: $A,R,n$, where $A$ is the center of the circle, $R$ its radius, and $n$ a normal vector to the plane of the circle.     \item The function \textbf{incircle3d(A,B,C)}, where $A$, $B$, and $C$ are three non-aligned 3D points, returns the circle inscribed in the triangle formed by these three points, as a sequence: $A,R,n$, where $A$ is the center of the circle, $R$ its radius, and $n$ a normal vector to the plane of the circle.
\end{itemize}

\subsection{Convex Hull: cvx\_hull3d()}

The function \textbf{cvx\_hull3d(L)}, where $L$ is a list of \textbf{distinct} 3D points, calculates and returns the convex hull of $L$ as a list of facets.

\begin{demo}{Using cvx\_hull3d()}
\begin{luadraw}{name=cvx_hull3d}
local g = graph3d:new{window={-2,4,-6,1},bbox=false,size={10,10}}
local L = {Origin, 4*vecI, M(4,4,0), 4*vecJ}
insert(L, shift3d(L,-3*vecK))
insert(L, {M(2,1,2), M(2,3,2)})
local V = cvx_hull3d(L)
local P = facet2poly(V)
g:Dpoly(P , {color="cyan",mode=mShadedHidden})
g:Show()
\end{luadraw}
\end{demo}

\subsection{Planes: plane(), planeEq(), orthoframe(), plane2ABC()}

A plane in space is a table of the form $\{A,n\}$ where $A$ is a point in the plane (3d point) and $n$ is a normal vector to the plane (non-zero 3d point).
\begin{itemize}
    \item The function \textbf{plane(A,B,C)} returns the plane passing through the three 3d points $A$, $B$, and $C$ (if they are not aligned, otherwise the result is \emph{nil}).
    \item The function \textbf{planeEq(a,b,c,d)} returns the plane whose Cartesian equation is $ax+by+cz+d=0$ (if the coefficients $a$, $b$, and $c$ are not all zero, otherwise the result is \emph{nil}).
    \item The function \textbf{plane2ABC(P)}, where $P=\{A,n\}$ denotes a plane, returns a sequence of three 3d points $A,B,C$, belonging to the plane, and such that $(A,\vec{AB},\vec{AC})$ is a direct orthonormal frame of this plane.
    \item The function \textbf{orthoframe(P)}, where $P=\{A,n\}$ denotes a plane, returns a sequence of three 3d points $A,u,v$, such that $(A,u,v)$ is a direct orthonormal frame of this plane.
\end{itemize}

\begin{demo}{Faces of a cube with holes in it and a regular hexagon}
\begin{luadraw}{name=plans}
local g = graph3d:new{window={-3,3,-3.25,3.25},margin={0,0,0,0},viewdir={20,60},bg="LightGray",size={10,10}}
Hiddenlines = true; Hiddenlinestyle = "dashed"
local p = polyreg(0,1,6)
local P = parallelep(M(-2,-2,-2),4*vecI,4*vecJ,4*vecK)
local V = g:Sortpolyfacet(P)
local list = {}
g:Filloptions("full","Crimson",1,true); -- true pour le mode evenodd
g:Lineoptions("solid","Gold",8)
for _, F in  ipairs(V) do
    local P1 = plane(isobar3d(F),F[1],F[2]) -- plan de la facette F
    local A, u, v = orthoframe(P1)  -- repère orthonormé sur la facette avec centre de gravité comme origine
    local p1 = map(function(z) return A+z.re*u+z.im*v end,p) -- hexagone reproduit sur la facette
    table.insert(p1,2,"m")
    local color = "Crimson"
    if not g:Isvisible(F) then  color = "Crimson!60!black" end
    g:Dpath3d( concat(F,{"l"},p1,{"l","cl"}),"fill="..color ) -- dessin de la facette "trouée" avec l'hexagone
end
g:Show()
\end{luadraw}
\end{demo}

\subsection{Circumscribed Sphere, Inscribed Sphere: circumsphere(), insphere()}

\begin{itemize}
    \item The function \textbf{circumsphere(A,B,C,D)}, where $A$, $B$, $C$, and $D$ are four non-coplanar 3d points, returns the sphere circumscribed within the tetrahedron formed by these four points, as a sequence: $A,R$, where $A$ is the center of the sphere, and $R$ its radius.
    \item The function \textbf{insphere(A,B,C,D)}, where $A$, $B$, $C$, and $D$ are four non-coplanar 3d points, returns the sphere inscribed within the tetrahedron formed by these four points, as a sequence: $A,R$, where $A$ is the center of the sphere, and $R$ its radius.
\end{itemize}

\subsection{Fixed-length tetrahedron: tetra\_len()}

The function \textbf{tetra\_len(ab,ac,ad,bc,bd,cd)} calculates the vertices $A,B,C,D$ of a tetrahedron whose edge lengths are given, i.e., such that $AB=ab$, $AC=ac$, $AD=ad$, $BC=bc$, $BD=bd$, and $CD=cd$. The function returns the sequence of four points $A,B,C,D$. Vertex $A$ is always the point $M(0,0,0)$ (\emph{Origin}) and vertex $B$ is always the point \emph{ab*vecI} and vertex $C$ in the $xOy$ plane. The tetrahedron as a polyhedron can then be constructed with the function \textbf{tetra(A,B-A,C-A,D-A)}.

\begin{demo}{A tetrahedron with fixed edge lengths}
\begin{luadraw}{name=tetra_len}
local g = graph3d:new{window={-4,4,-4,4},margin={0,0,0,0},viewdir={25,65},size={10,10}}
Hiddenlines = true; Hiddenlinestyle = "dashed"
require 'luadraw_spherical'
local R = 4
local A,B,C,D = tetra_len(R,R,R,R,R,R)
local T = tetra(A,B-A,C-A,D-A)
g:Define_sphere({radius=R})
g:DSpolyline( facetedges(T), {color="DarkGreen"})
g:DSbigcircle( {B,C},{color="Blue"} )
g:DSbigcircle( {B,D},{color="Blue"} )
g:DSbigcircle( {C,D},{color="Blue"}  )
g:DSlabel("$R$",(2*A+C)/3,{pos="S"})
g:Dspherical()
g:Ddots3d({A,B,C,D})
g:Dlabel3d("$A$",A,{pos="S"},"$B$",B,{pos="SW"},"$C$",C,{},"$D$",D,{pos="N"} )
g:Show()
\end{luadraw}
\end{demo}

\subsection{Triangles: sss\_triangle3d(), sas\_triangle3d(), asa\_triangle3d()}

These functions are the 3D version of the sss\_triangle(), sas\_triangle(), and asa\_triangle() functions already described.
\begin{itemize}
    \item The function \textbf{sss\_triangle3d(ab,bc,ca)}, where \emph{ab}, \emph{bc}, and \emph{ca} are three lengths, computes and returns a list of three 3D points $\{A,B,C\}$ forming the vertices of a direct triangle in the $xOy$ plane, whose side lengths are the arguments, i.e., $AB=ab$, $BC=bc$, and $CA=ca$, when possible. Vertex $A$ is always point $M(0,0,0)$ (\emph{Origin}) and vertex $B$ is always point \emph{ab*vecI}. This triangle can be drawn with the method \textbf{g:Dpolyline3d}.
    \item The function \textbf{sas\_triangle3d(ab,alpha,ca)} where \emph{ab} and \emph{ca} are two lengths, \emph{alpha} an angle in degrees, computes and returns a list of three 3d points $\{A,B,C\}$ forming the vertices of a triangle in the plane $xOy$ such that $AB=ab$, $CA=ca$, and such that the angle $(\vec{AB},\vec{AC})$ has measure \emph{alpha}, when possible. Vertex $A$ is always point $M(0,0,0)$ (\emph{Origin}) and vertex $B$ is always point \emph{ab*vecI}. This triangle can be drawn with the method \textbf{g:Dpolyline3d}.
    \item The function \textbf{asa\_triangle3d(alpha,ab,beta)} where \emph{ab} is a length, \emph{alpha} and \emph{beta} are two angles in degrees, computes and returns a list of three 3d points $\{A,B,C\}$ forming the vertices of a triangle in the $xOy$ plane such that $AB=ab$, such that angle $(\vec{AB},\vec{AC})$ has measure \emph{alpha}, and such that angle $(\vec{BA},\vec{BC})$ has measure \emph{beta}, when possible. Vertex $A$ is always point $M(0,0,0)$ (\emph{Origin}) and vertex $B$ is always point \emph{ab*vecI}. This triangle can be drawn with the \textbf{g:Dpolyline3d} method.
\end{itemize}
