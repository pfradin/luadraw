\section{Introduction}

\subsection{Prerequisites}

\begin{itemize}
    \item This document presents the use of the \emph{luadraw} package with the \emph{3d} global option:
\verb|\usepackage[3d]{luadraw}|.
    \item The package loads the \emph{luadraw\_graph2d.lua} module, which defines the \emph{graph} class and provides the \emph{luadraw} environment for creating graphs in Lua. Everything said in the previous chapter (Drawing 2d) therefore applies, and is assumed to be known here.
    \item The \emph{3d} global option also allows the loading of the \emph{luadraw\_graph3d.lua} module. This also defines the \emph{graph3d} class (which relies on the \emph{graph} class) for 3D drawings.
\end{itemize}

\subsection{Some reminders}

\begin{itemize}
    \item Another global package option: \emph{noexec}. When this global option is mentioned, the default value of the \emph{exec} option for the \emph{luadraw} environment will be false (and no longer true).

    \item When a graph is finished, it is exported in tikz format, so this package also loads the tikz package and the libraries:

    \begin{itemize}
        \item\emph{patterns}
        \item\emph{plotmarks}
        \item\emph{arrows.meta}
        \item\emph{decorations.markings}
    \end{itemize}
    
    \item Graphs are created in a luadraw environment, which calls luacode, so the textbf Lua language must be used in this environment.

    \item Saving the tkz file: the chart is exported in tikz format to a file (with the \emph{tkz} extension). By default, it is saved in the \emph{\_luadraw} folder, which is a subfolder of the current folder (containing the master document), but it is possible to specify a path to another subfolder. with the global option \emph{cachedir=}.

    \item The environment options are:
    
    \begin{itemize}
        \item \emph{name = \ldots{}}: allows you to name the resulting tikz file. It is given a name without an extension (the extension will be automatically added; it is \emph{.tkz}). If this option is omitted, then a default name is used, which is the name of the master file followed by a number.     \item \emph{exec = true/false}: Allows you to execute or not the Lua code included in the environment. By default, this option is true, \textbf{EXCEPT} if the global option \emph{noexec} was mentioned in the preamble with the package declaration. When a complex graph that requires a lot of calculations is developed, it may be useful to add the option \emph{exec=false}; this will avoid recalculating the same graph for future compilations.
        \item \emph{auto = true/false}: Allows you to automatically include or not the tikz file in place of the \emph{luadraw} environment when the \emph{exec} option is false. By default, the \emph{auto} option is true.
    \end{itemize}
\end{itemize}


\subsection{Creating a 3D Graph}

\begin{TeXcode}
\begin{luadraw}{ name=<filename>, exec=true/false, auto=true/false }
-- create a new graph and give it a local name
local g = graph3d:new{ window3d={x1,x2,y1,y2,z1,z2}, adjust2d=true/false, viewdir={30,60}, window={x1,x2,y1,y2,xscale,yscale}, margin={top,right,bottom,left}, size={width,height,ratio}, bg="color", border=true/false }
-- build graph g
graph instructions in Lua language ...
-- display graph g and save it in the file <filename>.tkz
g:Show()
-- or Save only in the <filename>.tkz file
g:Save()
\end{luadraw}
\end{TeXcode}

Creation is done in a \emph{luadraw} environment. This creation is done on the first line inside the environment by naming the graph:

\begin{Luacode}
local g = graph3d:new{ window3d={x1,x2,y1,y2,z1,z2}, adjust2d=true/false, viewdir={30,60}, window={x1,x2,y1,y2,xscale,yscale}, margin={left,right,top,bottom}, size={width,height,ratio}, bg="color", border=true/false }
\end{Luacode}

The \emph{graph3d} class is defined in the \emph{luadraw} package using the global option \emph{3d}. This class is instantiated by invoking its constructor and giving it a name (here it's \emph{g}). This is done locally so that the graph \emph{g} thus created will no longer exist once it leaves the environment (otherwise \emph{g} would remain in memory until the end of the document).

\begin{itemize}
    \item The (optional) parameter \emph{window3d} defines the $\mathbf R^3$ block corresponding to the graph: it is $[x_1,x_2]\times[y_1,y_2]\times[z_1,z_2]$. By default, it is $[-5,5]\times[-5,5]\times[-5,5]$.
    \item The (optional) \emph{adjust2d} parameter indicates whether the 2D window that will contain the orthographic projection of the 3D drawing should be determined automatically (false by default). This 2D window corresponds to the \emph{window} argument.

    \item The (optional) parameter \emph{viewdir} is a table that defines the two viewing angles (in degrees) used for the orthographic projection, which is the default projection (\emph{viewdir=\{30,60\}} by default). The following figure shows what these two angles correspond to.

\begin{center}
\begin{luadraw}{name=viewdir}
local g = graph3d:new{ size={8,8}, margin={0,0,0,0} }
local i = cpx.I
local O, A = Origin, M(4,4,4)
local B, C, D, E = pxy(A), px(A), py(A), pz(A)
g:Dpolyline3d( {{O,A},{-5*vecI,5*vecI},{-5*vecJ,5*vecJ},{-5*vecK,5*vecK}}, "->")
g:Dpolyline3d( {{E,A,B,O}, {C,B,D}}, "dashed")
g:Dpath3d( {C,O,B,2.5,1,"ca",O,"l","cl"}, "draw=none,fill=cyan,fill opacity=0.8")
g:Darc3d(C,O,B,2.5,1,"->")
g:Dpath3d( {E,O,A,2.5,1,"ca",O,"l","cl"}, "draw=none,fill=cyan,fill opacity=0.8")
g:Darc3d(E,O,A,2.5,1,"->")
g:Dballdots3d(O)
g:Labelsize("footnotesize")
g:Dlabel3d(
    "$x$", 5.25*vecI,{}, "$y$", 5.25*vecJ,{}, "$z$", 5.25*vecK,{},
    "vers observateur", A, {pos="E"},
    "$O$", O, {pos="NW"},
    "$\\theta$", (B+C)/2, {pos="N", dist=0.15},
    "$\\varphi$", (A+E)/2, {pos="S",dist=0.25}
)
g:Dlabel("viewdir=\\{$\\theta,\\varphi$\\} (en degrés)",-5*i,{pos="N"})
g:Show()            
\end{luadraw}
\captionof{figure}{Viewing Angles}\label{viewdir}
\end{center}

    \item The other parameters are those of the \emph{graph} class, described in Chapter 1.
\end{itemize}

\paragraph{Graph construction.}

\begin{itemize}
    \item The instantiated object (\emph{g} in the example) has all the methods of the \emph{graph} class, plus methods specific to 3D.
    \item The \emph{graph3d} class also provides a number of mathematical functions specific to 3D.
\end{itemize}

\subsection{Affine Projection Modes}

By default, \emph{luadraw} uses orthographic projection (orthogonal projection on the screen). This is defined by two angles given to the \emph{viewdir} option during creation, or to the \emph{g:Setviewdir()} method.

There are three other possible affine projection modes, but they are not orthogonal projections:

\begin{itemize}
    \item three cavalier perspectives: on the $yz$ plane, or on the $xz$ plane, or on the $xy$ plane. These are defined using two parameters: a positive number $k$ and an angle in degrees \emph{alpha}, which are explained in the following figure.
    \item an isometric perspective.
\end{itemize}

\begin{center}
\begin{luadraw}{name=perpectives}
local k = 0.65
local alpha = 60
local r = 3
local g = graph3d:new{
    window3d = {-4.5,4.5,-4.5,4.5,-4.5,4.5},
    window ={-5,5,-5,5},
    viewdir = perspective("yz",k,alpha),
    size={10,10},
    --bbox = false
}
g:Labelsize("footnotesize")
local draw = function()
    g:Dscene3d( g:addAxes(Origin, {arrows=1}) )
    g:Darc(1,0,Zp(1,alpha*deg),(r+0.5)*k,1,"->"); g:Ddots(r*k)
    g:Dlabel("$\\alpha$",Zp((r+1)*k,30*deg),{pos="NE"}, "$k$", r*k,{pos="SE"})
    g:Dcircle(0,r*k)
end
g:Dline(0,1); g:Dline(0,cpx.I)
-- top left
g:Saveattr(); g:Viewport(-5,0,0,5); g:Coordsystem(-4.5,5.5,-5,5)
draw()
g:Ddots3d(r*vecI); g:Dlabel3d("$x=1$",r*vecI,{pos="W",dist=0.1}); 
g:Dlabel("perspective on yz plane",Z(0.5,-5),{pos="N",node_options="fill=white"})
g:Restoreattr()

-- top right
g:Saveattr(); g:Viewport(0,5,0,5); g:Coordsystem(-4.75,5.5,-5,5)
g:Setviewdir(perspective("xz",k,alpha))
draw()
g:Ddots3d(r*vecJ); g:Dlabel3d("$y=1$",r*vecJ,{pos="NW"}); 
g:Dlabel("perspective on xz plane",Z(0.5,-5),{pos="N",node_options="fill=white"})
g:Restoreattr()

-- bottom left
g:Saveattr(); g:Viewport(-5,0,-5,0); g:Coordsystem(-4.5,5.5,-5,5.5)
g:Setviewdir(perspective("xy",k,alpha))
draw()
g:Ddots3d(r*vecK); g:Dlabel3d("$z=1$",r*vecK,{pos="W"}); 
g:Dlabel("perspective on xy plane",Z(0.5,-5),{pos="N",node_options="fill=white"})
g:Restoreattr()

-- bottom right
g:Saveattr(); g:Viewport(0,5,-5,0); g:Coordsystem(-4.5,5.5,-5,5.5)
g:Setviewdir(perspective("iso"))
g:Dscene3d( g:addAxes(Origin, {arrows=1}) ); g:Dcircle(0, cpx.abs(g:Proj3d(r*vecI)))
g:Ddots3d({r*vecI,r*vecJ,r*vecK}); 
g:Dlabel3d("$z=1$",r*vecK,{pos="NW"}, "$x=1$",r*vecI,{pos="NW"},"$y=1$",r*vecJ,{pos="NE"});
g:Dlabel("isometric perspective",Z(0.5,-5),{pos="N",node_options="fill=white"})
g:Restoreattr()
g:Show()
\end{luadraw}
\captionof{figure}{Affine Projection Modes}
\end{center}

These projection modes are accessed via the function \textbf{perspective(mode,k,alpha)}, where the argument \emph{mode} can be "yz", "xz", or "xy" (for the three cavalier perspectives) or "iso" for the isometric perspective. If \emph{mode} has an unrecognized value, then the orthographic projection is selected. For the first three modes, the values ​​of the parameters $k$ (0.5 by default) and \emph{alpha} (45 by default) are also provided; these values ​​are unnecessary for the fourth mode.

This function is used either with the \emph{viewdir} option when creating the graphic object, for example:

\begin{Luacode}
local g = graph3d:new{ viewdir = perspective("yz",0.65,60) }
\end{Luacode}

or during graph creation with the \emph{g:Setviewdir()} method:

\begin{Luacode}
g:Setviewdir(perspective("yz",0.65,60))
\end{Luacode}

\subsection{Central Projection}

Since version 2.4, \emph{luadraw} also offers central projection. Unlike previous modes, \textbf{this projection is not affine}, and furthermore, it is not defined for all points in space, which can lead to errors, thus requiring thought and adjustments. This projection is defined by:

\begin{itemize}
    \item A camera, which is a point in space stored in a variable called \emph{camera} and which should not be modified directly.
    \item A target, which is a point in space stored in a variable called \emph{target} and which should not be modified directly.
\end{itemize}
\section{}
The plane passing through the \emph{target} and orthogonal to the \emph{target} - \emph{camera} axis is the projection plane; it represents the screen. As with the previous modes, the central projection is accessed via the \emph{perspective} function:\par
\hfil\textbf{perspective("central",camera,target)},\hfil\par
or\par
\hfil\textbf{perspective("central",theta,phi,d,target)},\hfil\par
In the first case, the values ​​of \emph{camera} and \emph{target} are given (3D points; by default, \emph{target} is the origin). In the second case, the three arguments \emph{theta}, \emph{phi}, and \emph{d} are used to position the camera according to the following diagram:

\begin{center}
\begin{luadraw}{name=central_perspective}
local g = graph3d:new{window3d={-5,3,-5,3,-5,5}, window={-5,10,-7,7}, size={12,12}, margin={0,0,0,0}, bbox=false, viewdir=perspective("central",-60,65,35)}
g:Writeln("\\tikzset{->-/.style= {decoration={markings, mark=at position #1 with {\\arrow{stealth}}}, postaction={decorate}}}")
g:Labelsize("footnotesize")
local dcamera = function(pos,dir)
    local a, b, c, d, e = 0.25, 0.25, 0.1, 0.2, 0.1
    local u = cpx.normalize(dir)
    local v = cpx.I*u
    local chem, dep = {}, pos+e*u-b*v
    chem = {dep,dep+2*a*u,dep+2*a*u+2*b*v,dep+2*b*v,0.15,"cla", dep+(b-c)*v,"m",pos-d*v,pos+d*v,dep+(b+c)*v,"l"}
    g:Dpath(chem)
end
local d = 10
local N = pt3d.normalize(M(2,1,1.5))
local v = pt3d.normalize(pt3d.prod(N,vecJ))
local u = pt3d.prod(v,N)
local Cam = d*N
local A = pxy(Cam)
local B, C = M(0,2,2.5), M(3,-1,4)
local E = (B+C)/2+vecK
local F = isobar3d({B,C,E})+1.51*vecJ
local T = tetra(B,E-B,C-B,F-B)
local B1, C1, E1, F1 = proj3dO(B, {Origin,N}, Cam-B), proj3dO(C, {Origin,N}, Cam-C), proj3dO(E, {Origin,N}, Cam-E), proj3dO(F, {Origin,N}, Cam-F)
local x, y = 12,8
local D = Origin -x*u/2-y*v/2.5
local plan = {D,D+x*u,D+x*u+y*v,D+y*v}  
g:Dscene3d(
    g:addFacet(plan, {color="white", contrast=0.125, edge=true}),
    g:addPoly(T, {color="white", contrast=0.25, edge=true, hidden=true, hiddenstyle="dashed"}),
    g:addPolyline( {{(Cam+B)/2,B1},{C,C1},{E,E1},{F,F1}}, {width=2,color="gray", hidden=true}),
    g:addPolyline({{-5*vecK,5*vecK},{-5*vecI,4*vecI}}, {arrows=1, hidden=true})
)
g:Dballdots3d(Origin,nil,1.5)
g:Ddots3d({B1,C1, E1, F1},"gray")
g:Dpolyline3d({Origin, A, Cam},true,"dashed")
g:Dangle3d(Cam,A,Origin)
g:Dangle3d(Cam,Origin,v,0.3,"line width=0.8pt")
g:Darc3d(vecI,Origin,A,2.25,1,'-stealth'); g:Darc3d(vecK,Origin,Cam,2.25,1,'-stealth')
g:Dpolyline3d({{Cam,(B+Cam)/2},{Cam,C},{Cam,E},{Cam,F}},true,"->-=0.35,line width=0.1pt,gray")
g:Dpolyline3d({{B1,C1,E1,F1},{C1,F1}},true,"gray")
g:Dpolyline3d({B1,E1},true,"dotted,gray")
g:Ddots3d({B, C, E, F})
g:Dlabel3d("screen plane",D,{pos="NE",dir={u,v}})
g:Dlabel3d("target",Origin, {pos="S",dist=0.1}, "camera",Cam,{pos="SE"}, "$A$",B,{pos="N"},"$B$",C,{pos="S"},"$C$",E,{pos="N"},
"$D$",F,{},"$A'$",B1,{node_options="gray",dist=0}, "$B'$",C1,{pos="NW"}, "$C'$",E1,{pos="N"},"$D'$",F1,{}, "$\\theta$", 2.75*vecI,{pos="N",node_options="black"},"$\\varphi$", 1.3*(vecK+N), {}, "$z$",5*vecK,{},"$x$",4.5*vecI,{pos="center"})
local O = M(-3,-2,-4)
g:Ddots3d(O); g:Dpolyline3d({{O,O+vecI},{O,O+vecJ},{O,O+vecK}},'->')
g:Dlabel3d("$x$",O+1.25*vecI,{},"$y$",O+1.25*vecJ,{},"$z$",O+1.25*vecK,{},"Origin",O,{pos="S"})
u = g:Proj3dV(Cam)
g:Dpolyline3d( {Cam/2-0.4*vecK-Cam/3.5, Cam/2-0.4*vecK+Cam/3.5}, 'stealth-stealth')
g:Dlabel("$d$ = distance target - camera", g:Proj3d(Cam/2-0.4*vecK),{dir={u, cpx.I*u}, node_options="fill=white"})
dcamera(u,u)
g:Show()
\end{luadraw}
\captionof{figure}{Central projection}
\end{center}

The default values ​​are: \emph{theta=30} (degrees), \emph{phi=60}, \emph{d=15}, \emph{target=Origin}. This function is used either with the \emph{viewdir} option when creating the graph object, for example:
\begin{Luacode}
local g = graph3d:new{ viewdir = perspective("central",40,60) }
\end{Luacode}
or during graph creation with the \emph{g:Setviewdir()} method:
\begin{Luacode}
g:Setviewdir(perspective("central",40,60))
\end{Luacode}
