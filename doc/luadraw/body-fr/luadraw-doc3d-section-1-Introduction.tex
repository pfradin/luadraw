\section{Introduction}

\subsection{Prérequis}

\begin{itemize}
\item Ce document présente l'utilisation du package \emph{luadraw} avec l'option globale \emph{3d} :
\verb|\usepackage[3d]{luadraw}|.
\item Le paquet charge le module \emph{luadraw\_graph2d.lua} qui définit la classe \emph{graph}, et fournit l'environnement \emph{luadraw} qui permet de faire des graphiques en Lua. Tout ce qui est dit dans le précédent chapitre (Dessin 2d) s'applique donc, et est supposé connu ici.
\item L'option globale \emph{3d} permet en plus le chargement du module \emph{luadraw\_graph3d.lua}. Celui-ci définit en plus la classe \emph{graph3d} (qui s'appuie sur la classe \emph{graph}) pour des dessins en 3d. 
\end{itemize}

\subsection{Quelques rappels}

\begin{itemize}
    \item Autre option globale du paquet : \emph{noexec}. Lorsque cette option globale est mentionnée la valeur par défaut de l'option \emph{exec} pour l'environnement \emph{luadraw} sera false (et non plus true).

    \item Lorsqu'un graphique est terminé il est exporté au format tikz, donc ce paquet charge également le paquet \emph{tikz} ainsi que les librairies :
    \begin{itemize}
        \item\emph{patterns}
        \item\emph{plotmarks}
        \item\emph{arrows.meta}
        \item\emph{decorations.markings}
        \end{itemize}
    \item Les graphiques sont créés dans un environnement \emph{luadraw}, celui-ci appelle \emph{luacode}, c'est donc du \textbf{langage Lua} qu'il faut utiliser dans cet environnement.

    \item Sauvegarde du fichier \emph{.tkz} : le graphique est exporté au format tikz dans un fichier (avec l'extension \emph{tkz}), par défaut celui-ci est sauvegardé dans le dossier \emph{\_luadraw} qui est un sous-dossier du dossier courant (contenant le document maître), mais il est possible d'imposer un chemin vers un autre sous-dossier avec l'option globale \emph{cachedir=}.
    
    \item Les options de l'environnement sont :
    \begin{itemize}
    \item \emph{name = \ldots{}} : permet de donner un nom au fichier tikz produit, on donne un nom sans extension (celle-ci sera automatiquement ajoutée, c'est \emph{.tkz}). Si cette option est omise, alors il y a un nom par défaut, qui est le nom du fichier maître suivi d'un numéro.
    \item \emph{exec = true/false} : permet d'exécuter ou non le code Lua compris dans l'environnement. Par défaut cette option vaut true, \textbf{SAUF} si l'option globale \emph{noexec} a été mentionnée dans le préambule avec la déclaration du paquet. Lorsqu'un graphique complexe qui demande beaucoup de calculs est au point, il peut être intéressant de lui ajouter l'option \emph{exec=false}, cela évitera les recalculs de ce même graphique pour les compilations à venir.
    \item \emph{auto = true/false} : permet d'inclure ou non automatiquement le fichier tikz en lieu et place de l'environnement \emph{luadraw} lorsque l'option \emph{exec} est à false. Par défaut l'option \emph{auto} vaut true.
    \end{itemize}
\end{itemize}


\subsection{Création d'un graphe 3d}

\begin{TeXcode}
\begin{luadraw}{ name=<filename>, exec=true/false, auto=true/false }
-- création d'un nouveau graphique en lui donnant un nom local
local g = graph3d:new{ window3d={x1,x2,y1,y2,z1,z2}, adjust2d=true/false, viewdir={30,60}, window={x1,x2,y1,y2,xscale,yscale}, margin={top,right,bottom,left}, size={largeur,hauteur,ratio}, bg="color", border=true/false }
-- construction du graphique g
    instructions graphiques en langage Lua ...
-- affichage du graphique g et sauvegarde dans le fichier <filename>.tkz
g:Show()
-- ou bien sauvegarde uniquement dans le fichier <filename>.tkz
g:Save()
\end{luadraw}
\end{TeXcode}

La création se fait dans un environnement \emph{luadraw}, c'est à la première ligne à l'intérieur de l'environnement qu'est faite cette création en nommant le graphique :

\begin{Luacode}
local g = graph3d:new{ window3d={x1,x2,y1,y2,z1,z2}, adjust2d=true/false, viewdir={30,60}, window={x1,x2,y1,y2,xscale,yscale}, margin={left,right,top,bottom}, size={largeur,hauteur,ratio}, bg="color", border=true/false }
\end{Luacode}

La classe \emph{graph3d} est définie dans le paquet \emph{luadraw} grâce à l'option globale \emph{3d}. On instancie cette classe en invoquant son constructeur et en donnant un nom (ici c'est \emph{g}), on le fait en local de sorte que le graphique \emph{g} ainsi créé, n'existera plus une fois sorti de l'environnement (sinon \emph{g} resterait en mémoire jusqu'à la fin du document).

\begin{itemize}
 \item Le paramètre (facultatif) \emph{window3d} définit le pavé de $\mathbf R^3$ correspondant au graphique : c'est $[x_1,x_2]\times[y_1,y_2]\times[z_1,z_2]$. Par défaut c'est $[-5,5]\times[-5,5]\times[-5,5]$.
 \item Le paramètre (facultatif) \emph{adjust2d} indique si la fenêtre 2d qui va contenir la projection orthographique du dessin 3d, doit être déterminée automatiquement (false par défaut). Cette fenêtre 2d correspond à l'argument \emph{window}.
 
 \item Le paramètre (facultatif) \emph{viewdir} est une table qui définit les deux angles de vue (en degrés) utilisés pour la projection orthographique, qui est la projection par défaut (\emph{viewdir=\{30,60\}} par défaut). La figure suivante montre à quoi correspondent ces deux angles.
 
\begin{center}
\begin{luadraw}{name=viewdir}
local g = graph3d:new{ size={8,8}, margin={0,0,0,0} }
local i = cpx.I
local O, A = Origin, M(4,4,4)
local B, C, D, E = pxy(A), px(A), py(A), pz(A)
g:Dpolyline3d( {{O,A},{-5*vecI,5*vecI},{-5*vecJ,5*vecJ},{-5*vecK,5*vecK}}, "->")
g:Dpolyline3d( {{E,A,B,O}, {C,B,D}}, "dashed")
g:Dpath3d( {C,O,B,2.5,1,"ca",O,"l","cl"}, "draw=none,fill=cyan,fill opacity=0.8")
g:Darc3d(C,O,B,2.5,1,"->")
g:Dpath3d( {E,O,A,2.5,1,"ca",O,"l","cl"}, "draw=none,fill=cyan,fill opacity=0.8")
g:Darc3d(E,O,A,2.5,1,"->")
g:Dballdots3d(O)
g:Labelsize("footnotesize")
g:Dlabel3d(
    "$x$", 5.25*vecI,{}, "$y$", 5.25*vecJ,{}, "$z$", 5.25*vecK,{},
    "vers observateur", A, {pos="E"},
    "$O$", O, {pos="NW"},
    "$\\theta$", (B+C)/2, {pos="N", dist=0.15},
    "$\\varphi$", (A+E)/2, {pos="S",dist=0.25}
)
g:Dlabel("viewdir=\\{$\\theta,\\varphi$\\} (en degrés)",-5*i,{pos="N"})
g:Show()            
\end{luadraw}
\captionof{figure}{Angles de vue}\label{viewdir}
\end{center}

\item Les autres paramètres sont ceux de la classe \emph{graph}, ils ont été décrits dans le chapitre 1.
\end{itemize}



\paragraph{Construction du graphique.}

\begin{itemize}
    \item L'objet instancié possède toutes les méthodes de la classe \emph{graph}, plus des méthodes spécifiques à la 3d.
    \item La classe \emph{graph3d} amène aussi un certain nombre de fonctions mathématiques propres à la 3d.
\end{itemize}


\subsection{Modes de projection affine}

Par défaut \emph{luadraw} utilise la projection orthographique (projection orthogonale sur l'écran), celle-ci est définie par deux angles qui dont donnés à l'option \emph{viewdir} lors de la création, ou bien à la méthode \emph{g:Setviewdir()}.

Il y a trois autres modes de projection affine possible, mais qui ne sont pas des projections orthogonales:
\begin{itemize}
    \item trois perspectives cavalières : sur le plan $yz$, ou sur le plan $xz$, ou sur le plan $xy$. Celles-ci sont définies à l'aide de deux paramètres : un nombre positif $k$ et un angle en degré \emph{alpha}, qui sont mis en évidence dans la figure suivante.
    \item une perspective isométrique.
\end{itemize}

\begin{center}
\begin{luadraw}{name=perpectives}
local k = 0.65
local alpha = 60
local r = 3
local g = graph3d:new{
    window3d = {-4.5,4.5,-4.5,4.5,-4.5,4.5},
    window ={-5,5,-5,5},
    viewdir = perspective("yz",k,alpha),
    size={10,10},
    --bbox = false
}
g:Labelsize("footnotesize")
local draw = function()
    g:Dscene3d( g:addAxes(Origin, {arrows=1}) )
    g:Darc(1,0,Zp(1,alpha*deg),(r+0.5)*k,1,"->"); g:Ddots(r*k)
    g:Dlabel("$\\alpha$",Zp((r+1)*k,30*deg),{pos="NE"}, "$k$", r*k,{pos="SE"})
    g:Dcircle(0,r*k)
end
g:Dline(0,1); g:Dline(0,cpx.I)
-- top left
g:Saveattr(); g:Viewport(-5,0,0,5); g:Coordsystem(-4.5,5.5,-5,5)
draw()
g:Ddots3d(r*vecI); g:Dlabel3d("$x=1$",r*vecI,{pos="W",dist=0.1}); 
g:Dlabel("perspective on yz plane",Z(0.5,-5),{pos="N",node_options="fill=white"})
g:Restoreattr()

-- top right
g:Saveattr(); g:Viewport(0,5,0,5); g:Coordsystem(-4.75,5.5,-5,5)
g:Setviewdir(perspective("xz",k,alpha))
draw()
g:Ddots3d(r*vecJ); g:Dlabel3d("$y=1$",r*vecJ,{pos="NW"}); 
g:Dlabel("perspective on xz plane",Z(0.5,-5),{pos="N",node_options="fill=white"})
g:Restoreattr()

-- bottom left
g:Saveattr(); g:Viewport(-5,0,-5,0); g:Coordsystem(-4.5,5.5,-5,5.5)
g:Setviewdir(perspective("xy",k,alpha))
draw()
g:Ddots3d(r*vecK); g:Dlabel3d("$z=1$",r*vecK,{pos="W"}); 
g:Dlabel("perspective on xy plane",Z(0.5,-5),{pos="N",node_options="fill=white"})
g:Restoreattr()

-- bottom right
g:Saveattr(); g:Viewport(0,5,-5,0); g:Coordsystem(-4.5,5.5,-5,5.5)
g:Setviewdir(perspective("iso"))
g:Dscene3d( g:addAxes(Origin, {arrows=1}) ); g:Dcircle(0, cpx.abs(g:Proj3d(r*vecI)))
g:Ddots3d({r*vecI,r*vecJ,r*vecK}); 
g:Dlabel3d("$z=1$",r*vecK,{pos="NW"}, "$x=1$",r*vecI,{pos="NW"},"$y=1$",r*vecJ,{pos="NE"});
g:Dlabel("isometric perspective",Z(0.5,-5),{pos="N",node_options="fill=white"})
g:Restoreattr()
g:Show()
\end{luadraw}
\captionof{figure}{Modes de projection affine}
\end{center}

Ces modes de projection sont accessibles par la fonction \textbf{perspective(mode,k,alpha)}, où l'argument \emph{mode} peut valoir "yz", ou "xz" ou "xy" (pour les trois perspectives cavalières) ou "iso" pour la perspective isométrique, si \emph{mode} a une valeur non reconnue alors c'est la projection orthographique qui est sélectionnée. Pour les trois premiers modes, on donne également les valeurs des paramètres $k$ (0.5 par défaut) et \emph{alpha} (45 par défaut), ces valeurs sont inutiles pour le quatrième mode.

Cette fonction s'utilise soit avec l'option \emph{viewdir} à la création de l'objet graphique, par exemple :
\begin{Luacode}
local g = graph3d:new{ viewdir = perspective("yz",0.65,60) }
\end{Luacode}
ou bien pendant la création du graphique avec la méthode \emph{g:Setviewdir()} :
\begin{Luacode}
g:Setviewdir(perspective("yz",0.65,60))
\end{Luacode}

\subsection{Projection centrale}

Depuis la version 2.4, \emph{luadraw} propose également la projection centrale. À la différence des modes précédents, \textbf{cette projection n'est pas affine}, et d'autre part elle n'est pas définie pour tous les points de l'espace, ce qui peut conduire à des erreurs, cela demande donc de la réflexion et des ajustements. Cette projection est définie par :
\begin{itemize}
    \item Une camera, qui est un point de l'espace mémorisé dans une variable appelée \emph{camera} et qui ne  doit pas être modifiée directement.
    \item Une cible, qui est un point de l'espace mémorisé dans une variable appelée \emph{target} et qui ne  doit pas être modifiée directement.
\end{itemize}
Le plan passant par la \emph{target} et orthogonal à l'axe \emph{target} - \emph{camera} est le plan de la projection, il représente l'écran. Comme pour les modes précédents, la projection centrale est accessible par la fonction \emph{perspective} :\par
\hfil\textbf{perspective("central",camera,target)},\hfil\par
ou bien \par
\hfil\textbf{perspective("central",theta,phi,d,target)},\hfil\par
dans le premier cas on donne les valeurs de \emph{camera} et \emph{target} (points 3d, par défaut \emph{target} est l'origine). Dans le second cas, les trois arguments \emph{theta}, \emph{phi} et \emph{d} servent à positionner la caméra conformément au schéma suivant :

\begin{center}
\begin{luadraw}{name=central_perspective}
local g = graph3d:new{window3d={-5,3,-5,3,-5,5}, window={-5,10,-7,7}, size={12,12}, margin={0,0,0,0}, bbox=false, viewdir=perspective("central",-60,65,35)}
g:Writeln("\\tikzset{->-/.style= {decoration={markings, mark=at position #1 with {\\arrow{stealth}}}, postaction={decorate}}}")
g:Labelsize("footnotesize")
local dcamera = function(pos,dir)
    local a, b, c, d, e = 0.25, 0.25, 0.1, 0.2, 0.1
    local u = cpx.normalize(dir)
    local v = cpx.I*u
    local chem, dep = {}, pos+e*u-b*v
    chem = {dep,dep+2*a*u,dep+2*a*u+2*b*v,dep+2*b*v,0.15,"cla", dep+(b-c)*v,"m",pos-d*v,pos+d*v,dep+(b+c)*v,"l"}
    g:Dpath(chem)
end
local d = 10
local N = pt3d.normalize(M(2,1,1.5))
local v = pt3d.normalize(pt3d.prod(N,vecJ))
local u = pt3d.prod(v,N)
local Cam = d*N
local A = pxy(Cam)
local B, C = M(0,2,2.5), M(3,-1,4)
local E = (B+C)/2+vecK
local F = isobar3d({B,C,E})+1.51*vecJ
local T = tetra(B,E-B,C-B,F-B)
local B1, C1, E1, F1 = proj3dO(B, {Origin,N}, Cam-B), proj3dO(C, {Origin,N}, Cam-C), proj3dO(E, {Origin,N}, Cam-E), proj3dO(F, {Origin,N}, Cam-F)
local x, y = 12,8
local D = Origin -x*u/2-y*v/2.5
local plan = {D,D+x*u,D+x*u+y*v,D+y*v}  
g:Dscene3d(
    g:addFacet(plan, {color="white", contrast=0.125, edge=true}),
    g:addPoly(T, {color="white", contrast=0.25, edge=true, hidden=true, hiddenstyle="dashed"}),
    g:addPolyline( {{(Cam+B)/2,B1},{C,C1},{E,E1},{F,F1}}, {width=2,color="gray", hidden=true}),
    g:addPolyline({{-5*vecK,5*vecK},{-5*vecI,4*vecI}}, {arrows=1, hidden=true})
)
g:Dballdots3d(Origin,nil,1.5)
g:Ddots3d({B1,C1, E1, F1},"gray")
g:Dpolyline3d({Origin, A, Cam},true,"dashed")
g:Dangle3d(Cam,A,Origin)
g:Dangle3d(Cam,Origin,v,0.3,"line width=0.8pt")
g:Darc3d(vecI,Origin,A,2.25,1,'-stealth'); g:Darc3d(vecK,Origin,Cam,2.25,1,'-stealth')
g:Dpolyline3d({{Cam,(B+Cam)/2},{Cam,C},{Cam,E},{Cam,F}},true,"->-=0.35,line width=0.1pt,gray")
g:Dpolyline3d({{B1,C1,E1,F1},{C1,F1}},true,"gray")
g:Dpolyline3d({B1,E1},true,"dotted,gray")
g:Ddots3d({B, C, E, F})
g:Dlabel3d("screen plane",D,{pos="NE",dir={u,v}})
g:Dlabel3d("target",Origin, {pos="S",dist=0.1}, "camera",Cam,{pos="SE"}, "$A$",B,{pos="N"},"$B$",C,{pos="S"},"$C$",E,{pos="N"},
"$D$",F,{},"$A'$",B1,{node_options="gray",dist=0}, "$B'$",C1,{pos="NW"}, "$C'$",E1,{pos="N"},"$D'$",F1,{}, "$\\theta$", 2.75*vecI,{pos="N",node_options="black"},"$\\varphi$", 1.3*(vecK+N), {}, "$z$",5*vecK,{},"$x$",4.5*vecI,{pos="center"})
local O = M(-3,-2,-4)
g:Ddots3d(O); g:Dpolyline3d({{O,O+vecI},{O,O+vecJ},{O,O+vecK}},'->')
g:Dlabel3d("$x$",O+1.25*vecI,{},"$y$",O+1.25*vecJ,{},"$z$",O+1.25*vecK,{},"Origin",O,{pos="S"})
u = g:Proj3dV(Cam)
g:Dpolyline3d( {Cam/2-0.4*vecK-Cam/3.5, Cam/2-0.4*vecK+Cam/3.5}, 'stealth-stealth')
g:Dlabel("$d$ = distance target - camera", g:Proj3d(Cam/2-0.4*vecK),{dir={u, cpx.I*u}, node_options="fill=white"})
dcamera(u,u)
g:Show()
\end{luadraw}
\captionof{figure}{Projection centrale}
\end{center}

Les valeurs par défaut sont : \emph{theta=30} (degrés),  \emph{phi=60},  \emph{d=15},  \emph{target=Origin}. Cette fonction s'utilise soit avec l'option \emph{viewdir} à la création de l'objet graphique, par exemple :
\begin{Luacode}
local g = graph3d:new{ viewdir = perspective("central",40,60) }
\end{Luacode}
ou bien pendant la création du graphique avec la méthode \emph{g:Setviewdir()} :
\begin{Luacode}
g:Setviewdir(perspective("central",40,60))
\end{Luacode}
