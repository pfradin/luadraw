\section{Historique}

\subsection{Version 2.5}
Liste non exhaustive :
\begin{itemize}
    \item Ajout de la fonction \emph{read\_csv\_file()}\footnote{Sur une idée de Christophe BAL.} qui permet de lire un fichier \emph{csv} avec différentes options.
    \item L'extension \emph{luadraw\_palettes} est passée à la version 1.3.0 du projet \verb|@prism| de \href{https://github.com/projetmbc/for-writing/tree/main/@prism}{Christphe BAL}.
    \item Ajout de la méthode \emph{g:Dshadedpolyline()} qui permet de dessiner une ligne polygonale 2d avec un dégradé de couleurs en fonction de la méthode de calcul et de la palette choisies.
    \item Ajout de la méthode \emph{g:Dpolynames()} qui permet d'afficher un polyèdre avec le numéro des faces et/ou ceux des sommets.
    \item Ajout de l'extension \emph{luadraw\_cvx\_polyhedra\_nets} qui permet de déterminer un patron des polyèdres convexes.
    \item Correction de bug...
\end{itemize}

\subsection{Version 2.4}
Liste non exhaustive :
\begin{itemize}
    \item Ajout de la projection centrale.
    \item Ajout de l'option \emph{legendstyle} pour les axes, pour imposer un style de label ("auto", "N", "E", ...) pour les légendes lorsqu'il y en a (jusque là, le style était forcément "auto"). 
    \item Ajout de la méthode \emph{g:Labeldir()} qui permet de gérer globalement le sens de l'écriture.
    \item Ajout des fonctions \emph{interCS()} (intersection entre un cercle dans l'espace et une sphère), et de la fonction \emph{interSSS()} (intersection entre 3 sphères).
    \item Ajout de la fonction \emph{voronoi()} en complément de la triangulation de Delaunay, elle permet de faire des diagrammes de Voronoï.
    \item Ajout de la fonction \emph{parallel\_polyline()} qui renvoie une ligne polygonale parallèle.
    \item Ajout de la fonction \emph{tangent\_from()} et de la méthode \emph{g:Dtangent\_from()} qui permet de tracer les tangentes à une courbe donnée issues d'un point donné.
    \item Correction de bug...
\end{itemize}

\subsection{Version 2.3}
Liste non exhaustive :
\begin{itemize}
\item Ajout des projections en perspective cavalière: sur $yz$, sur $xz$ ou sur $xy$, ainsi que de la projection isométrique.
\item Ajout de la fonction \emph{section2tube()}.
\item Ajout du module \emph{luadraw\_compile\_tex}.
\item Ajout de la méthode \emph{Proj3dV} pour le calcul de la projection des vecteurs de l'espace sur le plan de l'écran.
\item Ajout des fonctions \emph{circumcircle()} et \emph{incircle()} en 2d, elles renvoient une séquence: centre et rayon.
\item Ajout de la fonction \emph{line2strip()} qui renvoie un chemin représentant une "bande" centrée sur une ligne polygonale donnée.
\item Ajout de la fonction \emph{delaunay()} qui fait une triangulation de Delaunay sur une liste de points et renvoie la liste des triangles obtenus.
\item Ajout de la fonction \emph{cpx.normalize(z)} qui renvoie le complexe $z$ divisé par son module (ou \emph{nil} s'il est nul).
\item Ajout de l'instruction \emph{whatis(variable, msg)} qui affiche dans le terminal le statut d'une \emph{variable} (accompagné du message \emph{msg}) et son contenu.
\item Correction de bug...    
\end{itemize}


\subsection{Version 2.2}
Liste non exhaustive :
\begin{itemize}
    \item Ajout de l'option \emph{clip} pour les méthodes : \emph{Dfacet()}, \emph{Dmixfacet()}, \emph{addFacet()}, \emph{addPoly()} et \emph{addPolyline()}, ainsi que pour les méthodes de dessin de nuages de points, et les méthodes de dessin "au trait" comme \emph{Dpolyline3d()}, \emph{Dparametric3d()}, \emph{Dpath3d()}, etc.
    \item Ajout de l'option \emph{xyzstep} pour la méthode \emph{Dboxaxes3d()}, cette option définit un pas commun aux trois axes ($1$ par défaut).
    \item Ajout des méthodes \emph{DSdots()}, \emph{DSstars()}, \emph{DSinvstereo\_curve()} et \emph{DSinvstereo\_polyline()} dans le module \emph{luadraw\_spherical}.
    \item Ajout du module \emph{luadraw\_palettes}.
    \item Ajout de la fonction \emph{interDC()} (intersection entre une droite et un cercle en 2d) et de la fonction \emph{interCC()} (intersection entre 2 cercles en 2d).
    \item Ajout des fonctions \emph{curvilinear\_param()} et \emph{curvilinear\_param3d()} qui permettent d'obtenir une paramétrisation d'une liste de points (2d pour l'une, et 3d pour l'autre) avec une fonction d'une variable $t$ entre $0$ et $1$.
    \item Ajout de la fonction \emph{cvx\_hull2d()} qui renvoie l'enveloppe convexe (ligne polygonale) d'une liste de points en 2d, et de la fonction \emph{cvx\_hull3d()} qui renvoie l'enveloppe convexe (liste de facettes) d'une liste de points en 3d.
    \item Ajout des méthodes \emph{g:Beginclip(<chemin>)} et \emph{g:Endclip()} qui facilitent la mise en place d'un clipping par tikz.
    \item Ajout des fonctions \emph{normal()}, \emph{normalC()}, \emph{normalI()} qui renvoient la normale à une courbe 2d en un point donné. Les méthodes graphiques correspondantes ont également été ajoutées.
    \item Ajout de la fonction \emph{isobar()} qui renvoie l'isobarycentre d'une liste de complexes.
    \item Ajout de l'option \emph{usepalette=\{palette,mode\}} pour les méthodes \emph{Dpoly}, \emph{Dfacet}, \emph{Dmixfacet}, \emph{addFacet}.
    \item Ajout de la fonction \emph{clipplane()} qui permet de clipper un plan avec un polyèdre convexe, la fonction renvoie la section, si elle existe, sous forme d'une facette.
    \item Ajout des fonctions \emph{cartesian3d()} et \emph{cylindrical\_surface()} qui calculent et renvoient des surfaces avec la possibilité d'ajouter ou non des cloisons séparatrices pour la méthode \emph{Dscene3d()}.
    \item Ajout de la fonction \emph{evalf(f,...)} qui permet une évaluation protégée de $f(...)$, elle renvoie le résultat de l'évaluation s'il n'y a pas d'erreur d'exécution de la part de Lua, sinon, elle renvoie \emph{nil} mais sans provoquer la fin de l'exécution du script.
    \item Ajout de la fonction \emph{split\_points\_by\_visibility()} (3d) pour séparer une courbe en deux parties : partie visible, partie cachée.
    \item Dans les méthodes \emph{g:Dfacet}, \emph{g:Dmixfacet}, \emph{g:Dpoly}, \emph{g:Dedges}, \emph{g:addFacet}, \emph{g:addPolyline}, \emph{g:addPoly}, les valeurs par défaut des options de tracé de lignes (épaisseur, couleur et style), sont les valeurs courantes en cours.
    \item Correction de bug...    
\end{itemize}

\subsection{Version 2.1}
Liste non exhaustive :
\begin{itemize}
    \item Par défaut, les fichiers tikz sont sauvegardés dans un sous-dossier appelé \emph{\_luadraw}. La nouvelle option de package \emph{cachedir} permet d'en changer.
    \item L'option \emph{line join = round} est automatiquement ajoutée à l'environnement \emph{tikzpicture}.
    \item Deux options supplémentaires pour l'environnement \emph{luadraw} : \emph{bbox} et \emph{pictureoptions}.
    \item Un certain nombre de fonctions de constructions géométriques supplémentaires en 2d et 3d.
    \item Les axes gradués (2d, 3d) utilisent le package \emph{siunitx}  pour formater les labels lorsque la variable globale \emph{siunitx} a la valeur \emph{true}.
    \item Ajout des cônes tronqués droits ou penchés (\textbf{frustum} et \textbf{Dfrustum}).
    \item Ajout des pyramides régulières (\textbf{regular\_pyramid} et pyramides tronquées \textbf{truncated\_pyramid}).
    \item Les cylindres et les cônes ne sont plus forcément droits, ils peuvent désormais être penchés.
    \item Ajout de la fonction \textbf{cutpolyline(L,D,close)}.    
    \item Dessin (élémentaire) d'ensembles (fonction \emph{set}) et opérations sur les ensembles (\emph{cap}, \emph{cup}, \emph{setminus}).
    \item Modification de l'argument \emph{mode} de la méthode \textbf{g:Dplane}.
    \item Ajout de l'option \emph{close} pour la méthode \textbf{g:addPolyline}.
    \item Correction de bug...
\end{itemize}

\subsection{Version 2.0}

\begin{itemize}
    \item Introduction du module \emph{luadraw\_graph3d.lua} pour les dessins en 3d.
    \item Introduction de l'option \emph{dir} pour la méthode \textbf{g:Dlabel}.
    \item Menus changements dans la gestion des couleurs.
\end{itemize}

\subsection{Version 1.0}
Première version.
