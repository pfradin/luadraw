\documentclass[%
10pt,%
oneside,
a4paper,%
%french,%
]%
{book}%


\usepackage{luatextra}
\usepackage{amsmath,amssymb,marvosym,stmaryrd,calrsfs,minted,setspace,array,caption,fvextra}%
\usepackage{unicode-math}
\usepackage{fourier-otf}%
\usepackage[svgnames]{xcolor}
\usepackage[margin=1.5cm,top=1cm,bottom=1.5cm,headheight=5mm,headsep=2mm]{geometry}
\usepackage[unicode,pdfstartview = FitH, colorlinks, linkcolor=blue]{hyperref}
\usepackage{babel,microtype,multicol}%
%\usepackage{minitoc}
\usepackage[Lenny]{fncychap}

\usepackage[noexec,3d,cachedir=tkz]{luadraw}
\def\version{2.3}

\hypersetup{
  pdfauthor={P. Fradin},
  pdflang={en},
  %hidelinks,
  pdfcreator={LuaLatex}
}

\title{\textbf{The package} \emph{luadraw} \\ version \version\\ 2d and 3d drawing with lua (and tikz).\\
{\small \url{https://github.com/pfradin/luadraw}} \\

\vspace{1cm}

\begin{minipage}{16cm}
\small

\hfil\textbf{Abstract}\hfil

The \emph{luadraw} package defines the environment of the same name, which lets you create mathematical graphs using the Lua language. These graphs are ultimately drawn by tikz (and automatically saved), so why make them in Lua? Because Lua brings all the power of a simple, efficient programming language with good computational capabilities.
\end{minipage}
}
\author{Patrick Fradin}
%\date{}


\onehalfspacing
\newminted{Lua}{bgcolor=Beige,ignorelexererrors=true,linenos,numbersep=6pt,breaklines,fontsize=\footnotesize}%
\newminted{TeX}{bgcolor=Gray!30,breaklines,fontsize=\footnotesize}%

%\columnseprule 0.4pt
\columnsep 25pt

\definecolor{bgcol}{RGB}{238,240,252}% couleur de fond des listings

\newenvironment*{demo}[2][]{%
\gdef\legende{#2}%
\gdef\lab{#1}%
\bgroup
\VerbatimOut{\jobname.tmp}%
}%
{%
\endVerbatimOut%
\egroup%
%\begin{center}
%\begin{tabular}{|m{10cm}|m{7cm}|}
%\hline
%\begin{minipage}{10cm}
\inputminted[ignorelexererrors=true,breaklines,bgcolor=Beige,linenos,numbersep=6pt,frame=single,fontsize=\footnotesize]{Lua}{\jobname.tmp}%
%\end{minipage}
%&
\begin{minipage}{0.9\textwidth}
%\par\smallskip
\begin{center}
\captionof{figure}{\legende}\label{\lab}%
\input{\jobname.tmp}%
%\end{center}
%\par\smallskip
%\hline
%\end{tabular}
\end{center}
\end{minipage}
}

\begin{document}
\maketitle

\setcounter{tocdepth}{3}%
\setcounter{secnumdepth}{2}%

\begin{multicols}{2}
\tableofcontents
\end{multicols}

\listoffigures

\renewcommand{\labelitemi}{$\bullet$}
\renewcommand{\labelitemii}{--}
\renewcommand{\labelitemiii}{$*$}
%\renewcommand{\thechapter}{\Roman{chapter}}
\renewcommand{\thesection}{\Roman{section}~}
\renewcommand{\thesubsection}{\arabic{subsection})}
\renewcommand{\thesubsubsection}{\arabic{subsection}.\arabic{subsubsection}}
\renewcommand{\thefigure}{\arabic{figure}}
\newcommand{\og}{"}
\newcommand{\fg}{"}

\chapter{2D Drawing}

\begin{center}
\input{tkz/trigo.tkz}%
\par
\captionof{figure}{A first example: three sub-figures in the same graph}
\end{center}


\section{Introduction}

\subsection{Prerequisites}

\begin{itemize}
\item In the preamble, you must declare the \emph{luadraw} package: \verb|\usepackage[global options]{luadraw}|
\item Compilation is done with LuaLatex \textbf{exclusively}.
\item The colors in the \emph{luadraw} environment are strings that must correspond to colors known to tikz. It is strongly recommended to use the \emph{xcolor} package with the \emph{svgnames} option.
\end{itemize}

Regardless of the global options chosen, this package loads the \emph{luadraw\_graph2d.lua} module, which defines the \emph{graph} class, and provides the \emph{luadraw} environment for creating graphs in Lua.

\paragraph{Global package options}: \emph{noexec}, \emph{3d}, and \emph{cachedir=}.

\begin{itemize}
\item \emph{noexec}: When this global option is specified, the default value of the \emph{exec} option for the \emph{luadraw} environment will be false (and no longer true).
\item \emph{3d}: When this global option is specified, the \emph{luadraw\_graph3d.lua} module is also loaded. This module also defines the \emph{graph3d} class (which relies on the \emph{graph} class) for 3D drawings.
\item \emph{cachedir = <folder>}: By default, the created files are saved in the \emph{\_luadraw} folder, which is a subfolder of the current folder (containing the master document). This folder can be changed with the \emph{cachedir} option, for example \emph{cachedir = \{tikz\}}.
\end{itemize}

\noindent\textbf{NB}: In this chapter, we will not discuss the \emph{3d} option. This is the subject of the next chapter. We will therefore only discuss the 2d version.

When a graph is finished, it is exported in tikz format, so this package also loads the tikz package and the libraries:
\begin{itemize}
\item\emph{patterns}
\item\emph{plotmarks}
\item\emph{arrows.meta}
\item\emph{decorations.markings}
\end{itemize}

Graphs are created in a luadraw environment, which calls luacode, so the lua language must be used in this environment:

\begin{TeXcode}
\begin{luadraw}{ name=<filename>, exec=true/false, auto=true/false }
-- create a new graph with a local name
local g = graph:new{ window={x1,x2,y1,y2,xscale,yscale}, margin={top,right,bottom,left},
size={width,height,ratio}, bg="color", border=true/false }
-- build the g chart
graphic instructions in Lua language ...
-- display the g chart and save it in the <filename>.tkz file
g:Show()
-- or save it only in the <filename>.tkz file
g:Save()
\end{luadraw}
\end{TeXcode}

\paragraph{Saving the \emph{.tkz} file}: the chart is exported in tikz format to a file (with the \emph{tkz} extension). By default, it is saved in the \emph{\_luadraw} folder, which is a subfolder of the current folder (containing the master document), but it is possible to specify a path to another subfolder. with the global option \emph{cachedir=}.

\subsection{Environment Options}

These are:
\begin{itemize}
\item \emph{name = \ldots{}}: Allows you to name the produced tikz file. It is given a name without an extension (this will be automatically added; it is \emph{.tkz}). If this option is omitted, then a default name is the name of the master file followed by a number.
\item \emph{exec = true/false}: Allows you to execute or not the Lua code included in the environment. By default, this option is true, \textbf{UNLESS} if the global option \emph{noexec} was mentioned in the preamble with the package declaration. When a complex graph that requires a lot of calculations is ready, it may be useful to add the \emph{exec=false} option. This will prevent recalculations of the same graph for future compilations.

\emph{auto = true/false}: Allows you to automatically include or exclude the tikz file in place of the \emph{luadraw} environment when the \emph{exec} option is set to false. By default, the \emph{auto} option is true.

\end{itemize}

\subsection{The cpx (complex) class}

It is automatically loaded by the \emph{luadraw\_graph2d} module and therefore when the \emph{luadraw} package is loaded. This class allows you to manipulate complex numbers and perform common calculations. We create a complex number with the function \textbf{Z(a,b)} for \(a + i\times b\), or with the function \textbf{Zp(r,theta)} for \(r\times e^{i\theta}\) in polar coordinates.

\begin{itemize}
\item Example: \emph{local z = Z(a,b)} will create the complex number corresponding to \(a + i\times b\) in the variable \emph{z}. We then access the real and imaginary parts of \emph{z} like this: \emph{z.re} and \emph{z.im}.
\item \textbf{Warning}: A real number \emph{x} is not considered complex by Lua. However, the functions provided for graphical constructions perform the verification and conversion from real to complex. However, we can use \emph{Z(x,0)} instead of \emph{x}.
\item The usual operators have been overloaded, allowing the use of the usual symbols, namely: +, x, -, /, as well as the equality test with =. When a calculation fails, the returned result should normally be equal to \emph{nil}.
In addition, the following functions are added (dot notation must be used in Lua):
    \begin{itemize}
    \item modulus: \textbf{cpx.abs(z)},
    \item modulus squared: \textbf{cpx.abs2(z)},
    \item normalization: \textbf{cpx.normalize(z)} (returns \emph{nil} if $z$ is null),
    \item norm 1: \textbf{cpx.N1(z)},
    \item main argument: \textbf{cpx.arg(z)},
    \item conjugate: \textbf{cpx.bar(z)},
    \item complex exponential: \textbf{cpx.exp(z)},
    \item scalar product: \textbf{cpx.dot(z1,z2)}, where the complex numbers represent vector affixes,
    \item determinant: \textbf{cpx.det(z1,z2)},
    \item the oriented angle (in radians) between two non-zero vectors: \textbf{cpx.angle(z1,z2)}
    \item rounding: \textbf{cpx.round(z, number of decimals)},
    \item the function: \textbf{cpx.isNul(z)} tests whether the real and imaginary parts of \emph{z} are in absolute value less than a variable \emph{epsilon} which is equal to \emph{1e-16} by default.
    \end{itemize}
\end{itemize}

The last function returns a Boolean, the bar, exponential, and round functions return a complex number, and the others return a real number.

We also have the constant \emph{cpx.I} which represents the pure imaginary \emph{i}.

Example:

\begin{Luacode}
local i = cpx.I
local A = 2+3*i
\end{Luacode}

The multiplication symbol is required.

\subsection{Displaying a Variable in the Terminal}

The instruction \textbf{whatis(variable,msg)} displays the type of the \emph{variable} and its contents in the terminal during compilation. Recognized types include the predefined types plus: \emph{complex number}, \emph{list of (complex) numbers}, and \emph{list of lists of (complex) numbers}. The argument \emph{msg} is an optional string (empty by default) which is displayed with the type to locate the variable in the terminal.

\subsection{Creating a Graph}

As seen above, creation is done in a \emph{luadraw} environment. This creation is done by naming the graph:

\begin{Luacode}
local g = graph:new{ window={x1,x2,y1,y2,xscale,yscale}, margin={left,right,top,bottom},
size={width,height,ratio}, bg="color", border=true/false, bbox=true/false, pictureoptions="" }
\end{Luacode}

The \emph{graph} class is defined in the \emph{luadraw} package. This class is instantiated by invoking its constructor and giving it a name (here it's \emph{g}). This is done locally so that the graph \emph{g} thus created will no longer exist once it leaves the environment (otherwise \emph{g} would remain in memory until the end of the document).

\begin{itemize}
\item The (optional) parameter \emph{window} defines the $\mathbf R^2$ block corresponding to the graph: it is $[x_1,x_2]\times[y_1,y_2]$. The \emph{xscale} and \emph{yscale} parameters are optional and set to $1$ by default; they represent the scale (cm per unit) on the axes. By default, we have \emph{window = \{-5.5,-5.5,1,1\}}.

\item The (optional) \emph{margin} parameter sets the margins around the graph in cm. By default, we have \emph{margin = \{0.5,0.5,0.5,0.5\}}.

\item The (optional) \emph{size} parameter allows you to impose a size (in cm, including margins) for the graph. The \emph{ratio} argument corresponds to the desired scale ratio (\emph{xscale}/\emph{yscale}). A ratio of 1 will result in an orthonormal coordinate system, and if the ratio is not specified, the default ratio is retained. Using this parameter will modify the values ​​of \emph{xscale} and \emph{yscale} to obtain the correct sizes. By default, the size is $11\times 11$ (in cm) with margins ($10\times 10$ without margins).

\item The (optional) \emph{bg} parameter allows you to define a background color for the graph. This color is a string representing a color for tikz. By default, this string is empty, meaning the background will not be painted.

\item The (optional) \emph{border} parameter indicates whether or not a frame should be drawn around the graph (including the margins). By default, this parameter is set to \emph{false}.

\item The (optional) \emph{bbox} parameter indicates whether a bounding box should be added to the graph so that it has the desired size. Everything outside of it is clipped by tikz. By default, this parameter is set to \emph{true}. With the value \emph{false}, no bounding box is added, but everything outside the 2D window, except for the paths, is clipped by luadraw. The graph size can be smaller than the requested size.

\item The (optional) parameter \emph{pictureoptions} is a string containing options that will be passed to \emph{tikzpicture} like this:
\begin{TeXcode}
\begin{tikzpicture}[line join=round <,pictureoptions>]
\end{TeXcode}
\end{itemize}

\paragraph{Graph construction.}

\begin{itemize}
\item The instantiated object (\emph{g} in the example) has several methods for drawing (segments, lines, curves, etc.). Drawing instructions are not sent directly to \TeX; they are stored as strings in a table that is a property of the \emph{g} object. The \textbf{g:Show()} method will send these instructions to \TeX while saving them in a text file.\footnote{This file will contain a \emph{tikzpicture} environment.} The \textbf{g:Save()} method saves the graph in the file designated by the (environment) option \emph{name} but without sending the instructions to \TeX.
\item The current graph can be saved to another file with the \textbf{g:Savetofile(<filename with extension>)} method.
\item A current graph can be reset, i.e., delete all elements already created, with the \textbf{g:Cleargraph()} method.
\item The \emph{luadraw} package also provides a number of mathematical functions, as well as functions for calculating lists (tables) of complex numbers, geometric transformations, etc.
\end{itemize}

\paragraph{Coordinate system. Location}

\begin{itemize}
\item The instantiated object (\emph{g} in the example) has:
\begin{enumerate}
\item An original view: this is the $\mathbf R^2$ block defined by the \emph{window} option at creation. This \textbf{must not be modified} subsequently.
\item A current view: this is a $\mathbf R^2$ block that must be included in the original view; anything outside this block will be clipped. By default, the current view is the original view. To retrieve the current view, you can use the \textbf{g:Getview()} method, which returns a table \verb|{x1,x2,y1,y2}|, representing the block $[x1,x2]\times [y1,y2]$.
\item A transformation matrix: this is initialized to the identity matrix. During a drawing instruction, the points are automatically transformed by this matrix before being sent to tikz.
\item A coordinate system (Cartesian coordinate system) linked to the current view; this is the user's coordinate system. By default, this is the canonical coordinate system of $\mathbf R^2$, but it is possible to change it. Let's say the current view is the $[-5,5]\times[-5,5]$ block. It is possible, for example, to decide that this block represents the $[-1,12]$ interval for the abscissas and the $[0,8]$ interval for the ordinates. The method that makes this change will modify the graph's transformation matrix, so that for the user, everything happens as if they were in the $[-1,12]\times [0,8]$ block. The intervals of the user's coordinate system can be retrieved using the methods: \textbf{g:Xinf(), g:Xsup(), g:Yinf(), and g:Ysup()}.
\end{enumerate}
\item Complex numbers are used to represent points or vectors in the user's Cartesian coordinate system.
\item In the tikz export, the coordinates will be different because the lower left corner (excluding margins) will have coordinates $(0,0)$, and the upper right corner (excluding margins) will have coordinates corresponding to the size (excluding margins) of the graph, and with $1$ cm per unit on both axes. This means that normally, tikz should only handle \og small\fg\ numbers. \item The conversion is done automatically with the \textbf{g:strCoord(x,y)} method, which returns a string of the form \emph{(a,b)}, where \emph{a} and \emph{b} are the coordinates for tikz, or with the \textbf{g:Coord(z)} method, which also returns a string of the form \emph{(a,b)} representing the tikz coordinates of the point with affix \emph{z} in the user's coordinate system.
\end{itemize}

\subsection{Can we use tikz directly in the \emph{luadraw} environment?}

Suppose we are creating a graph named \emph{g} in a \emph{luadraw} environment. It is possible to write a tikz instruction during this creation, but not using \verb|tex.sprint("<tikz instruction>")|, because then this instruction would not be part of the graph \emph{g}. To do this, you must use the method \textbf{g:Writeln("<tikz instruction>")}, with the constraint that \textbf{the backslashes must be doubled}, and without forgetting that the graphic coordinates of a point in \emph{g} are not the same for tikz. For example:
\begin{Luacode}
g:Writeln("\\draw"..g:Coord(Z(1,-1)).." node[red] {Text};")
\end{Luacode}

Or to change styles:
\begin{Luacode}
g:Writeln("\\tikzset{every node/.style={fill=white}}")
\end{Luacode}

In a Beamer presentation, this can also be used to insert pauses in a graph:
\begin{Luacode}
g:Writeln("\\pause")
\end{Luacode}
%
\section{Graphics Methods}

Polygonal lines, curves, paths, points, and labels can be created.

\subsection{Polygonal Lines}

\textbf{A polygonal line is a list (table) of connected components, and a connected component is a list (table) of complex numbers that represent the affixes of the points}. For example, the instruction:
\begin{Luacode}
local L = { {Z(-4,0), Z(0,2), Z(1,3)}, {Z(0,0), Z(4,-2), Z(1,-1)} }
\end{Luacode}
creates a polygonal line with two components in a variable \emph{L}.

\paragraph{Drawing a polygonal line.}

This is the \textbf{g:Dpolyline(L,close, draw\_options,clip)} method (where \emph{g} denotes the graphic being created), \emph{L} is a polygonal line, \emph{close} is an optional argument equal to \emph{true} or \emph{false} indicating whether the line should be closed or not (\emph{false} by default), and \emph{draw\_options} is a string that will be passed directly to the \emph{\textbackslash draw} instruction in the export. The argument \emph{clip} must contain either \emph{nil} (default value) or a table of the form \emph{\{x1,x2,y1,y2\}}, in the first case the line is clipped by the current 2d window \textbf{after} its transformation by the 2d matrix of the graph, in the second case the line is clipped by the window $[x_1,x_2]\times[y_1,y_2]$ \textbf{before} being transformed by the graph matrix.


\paragraph{Choosing options for drawing a polygonal line.}

Drawing options can be passed directly to the \emph{\textbackslash draw} instruction in the export, but they will have only a local effect. These options can be modified globally using the:
\par\hfil\textbf{g:Lineoptions(style,color,width,arrows)}\hfil\par
 method (when one of the arguments is nil, its default value is used):

\begin{itemize}
    \item \emph{color} is a string representing a color known to tikz (default is black),
    \item \emph{style} is a string representing the type of line to draw. This style can be:
\begin{itemize}
    \item \emph{"noline"}: undrawn line,
    \item \emph{"solid"}: solid line (default value),
    \item \emph{"dotted"}: dotted line,
    \item \emph{"dashed"}: dashed line,
    \item custom style: the \emph{style} argument can be a string of the form (example): \emph{"\{2.5pt\}\{2pt\}"} which means: a 2.5pt line followed by a 2pt space, the number of values ​​can be greater than 2, e.g.: \emph{"\{2.5pt\}\{2pt\}\{1pt\}\{2pt\}"} (succession of on, off).
\end{itemize}
    \item \emph{width} is a number representing the line thickness expressed in tenths of a point, for example $8$ for an actual thickness of 0.8pt (default value of $4$),
    \item \emph{arrows} is a string that specifies the type of arrow that will be drawn, this can be:
\begin{itemize}
    \item \emph{"-"} which means no arrow (default value),
    \item \emph{"->"} which means an arrow at the end,
    \item \emph{"<-"} which means an arrow at the beginning,
    \item \emph{"<->"} which means an arrow at each end.\\
\textbf{WARNING}: The arrow is not drawn when the argument \emph{close} is true. \end{itemize}
\end{itemize}

The options can be modified individually using the following methods:
\begin{itemize}
    \item \textbf{g:Linecolor(color)},
    \item \textbf{g:Linestyle(style)},
    \item \textbf{g:Linewidth(width)},
    \item \textbf{g:Arrows(arrows)},
    \item plus the following methods:
\begin{itemize}
    \item \textbf{g:Lineopacity(opacity)}, which sets the opacity of the line drawing. The \emph{opacity} argument must be a value between $0$ (fully transparent) and $1$ (fully opaque). The default value is $1$.     \item \textbf{g:Linecap(style)}, to adjust the line ends, the \emph{style} argument is a string that can be:
\begin{itemize}
    \item \emph{"butt"} (straight end at the breakpoint, default value),
    \item \emph{"round"} (rounded semicircle end),
    \item \emph{"square"} (rounded square end).
\end{itemize}

    \item \textbf{g:Linejoin(style)}, to adjust the join between segments, the style argument is a string that can be:
\begin{itemize}
    \item \emph{"miter"} (pointed corner, default value),
    \item \emph{"round"} (or rounded corner),
    \item \emph{"bevel"} (cut corner).
\end{itemize} 
\end{itemize}
\end{itemize}

\paragraph{Polygonal line fill options.}

This is the \textbf{g:Filloptions(style,color,opacity,evenodd)} method (which uses the tikz \emph{patterns} library, which is loaded with the package). When one of the arguments is \emph{nil}, its default value is used:

\begin{itemize}
    \item \emph{color} is a string representing a color known to tikz (default is black).     \item \emph{style} is a string representing the fill type. This style can be:
\begin{itemize}
    \item \emph{"none"}: no fill, this is the default value,
    \item \emph{"full"}: full fill,
    \item \emph{"bdiag"}: descending hatching from left to right,
    \item \emph{"fdiag"}: ascending hatching from left to right,
    \item \emph{"horizontal"}: horizontal hatching,
    \item \emph{"`vertical"}: vertical hatching,
    \item \emph{"hvcross"}: horizontal and vertical hatching,
    \item \emph{"diagcross"}: descending and ascending diagonals,
    \item \emph{"gradient"}: in this case, the fill is done with a gradient defined with the method \textbf{g:Gradstyle(string)}, this style is passed as is to the \emph{\textbackslash draw} instruction. By default, the string defining the gradient style is \emph{"left color = white, right color = red"}.

Any other style known from the \emph{patterns} library is also possible.

\end{itemize}
\end{itemize}

Some options can be modified individually with the methods:
\begin{itemize}

    \item \textbf{g:Fillopacity(opacity)},

    \item \textbf{g:Filleo(evenodd)}.

\end{itemize}

\begin{demo}{Vector field, integral curve of $y'= 1-xy^2$}
\begin{luadraw}{name=champ}
local g = graph:new{window={-5,5,-5,5},bg="Cyan!30",size={10,10}}
local f = function(x,y) -- éq. diff. y'= 1-x*y^2=f(x,y)
    return 1-x*y^2     
end
local A = Z(-1,1) -- A = -1+i
local deltaX, deltaY, long = 0.5, 0.5, 0.4
local champ = function(f)
    local vecteurs, v = {}
    for y = g:Yinf(), g:Ysup(), deltaY do
        for x = g:Xinf(), g:Xsup(), deltaX do
            v = Z(1,f(x,y)) -- coordonnées 1 et f(x,y)
            v = v/cpx.abs(v)*long -- normalisation de v
            table.insert(vecteurs, {Z(x,y), Z(x,y)+v} )
        end
    end 
    return vecteurs -- vecteurs est une ligne polygonale
end
g:Daxes( {0,1,1}, {labelpos={"none","none"}, arrows="->"} )
g:Dpolyline( champ(f), "->,blue")
g:Dodesolve(f, A.re, A.im, {t={-2.75,5},draw_options="red,line width=0.8pt"})
g:Dlabeldot("$A$", A, {pos="S"})
g:Show()
\end{luadraw}
\end{demo}
\label{field}


\subsection{Segments and Lines}

\textbf{A segment is a list (table) of two complex numbers representing the endpoints. A line is a list (table) of two complex numbers, the first representing a point on the line, and the second a direction vector of the line (this must be non-zero).}

\subsubsection{Dangle}
\begin{itemize}
    \item The \textbf{g:Dangle(B,A,C,r,draw\_options)} method draws the angle \(BAC\) with a parallelogram (only two sides are drawn). The optional argument \emph{r} specifies the length of one side (0.25 by default). The argument \emph{draw\_options} is a string (empty by default) that will be passed as is to the \emph{\textbackslash draw} instruction.     \item The function \textbf{angleD(B,A,C,r)} returns the list of points of this angle.
\end{itemize}

\subsubsection{Dbissec}
\begin{itemize}
    \item The method \textbf{g:Dbissec(B,A,C,interior,draw\_options)} draws a bisector of the geometric angle BAC, interior if the optional argument \emph{interior} is \emph{true} (default value), exterior otherwise. The argument \emph{draw\_options} is a string (empty by default) that will be passed as is to the instruction
\emph{\textbackslash draw}.
    \item The function \textbf{bissec(B,A,C,interior)} returns this bisector as a list \emph{\{A,u\}} where \emph{u} is a direction vector of the line. \end{itemize}

\subsubsection{Dhline}
The method \textbf{g:Dhline(d,draw\_options)} draws a half-line. The argument \emph{d} must be a list of complex \emph{\{A,B\}} variables. The half-line $[A,B)$ is drawn.

Variant: \textbf{g:Dhline(A,B,draw\_options)}. The argument \emph{draw\_options} is a string (empty by default) that will be passed as is to the \emph{\textbackslash draw} instruction.

\subsubsection{Dline}
The \textbf{g:Dline(d,draw\_options)} method draws the line \emph{d}, which is a list of type \emph{\{A,u\}} where \emph{A} represents a point on the line (a complex) and \emph{u} a direction vector (a non-zero complex).

Variant: the \textbf{g:Dline(A,B,draw\_options)} method draws the line passing through the points \emph{A} and \emph{B} (two complex numbers). The argument \emph{draw\_options} is a string (empty by default) that will be passed as is to the \emph{\textbackslash draw} instruction.

\subsubsection{DlineEq}
\begin{itemize}
    \item The \textbf{g:DlineEq(a,b,c,draw\_options)} method draws the line with the equation \(ax+by+c=0\). The \emph{draw\_options} argument is a string (empty by default) that will be passed as is to the \emph{\textbackslash draw} instruction.
    \item The \textbf{lineEq(a,b,c)} function returns the line with the equation \(ax+by+c=0\) as a list \emph{\{A,u\}} where \emph{A} is a point on the line and \emph{u} is a direction vector. \end{itemize}

\subsubsection{Dmarkarc}
The method \textbf{g:Dmarkarc(b,a,c,r,n,long,space)} marks the arc of a circle with center \emph{a}, radius \emph{r}, extending from \emph{b} to \emph{c}, with \emph{n} small segments. By default, the length (argument \emph{long}) is 0.25, and the spacing (argument \emph{space}) is 0.0625.

\subsubsection{Dmarkseg}
The method \textbf{g:Dmarkseg(a,b,n,long,space,angle)} marks the segment defined by \emph{a} and \emph{b} with \emph{n} small segments inclined at \emph{angle} degrees (45° by default). By default, the length (argument \emph{long}) is 0.25, and the spacing (argument \emph{space}) is 0.125.

\subsubsection{Dmed}
\begin{itemize}
    \item The \textbf{g:Dmed(A,B, draw\_options)} method draws the perpendicular bisector of the segment $[A;B]$.

Variant: \textbf{g:Dmed(seg,draw\_options)} where segment is a list of two points representing the segment. The \emph{draw\_options} argument is a string (empty by default) that will be passed as is to the \emph{\textbackslash draw} instruction.     \item The function \textbf{med(A,B)} (or \textbf{med(seg)}) returns the perpendicular bisector of the segment \emph{{[}A,B{]}} as a list \emph{\{C,u\}} where \emph{C} is a point on the line and \emph{u} is a direction vector.
\end{itemize}

\subsubsection{Dparallel}
\begin{itemize}
    \item The method \textbf{g:Dparallel(d,A,draw\_options)} draws the parallel to \emph{d} passing through \emph{A}. The argument \emph{d} can be either a line (a list consisting of a point and a direction vector) or a non-zero vector. The argument \emph{draw\_options} is a string (empty by default) that will be passed as is to the instruction \emph{\textbackslash draw}.
    \item The function \textbf{parallel(d,A)} returns the parallel to \emph{d} passing through \emph{A} as a list \emph{\{B,u\}} where \emph{B} is a point on the line and \emph{u} is a direction vector. \end{itemize}

\subsubsection{Dperp}
\begin{itemize}
    \item The method \textbf{g:Dperp(d,A,draw\_options)} draws the perpendicular to \emph{d} passing through \emph{A}. The argument \emph{d} can be either a line (a list consisting of a point and a direction vector) or a non-zero vector. The argument \emph{draw\_options} is a string (empty by default) that will be passed as is to the instruction \emph{\textbackslash draw}.
    \item The function \textbf{perp(d,A)} returns the perpendicular to \emph{d} passing through \emph{A} as a list \emph{\{B,u\}} where \emph{B} is a point on the line and \emph{u} is a direction vector. \end{itemize}

\subsubsection{Dseg}
The \textbf{g:Dseg(seg,scale,draw\_options)} method draws the segment defined by the \emph{seg} argument, which must be a list of two complex numbers. The optional \emph{scale} argument (1 by default) is a number that allows you to increase or decrease the segment length (the natural length is multiplied by \emph{scale}). The \emph{draw\_options} argument is a string (empty by default) that will be passed as is to the \emph{\textbackslash draw} instruction.

\subsubsection{Dtangent}
\begin{itemize}
    \item The \textbf{g:Dtangent(p,t0,long,draw\_options)} method draws the tangent to the curve parameterized by \(p: t \mapsto p(t)\) (with complex values), at the parameter point \(t0\). If the argument \emph{long} is \emph{nil} (the default value), then the entire line is drawn; otherwise, it is a segment of length \emph{long}. The argument \emph{draw\_options} is a string (empty by default) that will be passed as is to the \emph{\textbackslash draw} instruction.
    \item The function \textbf{tangent(p,t0,long)} returns either the line or a segment (depending on whether \emph{long} is \emph{nil} or not). \end{itemize}

\subsubsection{ DtangentC}
\begin{itemize}
    \item The method \textbf{g:DtangentC(f,x0,long,draw\_options)} draws the tangent to the Cartesian curve with equation \(y=f(x)\), at the abscissa point \emph{x0}. If the argument \emph{long} is \emph{nil} (the default value), then the entire line is drawn; otherwise, it is a segment of length \emph{long}. The argument \emph{draw\_options} is a string (empty by default) that will be passed as is to the instruction \emph{\textbackslash draw}.
    \item The function \textbf{tangentC(f,x0,long)} returns either the line or a segment (depending on whether \emph{long} is \emph{nil} or not).
\end{itemize}

\subsubsection{ DtangentI}
\begin{itemize}
    \item The method \textbf{g:DtangentI(f,x0,y0,long,draw\_options)} draws the tangent to the implicit curve with equation \(f(x,y)=0\), at the assumed point \emph{(x0,y0)} on the curve. If the argument \emph{long} is \emph{nil} (the default value), then the entire line is drawn; otherwise, it is a segment of length \emph{long}. The argument \emph{draw\_options} is a string (empty by default) that will be passed as is to the instruction \emph{\textbackslash draw}.
    \item The function \textbf{tangentI(f,x0,y0,long)} returns either the line or a segment (depending on whether \emph{long} is \emph{nil} or not).
\end{itemize}

\subsubsection{Dnormal}
\begin{itemize}
    \item The method \textbf{g:Dnormal(p,t0,long,draw\_options)} draws the normal to the curve parameterized by \(p: t \mapsto p(t)\) (with complex values), at the point parameter \(t0\). If the argument \emph{long} is \emph{nil} (the default value), then the entire line is drawn; otherwise, it is a segment of length \emph{long}. The argument \emph{draw\_options} is a string (empty by default) that will be passed as is to the instruction \emph{\textbackslash draw}.
    \item The function \textbf{normal(p,t0,long)} returns either the line or a segment (depending on whether \emph{long} is \emph{nil} or not).
\end{itemize}

\subsubsection{ DnormalC}
\begin{itemize}
    \item The \textbf{g:DnormalC(f,x0,long,draw\_options)} method draws the normal to the Cartesian curve with equation \(y=f(x)\), at the abscissa point \emph{x0}. If the argument \emph{long} is \emph{nil} (the default value), then the entire line is drawn; otherwise, it is a segment of length \emph{long}. The argument \emph{draw\_options} is a string (empty by default) that will be passed as is to the \emph{\textbackslash draw} instruction.
    \item The function \textbf{normalC(f,x0,long)} returns either the line or a segment (depending on whether \emph{long} is \emph{nil} or not). \end{itemize}

\subsubsection{DnormalI}
\begin{itemize}
    \item The \textbf{g:DnormalI(f,x0,y0,long,draw\_options)} method draws the normal to the implicit curve with equation \(f(x,y)=0\), at the assumed point \emph{(x0,y0)} on the curve. If the \emph{long} argument is \emph{nil} (the default value), then the entire line is drawn; otherwise, it is a segment of length \emph{long}. The \emph{draw\_options} argument is a string (empty by default) that will be passed as is to the \emph{\textbackslash draw} instruction.
    \item The function \textbf{normalI(f,x0,y0,long)} returns either a line or a segment (depending on whether \emph{long} is \emph{nil} or not).
\end{itemize}

\begin{demo}{Symmetric of the orthocenter}
\begin{luadraw}{name=orthocentre}
local g = graph:new{window={-5,5,-5,5},bg="cyan!30",size={10,10}}
local i = cpx.I
local A, B, C = 4*i, -2-2*i, 3.5
local h1, h2 = perp({B,C-B},A), perp({A,B-A},C) -- hauteurs
local A1, F = proj(A,{B,C-B}), proj(C,{A,B-A}) -- projetés
local H = interD(h1,h2) -- orthocentre
local A2 = 2*A1-H -- symétrique de H par rapport à BC
g:Dpolyline({A,B,C},true, "draw=none,fill=Maroon,fill opacity=0.3") -- fond du triangle
g:Linewidth(6); g:Filloptions("full", "blue", 0.2)
g:Dangle(C,A1,A,0.25); g:Dangle(B,F,C,0.25) -- angles droits
g:Linecolor("black"); g:Filloptions("full","cyan",0.5)
g:Darc(H,C,A2,1); g:Darc(B,A,A1,1) -- arcs
g:Filloptions("none","black",1) -- on rétablit l'opacité à 1
g:Dmarkarc(H,C,A1,1,2); g:Dmarkarc(A1,C,A2,1,2) -- marques
g:Dmarkarc(B,A,H,1,2)
g:Linewidth(8); g:Linecolor("black")
g:Dseg({A,B},1.25); g:Dseg({C,B},1.25); g:Dseg({A,C},1.25) -- côtés
g:Linecolor("red"); g:Dcircle(A,B,C) -- cercle
g:Linecolor("blue"); g:Dline(h1); g:Dline(h2) -- hauteurs
g:Dseg({A2,C}); g:Linecolor("red"); g:Dseg({H,A2}) -- segments
g:Dmarkseg(H,A1,2); g:Dmarkseg(A1,A2,2) -- marques
g:Labelcolor("blue") -- pour les labels
g:Dlabel("$A$",A, {pos="NW",dist=0.1}, "$B$",B, {pos="SW"}, "$A_2$",A2,{pos="E"}, "$C$", C, {pos="S" }, "$H$", H, {pos="NE"}, "$A_1$", A1, {pos="SW"})
g:Linecolor("black"); g:Filloptions("full"); g:Ddots({A,B,C,H,A1,A2}) -- dessin des points
g:Show(true)
\end{luadraw}
\end{demo}

\subsection{Geometric Figures}

\subsubsection{Darc}
\begin{itemize}
    \item The \textbf{g:Darc(B,A,C,r,sens,draw\_options)} method draws an arc of a circle with center \emph{A} (complex), radius \emph{r}, going from \emph{B} (complex) to \emph{C} (complex) counterclockwise if the argument \emph{sens} is 1, and counterclockwise otherwise. The argument \emph{draw\_options} is a string (empty by default) that will be passed as is to the \emph{\textbackslash draw} instruction.
    \item The \textbf{arc(B,A,C,r,sens)} function returns the list of points on this arc (polygonal line).
    \item The function \textbf{arcb(B,A,C,r,sens)} returns this arc as a path (see Dpath) using Bézier curves.
\end{itemize}

\subsubsection{Dcircle}
\begin{itemize}
    \item The method \textbf{g:Dcircle(c,r,d,draw\_options)} draws a circle. When the argument \emph{d} is nil, it is the circle with center \emph{c} (complex) and radius \emph{r}; when \emph{d} is specified (complex), it is the circle passing through the affix points \emph{c}, \emph{r}, and \emph{d}. The argument \emph{draw\_options} is a string (empty by default) that will be passed as is to the instruction \emph{\textbackslash draw}. Another possible syntax: \textbf{g:Dcircle(C,draw\_options)} where \emph{C=\{c,r,d\}}.
    \item The \textbf{circle(c,r,d)} function returns the list of points on this circle (polygonal line).
    \item The \textbf{circleb(c,r,d)} function returns this circle as a path (see Dpath) using Bézier curves. \end{itemize}

\subsubsection{Dellipse}
\begin{itemize}
    \item The \textbf{g:Dellipse(c,rx,ry,inclin,draw\_options)} method draws the ellipse centered at \emph{c} (complex). The arguments \emph{rx} and \emph{ry} specify the two radii (on x and y). The optional argument \emph{inclin} is an angle in degrees that indicates the inclination of the ellipse relative to the \(Ox\) axis (zero by default). The argument \emph{draw\_options} is a string (empty by default) that will be passed as is to the \emph{\textbackslash draw} instruction.
    \item The \textbf{ellipse(c,rx,ry,inclin)} function returns the list of points on this ellipse (polygonal line).
    \item The function \textbf{ellipseb(c,rx,ry,inclination)} returns this ellipse as a path (see Dpath) using Bézier curves.
\end{itemize}

\subsubsection{Dellipticarc}
\begin{itemize}
    \item The method \textbf{g:Dellipticarc(B,A,C,rx,ry,sens,inclination,draw\_options)} draws an arc of an ellipse centered at \emph{A} (complex) and radii at \emph{rx} and \emph{ry}, making an angle equal to \emph{inclination} with respect to the \(Ox\) axis (zero by default), going from \emph{B} (complex) to \emph{A} (complex) in the counterclockwise direction if the argument \emph{sens} is 1, and the opposite direction otherwise. The \emph{draw\_options} argument is a string (empty by default) that will be passed as is to the \emph{\textbackslash draw} instruction.
    \item The \textbf{ellipticarc(B,A,C,rx,ry,sens,inclination)} function returns the list of points of this arc (polygonal line).
    \item The \textbf{ellipticarcb(B,A,C,rx,ry,sens,inclination)} function returns this arc as a path (see Dpath) using Bézier curves.
\end{itemize}

\subsubsection{Dpolyreg}
\begin{itemize}
    \item The \textbf{g:Dpolyreg(vertex1,vertex2,number of points,direction,draw\_options)} or \par \textbf{g:Dpolyreg(center,vertex,number of points,draw\_options)} method draws a regular polygon. The \emph{draw\_options} argument is a string (empty by default) that will be passed as is to the \emph{\textbackslash draw} instruction.
    \item The \textbf{polyreg(vertex1,vertex2,number of points,direction)} function and the \textbf{polyreg(center,vertex,number of points)} function return the list of vertices of this regular polygon. \end{itemize}

\subsubsection{Drectangle}
\begin{itemize}
    \item The \textbf{g:Drectangle(a,b,c,draw\_options)} method draws a rectangle with consecutive vertices \emph{a} and \emph{b}, and whose opposite side passes through \emph{c}. The \emph{draw\_options} argument is a string (empty by default) that will be passed as is to the \emph{\textbackslash draw} instruction.
    \item The \textbf{rectangle(a,b,c)} function returns the list of vertices of this rectangle. \end{itemize}

\subsubsection{Dsequence}
\begin{itemize}
    \item The \textbf{g:Dsequence(f,u0,n,draw\_options)} method draws the "staircases" of the recurrent sequence defined by its first term \emph{u0} and the relation \(u_{k+1}=f(u_k)\). The argument \emph{f} must be a function of a real-valued variable, and the argument \emph{n} is the number of terms calculated. The argument \emph{draw\_options} is a string (empty by default) that will be passed as is to the \emph{\textbackslash draw} instruction.
    \item The \textbf{sequence(f,u0,n)} function returns the list of points constituting these "staircases".
\end{itemize}

\begin{demo}{Sequence $u_{n+1}=\cos(u_n)$}
\begin{luadraw}{name=sequence}
local g = graph:new{window={-0.1,1.7,-0.1,1.1},size={10,10,0}}
local i, pi, cos = cpx.I, math.pi, math.cos
local f = function(x) return cos(x)-x end
local ell = solve(f,0,pi/2)[1]
local L = sequence(cos,0.2,5) -- u_{n+1}=cos(u_n), u_0=0.2
local seg, z = {}, L[1]
for k = 2, #L do 
    table.insert(seg,{z,L[k]})
    z = L[k]
end -- seg est la liste des segments de l'escalier
g:Writeln("\\tikzset{->-/.style={decoration={markings, mark=at position #1 with {\\arrow{Stealth}}}, postaction={decorate}}}")
g:Daxes({0,1,1}, {arrows="-Stealth"})
g:DlineEq(1,-1,0,"line width=0.8pt,ForestGreen")
g:Dcartesian(cos, {x={0,pi/2},draw_options="line width=1.2pt,Crimson"})
g:Dpolyline(seg,false,"->-=0.65,blue")
g:Dlabel("$u_0$",0.2,{pos="S",node_options="blue"})
g:Dseg({ell, ell*(1+i)},1,"dashed,gray")
g:Dlabel("$\\ell\\approx"..round(ell,3).."$", ell,{pos="S"})
g:Ddots(ell*(1+i)); g:Labelcolor("Crimson")
g:Dlabel("${\\mathcal C}_{\\cos}$",Z(1,cos(1)),{pos="E"})
g:Labelcolor("ForestGreen"); g:Labelangle(g:Arg(1+i)*180/pi)
g:Dlabel("$y=x$",Z(0.4,0.4),{pos="S",dist=0.1}) 
g:Show()
\end{luadraw}
\end{demo}

The \textbf{g:Arg(z)} method calculates and returns the \textit{real} argument of the complex $z$, that is, its argument (in radians) when exported to the tikz coordinate system (to do this, you must apply the graph's transformation matrix to $z$, then convert the coordinate system to that of tikz). If the graph coordinate system is orthonormal and the transformation matrix is ​​the identity matrix, then the result is identical to that of \textbf{cpx.arg(z)} (this is not the case in the example above).

Similarly, the \textbf{g:Abs(z)} method calculates and returns the \textit{real} modulus of the complex $z$, that is, its modulus when exported to the tikz coordinate system; it is therefore a length in centimeters. If the graph coordinate system is orthonormal with 1 cm per unit on each axis, and if the transformation matrix is ​​an isometry, then the result is identical to that of \textbf{cpx.abs(z)}.

\subsubsection{Dsquare}
\begin{itemize}
    \item The \textbf{g:Dsquare(a,b,sens,draw\_options)} method draws the square with consecutive vertices \emph{a} and \emph{b}, in the counterclockwise direction when \emph{sens} is 1 (the default value). The argument \emph{draw\_options} is a string (empty by default) that will be passed as is to the \emph{\textbackslash draw} instruction.
    \item The \textbf{square(a,b,sens)} function returns the list of vertices of this square. \end{itemize}

\subsubsection{Dwedge}
The method \textbf{g:Dwedge(B,A,C,r,sens,draw\_options)} draws an angular sector with center \emph{A} (complex), radius \emph{r}, going from \emph{B} (complex) to \emph{C} (complex) counterclockwise if the argument \emph{sens} is 1, and counterclockwise otherwise. The argument \emph{draw\_options} is a string (empty by default) that will be passed as is to the instruction \emph{\textbackslash draw}.

\subsection{Curves}

\subsubsection{Parametric: Dparametric}

\begin{itemize}
    \item The function \textbf{parametric(p,t1,t2,nbdots,discont,nbdiv)} calculates the points and returns a polygonal line (no drawing).
\begin{itemize}
    \item The argument \emph{p} is the parameterization; it must be a function of a real variable \emph{t} and complex-valued variables, for example:
\mintinline{Lua}{local p = function(t) return cpx.exp(t*cpx.I) end}
    \item The arguments \emph{t1} and \emph{t2} are required with \(t1 < t2\); they form the bounds of the interval for the parameter.     \item The argument \emph{nbdots} is optional; it is the (minimum) number of points to calculate; it is 40 by default.
    \item The argument \emph{discont} is an optional Boolean that indicates whether there are discontinuities or not; it is false by default.
    \item The argument \emph{nbdiv} is a positive integer that is 5 by default and indicates the number of times the interval between two consecutive values ​​of the parameter can be cut in two (dichotomized) when the corresponding points are too far apart.
\end{itemize}

    \item The method \textbf{g:Dparametric(p,args)} calculates the points and draws the curve parametrized by \emph{p}. The \emph{args} parameter is a 6-field table:

\begin{TeXcode}
{ t={t1,t2}, nbdots=40, discont=true/false, nbdiv=5, draw_options=", clip={x1,x2,y1,y2} }
\end{TeXcode}

\begin{itemize}
    \item By default, the \emph{t} field is equal to \emph{\{g:Xinf(),g:Xsup()\}},
    \item the \emph{nbdots} field is equal to 40,
    \item the \emph{discont} field is equal to \emph{false},
    \item the \emph{nbdiv} field is equal to 5,
    \item the \emph{draw\_options} field is an empty string (this will be passed as is to the instruction \emph{\textbackslash draw}),
    \item the \emph{clip} field is either \emph{nil} (default value) or a table \emph{\{x1,x2,y1,y2\}}. In the first case, the line is clipped by the current 2D window \textbf{after} its transformation by the graph's 2D matrix. In the second case, the line is clipped by the window $[x_1,x_2]\times[y_1,y_2]$ \textbf{before} being transformed by the graph's matrix.
\end{itemize}
\end{itemize}


\subsubsection{Polars: Dpolar}

\begin{itemize}
    \item The function \textbf{polar(rho,t1,t2,nbdots,discont,nbdiv)} calculates the points and returns a polygonal line (no drawing). The argument \emph{rho} is the polar parameterization of the curve; it must be a function of a real variable \emph{t} and with real values, for example:

\mintinline{Lua}{local rho = function(t) return 4*math.cos(2*t) end}

The other arguments are identical to those for parameterized curves.
    \item The method \textbf{g:Dpolar(rho,args)} calculates the points and draws the polar curve parameterized by \emph{rho}. The \emph{args} parameter is a 6-field table:

\begin{TeXcode}
{ t={t1,t2}, nbdots=40, discont=true/false, nbdiv=5, draw_options=", clip={x1,x2,y1,y2} }
\end{TeXcode}

\begin{itemize}
    \item By default, the \emph{t} field is equal to $\{-\pi,\pi\}$,
    \item the \emph{nbdots} field is equal to 40,
    \item the \emph{discont} field is equal to \emph{false},
    \item the \emph{nbdiv} field is equal to 5,
    \item the \emph{draw\_options} field is an empty string (this will be passed as is to the \emph{\textbackslash instruction). draw}),
    \item the \emph{clip} field is either \emph{nil} (default value) or a table \emph{\{x1,x2,y1,y2\}}. In the first case, the line is clipped by the current 2D window \textbf{after} its transformation by the graph's 2D matrix; in the second case, the line is clipped by the window $[x_1,x_2]\times[y_1,y_2]$ \textbf{before} being transformed by the graph's matrix.
\end{itemize}
\end{itemize}

\subsubsection{Cartesian: Dcartesian}

\begin{itemize}
    \item The function \textbf{cartesian(f,x1,x2,nbdots,discont,nbdiv)} calculates the points and returns a polygonal line (no drawing). The argument \emph{f} is the function whose curve we want to obtain. It must be a function of a real variable \emph{x} and with real values, for example:

\mintinline{Lua}{local f = function(x) return 1+3*math.sin(x)*x end}

The arguments \emph{x1} and \emph{x2} are required and form the bounds of the interval for the variable. The other arguments are identical to those for parametric curves.

    \item The method \textbf{g:Dcartesian(f,args)} calculates the points and draws the curve of \emph{f}. The \emph{args} parameter is a 6-field table:

\begin{TeXcode}
{ x={x1,x2}, nbdots=40, discont=true/false, nbdiv=5, draw_options=", clip={x1,x2,y1,y2} }
\end{TeXcode}

\begin{itemize}
    \item By default, the \emph{x} field is equal to \emph{\{g:Xinf(),g:Xsup()\}},
    \item the \emph{nbdots} field is equal to 40,
    \item the \emph{discont} field is equal to \emph{false},
    \item the \emph{nbdiv} field is equal to 5,
    \item the \emph{draw\_options} field is an empty string (this will be passed as is to the instruction \emph{\textbackslash draw}),
    \item the \emph{clip} field is either \emph{nil} (default value) or a table \emph{\{x1,x2,y1,y2\}}. In the first case, the line is clipped by the current 2D window \textbf{after} its transformation by the 2D graph matrix. In the second case, the line is clipped by the window $[x_1,x_2]\times[y_1,y_2]$ \textbf{before} being transformed by the graph matrix.
\end{itemize}
\end{itemize}

\subsubsection{Periodic Functions: Dperiodic}

\begin{itemize}
    \item The function \textbf{periodic(f,period,x1,x2,nbdots,discont,nbdiv)} calculates the points and returns a polygonal line (no drawing).

\begin{itemize}
    \item The argument \emph{f} is the function whose curve we want; it must be a function of a real variable \emph{x} and with real values.
    \item The argument \emph{period} is a table of the type \emph{\{a,b\}}, with \(a<b\) representing a period of the function \emph{f}.
    \item The arguments \emph{x1} and \emph{x2} are required and form the bounds of the interval for the variable.
    \item The other arguments are identical to those for parametric curves. \end{itemize}
    \item The method \textbf{g:Dperiodic(f,period,args)} calculates the points and draws the curve of \emph{f}. The \emph{args} parameter is a 6-field table:

\begin{TeXcode}
{ x={x1,x2}, nbdots=40, discont=true/false, nbdiv=5, draw_options=", clip={x1,x2,y1,y2} }
\end{TeXcode}

\begin{itemize}
    \item By default, the \emph{x} field is equal to \emph{\{g:Xinf(),g:Xsup()\}},
    \item the \emph{nbdots} field is equal to 40,
    \item the \emph{discont} field is equal to \emph{false},
    \item the \emph{nbdiv} field is equal to 5,
    \item the \emph{draw\_options} field is an empty string (this will be passed as is to the instruction). \emph{\textbackslash draw}),
    \item the \emph{clip} field is either \emph{nil} (default value) or a table \emph{\{x1,x2,y1,y2\}}. In the first case, the line is clipped by the current 2D window \textbf{after} its transformation by the graph's 2D matrix; in the second case, the line is clipped by the window $[x_1,x_2]\times[y_1,y_2]$ \textbf{before} being transformed by the graph's matrix.

\end{itemize}

\end{itemize}

\subsubsection{Step Functions: Dstepfunction}

\begin{itemize}
    \item The \textbf{stepfunction(def,discont)} function calculates the points and returns a polygonal line (no drawing).

\begin{itemize}
    \item The \emph{def} argument defines the step function; it is a two-field table:

\begin{TeXcode}
{ {x1,x2,x3,...,xn}, {c1,c2,...} }
\end{TeXcode}

The first element \emph{\{x1,x2,x3,\ldots,xn\}} must be a subdivision of the segment \([x1,xn]\).

The second element \emph{\{c1,c2,\ldots\}} is the list of constants, with \emph{c1} for the segment \emph{{[}x1,x2{]}}, \emph{c2} for the segment \emph{{[}x2,x3{]}}, etc.

    \item The argument \emph{discont} is a Boolean that defaults to \emph{true}.
\end{itemize}

    \item The method \textbf{g:Dstepfunction(def,args)} calculates the points and draws the curve of the step function.

\begin{itemize}
    \item The argument \emph{def} is the same as the one described above (definition of the step function).     \item The \emph{args} argument is a three-field table:

\begin{TeXcode}
{ discont=true/false, draw_options=",clip={x1,x2,y1,y2} }
\end{TeXcode}

By default, the \emph{discont} field is true, and the \emph{draw\_options} field is an empty string (this will be passed as is to the \emph{\textbackslash draw} instruction). The \emph{clip} field is either \emph{nil} (default value) or a table \emph{\{x1,x2,y1,y2\}}. In the first case, the line is clipped by the current 2D window \textbf{after} its transformation by the graph's 2D matrix. In the second case, the line is clipped by the window $[x_1,x_2]\times[y_1,y_2]$ \textbf{before} being transformed by the graph's matrix.

\end{itemize}
\end{itemize}

\subsubsection{Piecewise Affine Functions: Daffinebypiece}

\begin{itemize}
    \item The function \textbf{affinebypiece(def,discont)} calculates the points and returns a polygonal line (no drawing).

\begin{itemize}
    \item The argument \emph{def} defines the step function; it is a two-field table:

\begin{TeXcode}
{ {x1,x2,x3,...,xn}, { {a1,b1}, {a2,b2},...} }
\end{TeXcode}

The first element \emph{\{x1,x2,x3,\ldots,xn\}} must be a subdivision of the segment \([x1,xn]\).

The second element \emph{\{ \{a1,b1\}, \{a2,b2\}, \ldots\}} means that on \emph{{[}x1,x2{]}} the function is \(x\mapsto a_1x+b_1\), on \emph{{[}x2,x3{]}} the function is
\(x\mapsto a_2x+b_2\), etc.

    \item The argument \emph{discont} is a boolean that defaults to \emph{true}.
\end{itemize}

    \item The method \textbf{g:Daffinebypiece(def,args)} calculates the points and draws the curve of the piecewise affine function.

\begin{itemize}
    \item The argument \emph{def} is the same as the one described above (definition of the piecewise affine function).
    \item The \emph{args} argument is a 3-field table:

\begin{TeXcode}
{ discont=true/false, draw_options=", clip={x1,x2,y1,y2} }
\end{TeXcode}

By default, the \emph{discont} field is set to \emph{true}, and the \emph{draw\_options} field is an empty string (this will be passed as is to the \emph{\textbackslash draw} instruction). The \emph{clip} field is either \emph{nil} (default value) or a table \emph{\{x1,x2,y1,y2\}}, in the first case the line is clipped by the current 2d window \textbf{after} its transformation by the 2d matrix of the graph, in the second case the line is clipped by the window $[x_1,x_2]\times[y_1,y_2]$ \textbf{before} being transformed by the graph matrix. \end{itemize}
\end{itemize}

\subsubsection{Differential Equations: Dodesolve}

\begin{itemize}
    \item The function \textbf{odesolve(f,t0,Y0,tmin,tmax,nbdots,method)} allows an approximate solution of the differential equation \(Y'(t)=f(t,Y(t))\) in the interval {[}tmin,tmax{]} which must contain \emph{t0}, with the initial condition $Y(t0)=Y0$.

\begin{itemize}
    \item The argument \emph{f} is a function \(f: (t,Y) -> f(t,Y)\) with values ​​in \(R^n\) and where \emph{Y} is also in \(R^n\): \emph{Y=\{y1, y2,\ldots, yn\}} (when $n=1$, \emph{Y} is a real number).
    \item The arguments \emph{t0} and \emph{Y0} give the initial conditions with \emph{Y0=\{y1(t0), \ldots, yn(t0)\}} (the yi numbers are real), or \emph{Y0=y1(t0)} when $n=1$.
    \item The arguments \emph{tmin} and \emph{tmax} define the resolution interval; this must contain \(t0\).
    \item The argument \emph{nbdots} indicates the number of points calculated on either side of \(t0\).
    \item The optional argument \emph{method} is a string that can be \emph{"rkf45"} (default), or \emph{"rk4"}. In the first case, we use the Runge Kutta-Fehlberg method (with variable step size), in the second case, it is the classic Runge-Kutta method of order 4.
    \item As output, the function returns the following matrix (list of lists of real numbers):

\begin{TeXcode}
{ {tmin,...,tmax}, {y1(tmin),...,y1(tmax)}, {y2(tmin),...,y2(tmax)},...}
\end{TeXcode}

The first component is the list of values ​​of \emph{t} (in ascending order), the second is the list of (approximate) values ​​of the component \emph{y1} corresponding to these values ​​of \emph{t}, ... etc.
\end{itemize}

    \item The method \textbf{g:DplotXY(X,Y,draw\_options,clip)}, where the arguments \emph{X} and \emph{Y} are two lists of real numbers of the same length, draws the polygonal line consisting of the points $(X[k],Y[k])$. The argument \emph{draw\_options} is a string (empty by default) that will be passed as is to the \emph{\textbackslash draw} instruction. The \emph{clip} field is either \emph{nil} (default value) or a table \emph{\{x1,x2,y1,y2\}}. In the first case, the line is clipped by the current 2D window \textbf{after} its transformation by the graph's 2D matrix. In the second case, the line is clipped by the window $[x_1,x_2]\times[y_1,y_2]$ \textbf{before} being transformed by the graph's matrix.

\begin{demo}{A Lokta-Volterra differential system}
\begin{luadraw}{name=lokta_volterra}
local g = graph:new{window={-5,50,-0.5,5},size={10,10,0}, border=true}
local i = cpx.I
local f = function(t,y) return {y[1]-y[1]*y[2],-y[2]+y[1]*y[2]} end
g:Labelsize("footnotesize")
g:Daxes({0,10,1},{limits={{0,50},{0,4}}, nbsubdiv={4,0}, legendsep={0.1,0}, originpos={"center","center"}, legend={"$t$",""}})
local y0 = {2,2}
local M = odesolve(f,0,y0,0,50,250) -- résolution approchée
-- M est une table à 3 éléments: t, x et y
g:Lineoptions("solid","blue",8)
g:Dseg({5+3.5*i,10+3.5*i}); g:Dlabel("$x$",10+3.5*i,{pos="E"})
g:DplotXY(M[1],M[2]) -- points (t,x(t))
g:Linecolor("red"); g:Dseg({5+3*i,10+3*i}); g:Dlabel("$y$",10+3*i,{pos="E"})
g:DplotXY(M[1],M[3])  -- points (t,y(t))
g:Lineoptions(nil,"black",4)
g:Saveattr(); g:Viewport(20,50,3,5) -- changement de vue
g:Coordsystem(-0.5,3.25,-0.5,3.25) -- nouveau repère associé
g:Daxes({0,1,1},{legend={"$x$","$y$"},arrows="->"})
g:Lineoptions(nil,"ForestGreen",8); g:DplotXY(M[2],M[3]) -- points (x(t),y(t))
g:Restoreattr() -- retour à l'ancienne vue
g:Dlabel("$\\begin{cases}x'=x-xy\\\\y'=-y+xy\\end{cases}$", 5+4.75*i,{})
g:Show()
\end{luadraw}
\end{demo}
    
    \item The method \textbf{g:Dodesolve(f,t0,Y0,args)} allows the drawing of a solution to the equation \(Y'(t)=f(t,Y(t))\).
\begin{itemize}
    \item The required argument \emph{f} is a function \(f: (t,Y) -> f(t,Y)\) with values ​​in \(R^n\) and where \emph{Y} is also in \(R^n\): \emph{Y=\{y1, y2,\ldots, yn\}} (when $n=1$, \emph{Y} is a real number).
    \item The arguments \emph{t0} and \emph{Y0} give the initial conditions with \emph{Y0=\{y1(t0), \ldots, yn(t0)\}} (the yi are real), or \emph{Y0=y1(t0)} when $n=1$.
    \item The \emph{args} argument (optional) allows you to define the parameters for the curve. It is a table with 6 fields:

\begin{TeXcode}
{ t={tmin,tmax}, out={i1,i2}, nbdots=50, method="rkf45"/"rk4", draw_options="", clip={x1,x2,y1,y2} }
\end{TeXcode}

\begin{itemize}
    \item The \emph{t} field determines the interval for the variable \(t\). By default, it is \emph{\{g:Xinf(), g:Xsup()\}}.
    \item The \emph{out} field is a table of two integers \{i1, i2\}. If \emph{M} denotes the matrix returned by the \emph{odesolve} function, the points drawn will have the M{[}i1{]} as abscissas and the M{[}i2{]} as ordinates. By default, we have \emph{i1=1} and \emph{i2=2}, which corresponds to the \emph{y1} function as a function of \emph{t}.
    \item The \emph{nbdots} field determines the number of points to calculate for the function (50 by default).
    \item The \emph{method} field determines the method to use; possible values ​​are \emph{"rkf45"} (default value), or \emph{"rk4"}.     \item The \emph{draw\_options} field is a string (empty by default) that will be passed as is to the \emph{\textbackslash draw} instruction.
    \item The \emph{clip} field is either \emph{nil} (default value) or a table \emph{\{x1,x2,y1,y2\}}. In the first case, the line is clipped by the current 2D window \textbf{after} its transformation by the graph's 2D matrix. In the second case, the line is clipped by the window $[x_1,x_2]\times[y_1,y_2]$ \textbf{before} its transformation by the graph's matrix.

\end{itemize}
\end{itemize}
\end{itemize}

\subsubsection{Implicit Curves: Dimplicit}

\begin{itemize}
    \item The function \textbf{implicit(f,x1,x2,y1,y2,grid)} calculates and returns a polygonal line constituting the implicit curve with equation $f(x,y)=0$ in the box $[x_1,x_2]\times[y_1,y_2]$. This box is split according to the parameter \emph{grid}.

\begin{itemize}
    \item The required argument \emph{f} is a function \(f: (x,y) -> f(x,y)\) with values ​​in \(R\).
    \item The arguments \emph{x1}, \emph{x2}, \emph{y1}, \emph{y2} define the plot window, which will be the $[x_1,x_2]\times[y_1,y_2]$ box. We must have \(x1<x2\) and \(y1<y2\).
    \item The argument \emph{grid} is a table containing two positive integers: \{n1,n2\}. The first integer indicates the number of subdivisions following $x$, and the second the number of subdivisions following $y$.
\end{itemize}

    \item The \textbf{g:Dimplicit(f,args)} method draws the implicit curve of equations $f(x,y)=0$.

\begin{itemize}
    \item The required argument \emph{f} is a function \(f: (x,y) -> f(x,y)\) with values ​​in \(R\).
    \item The argument \emph{args} defines the drawing parameters; it is a table with 3 fields:

\begin{TeXcode}
{ view={x1,x2,y1,y2}, grid={n1,n2}, draw_options="" }
\end{TeXcode}
\begin{itemize}
    \item The \emph{view} field determines the drawing area $[x_1,x_2]\times[y_1,y_2]$. By default, we have \emph{view=\{g:Xinf(), g:Xsup(), g:Yinf(), g:Ysup()\}},
    \item the \emph{grid} field determines the grid, this field defaults to \emph{\{50,50\}},
    \item the \emph{draw\_options} field is a string (empty by default) that will be passed as is to the \emph{\textbackslash draw} instruction.
\end{itemize}
\end{itemize}
\end{itemize}

\subsubsection{Contour Lines: Dcontour}

The \textbf{g:Dcontour(f,z,args)} method draws \textbf{contour lines} of the function \(f: (x,y) -> f(x,y)\) with real values.

\begin{itemize}
    \item The argument \emph{z} (required) is the list of different levels to plot.
    \item The \emph{args} argument (optional) allows you to define the drawing parameters. It is a 4-field table:

\begin{TeXcode}
{ view={x1,x2,y1,y2}, grid={n1,n2}, colors={"color1","color2",...}, draw_options="" }
\end{TeXcode}

\begin{itemize}
    \item The \emph{view} field determines the drawing area {[}x1,x2{]}x{[}y1,y2{]}. By default, we have \emph{view=\{g:Xinf(),g:Xsup(), g:Yinf(), g:Ysup()\}}.
    \item The \emph{grid} field determines the grid. By default, we have \emph{grid=\{50,50\}}.     \item The \emph{colors} field is the list of colors per level. By default, this list is empty and the current drawing color is used.
    \item The \emph{draw\_options} field is a string (empty by default) that will be passed as is to the \emph{\textbackslash draw} instruction.
\end{itemize}
\end{itemize}

\begin{demo}{Example with Dcontour}
\begin{luadraw}{name=Dcontour}
local g = graph:new{window={-1,6.5,-1.5,11},size={10,10,0}}
local i, sin, cos = cpx.I, math.sin, math.cos
local f = function(x,y) return (x+y)/(2+cos(x)*sin(y)) end
local Lz = range(1,10) -- niveaux à tracer
local Colors = getpalette(palRainbow,10)
g:Dgradbox({0,5+10*i,1,1},{legend={"$x$","$y$"}, grid=true, title="$z=\\frac{x+y}{2+\\cos(x)\\sin(y)}$"})
g:Linewidth(12); g:Dcontour(f,Lz,{view={0,5,0,10}, colors=Colors})
for k = 1, 10 do
    local y = (2*k+4)/3*i
    g:Dseg({5.25+y,5.5+y},1,"color="..Colors[k])
    g:Labelcolor(Colors[k])
    g:Dlabel("$z="..k.."$",5.5+y,{pos="E"})
end
g:Show()
\end{luadraw}
\end{demo}

\subsubsection{Parameterization of a Polygonal Line: \emph{curvilinear\_param}}
Let $L$ be a list of complex numbers representing a continuous \og \fg line. It is possible to obtain a parameterization of this line based on a parameter $t$ between $0$ and $1$ ($t$ is the curvilinear abscissa divided by the total length of $L$).

The function \textbf{curvilinear\_param(L,close)} returns a function of one variable $t\in[0;1]$ and values ​​on the line $L$ (complex numbers). The value at $t=0$ is the first point of $L$, and the value at $t=1$ is the last point; this function is followed by a number representing the total length of L. The optional argument \emph{close} indicates whether the line $L$ should be closed (\emph{false} by default).

\begin{demo}{Points distributed on a polygonal line}
\begin{luadraw}{name=curvilinear_param}
local g = graph:new{bbox=false,size={10,10}}
local i = cpx.I; g:Linewidth(8)
local L = cartesian(math.sin,-5,5)[1]
insert(L, {5-2*i, -5-2*i})
local f = curvilinear_param(L, true)
local I = map(f,linspace(0,1,20)) -- 20 points répartis sur L
g:Shift(4*i)
g:Lineoptions(nil,"ForestGreen",6); g:Dpolyline(L,true)
g:Filloptions("full","white"); g:Ddots(I) -- le premier et le dernier point sont confondus car L est fermée

-- autre exemple d'utilisation:
local nb = 16 --nb arrows
local t = linspace(0,1,3*nb+1)
g:Shift(-4*i)
for k = 0,nb-1 do
    g:Dparametric(f,{t={t[3*k+1],t[3*k+3]},nbdots=10,nbdiv=2,draw_options="-stealth"})
end
g:Show() 
\end{luadraw}
\end{demo}


\subsection{Domains related to Cartesian curves}

\subsubsection{Ddomain1}

\begin{itemize}
    \item The function \textbf{domain1(f,a,b,nbdots,discont,nbdiv)} returns a list of complex numbers that represents the contour of the part of the plane bounded by the curve of the function \emph{f} on an interval \([a;b]\), the \emph{Ox} axis, and the lines \(x=a\), \(x=b\).

    \item The method \textbf{g:Ddomain1(f,args)} draws this contour. The optional \emph{args} argument defines the parameters for the curve. It is a table with 5 fields:

\begin{TeXcode}
{ x={a,b}, nbdots=50, discont=false, nbdiv=5, draw_options="" }
\end{TeXcode}

\begin{itemize}
    \item The \emph{x} field determines the study interval; by default, it is \emph{\{g:Xinf(), g:Xsup()\}}.
    \item The \emph{nbdots} field determines the number of points to calculate for the function (50 by default).
    \item The \emph{discont} field indicates whether or not there are discontinuities for the function (\emph{false} by default).     \item The \emph{nbdiv} field is used in the method for calculating the curve points (5 by default).
    \item The \emph{draw\_options} field is a string (empty by default) that will be passed as is to the \emph{\textbackslash draw} instruction.
\end{itemize}
\end{itemize}

\subsubsection{Ddomain2}

\begin{itemize}
    \item The \textbf{domain2(f,g,a,b,nbdots,discont,nbdiv)} function returns a list of complex numbers that represents the contour of the part of the plane bounded by the curve of the function \emph{f}, the curve of the function \emph{g}, and the lines \(x=a\), \(x=b\).

    \item The \textbf{g:Ddomain2(f,g,args)} method draws this contour. The optional \emph{args} argument allows you to define the parameters for the curves. It is a table with 6 fields:

\begin{TeXcode}
{ x={a,b}, nbdots=50, discont=false, nbdiv=5, draw_options="" }
\end{TeXcode}

\begin{itemize}
    \item The \emph{x} field determines the study interval; by default, it is \emph{\{g:Xinf(), g:Xsup()\}}.
    \item The \emph{nbdots} field determines the number of points to calculate for the function (50 by default).
    \item The \emph{discont} field indicates whether or not there are discontinuities for the function (false by default).     \item The \emph{nbdiv} field is used in the method for calculating the curve points (5 by default).
    \item The \emph{draw\_options} field is a string (empty by default) that will be passed as is to the \emph{\textbackslash draw} instruction.
\end{itemize}
\end{itemize}

\subsubsection{Ddomain3}

\begin{itemize}
    \item The \textbf{domain3(f,g,a,b,nbdots,discont,nbdiv)} function returns a list of complex numbers that represents the contour of the part of the plane bounded by the curve of the function \emph{f} and that of the function \emph{g} (searching for intersection points in the interval $[a;b]$).

    \item The \textbf{g:Ddomain3(f,g,args)} method draws this contour. The optional \emph{args} argument defines the parameters for the curve. It is a table with 5 fields:

\begin{TeXcode}
{ x={a,b}, nbdots=50, discont=false, nbdiv=5, draw_options="" }
\end{TeXcode}

\begin{itemize}
    \item The \emph{x} field determines the study interval; by default, it is \emph{\{g:Xinf(), g:Xsup()\}}.
    \item The \emph{nbdots} field determines the number of points to calculate for the function (50 by default).
    \item The \emph{discont} field indicates whether or not there are discontinuities for the function (false by default).     \item The \emph{nbdiv} field is used in the curve point calculation method (5 by default).
    \item The \emph{draw\_options} field is a string (empty by default) that will be passed as is to the \emph{\textbackslash draw} instruction.
\end{itemize}
\end{itemize}

\begin{demo}{Integer part, Ddomain1 and Ddomain3 functions}
\begin{luadraw}{name=courbe}
local g = graph:new{ window={-5,5,-5,5}, bg="", size={10,10} }
local f = function(x) return (x-2)^2-2 end
local h = function(x) return 2*math.cos(x-2.5)-2.25 end
g:Daxes( {0,1,1},{grid=true,gridstyle="dashed", arrows="->"})
g:Filloptions("full","brown",0.3)
g:Ddomain1( math.floor, { x={-2.5,3.5} })
g:Filloptions("none","white",1); g:Lineoptions("solid","red",12)
g:Dstepfunction( {range(-5,5), range(-5,4)},{draw_options="arrows={Bracket-Bracket[reversed]},shorten >=-2pt"})
g:Labelcolor("red")
g:Dlabel("Partie entière",Z(-3,3),{node_options="fill=white"})
g:Ddomain3(f,h,{draw_options="fill=blue,fill opacity=0.6"})
g:Dcartesian(f, {x={0,5}, draw_options="blue"})
g:Dcartesian(h, {x={0,5}, draw_options="green"})
g:Show()
\end{luadraw}
\end{demo}

\subsection{Points (Ddots) and Labels (Dlabel)}

\begin{itemize}
    \item The method for drawing one or more points is: \textbf{g:Ddots(dots, mark\_options)}.

\begin{itemize}
    \item The argument \emph{dots} can be either a single point (i.e., a complex), a list (a table) of complex numbers, or a list of lists of complex numbers. The points are drawn in the current color of the line plot.     \item The \emph{mark\_options} argument is an optional string that will be passed as is to the \emph{\textbackslash draw} instruction (local modifications), for example:
\begin{TeXcode}
"color=green, line width=1.2, scale=0.25"
\end{TeXcode}

    \item Two methods to globally modify the appearance of points:
\begin{itemize}
    \item The \textbf{g:Dotstyle(style)} method defines the point style. The \emph{style} argument is a string that defaults to \emph{"*"}. The possible styles are those of the \emph{plotmarks} library.     \item The \textbf{g:Dotscale(scale)} method allows you to adjust the dot size. The \emph{scale} argument is a positive integer that defaults to $1$. It is used to multiply the default dot size. The current line width also affects the dot size. For "solid" dot styles (e.g., the \emph{triangle*} style), the current fill style and color are used by the library.
\end{itemize}
\end{itemize}

    \item The method for placing a label is:

\hfil\textbf{g:Dlabel(text1, anchor1, args1, text2, anchor2, args2, ...)}.\hfil

\begin{itemize}
    \item The arguments \emph{text1, text2, ...} are strings; they are the labels.     \item The arguments \emph{anchor1, anchor2,...} are complex numbers representing the anchor points of the labels.
    \item The arguments \emph{args1, arg2,...} allow you to locally define the label parameters; they are tables with 4 fields:
\begin{TeXcode}
{ pos=nil, dist=0, dir={dirX,dirY,dep}, node_options="" }
\end{TeXcode}
\begin{itemize}
    \item The \emph{pos} field indicates the position of the label relative to the anchor point. It can be \emph{"N"} for north, \emph{"NE"} for northeast, \emph{"NW"} for northwest, or \emph{"S"}, \emph{"SE"}, \emph{"SW"}. By default, it is set to \emph{center}, and in this case the label is centered on the anchor point.
    \item The \emph{dist} field is a distance in cm that defaults to $0$. It is the distance between the label and its anchor point when \emph{pos} is not equal to \emph{center}.
    \item \emph{dir=\{dirX,dirY,dep\}} is the writing direction (\emph{nil}, the default value, for the default direction). The three values ​​\emph{dirX}, \emph{dirY}, and \emph{dep} are three complex vectors representing three vectors: the first two indicate the writing direction, and the third a displacement (translation) of the label relative to the anchor point.     \item The \emph{node\_options} argument is a string (empty by default) intended to receive options that will be passed directly to tikz in the \emph{node{[}{]}} instruction.
    \item The labels are drawn in the current color of the document text, but the color can be changed with the \emph{node\_options} argument, for example, by setting: \emph{node\_options="color=blue"}.

\textbf{Warning}: The options chosen for a label also apply to subsequent labels if they are unchanged.
\end{itemize}
\end{itemize}

Global options for labels:

\begin{itemize}
    \item The \textbf{g:Labelstyle(position)} method allows you to specify the position of the labels relative to the anchor points. The \emph{position} argument is a string that can be: \emph{"N"} for north, \emph{"NE"} for northeast, \emph{"NW"} for northwest, or \emph{"S"}, \emph{"SE"}, \emph{"SW"}. By default, it
is set to \emph{center}, in which case the label is centered on the anchor point.
    \item The \textbf{g:Labelcolor(color)} method allows you to set the color of the labels. The \emph{color} argument is a string representing a color for tikz. By default, the argument is an empty string, which represents the current color of the document.
    \item The \textbf{g:Labelangle(angle)} method allows you to specify an angle (in degrees) for rotating the labels around the anchor point. This angle is zero by default.
    \item The \textbf{g:Labelsize(size)} method allows you to manage the size of the labels. The \emph{size} argument is a string that can be: \emph{"tiny"}, or \emph{"scriptsize"}, or \emph{"footnotesize"}, etc. By default, the argument is an empty string, which represents the \emph{"normalsize"} size.
\end{itemize}

    \item The \textbf{g:Dlabeldot(text,anchor,args)} method allows you to place a label and draw the anchor point at the same time.

\begin{itemize}
    \item The \emph{text} argument is a string; it is the label.
    \item The \emph{anchor} argument is a complex representing the label's anchor point.
    \item The \emph{args} argument (optional) allows you to define the label and point parameters; it is a 4-field table:

\begin{TeXcode}
{ pos=nil, dist=0, node_options="", mark_options="" }
\end{TeXcode}

The fields are identical to those of the \emph{Dlabel} method, plus the \emph{mark\_options} field, which is a string that will be passed as is to the \emph{\textbackslash draw} instruction when drawing the anchor point.
\end{itemize}
\end{itemize}

\subsection{Paths: Dpath, Dspline, and Dtcurve}

\begin{itemize}
    \item The \textbf{path( path )} function returns a polygonal line containing the points constituting the \emph{path}. The optional argument, \emph{nbdots}, is the minimum number of points calculated for each Bézier curve; its default value is the global variable \emph{bezier\_nbdots}, which is initialized to 8.
    
    The \emph{path} is a table of complex numbers and string instructions, for example:

\begin{TeXcode}
{ Z(-3,2),-3,-2,"l",0,2,2,-1,"ca",3,Z(3,3),0.5,"la",1,Z(-1,5),Z(-3,2),"b" }
\end{TeXcode}
with:
\begin{itemize}
    \item \emph{"m"} for moveto,
    \item \emph{"l"} for lineto,
    \item \emph{"b"} for Bezier (two control points are required),
    \item \emph{"s"} for a natural cubic spline passing through the points listed,
    \item \emph{"c"} for circle (one point and the center, or three points are required),
    \item \emph{"ca"} for a circular arc (requires 3 points, a radius, and a direction),
    \item \emph{"ea"} for an elliptical arc (requires 3 points, a radius rx, a radius ry, a direction, and possibly an inclination in degrees),
    \item \emph{"e"} for an ellipse (requires a point, the center, a radius rx, a radius ry, and possibly an inclination in degrees),
    \item \emph{"cl"} for close (closes the current component),
    \item \emph{"la"} for a line arc, i.e., a line with rounded corners (the radius must be specified just before the \emph{"la"} instruction),
    \item \emph{"cla"} for a closed line with rounded corners (the radius must be specified just before the \emph{"cla"} instruction). \end{itemize}

    \item The \textbf{g:Dpath(path,draw\_options)} method draws the \emph{path} (using Bézier curves as much as possible, including arcs, ellipses, etc.). The \emph{draw\_options} argument is a string that will be passed directly to the \emph{\textbackslash draw} instruction.
\begin{itemize}
    \item The \emph{path} argument was described above.
    \item The \emph{draw\_options} argument is a string (empty by default) that will be passed as is to the \emph{\textbackslash draw} instruction.
\end{itemize}

    \item The function \textbf{spline(points,v1,v2)} returns a path (to be drawn with Dpath) of the cubic spline passing through the points of the argument \emph{points} (which must be a list of complex vectors). The arguments \emph{v1} and \emph{v2} are tangent vectors imposed at the ends (constraints); when these are equal to \emph{nil}, a natural cubic spline (i.e., unconstrained) is calculated.

    \item The method \textbf{g:Dspline(points,v1,v2,draw\_options)} draws the spline described above. The argument \emph{draw\_options} is a string that will be passed directly to the instruction \emph{\textbackslash draw}.

\begin{demo}{Path and Spline}
\begin{luadraw}{name=path_spline}
local g = graph:new{window={-5,5,-5,5},size={10,10},bg="Beige"}
local i = cpx.I
local p = {-3+2*i,-3,-2,"l",0,2,2,1,"ca",3,3+3*i,0.5,"la",1,-1+5*i,-3+2*i,"b",-1,"m",0,"c"}
g:Daxes( {0,1,1} )
g:Filloptions("full","blue!30",1,true); g:Dpath(p,"line width=0.8pt")
g:Filloptions("none")
local A,B,C,D,E = -4-i,-3*i,4,3+4*i,-4+2*i
g:Lineoptions(nil,"ForestGreen",12); g:Dspline({A,B,C,D,E},nil,-5*i) -- contrainte en E
g:Ddots({A,B,C,D,E},"fill=white,scale=1.25")
g:Show()
\end{luadraw}
\end{demo}

    \item The function \textbf{tcurve(L} returns a curve passing through given points as a path, with tangent vectors (left and right) imposed at each point. \emph{L} is a table of the form:
\begin{Luacode}
L = {point1,{t1,a1,t2,a2}, point2,{t1,a1,t2,a2}, ..., pointN,{t1,a1,t2,a2}}
\end{Luacode}
\emph{point1}, ..., \emph{pointN} are the interpolation points of the curve (affixes), and each of them is followed by a table of the form \verb|{t1,a1,t2,a2}| which specifies the tangent vectors to the curve to the left of the point (with \emph{t1} and \emph{a1}) and to the right of the point (with \emph{t2} and \emph{a2}). The left tangent vector is given by the formula $V_g = t_1\times e^{ia_1\pi/180}$, so $t1$ represents the modulus and $a1$ is an argument \textbf{in degrees} of this vector. The same is true for \emph{t2} and \emph{a2} for the right tangent vector, \textbf{but these are optional}, and if not specified, they take the same values ​​as \emph{t1} and \emph{a1}.

Two consecutive points will be connected by a Bézier curve; the function calculates the control points to obtain the desired tangent vectors.

    \item The method \textbf{g:Dtcurve(L,options)} draws the path obtained by \emph{tcurve} described above. The argument \emph{options} is a two-field table:
\begin{itemize}
    \item \emph{showdots=true/false} (false by default). This option allows you to draw the given interpolation points as well as the calculated control points, allowing for visualization of the constraints.
    \item \emph{draw\_options=""}. This is a string that will be passed directly to the \emph{\textbackslash draw} instruction.
\end{itemize}
\end{itemize}

\begin{demo}{Interpolation curve with imposed tangent vectors}
\begin{luadraw}{name=tcurve}
local g = graph:new{window={-0.5,10.5,-0.5,6.5},size={10,10,0}}
local i = cpx.I
local L = {
    1+4*i,{2,-20},
    2+3*i,{2,-70},
    4+i/2,{3,0},
    6+3*i,{4,15},
    8+6*i,{4,0,4,-90}, -- point anguleux
    10+i,{3,-15}}
g:Dgrid({0,10+6*i},{gridstyle="dashed"})
g:Daxes(nil,{limits={{0,10},{0,6}},originpos={"center","center"}, arrows="->"})
g:Dtcurve(L,{showdots=true,draw_options="line width=0.8pt,red"})
g:Show()
\end{luadraw}
\end{demo}

\subsection{Paths and Clipping: Beginclip() and Endclip()}

A path can be used for clipping using two functions: \textbf{g:Beginclip(path,reverse)} and \textbf{g:Endclip()}. The first opens a \emph{scope} group and passes the \emph{path} as an argument to tikz's \emph{\textbackslash clip} function. The second closes the \emph{scope} group; it is essential (otherwise there will be a compilation error).
The \emph{reverse} argument is a Boolean that defaults to \emph{false}. When it has the value \emph{true}, the clipping is reversed, meaning that only what is outside the \emph{path} will be drawn, but for this to happen, the path must be counterclockwise.

\begin{demo}{Clipping Example}
\begin{luadraw}{name=polygon_with_different_line_color_and_rounded_corners}
local g = graph:new{window={-5,5,-5,5},size={10,10}}
local i = cpx.I
local Dcolored_polyreg = function(c,a,nb,r,wd,colors) 
-- c=center, a=vertice, nb=number of sides, r=radius, wd=width in point, colors=list of colors
    local L = polyreg(c,a,nb)
    insert(L,{r,"cla"}) --polygon width rounded corners (radius=r)
    local angle = 360/nb
    local b = a
    for k = 1, nb do
        a = b; b = rotate(a,angle,c)
        g:Beginclip({2*a-c,c,2*b-c,"l"})  -- définition d'un secteur angulaire pour clipper
        g:Dpath(L,"line width="..wd.."pt,color="..colors[k])
        g:Endclip()
    end
end
Dcolored_polyreg(3+2*i,5+2*i,5,0.8,12,{"red","blue","orange","green","yellow"}) -- pentagon
Dcolored_polyreg(-2.5-2*i,-5-2*i,7,1,24,getpalette(palGasFlame,7))  -- heptagon
g:Show()
\end{luadraw}
\end{demo}


\subsection{Axes and Grids}

Global variables used for axes and grids:
\begin{itemize}
    \item \emph{maxGrad = 100}: Maximum number of tick marks on an axis.
    \item \emph{defaultlabelshift = 0.125}: When a grid is drawn with the axes (option \emph{grid=true}), the labels are automatically shifted along the axis using this variable.
    \item \emph{defaultxylabelsep = 0}: Sets the default distance between labels and tick marks.
    \item \emph{defaultlegendsep = 0.2}: Sets the default distance between the legend and the axis.
    \item \emph{digits = 4}: Default number of decimal places in string conversions; terminal $0$s are removed.     \item \emph{dollar = true}: to add dollars around the tick labels.
    \item \emph{siunitx = false}: with the value \emph{true}, the labels are formatted with the macro \verb|\num{..}| of the \emph{siunitx} package, which allows you to use certain options of this package, such as replacing the decimal point with a comma by doing: \par
\begin{TeXcode}
\usepackage[local=FR]{siunitx}
\end{TeXcode}
or by doing:
\begin{TeXcode}
\usepackage{siunitx}
\sisetup{output-decimal-marker={,}}
\end{TeXcode}
\end{itemize}
For axes, in both 2D and 3D, all labels are formatted as strings with the \textbf{num(x)} function. This transforms the number $x$ into a string \emph{str} with the number of decimal places set by the global variable \emph{digits}. When the \emph{siunitx} variable has the value \emph{true}, the function returns \verb|"\num{str}"|, otherwise it simply returns \emph{str}. This also applies to 3D axes. Here is the code for this function:
\begin{Luacode}
function num(x) -- x is a real, returns a string
local rep = strReal(x) -- conversion to string with digits decimals max
if siunitx then rep = "\\num{"..rep.."}" end --needs \usepackage{siunitx}
return rep
end
\end{Luacode}

\subsubsection{Daxes}
\def\opt#1{\textcolor{blue}{\texttt{#1}}}%
The axes are plotted using the method \textbf{g:Daxes( \{A,xpas,ypas\}, options)}.
\begin{itemize}
    \item The first argument specifies the intersection point of the two axes (this is the complex \emph{A}), the graduation spacing on the $Ox$ axis (this is \emph{xpas}), and the graduation spacing on $Oy$ (this is \emph{ypas}). By default, the point \emph{A} is the origin $Z(0,0)$, and both spacings are equal to $1$.
    \item The argument \emph{options} is a table specifying the possible options. Here are these options with their default values:

\begin{itemize}
    \item \opt{showaxe={1,1}}. This option specifies whether or not the axes should be plotted ($1 or $0). The first value is for the 0x axis and the second for the 0y axis.

    \item \opt{arrows="-"}. This option allows you to add an arrow to the axes (no arrow by default; enter "->" to add an arrow).

    \item \opt{limits={"auto","auto"}}. This option specifies the extent of the two axes (first value for 0x, second value for 0y). The value "auto" means that it is the entire line, but you can specify the extreme abscissas, for example:
opt{limits={{-4,4},"auto"}}.
    \item \opt{gradlimits=\{"auto","auto"\}}. This option allows you to specify the range of the graduations on both axes (first value for $Ox$, second value for $Oy$). The value "auto" means that it is the entire line, but you can specify the extreme graduations, for example: \opt{gradlimits=\{\{-4.4\},\{-2.3\}\}}.
    \item \opt{unit=\{"",""\}}. This option allows you to specify the range of the graduations on the axes. The default value ("") means that the step value should be taken (\emph{xstep} on $Ox$, or \emph{ystep} on $Oy$), EXCEPT when the option \opt{labeltext} is not the empty string, in which case \emph{unit} takes the value $1$.
    \item \opt{nbsubdiv=\{0,0\}}. This option specifies the number of subdivisions between two main ticks on the axis.
    \item \opt{tickpos=\{0.5,0.5\}}. This option specifies the position of the ticks relative to each axis. These are two numbers between $0$ and $1$. The default value of $0.5$ means they are centered on the axis. ($0$ and $1$ represent the ends).
    \item \opt{tickdir=\{"auto","auto"\}}. This option specifies the direction of the ticks on the axis. This direction is a non-zero (complex) vector. The default value "auto" means the ticks are orthogonal to the axis.
    \item \opt{xyticks=\{0.2,0.2\}}. This option specifies the length of the ticks on the axis.
    \item \opt{xylabelsep=\{0,0\}}. This option specifies the distance between the labels and the tick marks on the axis.
    \item \opt{originpos=\{"right","top"\}}. This option specifies the position of the label at the origin on the axis. Possible values ​​are: "none", "center", "left", "right" for $Ox$, and "none", "center", "bottom", "top" for $Oy$.
    \item \opt{originnum=\{A.re,A.im\}}. This option specifies the tick mark value at the intersection of the axes (tick mark number $0$).

The formula that defines the label at tick mark number $n$ is: \textbf{(originnum + unit*n)"labeltext"/labelden}.

    \item \opt{originloc=A}. This option specifies the intersection point of the axes. 
    \item \opt{legend=\{"",""\}}. This option allows you to specify a legend for the axis.
    \item \opt{legendpos=\{0.975,0.975\}}. This option specifies the position (between $0$ and $1$) of the legend relative to each axis.
    \item \opt{legendsep=\{0.2,0.2\} }. This option specifies the distance between the legend and the axis. The legend is on the other side of the axis from the graduations.
    \item \opt{legendangle=\{"auto","auto"\}}. This option specifies the angle (in degrees) that the legend should make for the axis. The default value "auto" means that the legend must be parallel to the axis if the \emph{labelstyle} option is also set to "auto", otherwise the legend is horizontal.
    \item \opt{labelpos=\{"bottom","left"\}}. This option specifies the position of the labels relative to the axis. For the $Ox$ axis, the possible values ​​are: "none", "bottom", or "top", for the $Oy$ axis it is: "none", "right", or "left".
    \item \opt{labelden=\{1,1\}}. ​​This option specifies the denominator of the labels (integer) for the axis. The formula that defines the label at graduation number $n$ is: \textbf{(originnum + unit*n)"labeltext"/labelden}.
    \item \opt{labeltext=\{"",""\}}. This option defines the text that will be added to the numerator of the labels for the axis.
    \item \opt{labelstyle=\{"S","W"\}}. This option sets the label style for each axis. Possible values ​​are "auto","N", "NW", "W", "SW", "S", "SE", "E".
    \item \opt{labelangle=\{0,0\}}. This option sets the angle of the labels in degrees from the horizontal for each axis.
    \item \opt{labelcolor=\{"",""\}}. This option allows you to choose a color for the labels on each axis. The empty string represents the default color.
    \item \opt{labelshift=\{0,0\}}. This option allows you to set a systematic offset for the labels on the axis (along axis offset).
    \item \opt{nbdeci=\{2,2\}}. This option specifies the number of decimal places for numeric values ​​on the axis.     \item \opt{numericFormat=\{0,0\}}. This option specifies the type of digital display (not yet implemented).
    \item \opt{myxlabels=""}. This option allows you to impose custom labels on the $Ox$ axis. When any are present, the value passed to the option must be a list of the type: \verb|{pos1,"text1", pos2,"text2",...}|. The number \emph{pos1} represents an abscissa in the (A,xpas) coordinate system, which corresponds to the affix point $A+$pos1$*$xpas.
    \item \opt{myylabels=""}. This option allows you to impose custom labels on the $Oy$ axis. When any are present, the value passed to the option must be a list of the type: \verb|{pos1,"text1", pos2,"text2",...}|. The number \emph{pos1} represents an abscissa in the coordinate system (A,i*ypas), which corresponds to the affix point $A+$pos1$*$ypas$*i$.
    \item \opt{grid=false}. This option allows you to add a grid or not.
    \item \opt{drawbox=false}. This option draws the axes as a box; in this case, the graduations are on the left and bottom sides.
    \item \opt{gridstyle="solid"}. This option sets the line style for the primary grid.
    \item \opt{subgridstyle="solid"}. This option sets the line style for the secondary grid. A secondary grid appears when there are subdivisions on one of the axes.
    \item \opt{gridcolor="gray"}. This sets the color of the primary grid.     \item \opt{subgridcolor="lightgray"}. This sets the color of the secondary grid.
    \item \opt{gridwidth=4}. Line thickness of the primary grid (which is 0.4pt).
    \item \opt{subgridwidth=2}. Line thickness of the secondary grid (which is 0.2pt).
\end{itemize}
\end{itemize}


\begin{demo}{Example with axes with grid}
\begin{luadraw}{name=axes_grid}
local g = graph:new{window={-6.5,6.5,-3.5,3.5}, size={10,10,0}}
local i, pi, a = cpx.I, math.pi, math.sqrt(2)
local f = function(x) return 2*a*math.sin(x) end
g:Labelsize("footnotesize"); g:Linewidth(8)
g:Daxes({0,pi/2,a},{labeltext={"\\pi","\\sqrt{2}"}, labelden={2,1},nbsubdiv={1,1},grid=true,arrows="->"})
g:Lineoptions("solid","Crimson",12); g:Dcartesian(f, {x={-2*pi,2*pi}})
g:Show()
\end{luadraw}
\end{demo}

\subsubsection{DaxeX and DaxeY}

The methods \textbf{g:DaxeX(\{A,xsteps\}, options)} and \textbf{g:DaxeY(\{A,ysteps\}, options)} allow you to plot the axes separately.
\begin{itemize}
    \item The first argument specifies the point serving as the origin (the complex \emph{A}) and the step size of the tick marks on the axis. By default, the point \emph{A} is the origin $Z(0,0)$, and the step size is equal to $1$.
    \item The argument \emph{options} is a table specifying the possible options. Here are these options with their default values:
\begin{itemize}
    \item \opt{showaxe=1}. This option specifies whether or not the axis should be plotted ($1$ or $0$).
    \item \opt{arrows="-"}. This option allows you to add an arrow to the axis (no arrow by default; enter "->" to add an arrow).
    \item \opt{limits="auto"}. This option allows you to specify the range of the two axes. The value "auto" means that it is the entire line, but you can specify the extreme abscissas, for example: \opt{limits=\{-4.4\}}.
    \item \opt{gradlimits="auto"}. This option allows you to specify the range of the graduations on both axes. The value "auto" means that it is the entire line, but you can specify the extreme graduations, for example: \opt{gradlimits=\{-2.3\}}.
    \item \opt{unit=""}. This option allows you to specify the range of the graduations on the axis. The default value ("") means to take the step value, EXCEPT when the \opt{labeltext} option is not the empty string, in which case \emph{unit} takes the value $1$.
    \item \opt{nbsubdiv=0}. This option specifies the number of subdivisions between two main tick marks.
    \item \opt{tickpos=0.5}. This option specifies the position of the tick marks relative to the axis. These are two numbers between $0$ and $1$. The default value of $0.5$ means they are centered on the axis. ($0$ and $1$ represent the ends).
    \item \opt{tickdir="auto"}. This option indicates the direction of the tick marks on the axis. This direction is a non-zero (complex) vector. The default value "auto" means the tick marks are orthogonal to the axis.     \item \opt{xyticks=0.2}. This option specifies the length of the tick marks.
    \item \opt{xylabelsep=0}. This option specifies the distance between the labels and the tick marks.
    \item \opt{originpos="center"}. This option specifies the position of the label at the origin on the axis. Possible values ​​are: "none", "center", "left", "right" for $Ox$, and "none", "center", "bottom", "top" for $Oy$.
    \item \opt{originnum=A.re} for $Ox$ and \opt{originnum=A.im} for $Oy$. This option specifies the value of the tick mark at the origin (tick mark number $0$).

The formula that defines the label at tick mark number $n$ is: \textbf{(originnum + unit*n)"labeltext"/labelden}.

    \item \opt{legend=""}. This option allows you to specify a legend for the axis.
    \item \opt{legendpos=0.975}. This option specifies the position (between $0$ and $1$) of the legend relative to the axis.
    \item \opt{legendsep=0.2}. This option specifies the distance between the legend and the axis. The legend is on the other side of the axis from the graduations.
    \item \opt{legendangle="auto"}. This option specifies the angle (in degrees) that the legend should make for the axis. The default value "auto" means that the legend must be parallel to the axis if the \emph{labelstyle} option is also set to "auto", otherwise the legend is horizontal.
    \item \opt{labelpos="bottom"} for $Ox$ and \opt{labelpos="left"} for $Oy$. This option specifies the position of the labels relative to the axis. For the $Ox$ axis, the possible values ​​are: "none", "bottom", or "top", for the $Oy$ axis it is: "none", "right", or "left".
    \item \opt{labelden=1}. This option specifies the denominator of the labels (integer) for the axis. The formula that defines the label at graduation number $n$ is: \textbf{(originnum + unit*n)"labeltext"/labelden}.
    \item \opt{labeltext=""}. This option defines the text that will be added to the numerator of the labels.
    \item \opt{labelstyle="S"} for $Ox$ and \opt{labelstyle="W"} for $Oy$. This option defines the style of the labels. The possible values ​​are "auto", "N", "NW", "W", "SW", "S", "SE", "E".
    \item \opt{labelangle=0}. This option sets the angle of the labels in degrees from the horizontal.
    \item \opt{labelcolor=""}. This option allows you to choose a color for the labels. The empty string represents the current text color.
    \item \opt{labelshift=0}. This option allows you to set a systematic offset for labels on the axis (along-axis offset).
    \item \opt{nbdeci=2}. This option specifies the number of decimal places for numeric labels.
    \item \opt{numericFormat=0}. This option specifies the type of numeric display (not yet implemented).
    \item \opt{mylabels=""}. This option allows you to impose custom labels. When there are any, the value passed to the option must be a list of the type: \verb|{pos1,"text1", pos2,"text2",...}|. The number \emph{pos1} represents an abscissa in the coordinate system (A,xpas) for $Ox$, or (A,ypas$*$i) for $Oy$, which corresponds to the affix point $A+$pos1$*$xpas for $Ox$, and $A+$pos1$*$ypas$*i$ for $Oy$.
\end{itemize}
\end{itemize}

\subsubsection{Dgradline}

The axis plotting methods are based on the method \textbf{g:Dgradline(\{A,u\}, options)}, where \emph{\{A,u\}} represents the line passing through $A$ (a complex) and directed by the vector $u$ (a non-zero complex). The pair (A,u) serves as a reference point on this line (and orients this line), so each point $M$ on this line has an abscissa $x$ such that $M=A+xu$. This method allows you to draw this graduated line. The argument \emph{options} is a table specifying the possible options, which are (with their default values):
\begin{itemize}
    \item \opt{showaxe=1}. This option specifies whether or not the axis should be plotted ($1$ or $0$).     \item \opt{arrows="-"}. This option allows you to add an arrow to the axis (no arrow by default; enter "->" to add an arrow).
    \item \opt{limits="auto"}. This option allows you to specify the range of the two axes. The value "auto" means that it is the entire line, but you can specify the extreme abscissas, for example: \opt{limits=\{-4.4\}}.
    \item \opt{gradlimits="auto"}. This option allows you to specify the range of the graduations on both axes. The value "auto" means that it is the entire line, but you can specify the extreme graduations, for example: \opt{gradlimits=\{-2.3\}}.
    \item \opt{unit=1}. This option allows you to specify the number of graduations on the axis.
    \item \opt{nbsubdiv=0}. This option specifies the number of subdivisions between two main ticks.
    \item \opt{tickpos=0.5}. This option specifies the position of the ticks relative to the axis. These are two numbers between $0$ and $1$. The default value of $0.5$ means they are centered on the axis. ($0$ and $1$ represent the ends).
    \item \opt{tickdir="auto"}. This option specifies the direction of the ticks on the axis. This direction is a non-zero (complex) vector. The default value "auto" means the ticks are orthogonal to the axis.
    \item \opt{xyticks=0.2}. This option specifies the length of the ticks.
    \item \opt{xylabelsep=defaultxylabelsep}. This option specifies the distance between the labels and the tick marks. \emph{defaultxylabelsep} is a global variable with a default value of $0$.
    \item \opt{originpos="center"}. This option specifies the position of the label at the origin on the axis. Possible values ​​are: "none", "center", "left", "right".
    \item \opt{originnum=0}. This option specifies the value of the tick mark at the origin $A$ (tick mark number $0$).
    
The formula that defines the label at graduation number $n$ (at point $A+nu$) is: \textbf{(originnum + unit*n)"labeltext"/labelden}.

    \item \opt{legend=""}. This option allows you to specify a legend for the axis.
    \item \opt{legendpos=0.975}. This option specifies the position (between $0$ and $1$) of the legend relative to the axis.
    \item \opt{legendsep=defaultlegendsep}. This option specifies the distance between the legend and the axis. The legend is on the other side of the axis from the graduations; \emph{defaultlegendsep} is a global variable that defaults to 0.2.
    \item \opt{legendangle="auto"}. This option specifies the angle (in degrees) that the legend should form for the axis. The default value "auto" means that the legend must be parallel to the axis if the \emph{labelstyle} option is also set to "auto", otherwise the legend is horizontal.
    \item \opt{legendstyle="auto"}. Specifies the position of the legend relative to the axis. Possible values ​​are: "auto", "top", or "bottom".
    \item \opt{labelpos="bottom"}. This option specifies the position of the labels relative to the axis. Possible values ​​are: "none", "bottom", or "top".
    \item \opt{labelden=1}. This option specifies the denominator of the labels (integer) for the axis. The formula that defines the label at graduation number $n$ is: \textbf{(originnum + unit*n)"labeltext"/labelden}.
    \item \opt{labeltext=""}. This option defines the text that will be added to the numerator of the labels.     \item \opt{labelstyle="auto"}. This option defines the label style. Possible values ​​are "auto", "N", "NW", "W", "SW", "S", "SE", "E".
    \item \opt{labelangle=0}. This option defines the angle of the labels in degrees from the horizontal.
    \item \opt{labelcolor=""}. This option allows you to choose a color for the labels. The empty string represents the current text color.
    \item \opt{labelshift=0}. This option allows you to define a systematic offset for the labels on the axis (along axis offset).
    \item \opt{nbdeci=2}. This option specifies the number of decimal places for numeric labels.
    \item \opt{numericFormat=0}. This option specifies the type of numeric display (not yet implemented).
    \item \opt{mylabels=""}. This option allows you to impose custom labels. When any are present, the value passed to the option must be a list of the type: \verb|{x1,"text1", x2,"text2",...}|. The numbers \emph{x1, x2, ...} represent abscissas in the $(A,u)$ coordinate system.
\end{itemize}

\begin{demo}{Examples of numbered lines}
\begin{luadraw}{name=gradline}
local g = graph:new{window={-5,5,-5,5},size={10,10}}
g:Labelsize("footnotesize")
local i = cpx.I
g:Dgradline({3.25*i,1+i/2}, {limits={-4,4}, legend="Axe", legendpos=0.5, arrows="-stealth"})
g:Dgradline({-3,1}, {legend="demo", labeltext="\\pi", labelden=3, unit=2, nbsubdiv=1, arrows="-latex"})
g:Dgradline({3-4*i,-1.25+i/5}, {legend="A", labelstyle="N", gradlimits={-1,5}, nbsubdiv=3, unit=1.411, nbdeci=3, arrows="-Latex"})
g:Show()
\end{luadraw}
\end{demo}

\subsubsection{Dgrid}

The \textbf{g:Dgrid(\{A,B\},options} method allows you to draw a grid.
\begin{itemize}
    \item The first argument is mandatory; it specifies the lower-left corner (this is the \emph{A} complex) and the upper-right corner (this is the $B$ complex) of the grid.
    \item The \emph{options} argument is a table specifying the possible options. Here are these options with their default values:
\begin{itemize}
    \item \opt{unit=\{1,1\}}. ​​This option defines the units on the axes for the main grid.
    \item \opt{gridwidth=4}. This option defines the line thickness of the main grid (0.4pt by default).
    \item \opt{gridcolor="gray"}. Grid color of the main grid.
    \item \opt{gridstyle="solid"}. Line style for the primary grid.
    \item \opt{nbsubdiv={0,0}}. Number of subdivisions (for each axis) between two lines of the primary grid. These subdivisions determine the secondary grid.
    \item \opt{subgridcolor="lightgray"}. Color of the secondary grid.
    \item \opt{subgridwidth=2}. Line thickness of the secondary grid (0.2pt by default).
    \item \opt{subgridstyle="solid"}. Line style for the secondary grid.
    \item \opt{originloc=A}. Location of the grid origin.
\end{itemize}
\end{itemize}

\paragraph{Example:} It is possible to work in a non-orthogonal coordinate system. Here is an example where the $Ox$ axis is retained, but the first bisector becomes the new axis. $Oy$, we modify the graph's transformation matrix. Based on this modification, the affixes represent the coordinates in the new coordinate system.

\begin{demo}{Example of a non-orthogonal coordinate system}
\begin{luadraw}{name=axes_non_ortho}
local g = graph:new{window={-5.25,5.25,-4,4},size={10,10}}
local i, pi = cpx.I, math.pi
local f = function(x) return 2*math.sin(x) end
g:Setmatrix({0,1,1+i}); g:Labelsize("small")
g:Dgrid({-5-4*i,5+4*i},{gridstyle="dashed"})
g:Daxes({0,1,1}, {arrows="-Stealth"})
g:Lineoptions("solid","ForestGreen",12); g:Dcartesian(f,{x={-5,5}})
g:Dcircle(0,3,"Crimson")
g:DlineEq(1,0,3,"Navy") -- droite d'équation x=-3
g:Lineoptions("solid","black",8); g:DtangentC(f,pi/2,1.5,"<->")
g:Dpolyline({pi/2,pi/2+2*i,2*i},"dotted")
g:Ddots(Z(pi/2,2))
g:Dlabeldot("$\\frac{\\pi}2$",pi/2,{pos="SW"})
g:Show()
\end{luadraw}
\end{demo}

\subsubsection{Dgradbox}

The \textbf{g:Dgradbox(\{A,B,xpas,ypas\},options} method allows you to draw a graduated box.
\begin{itemize}
    \item The first argument is mandatory; it specifies the lower-left corner (this is the \emph{A} complex) and the upper-right corner (this is the $B$ complex) of the box, as well as the step on each axis.
    \item The \emph{options} argument is a table specifying the possible options. These are the same as for the axes, except for some default values. In addition, the following option is added: \opt{title=""}, which allows you to add a title at the top of the box; however, be careful to leave enough space for this.
\end{itemize}

\begin{demo}{Using Dgradbox}
\begin{luadraw}{name=gradbox}
local g = graph:new{window={-5,4,-5.5,5},size={10,10}}
local i, pi = cpx.I, math.pi
local h = function(x) return x^2/2-2 end
local f = function(x) return math.sin(3*x)+h(x) end
g:Dgradbox({-pi-4*i,pi+4*i,pi/3,1},{grid=true,originloc=0, originnum={0,0},labeltext={"\\pi",""},labelden={3,1}, title="\\textbf{Title}",legend={"Legend $x$","Legend $y$"}})
g:Saveattr(); g:Viewport(-pi,pi,-4,4) -- on limite la vue (clip)
g:Filloptions("full","blue",0.6); g:Linestyle("noline"); g:Ddomain2(f,h,{x={-pi/2,2*pi/3}})
g:Filloptions("none",nil,1); g:Lineoptions("solid",nil,8); g:Dcartesian(h,{x={-pi,pi}, draw_options="DarkBlue"})
g:Dcartesian(f,{x={-pi,pi},draw_options="Crimson"})
g:Restoreattr()
g:Show()
\end{luadraw}
\end{demo}

\subsection{Set Drawings (Venn Diagrams)}

\subsubsection{Drawing a Set}

The function \textbf{set(center,angle,scale)} returns a path representing a set (egg-shaped), with the center being \emph{center} (complex), the argument \emph{angle} representing the inclination (in degrees) of the set's vertical axis (0 by default), and the argument \emph{scale} being a scale factor to modify the size of the set (1 by default). Such a path can be drawn with the method \textbf{g:Dpath()}.

\begin{demo}{Drawing a Set}
\begin{luadraw}{name=set}
local g = graph:new{window={-5.25,5.25,-5,5},size={10,10}}
local i = cpx.I
local A, B, C = set(i,0), set(-2-i,25), set(2-i,-25)
g:Fillopacity(0.3)
g:Dpath(A,"fill=orange"); g:Dpath(B,"fill=blue")
g:Dpath(C,"fill=green")
g:Fillopacity(1)
g:Dlabel("$A$",5*i,{pos="N"},"$B$",-4+3*i,{pos="W"},"$C$",4+3*i,{pos="E"})
g:Show()
\end{luadraw}
\end{demo}

\subsubsection{Operations on Sets}

Let $C_1$ and $C_2$ be two lists of complex numbers representing the contour of two sets (simple closed curves, all in one piece).
There are three possible operations:
\begin{itemize}
    \item The function \textbf{cap(C1,C2)} returns a list of complex numbers representing the contour of the intersection of the sets corresponding to $C_1$ and $C_2$.
    \item The function \textbf{cup(C1,C2)} returns a list of complex numbers representing the contour of the union of the sets corresponding to $C_1$ and $C_2$.
    \item The function \textbf{setminus(C1,C2)} returns a list of complex numbers representing the contour of the difference of the sets corresponding to $C_1$ and $C_2$ ($C_1\setminus C_2$).
\end{itemize}
The result of these operations, being a list of complex numbers, can be drawn with the \textbf{g:Dpolyline()} method.

\begin{demo}{Set Operations}
\begin{luadraw}{name=cap_and_cup}
local g = graph:new{window={-5.5,5.5,-5,5},size={10,10}}
local i = cpx.I
local A, B, C = set(i,0), set(-2-i,25), set(2-i,-25)
g:Fillopacity(0.3)
g:Dpath(A,"fill=orange"); g:Dpath(B,"fill=blue"); g:Dpath(C,"fill=green")
g:Fillopacity(1)
local C1, C2, C3 = path(A), path(B), path(C) -- conversion chemin -> liste de complexes
local I = cap(cup(C1,C2),C3)
g:Linecolor("red"); g:Filloptions("full","white")
g:Dpolyline(I,true,"line width=0.8pt,fill opacity=0.8")
g:Dlabel("$A$",5*i,{pos="N"},"$B$",-4+3*i,{pos="W"},"$C$",4+3*i,{pos="E"},
"$(A\\cup B) \\cap C$",-i,{pos="NE",node_options="red,draw"})
g:Show()
\end{luadraw}
\end{demo}

\paragraph{NB}: The result is not always satisfactory when the contours become too complex, or when the contours share common sections.

\subsection{Color Calculations}

In the \emph{luadraw} environment, colors are character strings that must correspond to colors known to tikz. The \emph{xcolor} package is strongly recommended so as not to be limited to basic colors.

To be able to manipulate colors, they have been defined (in the \emph{luadraw\_colors.lua} module) as tables of three components: red, green, blue, each component being a number between $0$ and $1$, and with their names in the \emph{svgnames} format of the \emph{xcolor} package. For example, we find (among others) the declarations:

\begin{Luacode}
AliceBlue = {0.9412, 0.9725, 1}
AntiqueWhite = {0.9804, 0.9216, 0.8431}
Aqua = {0.0, 1.0, 1.0}
Aquamarine = {0.498, 1.0, 0.8314}
\end{Luacode}
You can refer to the \emph{xcolor} documentation for a list of these colors.

To use these in the \emph{luadraw} environment, you can:
\begin{itemize}
    \item either use them by name if you have declared them in the preamble: \verb|\usepackage[svgnames]{xcolor}|, for example: \mintinline{Lua}{g:Linecolor("AliceBlue")},

    \item or use them with the \emph{luadraw} \textbf{rgb()} function, for example: \mintinline{Lua}{g:Linecolor(rgb(AliceBlue))}. However, with this \emph{rgb()} function, to change the color locally, you must do the following (example): \par
\mintinline{Lua}{g:Dpolyline(L,"color="..rgb(AliceBlue))}, or \mintinline{Lua}{g:Dpolyline(L,"fill="..rgb(AliceBlue))}. Because the \emph{rgb()} function does not return a color name, but a color definition.
\end{itemize}

\paragraph{Functions for color management:}
\begin{itemize}
    \item The \textbf{rgb(r,g,b)} or \textbf{rgb(\{r,g,b\})} function returns the color as a string understandable by tikz in the \verb|color=...| and \verb|fill=...| options. The values ​​of $r$, $g$, and $b$ must be between $0$ and $1$.

    \item The \textbf{hsb(h,s,b,table)} function returns the color as a string understandable by tikz. The $h$ (hue) argument must be a number between $0$ and $360$, the $s$ (saturation) argument must be between $0$ and $1$, and the $b$ (brightness) argument must also be between $0$ and $1$.

The (optional) argument \emph{table} is a boolean (false by default) that indicates whether the result should be returned as a table \verb|{r,g,b}| or not (by default it is as a string).

    \item The function \textbf{mixcolor(color1,proportion1 color2,proportion1,...,colorN,proportionN)} mixes the colors \emph{color1}, ...,\emph{colorN} in the requested proportions and returns the resulting color as a string understandable by tikz, followed by the same color as a table \verb|{r,g,b}| . Each color must be a table of three components \verb|{r,g,b}|.

    \item The function \textbf{palette(colors,pos,table)}: the argument \emph{colors} is a list (table) of colors in the format \verb|{r,b,g}|, the argument \emph{pos} is a number between $0$ and $1$, the value $0$ corresponds to the first color in the list and the value $1$ to the last. The function calculates and returns the color corresponding to the position \emph{pos} in the list by linear interpolation. The (optional) argument \emph{table} is a boolean (false by default) that indicates whether the result should be returned as a table \verb|{r,g,b}| or not (by default it is as a string).

    \item The \textbf{getpalette(colors,nb,table)} function: the \emph{colors} argument is a list (table) of colors in \verb|{r,b,g}| format, the \emph{nb} argument indicates the desired number of colors. The function returns a list of \emph{nb} colors evenly distributed in \emph{colors}. The (optional) \emph{table} argument is a boolean (false by default) that indicates whether the colors are returned as \verb|{r,g,b}| tables or not (by default, they are as strings).

    \item The \textbf{g:Newcolor(name,rgbtable)} method allows you to define a new color in the tikz export in rgb format, whose name will be \emph{name} (string). \emph{rgbtable} is a table of three components: red, green, blue (between 0 and 1) defining this color.

\end{itemize}
You can also use all of TikZ's usual color management features.

By default, there are five color palettes.\footnote{A palette is a table of colors; these are themselves tables of numbers between $0$ and $1$ representing the red, green, and blue components.}.

\begin{demo}{The five default palettes}
\begin{luadraw}{name=palettes}
local g = graph:new{window={-5,5,-5,5},bbox=false, border=true}
g:Linewidth(1)
local Dpalette = function(pal,A,h,L,N,name)
    local dl = L/N
    for k = 1, N do
        local color = palette(pal,(k-1)/(N-1))
        g:Drectangle(A,A+h,A+h+dl,"color="..color..",fill="..color)
        A = A+dl
    end
    g:Drectangle(A,A+h,A+h-L); g:Dlabel(name,A+h/2,{pos="E"})
end
local A, h, dh, L, N = Z(-5,4), Z(0,-1), Z(0,-1.1), 5, 100
Dpalette(rainbow,A,h,L,N,"rainbow"); A = A+dh
Dpalette(palAutumn,A,h,L,N,"palAutumn"); A = A+dh
Dpalette(palGistGray,A,h,L,N,"palGistGray"); A = A+dh
Dpalette(palGasFlame,A,h,L,N,"palGasFlame"); A = A+dh
Dpalette(palRainbow,A,h,L,N,"palRainbow")
g:Show()
\end{luadraw}
\end{demo}
%
\section{Geometric Constructions}

This section groups together functions that construct geometric figures without a corresponding dedicated graphical method.

\subsection{circumcircle(), incircle()}

\begin{itemize}
\item The function \textbf{circumcircle(a,b,c)} (or \textbf{circumcircle(\{a,b,c\})}), where $a$, $b$, and $c$ are three points (three complex numbers), returns the circumcircle of the triangle formed by these three points, in the form of a sequence: $C,r$, where $C$ is the center of the circle (a complex number), and $r$ is its radius.
\item The function \textbf{incircle3d(a,b,c)} (or \textbf{incircle3d(\{a,b,c\})}), where $a$, $b$, and $c$ are three points (three complex numbers), returns the incircle of the triangle formed by these three points, as a sequence: $C,r$, where $C$ is the center of the circle (a complex number), and $r$ is its radius.
\end{itemize}

\subsection{cvx\_hull2d()}

The function \textbf{cvx\_hull2d(L)}, where $L$ is a list of complex numbers, computes and returns a list of complex numbers representing the convex hull of $L$.

\subsection{delaunay()}

The function \textbf{delaunay(L)} where \emph{L} is a list of \textbf{distinct} complex numbers, returns a list of triangles (a triangle being a list of three complex numbers) obtained by Delaunay triangulation of the points of \emph{L} (the circumcircle of each triangle does not contain any of the other points).

\begin{demo}{Delaunay Triangulation}
\begin{luadraw}{name=delaunay}
local g = graph3d:new{bbox=false, pictureoptions="scale=2"}
local i = cpx.I; g:Linewidth(6)
local L = {0.285+1.46*i,1.556-0.142*i,2.344+1.313*i,-2.38+1.218*i,1.548-0.624*i,0.969+1.819*i, -0.086-2.191*i,-0.477+1.834*i,-0.904+1.322*i,-2.892+0.025*i}
local T = delaunay(L)
local n = #T
local num = 7 -- we choose a triangle
local colors = getpalette(palGasFlame,n)
for k = 1, n do
    g:Dpolyline(T[k],true,'fill='..colors[k])
end
g:Ddots(L)
g:Dcircle( {circumcircle(T[num])}, "line width=0.4pt,gray,dashed" )
g:Ddots(T[num],"gray")
g:Show()
\end{luadraw}
\end{demo}

\subsection{voronoi()}

The function \textbf{voronoi(L, window)}, where \emph{L} is a list of distinct complex numbers, determines the Voronoi diagram of the points in the list $L$. This function returns a list of elements of the form \emph{\{A,polygon\}}, where \emph{A} is a point in the list $L$, and \emph{polygon} is a list of complex numbers representing the vertices of the cell associated with \emph{A}. Thus, there is one cell for each point in $L$. The cell for point $A$ contains the points in the plane that are closer to $A$ than to other points in $L$. This function uses Delaunay triangulation. The optional argument \emph{window}, which defaults to \emph{\{-5,5,-5,5\}}, is used to clip Voronoi cells that are unbounded; this window is automatically enlarged if necessary to contain all the points of $L$ as well as all the centers of the circles circumscribed about the Delaunay triangles (note: this does not change the 2d window of the current graph).

\begin{demo}{Voronoï diagram}
\begin{luadraw}{name=voronoi}
local g = graph:new{ bbox=true, margin={0,0,0,0}, size={10,10}}
local i = cpx.I
local S = {0.285+1.46*i,1.556-0.142*i,2.344+1.313*i,-2.38+1.218*i,1.548-0.624*i,
    0.969+1.819*i,-0.086-2.191*i,-0.477+1.834*i,-0.904+1.322*i,-2.892+0.025*i}
local V = voronoi(S)
local colors = getpalette(rainbow,#V)
for k,T in ipairs(V) do
    local A, polygon = table.unpack(T)
    g:Dpolyline(polygon,true,"color=white, line width=1.2pt,fill="..colors[k])
    g:Ddots(A,"mark=x,white,scale=2,line width=1.2pt")-- A is one of the points of S
end
g:Dpolyline(delaunay(S),true,"dotted,line width=0.6pt") -- Delaunay triangles
g:Show()
\end{luadraw}
\end{demo}


\subsection{line2strip()}

The function \textbf{line2strip(L,width,close,ends)} where \emph{L} is a list of complex numbers, or a list of lists of complex numbers, returns a path representing a "strip" centered on \emph{L} and of width \emph{width}. The optional argument \emph{close} is a boolean that indicates whether \emph{L} should be closed (\emph{false} by default). The optional argument \emph{ends} is a boolean that indicates whether both ends of the strip should be drawn (\emph{true} by default, except when the argument \emph{close} is \emph{true}).

\begin{demo}{Example with \emph{line2strip}}
\begin{luadraw}{name=line2strip}
local g = graph:new{bbox=false, bg="lightgray"}
local i = cpx.I; g:Linewidth(8)
local p = {-3+3*i,-3,"l",0,3,3,1,"ca", 3+3*i,"l"} 
g:Setmatrix({-3+3*i,0.5,0.5*i})
local L = line2strip(path(p),1,true) -- p is first converted to polyline
g:Dpath(L,"Crimson,fill=Gold"); g:Dpath(p,"gray,dashed")
g:Dlabel("close=true",i,{})
g:Setmatrix({3+3*i,0.5,0.5*i})
local L = line2strip(path(p),1,false) -- p is first converted to polyline
g:Dpath(L,"Crimson,fill=Gold"); g:Dpath(p,"gray,dashed")
g:Dlabel("close=false",i,{}) 
g:Setmatrix({-i,0.5,0.5*i})
local L = line2strip(path(p),1,false,false) -- p is first converted to polyline
g:Dpath(L,"Crimson,fill=Gold"); g:Dpath(p,"gray,dashed")
g:Dlabel("close=false",1.5*i,{}); g:Dlabel("ends=false",0.5*i,{}) 
g:Show()
\end{luadraw}
\end{demo}

\subsection{parallel\_polyline()}

The function \textbf{parallel\_polyline(L,width,close)} where \emph{L} is a list of complex numbers, or a list of lists of complex numbers, returns a polygonal line parallel to \emph{L} and located at a "distance" equal to \emph{width}. The argument \emph{width} can be positive or negative to be on one side or the other of \emph{L} (this depends on the direction of traversal of \emph{L}). The optional argument \emph{close} is a boolean that indicates whether \emph{L} should be closed (\emph{false} by default).


\subsection{sss\_triangle()}

The function \textbf{sss\_triangle(ab,bc,ca)}, where \emph{ab}, \emph{bc}, and \emph{ca} are three lengths, computes and returns a list of three points (3 complex numbers) $\{A,B,C\}$ forming the vertices of a direct triangle whose side lengths are the arguments, i.e., $AB=ab$, $BC=bc$, and $CA=ca$, when possible. Vertex A is always the complex 0 and vertex B is always the complex ab. This triangle can be drawn with the \textbf{g:Dpolyline} method.

\subsection{sas\_triangle()}

The function \textbf{sas\_triangle(ab,alpha,ca)}, where \emph{ab} and \emph{ca} are two lengths and \emph{alpha} is an angle in degrees, calculates and returns a list of three points (3 complex numbers) A, B, C forming the vertices of a triangle such that AB=ab, CA=ca, and angle (AB, AC) has a measure of alpha, whenever possible. Vertex A is always the complex 0 and vertex B is always the complex ab. This triangle can be drawn with the \textbf{g:Dpolyline} method.

\subsection{asa\_triangle()}

The function \textbf{asa\_triangle(alpha,ab,beta)}, where \emph{ab} is a length, \emph{alpha} and \emph{beta} are two angles in degrees, calculates and returns a list of three points (3 complex numbers) $\{A,B,C\}$ forming the vertices of a triangle such that $AB=ab$, such that angle $(\vec{AB},\vec{AC})$ has measure \emph{alpha}, and such that angle $(\vec{BA},\vec{BC})$ has measure \emph{beta}, whenever possible. Vertex $A$ is always complex $0$ and vertex $B$ is always complex $ab$. This triangle can be drawn using the \textbf{g:Dpolyline} method.

\begin{demo}{sss\_triangle, sas\_triangle and asa\_triangle}
\begin{luadraw}{name=sss_triangles_and_co}
local g = graph:new{window={-5,5,-3,5},size={10,10}}
g:Labelsize("footnotesize"); g:Linewidth(8)
local i = cpx.I
local T1 = shift( sss_triangle(4,5,3), 2*i-2)
local T2 = shift( sas_triangle(4,60,2), -4-2*i)
local T3 = shift( asa_triangle(30,4,50), 0.5-i)
g:Dpolyline({T1,T2,T3}, true)
g:Linewidth(4)
g:Darc(T2[2],T2[1],T2[3],0.5,1,"->")
g:Darc(T3[2],T3[1],T3[3],0.75,1,"->")
g:Darc(T3[1],T3[2],T3[3],0.75,-1,"->")
g:Dlabel( 
    "$4$",(T1[1]+T1[2])/2,{pos="N"}, "$5$",(T1[2]+T1[3])/2,{pos="NE"},"$3$",(T1[1]+T1[3])/2,{pos="W"},
    "$4$",(T2[1]+T2[2])/2,{pos="N"}, "$60^\\circ$",T2[1]+Zp(0.9,30*deg),{pos="center"},"$2$",(T2[1]+T2[3])/2,{pos="W"},
    "$4$",(T3[1]+T3[2])/2,{pos="N"}, "$30^\\circ$",T3[1]+Zp(1.15,15*deg),{pos="center"},
    "$50^\\circ$",T3[2]+Zp(1.15,155*deg),{pos="center"},
    "sss\\_triangle(4,5,3)",(T1[1]+T1[2])/2,{pos="S"}, "sas\\_triangle(4,60,2)",(T2[1]+T2[2])/2,{}, "asa\\_triangle(30,4,50)",(T3[1]+T3[2])/2,{})
for _,T in ipairs({T1,T2,T3}) do
    g:Dlabel("$A$",T[1],{pos="SW"}, "$B$",T[2],{pos="SE"},"$C$",T[3],{pos="N"})
end
g:Show()
\end{luadraw}
\end{demo}
%
\section{Computations on Lists}

\subsection{concat}
The function \textbf{concat\{table1, table2, \ldots{} \}} concatenates all the tables passed as arguments and returns the resulting table.

\begin{itemize}
    \item Each argument can be a real number, a complex number, or a table.
    \item Example: The instruction \mintinline{Lua}{concat( 1,2,3,{4,5,6},7 )} returns the table \emph{\{1,2,3,4,5,6,7\}}.
\end{itemize}

\subsection{cut}
The function \textbf{cut(L,A,before)} cuts \emph{L} at the point \emph{A}, which is assumed to be located on the line \emph{L} (\emph{L} is either a list of complex numbers or a polygonal line, i.e., a list of lists of complex numbers). If the argument \emph{before} is \emph{false} (the default value), then the function returns the part before \emph{A}, followed by the part after \emph{A}; otherwise, the reverse is true.

\subsection{cutpolyline}
The function \textbf{cutpolyline(L,D,close)} cuts the polygonal line \emph{L} with the straight line \emph{D}. The argument \emph{L} must be a list of complex numbers or a list of lists of complex numbers, the argument \emph{D} is a list of the form \emph{\{A,u\}} where is a complex (point on the line) and $u$ is a non-zero complex (direction vector of the line). The argument \emph{close} indicates whether the line \emph{L} should be closed (false by default). The function returns three things:
\begin{itemize}
    \item The part of \emph{L} that is in the half-plane defined by the line to the "left" of $u$ (i.e., containing the point $A+iu$) (it is a polygonal line),
    \item followed by the part of \emph{L} that is in the other half-plane (polygonal line),
    \item followed by the list of intersection points between \emph{L} and the line. \end{itemize}

\begin{demo}{Illustrate a linear programming exercise}
\begin{luadraw}{name=cutpolyline}
local g = graph:new{window={-5,5,-5,5}, size={10,10},margin={0,0,0,0}}
g:Linewidth(6)
local i = cpx.I
local P = g:Box2d() -- polygon representing the 2d window
local D1, D2, D3 = {0,1+i}, {2.5,-i}, {-3*i,-1-i/4}  -- three lines
local P1 = cutpolyline(P,D1,true)
local P2 = cutpolyline(P,D2,true)
local P3 = cutpolyline(P,D3,true)
g:Daxes({0,1,1},{grid=true,gridcolor="LightGray",arrows="->",legend={"$x$","$y$"}})
g:Filloptions("horizontal","blue"); g:Dpolyline(P1,true,"draw=none")
g:Filloptions("fdiag","red"); g:Dpolyline(P2,true,"draw=none")
g:Filloptions("bdiag","green"); g:Dpolyline(P3,true,"draw=none")
g:Filloptions("none","black",1)
g:Linewidth(8)
g:Dline(D1,"blue"); g:Dline(D2,"red"); g:Dline(D3,"green")
g:Dlabel(
    "$x-y\\leqslant 0$",-3-3*i,{pos="N",dir={1+i,-1+i},dist=0.1,node_options="fill=white,fill opacity=0.8"},
    "$x-2.5\\geqslant0$", 2.5+i,{dir={-i,1}},
    "$-\\frac{x}{4}+y+3\\leqslant0$", -3-15/4*i,{pos="S",dir={1+i/4,i-1/4}}
)
g:Show()
\end{luadraw}
\end{demo}

\subsection{getbounds}
\begin{itemize}
    \item The function \textbf{getbounds(L)} returns the bounds xmin, xmax, ymin, ymax of the polygonal line \emph{L}.
    \item Example: \mintinline{Lua}{ local xmin, xmax, ymin, ymax = getbounds(L)} (where \emph{L} denotes a polygonal line).
\end{itemize}

\subsection{getdot}
The function \textbf{getdot(x,L)} returns the point with abscissa \emph{x} (real between $0$ and $1$) along the connected component \emph{L} (list of complex numbers). The abscissa $0$ corresponds to the first point and the abscissa $1$ to the last. More generally, \emph{x} corresponds to a percentage of the length of \emph{L}.

\subsection{insert}
The function \textbf{insert(table1, table2, pos)} inserts the elements of \emph{table2} into \emph{table1} at position \emph{pos}.

\begin{itemize}
    \item The argument \emph{table2} can be a real number, a complex number, or a table.
    \item The argument \emph{table1} must be a variable that designates a table; this will be modified by the function.
    \item If the argument \emph{pos} is nil, the insertion is performed at the end of \emph{table1}.
    \item Example: If a variable \emph{L} is equal to \emph{\{1,2,6\}}, then after the instruction \mintinline{Lua}{insert(L, {3,4,5},3)}, the variable \emph{L} will be equal to \emph{\{1,2,3,4,5,6\}}.
\end{itemize}

\subsection{interCC}
The function \textbf{interCC(C1,C2)} returns the intersection of circle \emph{C1} with circle \emph{C2}, where \emph{C1=\{O1,r1\}} (circle with center $O$1 and radius $r$1), and \emph{C2=\{O2,r2\}} (circle with center $O2$ and radius $r2$). The function returns a list containing 1 or 2 points or the entire circle. If the intersection is not empty, it returns nil.


\begin{demo}{Tangents to a circle {O,2} and to an ellipse {O,3,2} from a point}
\begin{luadraw}{name=interCC}
local g = graph:new{window={-10,10,-5,5}, margin={0,0,0,0},size={16,8}}
local i = cpx.I
-- pour le cercle {O,2}
g:Saveattr(); g:Viewport(-10,0,-5,5); g:Coordsystem(-4,6,-5,5)
local O = -1 
local C1, I = {O, 2}, 4-i
local C2 = {(O+I)/2,cpx.abs(I-O)/2}
local rep = interCC(C1,C2) -- points de tangence
g:Dcircle(C1,"blue"); g:Dcircle(C2,"dashed")
g:Dhline(I,rep[1],"red"); g:Dhline(I,rep[2],"red")  --demi- tangentes
g:Ddots(rep); g:Ddots({O,I}); g:Dlabel("$I$",I,{pos="SE"},"$O$",O,{pos="W"},
    "tangentes au cercle issues de $I$",1-5*i,{pos="N"})
g:Restoreattr()

-- pour l'ellipse (E) : {O,3,2}
g:Saveattr(); g:Viewport(0,10,-5,5); g:Coordsystem(-4,6,-5,5)
local mat = {0,1.5,i} -- cette matrice transforme un cercle {01,2} en l'ellipse (E)
local inv_mat = invmatrix(mat) -- matrice inverse
local O1, I1 = table.unpack( mtransform({O,I},inv_mat) ) -- antécédents de O et de I
C1 = {O1, 2}
C2 = {(O1+I1)/2,cpx.abs(I1-O1)/2}
rep = interCC(C1,C2) -- points de tangence (tangentes issues de I1)
g:Composematrix(mat) -- on applique la matrice pour retrouver l'ellipse, la tangence est conservée
g:Dcircle(C1,"blue"); g:Dcircle(C2,"dashed")
g:Dhline(I1,rep[1],'red'); g:Dhline(I1,rep[2],"red")
g:Ddots(rep); g:Ddots({O1,I1}); g:Dlabel("$I$",I1,{pos="SE"},"$O$",O1,{pos="W"},
    "tangentes à l'ellipse issues de $I$",1-5*i,{pos="N"})
g:Restoreattr()
g:Show()
\end{luadraw}
\end{demo}

\subsection{interD}
The function \textbf{interD(d1,d2)} returns the intersection point of the lines \emph{d1} and \emph{d2}. A line is a list of two complex numbers: a point on the line and a direction vector.

\subsection{interDC}
The function \textbf{interDC(d,C)} returns the intersection of the line \emph{d} with the circle \emph{C}, where \emph{d=\{A,u\}} (a line passing through $A$ and directed by $u$), and \emph{C=\{O,r\}} (a circle with center $O$ and radius $r$). The function returns a list containing $1$ or $2$ points if the intersection is not empty; otherwise, it returns \emph{nil}.

\subsection{interDL}
The function \textbf{interDL(d,L)} returns the list of intersection points between the straight line \emph{d} and the polygonal line \emph{L}.

\subsection{interL}
The function \textbf{interL(L1,L2)} returns the list of intersection points of the polygonal lines defined by \emph{L1} and \emph{L2}. These two arguments are two lists of complex numbers or two lists of lists of complex numbers.

\subsection{interP}
The function \textbf{interP(P1,P2)} returns the list of intersection points of the paths defined by \emph{P1} and \emph{P2}. These two arguments are two lists of complex numbers and instructions (see \emph{Dpath}).

\subsection{isobar}
The function \textbf{isobar(L)}, where \emph{L} is a list of complex numbers, returns the isobarycenter of these numbers. If \emph{L} contains elements that are not real or complex numbers, they are ignored.

\subsection{linspace}
The function \textbf{linspace(a,b,nbdots)} returns a list of \emph{nbdots} equally distributed numbers from \emph{a} to \emph{b}. By default, \emph{nbdots} is 50.

\subsection{map}
The function \textbf{map(f,list)} applies the function \emph{f} to each element of the \emph{list} and returns the table of results. When a result is \emph{nil}, the complex \emph{cpx.Jump} is inserted into the list.

\subsection{merge}
The function \textbf{merge(L)} reassembles, if possible, the connected components of \emph{L}, which must be a list of lists of complex numbers. The function returns the result.

\subsection{range}
The function \textbf{range(a,b,step)} returns the list of numbers from \emph{a} to \emph{b} with a step equal to \emph{step}, which is 1 by default.

\subsection{Clipping Functions}

\begin{itemize}
    \item The function \textbf{clipseg(A,B,xmin,xmax,ymin,ymax)} clips the segment \emph{{[}A,B{]}} with the window \emph{{[}xmin,xmax{]}x{[}ymin,ymax{]}} and returns the result.
    \item The function \textbf{clipline(d,xmin,xmax,ymin,ymax)} clips the line \emph{d} with the window \emph{{[}xmin,xmax{]}x{[}ymin,ymax{]}} and returns the result. The line \emph{d} is a list of two complex numbers: a point and a direction vector.
    \item The function \textbf{clippolyline(L,xmin,xmax,ymin,ymax,close)} clips the polygonal line \emph{L} with \emph{{[}xmin,xmax{]}x{[}ymin,ymax{]}} and returns the result. The argument \emph{L} is a list of complex numbers or a list of lists of complex numbers. The optional argument \emph{close} (false by default) indicates whether the polygonal line should be closed.
    \item The function \textbf{clipdots(L,xmin,xmax,ymin,ymax)} clips the point list \emph{L} with the window \emph{{[}xmin,xmax{]}x{[}ymin,ymax{]}} and returns the result (exterior points are simply excluded). The argument \emph{L} is a list of complex numbers or a list of lists of complex numbers.
\end{itemize}

\subsection{Adding Mathematical Functions}
In addition to the functions associated with graphics methods that perform calculations and return a polygonal line (such as \emph{cartesian}, \emph{periodic}, \emph{implicit}, \emph{odesolve}, etc.), the \emph{luadraw} package adds some mathematical functions that are not natively provided in the \emph{math} module.

\subsubsection{Protected Evaluation: evalf}
The \textbf{evalf(f,...)} function allows you to evaluate \emph{f(...)} and return the result if there is no runtime error in Lua; otherwise, the function returns \emph{nil}. For example, executing:
\begin{Luacode}
local f = function(a,b)
    return 2*Z(a,1/b)
end
print(f(1,0))
\end{Luacode}
causes the runtime error \verb|attempt to perform arithmetic on a nil value| (in the console), because here \emph{Z(1,1/0)} returns \emph{nil}, and Lua does not accept an argument equal to \emph{nil} in a calculation. On the other hand, executing:
\begin{Luacode}
local f = function(a,b)
    return 2*Z(a,1/b)
end
print(evalf(f,1,0))
\end{Luacode}
does not cause an error from Lua, and there is no output to the console either since the value to be displayed is \emph{nil}.

\subsubsection{int}
The function \textbf{int(f,a,b)} returns an approximate value of the integral of the function \emph{f} over the interval $[a;b]$. The function \emph{f} is a real variable and has real or complex values. The method used is Simpson's method, accelerated twice with the Romberg method.

\paragraph{Example:}
\begin{TeXcode}
$\int_0^1 e^{t^2}\mathrm d t \approx \directlua{tex.sprint(int(function(t) return math.exp(t^2) end, 0, 1))}$
\end{TeXcode}
\paragraph{Result:} $\int_0^1 e^{t^2}\mathrm d t \approx \directlua{tex.sprint(int(function(t) return math.exp(t^2) end, 0, 1))}$.

\subsubsection{gcd}
The function \textbf{gcd(a,b)} returns the greatest common divisor between $a$ and $b$.

\subsubsection{lcm}
The function \textbf{lcm(a,b)} returns the smallest positive common divisor between $a$ and $b$.

\subsubsection{solve}
The function \textbf{solve(f,a,b,n)} numerically solves the equation $f(x)=0$ in the interval $[a;b]$, which is subdivided into $n$ pieces ($n$ is $25$ by default). The function returns a list of results or \emph{nil}. The method used is a Newtonian variant.

\paragraph{Example 1:}
\begin{TeXcode}
\begin{luacode}
resol = function(f,a,b) 
    local y = solve(f,a,b) 
    if y == nil then tex.sprint("\\emptyset") 
    else 
        local str = y[1] 
        for k = 2, #y do 
            str = str..", "..y[k] 
        end 
        tex.sprint(str) 
    end
end
\end{luacode}
\def\solve#1#2#3{\directlua{resol(#1,#2,#3)}}%
\begin{luacode}
f1 = function(x) return math.cos(x)-x end
f2 = function(x) return x^3-2*x^2+1/2 end
\end{luacode}
Solving the equation $\cos(x)=x$ in $[0;\frac{\pi}2]$ gives $\solve{f1}{0}{math.pi/2}$.\par
Solving the equation $\cos(x)=x$ in $[\frac{\pi}2;\pi]$ gives $\solve{f1}{math.pi/2}{math.pi}$.\par
Solving the equation $x^3-2x^2+\frac12=0$ in $[-1;2]$ gives: $\{\solve{f2}{-1}{2}\}$.
\end{TeXcode}
\paragraph{Result:}\ \par

\begin{luacode}
resol = function(f,a,b) 
    local y = solve(f,a,b) 
    if y == nil then tex.sprint("\\emptyset") 
    else 
        local str = y[1] 
        for k = 2, #y do 
            str = str..", "..y[k] 
        end 
        tex.sprint(str) 
    end
end
\end{luacode}
\def\solve#1#2#3{\directlua{resol(#1,#2,#3)}}%
\begin{luacode}
f1 = function(x) return math.cos(x)-x end
f2 = function(x) return x^3-2*x^2+1/2 end
\end{luacode}

Solving the equation $\cos(x)=x$ in $[0;\frac{\pi}2]$ gives $\solve{f1}{0}{math.pi/2}$.\par
Solving the equation $\cos(x)=x$ in $[\frac{\pi}2;\pi]$ gives $\solve{f1}{math.pi/2}{math.pi}$.\par
Solving the equation $x^3-2x^2+\frac 12=0$ in $[-1;2]$ gives: $\{\solve{f2}{-1}{2}\}$.

\paragraph{Example 2:} We want to plot the curve of the function $f$ defined by the condition:
\[\forall x\in \mathbf R,\ \int_x^{f(x)} \exp(t^2)\mathrm d t = 1.\]
We have two possible methods:
\begin{enumerate}
    \item We consider the function $G\colon (x,y) \mapsto \int_x^y \exp(t^2)\mathrm d t-1$, and we draw the implicit curve with equation $G(x,y)=0$.
    \item We determine a real number $y_0$ such that $\int_0^{y_0}\exp(t^2)\mathrm d t = 1$ and we draw the solution to the differential equation $y'=e^{x^2-y^2}$ satisfying the initial condition $y(0)=y_0$.
\end{enumerate}
Let's draw both:
\begin{demo}{Function $f$ defined by $\int_x^{f(x)} \exp(t^2)\mathrm d t = 1$.}
\begin{luadraw}{name=int_solve}
local g = graph:new{window={-3,3,-3,3},size={10,10}}
local h = function(t) return math.exp(t^2) end
local G = function(x,y) return int(h,x,y)-1 end
local H = function(y) return G(0,y) end
local F = function(x,y) return math.exp(x^2-y^2) end
local y0 = solve(H,0,1)[1] -- solution de H(x)=0
g:Daxes({0,1,1}, {arrows="->"})
g:Dimplicit(G, {draw_options="line width=4.8pt,Pink"})
g:Dodesolve(F,0,y0,{draw_options="line width=0.8pt"}) 
g:Lineoptions("dashed","gray",4); g:DlineEq(1,-1,0); g:DlineEq(1,1,0) -- bissectrices
g:Dlabel("${\\mathcal C}_f$",Z(2.15,2),{pos="S"})
g:Show()
\end{luadraw}
\end{demo}

We see that the two curves overlap well, however the first method (implicit curve) is much more computationally intensive, so method 2 is preferable.
%
\section{Transformations}
In the following:
\begin{itemize}
    \item the argument \emph{L} is either a complex number, a list of complex numbers, or a list of lists of complex numbers,
    \item the line \emph{d} is a list of two complex numbers: a point on the line and a direction vector.
\end{itemize}

\subsection{affin}
The function \textbf{affin(L,d,v,k)} returns the image of \emph{L} by the affinity of base line \emph{d}, parallel to the vector \emph{v} and of ratio \emph{k}.

\subsection{ftransform}
The function \textbf{ftransform(L,f)} returns the image of \emph{L} by the function \emph{f}, which must be a function of the complex variable. If one of the elements of \emph{L} is the complex number \emph{cpx.Jump}, then it is returned as is in the result.

\subsection{hom}
The function \textbf{hom(L,factor,center)} returns the image of \emph{L} by the homothety with center \emph{center} and ratio \emph{factor}. By default, the argument \emph{center} is 0.

\subsection{inv}
The function \textbf{inv(L, radius, center)} returns the image of \emph{L} by the inversion with respect to the circle with center \emph{center} and radius \emph{radius}. By default, the argument \emph{center} is 0.

\subsection{proj}
The function \textbf{proj(L,d)} returns the image of \emph{L} by the orthogonal projection onto the line \emph{d}.

\subsection{projO}
The function \textbf{projO(L,d,v)} returns the image of \emph{L} by projection onto the line \emph{d} parallel to the vector \emph{v}.

\subsection{rotate}
The function \textbf{rotate(L,angle,center)} returns the image of \emph{L} by rotation with center \emph{center} and angle \emph{angle} (in degrees). By default, the argument \emph{center} is 0.

\subsection{shift}
The function \textbf{shift(L,u)} returns the image of \emph{L} by translation of vector \(u\).

\subsection{simil}
The function \textbf{simil(L,factor,angle,center)} returns the image of \emph{L} by the similarity of center \emph{center}, ratio \emph{factor}, and angle \emph{angle} (in degrees). By default, the argument \emph{center} is 0.

\subsection{sym}
The function \textbf{sym(L,d)} returns the image of \emph{L} by the orthogonal symmetry of axis \emph{d}.

\subsection{symG}
The function \textbf{symG(L,d,v)} returns the image of \emph{L} by the symmetry about the line \emph{d} followed by the translation of vector \emph{v} (sliding symmetry).

\subsection{symO}
The function \textbf{symO(L,d)} returns the image of \emph{L} by symmetry with respect to the line \emph{d} and parallel to the vector \emph{v} (oblique symmetry).

\begin{demo}{Using Transformations}
\begin{luadraw}{name=Sierpinski}
local g = graph:new{window={-5,5,-5,5},size={10,10}}
local i = cpx.I
local rand = math.random
local A, B, C = 5*i, -5-5*i, 5-5*i -- triangle initial
local T, niv = {{A,B,C}}, 5
for k = 1, niv do
    T = concat( hom(T,0.5,A), hom(T,0.5,B), hom(T,0.5,C) )
end
for _,cp in ipairs(T) do
    g:Filloptions("full", rgb(rand(),rand(),rand()))
    g:Dpolyline(cp,true)
end
g:Show()
\end{luadraw}
\end{demo}
%
\section{Matrix Calculus}

If $f$ is an affine application of the complex plane, we will call the list (table) of $f$ the matrix:
\begin{Luacode}
{ f(0), Lf(1), Lf(i) }
\end{Luacode}
where $Lf$ denotes the linear part of $f$ (we have $Lf(1) = f(1)-f(0)$ and $Lf(i) = f(i)-f(0)$). The identity matrix is ​​denoted \emph{ID} in the \emph{luadraw} package; it simply corresponds to the list \mintinline{Lua}{ {0,1,i} }.

\subsection{Matrix Calculations}

\subsubsection{applymatrix and applyLmatrix}
\begin{itemize}
    \item The function \textbf{applymatrix(z,M)} applies the matrix $M$ to the complex $z$ and returns the result (which is equivalent to calculating $f(z)$ if $M$ is the matrix of $f$). When $z$ is the complex \emph{cpx.Jump} then the result is \emph{cpx.Jump}. When $z$ is a string then the function returns $z$.
    \item The function \textbf{applyLmatrix(z,M)} applies the linear part of the matrix $M$ to the complex $z$ and returns the result (which is equivalent to calculating $Lf(z)$ if $M$ is the matrix of $f$). When $z$ is the complex \emph{cpx.Jump} then the result is \emph{cpx.Jump}.
\end{itemize}

\subsubsection{composematrix}
The function \textbf{composematrix(M1,M2)} performs the matrix product $M1\times M2$ and returns the result.

\subsubsection{invmatrix}
The function \textbf{invmatrix(M)} calculates and returns the inverse of the matrix $M$ when possible.

\subsubsection{matrixof}
\begin{itemize}
    \item The function \textbf{matrixof(f)} calculates and returns the matrix of $f$ (which must be an affine application of the complex plane.
    \item Example: \mintinline{Lua}{ matrixof( function(z) return proj(z,{0,Z(1,-1)}) end )} returns \par
\mintinline{Lua}{{0,Z(0.5,-0.5),Z(-0.5,0.5)}} (matrix of the orthogonal projection onto the second bisector).
\end{itemize}

\subsubsection{mtransform and mLtransform}
\begin{itemize}
    \item The function \textbf{mtransform(L,M)} applies the matrix $M$ to the list $L$ and returns the result. $L$ must be a list of complex numbers or a list of lists of complex numbers. If one of them is the complex \emph{cpx.Jump} or a string, then it is unchanged (and therefore returned as is).
    \item The function \textbf{mLtransform(L,M)} applies the linear part of the matrix $M$ to the list $L$ and returns the result. $L$ must be a list of complex numbers. If one of them is the complex \emph{cpx.Jump}, then it is unchanged.
\end{itemize}

\subsection{Matrix associated with the graph}

When creating a graph in the \emph{luadraw} environment, for example:
\begin{Luacode}
local g = graph:new{window={-5,5,-5,5},size={10,10}}
\end{Luacode}
The created \emph{g} object has a transformation matrix that is initially the identity. All graphics methods used automatically apply the graph's transformation matrix. This matrix is ​​designated \mintinline{Lua}{g.matrix}, but to manipulate it, the following methods are available.

\subsubsection{g:Composematrix()}
The \textbf{g:Composematrix(M)} method multiplies the graph matrix \emph g by the matrix \emph{M} (with \emph{M} on the right) and assigns the result to the graph matrix. The argument \emph{M} must therefore be a matrix.

\subsubsection{g:Det2d()()}
The \textbf{g:Det2d()} method returns $1$ when the transformation matrix has a positive determinant, and $-1$ otherwise. This information is useful when we need to know whether the plane orientation has been changed or not.

\subsubsection{g:IDmatrix()}
The \textbf{g:IDmatrix()} method reassigns the identity to the graph matrix \emph g.

\subsubsection{g:Mtransform()}
The \textbf{g:Mtransform(L)} method applies the graph matrix \emph g to \emph{L} and returns the result. The argument \emph L must be a list of complex numbers, or a list of lists of complex numbers.

\subsubsection{g:MLtransform()}
The \textbf{g:MLtransform(L)} method applies the linear part of the graph matrix \emph g to \emph{L} and returns the result. The argument \emph L must be a list of complex numbers, or a list of lists of complex numbers.

\begin{demo}{Using the Graph Matrix}
\begin{luadraw}{name=Pythagore}
local g = graph:new{window={-15,15,0,22},size={10,10}}
local a, b, c = 3, 4, 5 -- un triplet de Pythagore
local i, arccos, exp = cpx.I, math.acos, cpx.exp
local f1 = function(z)
        return (z-c)*a/c*exp(-i*arccos(a/c))+c+i*c end
local M1 = matrixof(f1)
local f2 = function(z)
        return z*b/c*exp(i*arccos(b/c))+i*c end
local M2 = matrixof(f2)
local arbre
arbre = function(n)
    local color = mixcolor(ForestGreen,1,Brown,n)
    g:Linecolor(color); g:Dsquare(0,c,1,"fill="..color)
    if n > 0 then
        g:Savematrix(); g:Composematrix(M1); arbre(n-1)
        g:Restorematrix(); g:Savematrix(); g:Composematrix(M2)
        arbre(n-1); g:Restorematrix()
    end
end
arbre(8)
g:Show()
\end{luadraw}
\end{demo}


\subsubsection{g:Rotate()}
The \textbf{g:Rotate(angle, center)} method modifies the transformation matrix of the graph \emph g by composing it with the rotation matrix with angle \emph{angle} (in degrees) and center \emph{center}. The argument \emph{center} is a complex matrix that defaults to $0$.

\subsubsection{g:Scale()}
The \textbf{g:Scale(factor, center)} method modifies the transformation matrix of the graph \emph g by composing it with the homothety matrix with ratio \emph{factor} and center \emph{center}. The argument \emph{center} is a complex matrix that defaults to $0$.

\subsubsection{g:Savematrix() and g:Restorematrix()}
\begin{itemize}
    \item The \textbf{g:Savematrix()} method saves the transformation matrix of graph \emph g to a stack.
    \item The \textbf{g:Restorematrix()} method restores the transformation matrix of graph \emph g to its last saved value.
\end{itemize}

\subsubsection{g:Setmatrix()}
The \textbf{g:Setmatrix(M)} method assigns matrix \emph M to the transformation matrix of graph \emph g.

\subsubsection{g:Shift()}
The \textbf{g:Shift(v)} method modifies the transformation matrix of graph \emph g by compositing it with the translation matrix of vector \emph{v}, which must be a complex matrix.

\begin{demo}{Using Shift, Rotate and Scale}
\begin{luadraw}{name=free_art}
local du = math.sqrt(2)/2
local g = graph:new{window={1-du,4+du,1-du,4+du},
            margin={0,0,0,0},size={7,7}}
local i = cpx.I
g:Linestyle("noline")
g:Filloptions("full","Navy",0.1)
for X = 1, 4 do
    for Y = 1, 4 do
        g:Savematrix()
        g:Shift(X+i*Y); g:Rotate(45)
        for k = 1, 25 do
            g:Dsquare((1-i)/2,(1+i)/2,1)
            g:Rotate(7); g:Scale(0.9)
        end
        g:Restorematrix()
    end
end
g:Show()
\end{luadraw}
\end{demo}

\subsection{View change. Change of coordinate system}

\paragraph{View change: } when creating a new graph, for example:
\begin{Luacode}
local g = graph:new{window={-5,5,-5,5},size={10,10}}
\end{Luacode}
The option \emph{window=\{xmin,xmax,ymin,ymax\}} sets the view for graph \emph{g}. This will be the \emph{[xmin, xmax]x [ymin, ymax]} block of $\mathbf R^2$, and all plots will be clipped by this window (except labels, which can overflow into the margins, but not beyond). Within this box, it is possible to define another box to create a new view, using the \textbf{g:Viewport(x1,x2,y1,y2)} method. The values ​​of \emph{x1}, \emph{x2}, \emph{y1}, \emph{y2} refer to the initial window defined by the \emph{window} option. From then on, everything outside this new area will be clipped, and the graph matrix will be reset to the same value. Therefore, you must first save the current graphics settings:
\begin{Luacode}
g:Saveattr()
g:Viewport(x1,x2,y1,y2)
\end{Luacode}
To return to the previous view with the previous matrix, simply restore the graphics settings with the \textbf{g:Restoreattr()} method.

\paragraph{Warning: } Each \emph{Saveattr()} instruction must correspond to a \emph{Restoreattr()} instruction, otherwise a compilation error will occur.

\paragraph{Changing the coordinate system: } The coordinate system of the current view can be changed with :\par
\hfil\textbf{g:Coordsystem(x1,x2,y1,y2,ortho)}.\hfil\par
 This method will modify the graph matrix so that everything occurs as if the current view corresponded to the $[x1,x2]\times[y1,y2]$ box. The optional Boolean argument \emph{ortho} indicates whether the new coordinate system should be orthonormal or not (false by default). Since the graph matrix is ​​modified, it is best to save the graphical parameters first and restore them later. This can be used, for example, to create multiple figures in the current graph.

\begin{demo}{Classification of the points of a parametric curve}
\begin{luadraw}{name=viewport_changewin}
local g = graph:new{window={-5,5,-5,5},size={10,10}}
local i = cpx.I
g:Labelsize("tiny") 
g:Writeln("\\tikzset{->-/.style={decoration={markings, mark=at position #1 with {\\arrow{>}}}, postaction={decorate}}}")
g:Dline({0,1},"dashed,gray"); g:Dline({0,i},"dashed,gray")
local legende = {"Point ordinaire", "Point d'inflexion", "Rebroussement 1ère espèce", "Rebroussement 2ème espèce"}
local A, B, C =(1+i)*0.75, 0.75, 0
local A2, B2 ={-1.25+i*0.5,-0.75-i*0.5,1.25-0.5*i, 0.5+i}, {-0.75,-0.75,0.75,0.75}
local u = {Z(-5,0),Z(0,0),-5-5*i,-5*i}
for k = 1, 4 do
    g:Saveattr(); g:Viewport(u[k].re,u[k].re+5,u[k].im,u[k].im+5)
    g:Coordsystem(-1.4,2.25,-1,1.25)
    g:Composematrix({0,1,1+i}) -- pour pencher l'axe Oy
    g:Dpolyline({{-1,1},{-i*0.5,i}}) -- axes
    g:Lineoptions(nil,"blue",8)
    g:Dpath({A2[k],(B2[k]+2*A2[k])/3,(C+5*B2[k])/6, C,"b"},"->-=0.5")
    g:Dpath({C,(C+5*B)/6,(B+2*A)/3,A,"b"},"->-=0.75")
    g:Dpolyline({{0,0.75},{0,0.75*i}},false,"->,red")
    g:Dlabel(
        legende[k],0.75-0.5*i, {pos="S"},
        "$f^{(p)}(t_0)$",1,{pos="E",node_options="red"},
        "$f^{(q)}(t_0)$",0.75*i,{pos="W",dist=0.05})
    g:Restoreattr()
end
g:Show()
\end{luadraw}
\end{demo}
%
\section{Adding Your Own Methods}

Without having to modify the Lua source files associated with the \emph{luadraw} package, you can add your own methods to the \emph{graph} class, or modify an existing method. This is only useful if these modifications will be used in different graphs and/or different documents (otherwise, you can simply write a function locally in the graph where it's needed).

\subsection{An Example}
In the graph on page \pageref{field}, we drew a vector field. To do this, we wrote a function that calculates the vectors before drawing, but this function is local. We could make it a global function (by removing the \emph{local} keyword), which would then be usable throughout the document, but not in another document!

To generalize this function, we will need to create a Lua file that can then be imported into documents if necessary. To make the example a bit more consistent, we'll create a file that defines a function that calculates the vectors of a field, and that will add two new methods to the \emph{graph} class: one to draw a vector field of a function $f\colon(x,y)\to(x,y)\in \mathbf R^2$, we'll name it \emph{graph:Dvectorfield}, and another to draw a gradient field of a function $f\colon(x,y)\to\mathbf R$, we'll name it \emph{graph:Dgradientfield}. Therefore, we'll call this file: \emph{luadraw\_fields.lua}.

\paragraph{File contents:}
\begin{Luacode}
-- luadraw_fields.lua
-- added methods to the graph class of the luadraw package
-- to draw vector or gradient fields
function field(f,x1,x2,y1,y2,grid,long)  -- mathematical function, independent of the graph
-- calcule un champ de vecteurs dans le pavé [x1,x2]x[y1,y2]
-- f fonction de deux variables à valeurs dans R^2
-- grid = {nbx, nby} : nombre de vecteurs suivant x et suivant y
-- long = longueur d'un vecteur
    if grid == nil then grid = {25,25} end
    local deltax, deltay = (x2-x1)/(grid[1]-1), (y2-y1)/(grid[2]-1) -- x and y step
    if long == nil then long = math.min(deltax,deltay) end -- default length
    local vectors = {} -- will contain the list of vectors
    local x, y, v = x1 
    for _ = 1, grid[1] do -- a loop on x
        y = y1
        for _ = 1, grid[2] do -- a loop on y
            v = f(x,y) -- we assume that v is well defined
            v = Z(v[1],v[2]) -- to complex number
            if not cpx.isNul(v) then
                v = v/cpx.abs(v)*long -- normalization
                table.insert(vectors, {Z(x,y), Z(x,y)+v} ) -- we add the vector
            end
            y = y+deltay
        end
        x = x+deltax
    end
    return vectors -- we return the result (polygonal line)
end

function graph:Dvectorfield(f,args) -- added a method to the graph class
-- draws a vector field
-- f is a function of two variables with values ​​in R^2
-- args is a 4-field table:
-- { view={x1,x2,y1,y2}, grid={nbx,nby}, long=, draw_options=""}
    args = args or {}
    local view = args.view or {self:Xinf(),self:Xsup(),self:Yinf(),self:Ysup()} -- default user reference
    local vectors = field(f,view[1],view[2],view[3],view[4],args.grid,args.long) -- field calculation
    self:Dpolyline(vectors,false,args.draw_options) -- the drawing (non-closed polygonal line)
end

function graph:Dgradientfield(f,args) -- added another method to the graph class
-- draws a gradient field
-- f is a function of two variables with values ​​in R
-- args is a 4-field table:
-- { view={x1,x2,y1,y2}, grid={nbx,nby}, long=, draw_options=""}
    local h = 1e-6
    local grad_f = function(x,y) -- gradient function of f
        return { (f(x+h,y)-f(x-h,y))/(2*h), (f(x,y+h)-f(x,y-h))/(2*h) }
    end
    self:Dvectorfield(grad_f,args) -- we use the previous method
end
\end{Luacode}

\subsection{How to import the file}

There are two methods for this:

\begin{enumerate}
\item With the Lua instruction \emph{dofile}. This can be written, for example, in the preamble after the package declaration:
\begin{TeXcode}
\usepackage[]{luadraw}
\directlua{dofile("<path>/luadraw_fields.lua")}
\end{TeXcode}
Of course, you will need to replace \verb|<path>| with the path to this file.

The instruction \verb|\directlua{dofile("<path>/luadraw_fields.lua")}| can be placed elsewhere in the document, as long as it is after the package has been loaded (otherwise the \emph{graph} class will not be recognized when reading the file). We can also place the instruction \verb|dofile("<path>/luadraw_fields.lua")| in a \emph{luacode} environment, and therefore in particular in a \emph{luadraw} environment.

As soon as the file is imported, the new methods are available for the rest of the document.

This approach has at least two drawbacks: it must be remembered each time \verb|<path>| is used, and secondly, the \emph{dofile} instruction does not check whether the file has already been read. For these reasons, we prefer the following method.

\item With the Lua instruction \emph{require}. For example, we can write it in the preamble after the package declaration:
\begin{TeXcode}
\usepackage[]{luadraw}
\directlua{require "luadraw_fields"}
\end{TeXcode}
Note the absence of the path (and the lua extension is unnecessary).

The \verb|\directlua{require "luadraw_fields"}| instruction can be placed elsewhere in the document, provided it is after the package has been loaded (otherwise the \emph{graph} class will not be recognized when reading the file). We can also place the \verb|require "luadraw_fields"| instruction in a \emph{luacode} environment, and therefore in particular in a \emph{luadraw} environment.

The \emph{require} instruction checks whether the file has already been loaded or not, which is preferable. However, Lua must be able to find this file, and the easiest way to do this is for it to be somewhere in a tree structure known to TeX. For example, you can create the following path in your local \emph{texmf}:
\begin{TeXcode}
texmf/tex/lualatex/myluafiles/
\end{TeXcode}
then copy the file \emph{luadraw\_fields.lua} into the \emph{myluafiles} folder.
\end{enumerate}

\begin{demo}{Using the new methods}
\begin{luadraw}{name=fields}
require "luadraw_fields" -- import des nouvelles méthodes
local g = graph:new{window={0,21,0,10},size={16,10}}
local i = cpx.I
g:Labelsize("footnotesize")
local f = function(x,y) return {x-x*y,-y+x*y} end -- Volterra
local F = function(x,y) return x^2+y^2+x*y-6 end
local H = function(t,Y) return f(Y[1],Y[2]) end
-- graphique du haut
g:Saveattr();g:Viewport(0,10,0,10);g:Coordsystem(-5,5,-5,5)
g:Dgradbox({-4.5-4.5*i,4.5+4.5*i,1,1}, {originloc=0,originnum={0,0},grid=true,title="gradient field, $f(x,y)=x^2+y^2+xy-6$"}) 
g:Arrows("->"); g:Lineoptions(nil,"blue",6)
g:Dgradientfield(F,{view={-4,4,-4,4},grid={15,15},long=0.5})
g:Arrows("-"); g:Lineoptions(nil,"Crimson",12); g:Dimplicit(F, {view={-4,4,-4,4}})
g:Restoreattr()
-- graphique du bas
g:Saveattr();g:Viewport(11,21,0,10);g:Coordsystem(-5,5,-5,5)
g:Dgradbox({-4.5-4.5*i,4.5+4.5*i,1,1}, {originloc=0,originnum={0,0},grid=true,title="vector field, $f(x,y)=(x-xy,-y+xy)$"}) 
g:Arrows("->"); g:Lineoptions(nil,"blue",6); g:Dvectorfield(f,{view={-4,4,-4,4}})
g:Arrows("-");g:Lineoptions(nil,"Crimson",12)
g:Dodesolve(H,0,{2,3},{t={0,50},out={2,3},nbdots=250})
g:Restoreattr()
g:Show()
\end{luadraw}
\end{demo}

\subsection{Modifying an existing method}

Let's take for example the method \emph{DplotXY(X,Y,draw\_options)}, which takes two lists (tables) of real numbers as arguments and draws the polygonal line formed by the points with coordinates $(X[k],Y[k])$. We'll modify it to account for the case where \emph{X} is a list of names (strings). In this case, the names will be displayed below the x-axis (with x-axis $k$ for the kth name) and the polygonal line formed by the points with coordinates $(k,Y[k])$ will be drawn. Otherwise, we'll use the same method as the old method. To do this, simply rewrite the method (in a Lua file so that it can be imported later):
\begin{Luacode}
function graph:DplotXY(X,Y,draw_options)
-- X is a list of real numbers or strings
-- Y is a list of real numbers of the same length as X    local L = {} -- liste des points à dessiner
    if type(X[1]) == "number" then -- list of real numbers
        for k,x in ipairs(X) do
            table.insert(L,Z(x,Y[k]))
        end
    else
        local noms = {} -- list of labels to place
        for k = 1, #X do
            table.insert(L,Z(k,Y[k]))
            insert(noms,{X[k],k,{pos="E",node_options="rotate=-90"}})
        end
        self:Dlabel(table.unpack(noms)) -- drawing labels
    end
    self:Dpolyline(L,draw_options) -- drawing the curve
end
\end{Luacode}

As soon as the file is imported, this new definition will overwrite the old one (for the rest of the document). Of course, you could imagine adding other options to the drawing style, for example (lines, bars, dots, etc.).

\begin{demo}{Modifying an existing method}
\begin{luadraw}{name=newDplotXY}
require "luadraw_fields" -- import of the modified method
local g = graph:new{window={-0.5,11,-1,20}, margin={0.5,0.5,0.5,1}, size={10,10,0}}
g:Labelsize("scriptsize")
local X, Y = {}, {} -- we define two lists X and Y, we could also read them in a file
for k = 1, 10 do
    table.insert(X,"nom"..k)
    table.insert(Y,math.random(1,20))
end
defaultlabelshift = 0
g:Daxes({0,1,2},{limits={{0,10},{0,20}}, labelpos={"none","left"},arrows="->", grid=true})
g:DplotXY(X,Y,"line width=0.8pt, blue")
g:Show()
\end{luadraw}
\end{demo}
%

\chapter{3D Drawing}

\begin{center}
\captionof{figure}{Col point at $M(0,0,0)$ ($z=x^2-y^2$)}\label{pointcol}\par
\begin{luadraw}{name=point_col}
local g = graph3d:new{window3d={-2,2,-2,2,-4,4}, window={-4,3.5,-5.5,5}, size={8,9,0}, viewdir=perspective("central",120,60)}
local S = cartesian3d(function(u,v) return u^2-v^2 end, -2,2,-2,2,{20,20})
local P = facet2poly(S) -- conversion en polyèdre
local Tx = g:Intersection3d(P, {Origin,vecI}) --intersection de P avec le plan yOz
local Ty = g:Intersection3d(P, {Origin,vecJ}) --intersection de P avec le plan xOz
g:Dboxaxes3d({grid=true,gridcolor="gray",fillcolor="LightGray",drawbox=true})
g:Dfacet(S,{mode=mShadedOnly,color="ForestGreen"}) -- dessin de la surface
g:Dedges(Tx, {color="Crimson", hidden=true, width=8}) -- intersection avec yOz
g:Dedges(Ty, {color="Navy",hidden=true, width=8}) -- intersection avec xOz
g:Dpolyline3d( {M(2,0,4),M(-2,0,4),M(-2,0,-4)}, "Navy,line width=.8pt")
g:Dpolyline3d( {M(0,-2,4),M(0,2,4),M(0,2,-4)}, "Crimson,line width=.8pt")
g:Show()
\end{luadraw}
\end{center}


\section{Introduction}

\subsection{Prerequisites}

\begin{itemize}
    \item This document presents the use of the \emph{luadraw} package with the \emph{3d} global option:
\verb|\usepackage[3d]{luadraw}|.
    \item The package loads the \emph{luadraw\_graph2d.lua} module, which defines the \emph{graph} class and provides the \emph{luadraw} environment for creating graphs in Lua. Everything said in the previous chapter (Drawing 2d) therefore applies, and is assumed to be known here.
    \item The \emph{3d} global option also allows the loading of the \emph{luadraw\_graph3d.lua} module. This also defines the \emph{graph3d} class (which relies on the \emph{graph} class) for 3D drawings.
\end{itemize}

\subsection{Some reminders}

\begin{itemize}
    \item Another global package option: \emph{noexec}. When this global option is mentioned, the default value of the \emph{exec} option for the \emph{luadraw} environment will be false (and no longer true).

    \item When a graph is finished, it is exported in tikz format, so this package also loads the tikz package and the libraries:

    \begin{itemize}
        \item\emph{patterns}
        \item\emph{plotmarks}
        \item\emph{arrows.meta}
        \item\emph{decorations.markings}
    \end{itemize}
    
    \item Graphs are created in a luadraw environment, which calls luacode, so the textbf Lua language must be used in this environment.

    \item Saving the tkz file: the graph is exported in tikz format to a file (with the tkz extension). By default, this file is saved in the current folder. However, it is possible to impose a specific path by defining the \verb|\luadrawTkzDir| command in the document, for example: \verb|\def\luadrawTkzDir{tikz/}|, which will save the \emph{*.tkz} files in the \emph{tikz} subfolder of the current folder, provided that this subfolder exists!

    \item The environment options are:
    
    \begin{itemize}
        \item \emph{name = \ldots{}}: allows you to name the resulting tikz file. It is given a name without an extension (the extension will be automatically added; it is \emph{.tkz}). If this option is omitted, then a default name is used, which is the name of the master file followed by a number.     \item \emph{exec = true/false}: Allows you to execute or not the Lua code included in the environment. By default, this option is true, \textbf{EXCEPT} if the global option \emph{noexec} was mentioned in the preamble with the package declaration. When a complex graph that requires a lot of calculations is developed, it may be useful to add the option \emph{exec=false}; this will avoid recalculating the same graph for future compilations.
        \item \emph{auto = true/false}: Allows you to automatically include or not the tikz file in place of the \emph{luadraw} environment when the \emph{exec} option is false. By default, the \emph{auto} option is true.
    \end{itemize}
\end{itemize}


\subsection{Creating a 3D Graph}

\begin{TeXcode}
\begin{luadraw}{ name=<filename>, exec=true/false, auto=true/false }
-- create a new graph and give it a local name
local g = graph3d:new{ window3d={x1,x2,y1,y2,z1,z2}, adjust2d=true/false, viewdir={30,60}, window={x1,x2,y1,y2,xscale,yscale}, margin={left,right,top,bottom}, size={width,height,ratio}, bg="color", border=true/false }
-- build graph g
graph instructions in Lua language ...
-- display graph g and save it in the file <filename>.tkz
g:Show()
-- or Save only in the <filename>.tkz file
g:Save()
\end{luadraw}
\end{TeXcode}

Creation is done in a \emph{luadraw} environment. This creation is done on the first line inside the environment by naming the graph:

\begin{Luacode}
local g = graph3d:new{ window3d={x1,x2,y1,y2,z1,z2}, adjust2d=true/false, viewdir={30,60}, window={x1,x2,y1,y2,xscale,yscale}, margin={left,right,top,bottom}, size={width,height,ratio}, bg="color", border=true/false }
\end{Luacode}

The \emph{graph3d} class is defined in the \emph{luadraw} package using the global option \emph{3d}. This class is instantiated by invoking its constructor and giving it a name (here it's \emph{g}). This is done locally so that the graph \emph{g} thus created will no longer exist once it leaves the environment (otherwise \emph{g} would remain in memory until the end of the document).

\begin{itemize}
    \item The (optional) parameter \emph{window3d} defines the $\mathbf R^3$ block corresponding to the graph: it is $[x_1,x_2]\times[y_1,y_2]\times[z_1,z_2]$. By default, it is $[-5,5]\times[-5,5]\times[-5,5]$.
    \item The (optional) \emph{adjust2d} parameter indicates whether the 2D window that will contain the orthographic projection of the 3D drawing should be determined automatically (false by default). This 2D window corresponds to the \emph{window} argument.

    \item The (optional) \emph{viewdir} parameter is a table that defines the two viewing angles (in degrees) for the orthographic projection. The default is the \{30,60\} table.

\begin{center}
\captionof{figure}{Viewing Angles}\label{viewdir}
\begin{luadraw}{name=viewdir}
local g = graph3d:new{ size={8,8} }
local i = cpx.I
local O, A = Origin, M(4,4,4)
local B, C, D, E = pxy(A), px(A), py(A), pz(A)
g:Dpolyline3d( {{O,A},{-5*vecI,5*vecI},{-5*vecJ,5*vecJ},{-5*vecK,5*vecK}}, "->")
g:Dpolyline3d( {{E,A,B,O}, {C,B,D}}, "dashed")
g:Dpath3d( {C,O,B,2.5,1,"ca",O,"l","cl"}, "draw=none,fill=cyan,fill opacity=0.8")
g:Darc3d(C,O,B,2.5,1,"->")
g:Dpath3d( {E,O,A,2.5,1,"ca",O,"l","cl"}, "draw=none,fill=cyan,fill opacity=0.8")
g:Darc3d(E,O,A,2.5,1,"->")
g:Dballdots3d(O)
g:Labelsize("footnotesize")
g:Dlabel3d(
    "$x$", 5.25*vecI,{}, "$y$", 5.25*vecJ,{}, "$z$", 5.25*vecK,{},
    "vers observateur", A, {pos="E"},
    "$O$", O, {pos="NW"},
    "$\\theta$", (B+C)/2, {pos="N", dist=0.15},
    "$\\varphi$", (A+E)/2, {pos="S",dist=0.25}
)
g:Dlabel("viewdir=\\{$\\theta,\\varphi$\\} (en degrés)",-5*i,{pos="N"})
g:Show()            
\end{luadraw}
\end{center}

    \item The other parameters are those of the \emph{graph} class, described in Chapter 1.
\end{itemize}

\paragraph{Graph construction.}

\begin{itemize}
    \item The instantiated object (\emph{g} in the example) has all the methods of the \emph{graph} class, plus methods specific to 3D.
    \item The \emph{graph3d} class also provides a number of mathematical functions specific to 3D.
\end{itemize}
%
\section{The pt3d Class}

\subsection{Representation of Points and Vectors}

\begin{itemize}
    \item The usual space is $\mathbf R^3$, so points and vectors are triplets of real numbers (called 3d points). Four triplets have specific names (predefined variables), namely:
\begin{itemize}
    \item \textbf{Origin}, which represents the triplet $(0,0,0)$.
    \item \textbf{vecI}, which represents the triplet $(1,0,0)$.
    \item \textbf{vecJ}, which represents the triplet $(0,1,0)$.
    \item \textbf{vecK}, which represents the triplet $(0,0,1)$.
\end{itemize}
Added to this is the variable \textbf{ID3d}, which is the table \emph{\{Origin, vecI, vecJ, vecK\}} representing the 3D unit matrix. By default, it is the transformation matrix of the 3D graph.
    \item The class \emph{pt3d} (which is automatically loaded) defines the real triplets, the possible operations, and a number of methods. To create a 3D point, there are three methods:
\begin{itemize}
    \item Cartesian definition: the function \textbf{M(x,y,z)} returns the triplet $(x,y,z)$. This triplet can also be obtained by doing: \emph{x*vecI+y*vecJ+z*vecK}.
    \item Cylindrical definition: the function \textbf{Mc(r,$\theta$,z)} (angle expressed in radians) returns the triplet $(r\cos(\theta),r\sin(\theta),z)$.
    \item Spherical definition: the function \textbf{Ms(r,$\theta$,$\varphi$)} returns the triplet $(r\cos(\theta)\sin(\varphi), r\sin(\theta)\sin(\varphi),r\cos(\varphi))$ (angles expressed in radians).
\end{itemize}
Accessing the components of a 3D point: if a variable $A$ denotes a 3D point, then its three components are $A.x$, $A.y$, and $A.z$.

To test whether a variable $A$ designates a 3D point, we use the \textbf{isPoint3d()} function, which returns a Boolean.

Conversion: To convert a real or complex number into a 3D point, we use the \textbf{toPoint3d()} function.

\end{itemize}


\subsection{Operations on 3D Points}

These operations are the usual operations with the usual symbols:
\begin{itemize}
    \item Addition (+), difference (-), and negative (-).
    \item The product by a scalar, if k is a real number, \emph{k*M(x,y,z)} returns \emph{M(ka,ky,kz)}.
    \item A 3D point can be divided by a scalar; for example, if $A$ and $B$ are two 3D points, then the midpoint is simply written $(A+B)/2$.
    \item The equality of two 3D points can be tested with the symbol =.
\end{itemize}

\subsection{Methods of the class \emph{pt3d}}

These are:
\begin{itemize}
    \item \textbf{pt3d.abs(u)}: Returns the Euclidean norm of the 3d point $u$.
    \item \textbf{pt3d.abs2(u)}: Returns the squared Euclidean norm of the 3d point $u$.
    \item \textbf{pt3d.N1(u)}: Returns the 1-norm of the 3d point $u$. If $u=M(x,y,z)$, then \emph{pt3d.N1(u)} returns $|x|+|y|+|z|$.
    \item \textbf{pt3d.dot(u,v)}: Returns the dot product between the vectors (3d points) $u$ and $v$.
    \item \textbf{pt3d.det(u,v,w)}: Returns the determinant between the vectors (3D points) $u$, $v$, and $w$.
    \item \textbf{pt3d.prod(u,v)}: Returns the cross product between the vectors (3D points) $u$ and $v$.
    \item \textbf{pt3d.angle3d(u,v,epsilon)}: Returns the angular difference (in radians) between the vectors (3D points) $u$ and $v$, assumed to be non-zero. The (optional) argument \emph{epsilon} is $0$ by default; it indicates how close a given equality test is to a floating point.

    \item \textbf{pt3d.normalize(u)}: Returns the normalized vector (3D point) $u$ (returns \emph{nil} if $u$ is zero).
    \item \textbf{pt3d.round(u,nbDeci)}: Returns a 3D point whose components are those of the 3D point $u$ rounded to \emph{nbDeci} decimal places.
\end{itemize}

\subsection{Mathematical Functions}

In the file defining the \emph{pt3d} class, some mathematical functions are introduced:
\begin{itemize}
    \item \textbf{isobar3d(L)}: Returns the isobarycenter of the 3D points in the list (table) $L$ (elements of $L$ that are not 3D points are ignored).
    \item \textbf{insert3d(L,A,epsilon)}: This function inserts the 3D point $A$ into the list $L$, which must be a \textbf{variable} (and which will therefore be modified). Point $A$ is inserted \textbf{without duplicates} and the function returns its position (index) in list $L$ after insertion. The (optional) argument \emph{epsilon} is $0$ by default, indicating how closely the comparisons are made.
\end{itemize}
%
\section{Graphics Methods}

All 2D graphics methods apply. Added to this is the ability to draw polygonal lines, segments, straight lines, curves, paths, points, labels, planes, and solids in space. With solids, we also have the concept of facets, which was not found in 2D.

3D graphics methods will automatically calculate the projection onto the screen plane. After applying the 3D transformation matrix associated with the graphic (which is the default identity) to the objects, the 2D graphics methods will then take over.

The method that applies the 3D matrix and performs the projection onto the screen (plane passing through the origin and normal to the unit vector directed towards the observer and defined by the viewing angles) is: \textbf{g:Proj3d(L)} where $L$ is either a 3D point, a list of 3D points, or a list of lists of 3D points. This function returns complex numbers (affixes of the projected points onto the screen).

\textbf{Warning}: when the 3d matrix of the graph is not a linear transformation, the projection onto the screen of a vector $u$ in space is not \textbf{g:Proj3d(u)}, but \textbf{g:Proj3d(A+u)-\textbf{g:Proj3d(A)}} where $A$ denotes any point in space. To avoid these calculations, the method \textbf{g:Proj3dV()} has been introduced, it projects the \textbf{vectors} onto the screen, and returns complex numbers (affixes of the projected vectors onto the screen).

\subsection{Line Drawing}

\subsubsection{Polygonal Line: Dpolyline3d}

The method \textbf{g:Dpolyline3d(L,close,draw\_options,clip)} (where \emph{g} denotes the graph being created), \emph{L} is a 3D polygonal line (list of 3D point lists), \emph{close} is an optional argument that is \emph{true} or \emph{false} indicating whether the line should be closed or not (\emph{false} by default), and \emph{draw\_options} is a string that will be passed directly to the \emph{\textbackslash draw} instruction in the export. The \emph{clip} argument is set to \emph{false} by default. It indicates whether the line \emph{L} should be clipped to the current 3D window.

\subsubsection{Right Angle: Dangle3d}

The \textbf{g:Dangle3d(B,A,C,r,draw\_options,clip)} method draws the angle \(BAC\) with a parallelogram (only two sides are drawn). The optional argument \emph{r} specifies the length of one side (0.25 by default). The parallelogram is in the plane defined by points $A$, $B$, and $C$, so they should not be aligned. The \emph{draw\_options} argument is a string (empty by default) that will be passed as is to the \emph{\textbackslash draw} instruction. The \emph{clip} argument is set to \emph{false} by default; it indicates whether the plot should be clipped to the current 3D window.

\subsubsection{Segment: Dseg3d}

The \textbf{g:Dseg3d(seg,scale,draw\_options,clip)} method draws the segment defined by the \emph{seg} argument, which must be a list of two 3D points. The optional \emph{scale} argument (1 by default) is a number that allows you to increase or decrease the length of the segment (the natural length is multiplied by \emph{scale}). The \emph{draw\_options} argument is a string (empty by default) that will be passed as is to the \emph{\textbackslash draw} instruction. The \emph{clip} argument is set to \emph{false} by default; it indicates whether the plot should be clipped to the current 3D window.

    
\subsubsection{Line: Dline3d}

The method \textbf{g:Dline3d(d,draw\_options,clip)} draws the line \emph{d}, which is a list of type \emph{\{A,u\}} where \emph{A} represents a point on the line (3d point) and \emph{u} a direction vector (a non-zero 3d point).

Variant: the method \textbf{g:Dline3d(A,B,draw\_options,clip)} draws the line passing through the points \emph{A} and \emph{B} (two 3d points). The argument \emph{draw\_options} is a string (empty by default) that will be passed as is to the \emph{\textbackslash draw} instruction. The \emph{clip} argument is set to \emph{false} by default; it indicates whether the plot should be clipped to the current 3D window.

The \textbf{g:Line3d2seg(d,scale)} method returns a table consisting of two 3D points representing a segment. This segment is the portion of the line \emph{d} inside the current 3D window. The \emph{scale} argument (1 by default) allows you to vary the size of this segment. When the window is too small, the intersection may be empty.

\subsubsection{Circular arc: Darc3d}

\begin{itemize}
    \item The method \textbf{g:Darc3d(B,A,C,r,sens,normal,draw\_options,clip)} draws a circular arc with center \emph{A} (3d point), radius \emph{r}, going from \emph{B} (3d point) to \emph{C} (3d point) in the forward direction if the argument \emph{sens} is 1, and in the reverse direction otherwise. This arc is drawn in the plane containing the three points $A$, $B$, and $C$. When these three points are aligned, the argument \emph{normal} (non-zero 3d point) must be specified, which represents a vector normal to the plane. This plane is oriented by the vector product $\vec{AB}\wedge\vec{AC}$ or by the vector \emph{normal} if it is specified. The \emph{draw\_options} argument is a string (empty by default) that will be passed as is to the \emph{\textbackslash draw} instruction. The \emph{clip} argument is set to \emph{false} by default; it indicates whether the path should be clipped to the current 3D window.

    \item The \textbf{arc3d(B,A,C,r,sense,normal)} function returns the list of points of this arc (3D polygonal line).

    \item The \textbf{arc3db(B,A,C,r,sense,normal)} function returns this arc as a 3D path (see Dpath3d) using Bézier curves. \end{itemize}

\subsubsection{Circle: Dcircle3d}

\begin{itemize}
    \item The method \textbf{g:Dcircle3d(I,R,normal,draw\_options,clip)} draws the circle with center $I$ (3D point) and radius $R$, in the plane containing $I$ and normal to the vector defined by the argument \emph{normal} (non-zero 3D point). The argument \emph{draw\_options} is a string (empty by default) that will be passed as is to the instruction \emph{\textbackslash draw}. The argument \emph{clip} is set to \emph{false} by default; it indicates whether the plot should be clipped to the current 3D window. Another possible syntax: \textbf{g:Dcircle3d(C,draw\_options,clip)} where \emph{C=\{I,R,normal\}}.

    \item The \textbf{circle3d(I,R,normal)} function returns the list of points on this circle (3D polygonal line).

    \item The \textbf{circle3db(I,R,normal)} function returns this circle as a 3D path (see Dpath3d) using Bézier curves.
\end{itemize}

\subsubsection{3D Path: Dpath3d}

The \textbf{g:Dpath3d(path,draw\_options,clip)} method draws the \emph{path}. The \emph{draw\_options} argument is a string that will be passed directly to the \emph{\textbackslash draw} instruction. The \emph{clip} argument is set to \emph{false} by default; it indicates whether the path should be clipped to the current 3D window. The argument \emph{path} is a list of 3D points followed by instructions (strings) that work on the same principle as in 2D. The instructions are:
\begin{itemize}
    \item \emph{"m"} for moveto,
    \item \emph{"l"} for lineto,
    \item \emph{"b"} for Bezier (two control points are required),
    \item \emph{"c"} for circle (one point on the circle, the center, and a normal vector are required),
    \item \emph{"ca"} for arc (three points, a radius, a direction, and possibly a normal vector are required),
    \item \emph{"cl"} for close (closes the current component).
\end{itemize}

Here, for example, is the code in figure \ref{viewdir}.

\begin{Luacode}
\begin{luadraw}{name=viewdir}
local g = graph3d:new{ size={8,8} }
local i = cpx.I
local O, A = Origin, M(4,4,4)
local B, C, D, E = pxy(A), px(A), py(A), pz(A) --projeté de A sur le plan xOy et sur les axes
g:Dpolyline3d( {{O,A},{-5*vecI,5*vecI},{-5*vecJ,5*vecJ},{-5*vecK,5*vecK}}, "->") -- axes
g:Dpolyline3d( {{E,A,B,O}, {C,B,D}}, "dashed")
g:Dpath3d( {C,O,B,2.5,1,"ca",O,"l","cl"}, "draw=none,fill=cyan,fill opacity=0.8") --secteur angulaire
g:Darc3d(C,O,B,2.5,1,"->") -- arc de cercle pour theta
g:Dpath3d( {E,O,A,2.5,1,"ca",O,"l","cl"}, "draw=none,fill=cyan,fill opacity=0.8") --secteur angulaire
g:Darc3d(E,O,A,2.5,1,"->") -- arc de cercle pour phi
g:Dballdots3d(O) -- le point origine sous forme d'une petite sphère
g:Labelsize("footnotesize")
g:Dlabel3d(
    "$x$", 5.25*vecI,{}, "$y$", 5.25*vecJ,{}, "$z$", 5.25*vecK,{},
    "vers observateur", A, {pos="E"},
    "$O$", O, {pos="NW"},
    "$\\theta$", (B+C)/2, {pos="N", dist=0.15},
    "$\\varphi$", (A+E)/2, {pos="S",dist=0.25}
)
g:Dlabel("viewdir=\\{$\\theta,\\varphi$\\} (en degrés)",-5*i,{pos="N"}) -- label 2d
g:Show()   
\end{luadraw}      
\end{Luacode}

\subsubsection{Plane: Dplane}

The method \textbf{g:Dplane(P,V,L1,L2,mode,draw\_options)} draws the edges of the plane $P=\{A,u\}$ where $A$ is a point in the plane and $u$ is a normal vector to the plane ($P$ is therefore a table of two 3D points). The argument $V$ must be a non-zero vector in the plane $P$, $L_1$ and $L_2$ are two lengths. The method constructs a parallelogram centered on $A$, with one side $L_1\frac{V}{\|V\|}$ and the other $L_2\frac{W}{\|W\|}$ where $W = u\wedge V$. The argument \emph{mode} is a natural number indicating the edges to draw. To calculate this integer, we use the predefined variables: \emph{top} (=8), \emph{right} (=4), \emph{bottom} (=2), \emph{left} (=1), and \emph{all} (=15), which can be added together, for example:
\begin{itemize}
    \item mode = bottom+left: for the bottom and left sides
    \item mode = top+right+bottom: for the top, right, and bottom sides
    \item etc
\end{itemize}
By default, the mode is \emph{all}, which corresponds to \emph{top+right+bottom+left}.

\begin{demo}{Dplane, example with mode = left+bottom}
\begin{luadraw}{name=Dplane}
local g = graph3d:new{size={8,8},window={-5.25,3,-2.5,2.5},margin={0,0,0,0},border=true}
local i = cpx.I
g:Labelsize("footnotesize")
local A = Origin
local P = {A, vecK}
g:Dplane(P, vecJ, 6, 6, left+bottom)
g:Dcrossdots3d({A,vecK},nil,0.75)
g:Dseg3d({A,A+2*vecK},"->")
g:Dangle3d(-vecJ,A,vecK,0.25)
g:Dpolyline3d({{M(3.5,-3,0),M(3.5,3,0)},{M(3,-3.5,0), M(-3,-3.5,0)}}, "->,line width=0.8pt")
g:Dlabel3d("$A$",A,{pos="E"}, 
    "$u$",2*vecK,{},
    "$P$", M(3,-3,0),{pos="NE", dir={vecJ,-vecI}},
    "$L_1\\frac{V}{\\|V\\|}$ (bottom)", M(3.5,0,0), {pos="S"},
    "$L_2\\frac{W}{\\|W\\|}$ (left)", M(0,-3.5,0), {pos="N",dir={-vecI,-vecJ}}
)
g:Show()
\end{luadraw}
\end{demo}

\paragraph{Warning}: The concepts of top, right, bottom, and left are relative! They depend on the direction of the vectors $u$ (vector normal to the plane) and $V$ (vector given in the plane). The third vector $W$ is the cross product $u\wedge V$.

\subsubsection{Parametric curve: Dparametric3d}

\begin{itemize}
    \item The function \textbf{parametric3d(p,t1,t2,nbdots,discont,nbdiv)} calculates the points of the curve and returns a 3D polygonal line (no drawing).
\begin{itemize}
    \item The argument \emph{p} is the parameterization. It must be a function of a real variable \emph{t} with values ​​in $\mathbf R^3$ (the images are 3D points), for example:
\mintinline{Lua}{local p = function(t) return Mc(3,t,t/3) end}

    \item The arguments \emph{t1} and \emph{t2} are mandatory with \(t1 < t2\); they form the bounds of the interval for the parameter.

    \item The argument \emph{nbdots} is optional; it is the (minimum) number of points to calculate; it is 40 by default.

    \item The argument \emph{discont} is an optional boolean that indicates whether there are discontinuities or not. It is \emph{false} by default.

    \item The argument \emph{nbdiv} is a positive integer equal to 5 by default and indicates the number of times the interval between two consecutive parameter values ​​can be split in two (dichotomized) when the corresponding points are too far apart.
\end{itemize}

    \item The method \textbf{g:Dparametric3d(p,args)} calculates the points and draws the curve parameterized by \emph{p}. The \emph{args} parameter is a 6-field table:

\begin{TeXcode}
{ t={t1,t2}, nbdots=40, discont=true/false, clip=true/false, nbdiv=5, draw_options="" }
\end{TeXcode}

\begin{itemize}
    \item By default, the \emph{t} field is equal to \emph{\{g:Xinf(),g:Xsup()\}},
    \item the \emph{nbdots} field is equal to 40,
    \item the \emph{discont} field is equal to \emph{false},
    \item the \emph{nbdiv} field is equal to 5,
    \item the \emph{clip} field is equal to \emph{false}, it indicates whether the curve should be clipped with the current 3D window.     \item the \emph{draw\_options} field is an empty string (this will be passed as is to the \emph{\textbackslash draw} instruction).
\end{itemize}
\end{itemize}

\begin{demo}{A curve and its projections onto three planes}
\begin{luadraw}{name=Dparametric3d}
local g = graph3d:new{window3d={-4,4,-4,4,-3,3}, window={-7.5,6.5,-7,6}, size={8,8}}
local pi = math.pi
g:Labelsize("footnotesize")
local p = function(t) return Mc(3,t,t/3) end
local L = parametric3d(p,-2*pi,2*pi,25,false,2)
g:Dboxaxes3d({grid=true,gridcolor="gray",fillcolor="LightGray"})
g:Lineoptions("dashed","red",2)
-- projection sur le plan y=-4
g:Dpolyline3d(proj3d(L,{M(0,-4,0),vecJ}))
-- projection sur le plan x=-4
g:Dpolyline3d(proj3d(L,{M(-4,0,0),vecI}))
-- projection sur le plan z=-3
g:Dpolyline3d(proj3d(L,{M(0,0,-3),vecK}))
-- dessin de la courbe
g:Lineoptions("solid","Navy",8)
g:Dparametric3d(p,{t={-2*pi,2*pi}})
g:Show()
\end{luadraw}
\end{demo}

\subsubsection{Parameterization of a Polygonal Line: \emph{curvilinear\_param3d}}
Let $L$ be a list of 3D points representing a continuous \og \fg line. It is possible to obtain a parameterization of this line based on a parameter $t$ between $0$ and $1$ ($t$ is the curvilinear abscissa divided by the total length of $L$).

The function \textbf{curvilinear\_param3d(L,close)} returns a function of one variable $t\in[0;1]$ and values ​​on the line $L$ (3D points). The value at $t=0$ is the first point of $L$, and the value at $t=1$ is the last point; This function is followed by a number representing the total length of L. The optional argument \emph{close} indicates whether the line $L$ should be closed (\emph{false} by default).

\subsubsection{The reference: Dboxaxes3d}

The \textbf{g:Dboxaxes3d( args )} method allows you to draw the three axes, with a number of options defined in the \emph{args} table. These options are:
\def\opt#1{\textcolor{blue}{\texttt{#1}}}%
\begin{itemize}
    \item \opt{xaxe=true/false}, \opt{yaxe=true/false}, and \opt{zaxe=true/false}: Indicates whether the corresponding axes should be drawn or not (true by default).

    \item \opt{drawbox=true/false}: Indicates whether a box should be drawn with the axes (false by default).

    \item \opt{grid=true/false}: Indicates whether a grid should be drawn (one for $x$, one for $y$, and one for $z$). When this option is true, the following options can also be used:
\begin{itemize}
    \item \opt{gridwidth} (=1 by default) indicates the grid line thickness in tenths of a point.
    \item \opt{gridcolor} (black by default) indicates the grid color.
    \item \opt{fillcolor} ("" by default) allows you to paint the grid background or not.
\end{itemize}

    \item \opt{xlimits=\{x1,x2\}}, \opt{ylimits=\{y1,y2\}}, \opt{zlimits=\{z1,z2\}}: Allows you to define the three intervals used for the axis lengths. By default, these are the values ​​provided to the \opt{window3d} argument when creating the graph.

    \item \opt{xgradlimits=\{x1,x2\}}, \opt{ygradlimits=\{y1,y2\}}, \opt{zgradlimits=\{z1,z2\}}: Allows you to define the three graduation intervals on the axes. By default, these options are set to "auto," meaning they take the same values ​​as \opt{xlimits}, \opt{ylimits}, and \opt{zlimits}.
    \item \opt{xyzstep}: Specifies the tick step on all three axes (1 by default).
    \item \opt{xstep}, \opt{ystep}, \opt{zstep}: Specifies the tick step on each axis (value of \opt{xyzstep} by default).

    \item \opt{xyzticks} (0.2 by default): Specifies the length of the tick marks.

    \item \opt{labels} (true by default): Specifies whether or not to display the tick mark values.

    \item \opt{xlabelsep}, \opt{ylabelsep}, \opt{zlabelsep}: Specifies the distance between the labels and the graduations (0.25 by default).

    \item \opt{xlabelstyle}, \opt{ylabelstyle}, \opt{zlabelstyle}: Specifies the label style, i.e., the position relative to the anchor point. By default, the current style is applied.

    \item \opt{xlegend} ("x" by default), \opt{ylegend} ("y" by default), \opt{zlegend} ("z" by default): Allows you to define a legend for the axes.

    \item \opt{xlegendsep}, \opt{ylegendsep}, \opt{zlegendsep}: Specifies the distance between the legends and the graduations (0.5 by default).
\end{itemize}

\subsection{Points and Labels}

\subsubsection{3D Points: Ddots3d, Dballdots3d, Dcrossdots3d}

There are three ways to draw 3D points. For the first two, the argument \emph{L} can be either a single 3D point, a list (a table) of 3D points, or a list of lists of 3D points:

\begin{itemize}
    \item The \textbf{g:Ddots3d(L, mark\_options,clip)} method. The principle is the same as in the 2D version; the points are drawn in the current line color with the current style. The \emph{mark\_options} argument is an optional string that will be passed as is to the \emph{\textbackslash draw} instruction (local modifications). The \emph{clip} argument is set to \emph{false} by default; it indicates whether the plot should be clipped to the current 3D window.

    \item The \textbf{g:Dballdots3d(L,color,scale,clip)} method draws the points of \emph{L} as a sphere. The optional \emph{color} argument specifies the color of the sphere (black by default), and the optional \emph{scale} argument allows you to change the size of the sphere (1 by default).

    \item The \textbf{g:Dcrossdots3d(L,color,scale,clip)} method draws the points of \emph{L} as a plane cross. The argument \emph{L} is a list of the form \{3D point, normal vector\} or \{ \{3D point, normal vector\}, \{3D point, normal vector\}, ...\}. For each 3D point, the associated normal vector is used to determine the plane containing the cross. The optional argument \emph{color} specifies the color of the cross (black by default), and the optional argument \emph{scale} allows you to change the size of the cross (1 by default).
\end{itemize}

\begin{demo}{A tetrahedron and the centers of gravity of each face}
\begin{luadraw}{name=Ddots3d}
local g = graph3d:new{viewdir={15,60},bbox=false,size={8,8}}
local A, B, C, D = 4*M(1,0,-0.5), 4*M(-1/2,math.sqrt(3)/2,-0.5), 4*M(-1/2,-math.sqrt(3)/2,-0.5), 4*M(0,0,1)
local u, v, w = B-A, C-A, D-A
-- centres de gravité faces cachées
for _, F in ipairs({{A,B,C},{B,C,D}}) do
    local G, u = isobar3d(F), pt3d.prod(F[2]-F[1],F[3]-F[1])
    g:Dcrossdots3d({G,u}, "blue",0.75)
    g:Dpolyline3d({{F[1],G,F[2]},{G,F[3]}},"dotted")
end
-- dessin du tétraèdre construit sur A, B, C et D
g:Dpoly(tetra(A,u,v,w),{mode=mShaded,opacity=0.7,color="Crimson"})
-- centres de gravité faces visibles
for _, F in ipairs({{A,B,D},{A,C,D}}) do
    local G, u = isobar3d(F), pt3d.prod(F[2]-F[1],F[3]-F[1])
    g:Dcrossdots3d({G,u}, "blue",0.75)
    g:Dpolyline3d({{F[1],G,F[2]},{G,F[3]}},"dotted")
end
g:Dballdots3d({A,B,C,D}, "orange") --sommets
g:Show()
\end{luadraw}
\end{demo}

\subsubsection{3D Labels: Dlabel3d}

The method for placing a label in space is:

\hfil\textbf{g:Dlabel3d(text1, anchor1, args1, text2, anchor2, args2, ...)}.\hfil

\begin{itemize}
    \item The arguments \emph{text1, text2,...} are strings; they are the labels.
    \item The arguments \emph{anchor1, anchor2,...} are 3D points representing the anchor points of the labels.     \item The arguments \emph{args1, arg2,...} allow you to locally define the label parameters. They are 4-field tables:
\begin{TeXcode}
{ pos=nil, dist=0, dir={dirX,dirY,dep}, node_options="" }
\end{TeXcode}
\begin{itemize}
    \item The \emph{pos} field indicates the label's position in the screen plane relative to the anchor point. It can be \emph{"N"} for north, \emph{"NE"} for northeast, \emph{"NW"} for northwest, or \emph{"S"}, \emph{"SE"}, \emph{"SW"}. By default, it is \emph{center}, and in this case, the label is centered on the anchor point.     \item The \emph{dist} field is a distance in cm (in the screen plane) that is $0$ by default. It is the distance between the label and its anchor point when \emph{pos} is not equal to \emph{center}.
    \item \emph{dir=\{dirX,dirY,dep\}} is the direction of writing in space (\emph{nil}, the default value, for the default direction). The three values ​​\emph{dirX}, \emph{dirY} and \emph{dep} are three 3D points representing three vectors. The first two indicate the direction of writing, the third a displacement (translation) of the label relative to the anchor point.
    \item The \emph{node\_options} argument is a string (empty by default) intended to receive options that will be passed directly to tikz in the \emph{node{[}{]}} instruction.
    \item The labels are drawn in the current color of the document text, but the color can be changed with the \emph{node\_options} argument, for example, by setting: \emph{node\_options="color=blue"}.

\textbf{Warning}: The options chosen for a label also apply to subsequent labels if they are unchanged.
\end{itemize}
\end{itemize}

\subsection{Basic Solids (Without Facets)}

\subsubsection{Cylinder: Dcylinder}

Draw a cylinder with a circular base (right or tilted). Several possible syntaxes:
\begin{itemize}
    \item Old syntax: \textbf{g:Dcylinder(A,V,r,args)} draws a right cylinder, where \emph{A} is a 3d point representing the center of one of the circular faces, \emph{V} is a 3d point, a vector representing the axis of the cone, the center of the opposite circular face is the point $A+V$ (this face is orthogonal to $V$), and \emph{r} is the radius of the circular base.
    \item Syntax: \textbf{g:Dcylinder(A,r,B,args)} draws a right cylinder, where \emph{A} is a 3d point representing the center of one of the circular faces, \emph{B} is the center of the opposite face, and \emph{r} is the radius. The cylinder is right, meaning that the circular faces are orthogonal to the axis $(AB)$.
    \item For a tilted cylinder: \textbf{g:Dcylinder(A,r,V,B,args)}, where \emph{A} is a 3D point representing the center of one of the circular faces, \emph{B} is the center of the opposite circular face, \emph{r} is the radius, and \emph{V} is a non-zero 3D vector orthogonal to the plane of the circular faces.
\end{itemize}
For all three syntaxes, \emph{args} is a 5-field table for defining plotting options. These options are:
\begin{itemize}
    \item \emph{mode=mWireframe or mGrid} (\emph{mWireframe} by default). In \emph{mWireframe} mode, this is a wireframe drawing; in \emph{mGrid} mode, this is a grid drawing (as if there were facets).
\emph{hiddenstyle}, sets the line style for hidden areas (set it to "noline" to hide them). By default, this option has the value of the global variable \emph{Hiddenlinestyle}, which is itself initialized with the value \emph{"dotted"}.
\emph{hiddencolor}, sets the color of hidden lines (equal to edgecolor by default).
\emph{edgecolor}, sets the color of the lines (current color by default).
\emph{color=""}, when this option is an empty string (the default value), there is no fill; when it is a color (as a string), there is a fill with a linear gradient.
    \item \emph{opacity=1}, sets the transparency of the drawing.
\end{itemize}


\subsubsection{Cone: Dcone}

Draw a circular cone (right or inclined). Several possible syntaxes:
\begin{itemize}
    \item Old syntax: \textbf{g:Dcone(A,V,r,args)} draws a right cone, where \emph{A} is a 3D point representing the cone's vertex, \emph{V} is a 3D point, a vector representing the cone's axis, the center of the circular face is point $A+V$ (this face is orthogonal to $V$), and \emph{r} is the radius of the circular base.
    \item Syntax: \textbf{g:Dcone(C,r,A,args)} draws a right cone, where \emph{A} is a 3D point representing the cone's vertex, \emph{C} is the center of the circular face, and \emph{r} is the radius. The cone is right, meaning that the circular face is orthogonal to the axis $(AC)$.
    \item For a tilted cone: \textbf{g:Dcone(C,r,V,A,args)}, where \emph{A} is a 3D point representing the cone's vertex, \emph{C} is the center of the circular face, \emph{r} is the radius, and \emph{V} is a non-zero 3D vector orthogonal to the plane of the circular face.
\end{itemize}
For all three syntaxes, \emph{args} is a 5-field table for defining plotting options. These options are:
\begin{itemize}
    \item \emph{mode=mWireframe or mGrid} (\emph{mWireframe} by default). In \emph{mWireframe} mode, this is a wireframe drawing; in \emph{mGrid} mode, this is a grid drawing (as if there were facets).
\emph{hiddenstyle}, sets the line style for hidden areas (set it to "noline" to hide them). By default, this option has the value of the global variable \emph{Hiddenlinestyle}, which is itself initialized with the value \emph{"dotted"}.
\emph{hiddencolor}, sets the color of hidden lines (equal to edgecolor by default).
\emph{edgecolor}, sets the color of the lines (current color by default).
\emph{color=""}, when this option is an empty string (the default value), there is no fill; when it is a color (as a string), there is a fill with a linear gradient.
    \item \emph{opacity=1}, sets the transparency of the drawing.
\end{itemize}

\subsubsection{Frustum: Dfrustum}

Draw a truncated cone with a circular base (right or slanted). Two possible syntaxes:
The method \textbf{g:Dfrustum(A,V,R,r,args)} draws a truncated cone with circular bases.

\begin{itemize}
    \item The syntax: \textbf{g:Dfrustum(A,R,r,V,args)} for a right truncated cone, \emph{A} is a 3D point representing the center of the face with radius \emph{R}, \emph{V} is a 3D vector representing the axis of the truncated cone, the center of the second circular face is the point $A+V$, and its radius is \emph{r} (the faces are orthogonal to $V$). When $R=r$ we simply have a cylinder.
    \item Syntax: \textbf{g:Dfrustum(A,R,r,V,B,args)} for a tilted cone frustum, \emph{A} is a 3D point representing the center of the face with radius \emph{R}, \emph{V} is a 3D vector representing a normal vector to the circular faces, the center of the second circular face is point $B$, and its radius is \emph{r}. When $R=r$ we have a tilted cylinder.
\end{itemize}
In both cases, \emph{args} is a 5-field table for defining plotting options. These options are:
\begin{itemize}
    \item \emph{mode=mWireframe or mGrid} (\emph{mWireframe} by default). In \emph{mWireframe} mode, this is a wireframe drawing; in \emph{mGrid} mode, this is a grid drawing (as if there were facets).
\emph{hiddenstyle}, sets the line style for hidden areas (set it to "noline" to hide them). By default, this option has the value of the global variable \emph{Hiddenlinestyle}, which is itself initialized with the value \emph{"dotted"}.
\emph{hiddencolor}, sets the color of hidden lines (equal to edgecolor by default).
\emph{edgecolor}, sets the color of the lines (current color by default).
\emph{color=""}, when this option is an empty string (the default value), there is no fill; when it is a color (as a string), there is a fill with a linear gradient.
    \item \emph{opacity=1}, sets the transparency of the drawing.
\end{itemize}


\subsubsection{ Sphere: Dsphere}

The \textbf{g:Dsphere(A,r,args)} method draws a sphere.

\begin{itemize}
    \item \emph{A} is a 3D point representing the center of the sphere.
    \item \emph{r} is the radius of the sphere.
    \item \emph{args} is a 5-field table for defining drawing options. These options are:
\begin{itemize}
    \item \emph{mode=mWireframe or mGrid or mBorder} (\emph{mWireframe} by default). In \emph{mWireframe} mode, the outline (circle) and the equator are drawn; in \emph{mGrid} mode, the outline with meridians and spindles (grid) is drawn; and in \emph{mBorder} mode, the outline only is drawn.
    \item \emph{hiddenstyle}, defines the line style for hidden areas (set it to "noline" to hide them). By default, this option has the value of the global variable \emph{Hiddenlinestyle}, which is itself initialized with the value \emph{"dotted"}.
    \item \emph{hiddencolor}, defines the color of hidden lines (equal to edgecolor by default).
    \item \emph{color=""}, when this option is an empty string (the default value), there is no fill; when it is a color (as a string), there is a fill with a "ball color".
    \item \emph{opacity=1}, defines the transparency of the drawing.
    \item \emph{edgestyle}, defines the line style for visible edges; by default, it is the current style.
    \item \emph{edgecolor}, sets the color of the visible edges (current color by default).
    \item \emph{edgewidth}, sets the thickness of the visible edges in tenths of a point (current thickness by default).
\end{itemize}
\end{itemize}

\begin{demo}{Cylinders, Cones, and Spheres}
\begin{luadraw}{name=cylindre_cone_sphere}
local g = graph3d:new{ size={10,10} }
local dessin = function(args)
    g:Dsphere(M(-1,-2.5,1),2.5, args)
    g:Dcone(M(-1,2.5,5),-5*vecK,2, args)
    g:Dcylinder(M(3,-2,0),6*vecJ,1.5, args)
end
-- en haut à gauche, options par défaut
g:Saveattr(); g:Viewport(-5,0,0,5); g:Coordsystem(-5,5,-5,5,true); dessin(); g:Restoreattr()
-- en haut à droite
g:Saveattr(); g:Viewport(0,5,0,5); g:Coordsystem(-5,5,-5,5,true)
dessin({mode=mGrid, hiddenstyle="solid", hiddencolor="LightGray"}); g:Restoreattr()
-- en bas à gauche
g:Saveattr(); g:Viewport(-5,0,-5,0); g:Coordsystem(-5,5,-5,5,true)
dessin({mode=Border, color="orange"}); g:Restoreattr()
-- en bas à droite
g:Saveattr(); g:Viewport(0,5,-5,0); g:Coordsystem(-5,5,-5,5,true)
dessin({mode=mGrid,opacity=0.8,hiddenstyle="noline",color="LightBlue"}); g:Restoreattr()
g:Show()            
\end{luadraw}
\end{demo}
%
\section{Faceted Solids}

\subsection{Definition of a Solid}

There are two ways to define a solid:
\begin{enumerate}
    \item As a list (table) of facets. A facet is itself a list of 3D points (at least 3) that are coplanar and unaligned, which are the vertices. Facets are assumed to be convex and are oriented by the order of appearance of the vertices. That is, if $A$, $B$, and $C$ are the first three vertices of a facet $F$, then the facet is oriented with the normal vector $\vec{AB}\wedge\vec{AC}$. If this normal vector is directed toward the observer, then the facet is considered visible. In the definition of a solid, the normal vectors to the facets must be directed \textbf{outside} the solid for the orientation to be correct.

    \item In the form of a \textbf{polyhedron}, that is to say a table with two fields, a first field called \emph{vertices} which is the list of vertices of the polyhedron (3d points), and a second field called \emph{facets} which is the list of facets, but here, in the definition of the facets, the vertices are replaced by their index in the \emph{vertices} list. The facets are oriented in the same way as before. \end{enumerate}

For example, let's consider the four points $A=M(-2,-2,0)$, $B=M(3,0,0)$, $C=M(-2,2,0)$, and $D=M(0,0,4)$. We can then define the tetrahedron constructed on these four points:
\begin{itemize}
    \item either as a list of facets: \emph{T=\{\{A,B,D\},\{B,C,D\},\{C,A,D\},\{A,C,B\}\}} (pay attention to the orientation),
    \item or as a polyhedron:
\emph{T=\{vertices=\{A,B,C,D\}, facets=\{\{1,2,4\},\{2,3,4\},\{3,1,4\},\{1,3,2\}\}\}}.
\end{itemize}

\subsection{Drawing a Polyhedron: Dpoly}

The function \textbf{g:Dpoly(P,options)} allows you to represent the polyhedron $P$ (using the naive painter's algorithm). The argument \emph{options} is a table containing the options:
\begin{itemize}
    \item \opt{mode=}: Sets the representation mode.
\begin{itemize}
    \item \emph{mode=mWireframe}: Wireframe mode, draws both visible and hidden edges.
    \item \emph{mode=mFlat}: Draws solid-color faces, as well as visible edges.
    \item \emph{mode=mFlatHidden}: Draws solid-color faces, visible edges, and hidden edges.
    \item \emph{mode=mShaded}: Draws the faces in shaded color based on their inclination, as well as the visible edges. This is the default mode.
    \item \emph{mode=mShadedHidden}: Draws the faces in shaded color based on their inclination, with both visible and hidden edges.
    \item \emph{mode=mShadedOnly}: Draws the faces in shaded color based on their inclination, but not the edges.
\end{itemize}
    \item \opt{contrast}: This is a number that defaults to 1. This number allows you to accentuate or diminish the shade of the facet colors in the \emph{mShaded}, \emph{mShadedHidden}, and \emph{mShadedOnly} modes.
    \item \opt{edgestyle}: This is a string that defines the line style of the edges. This is the current style by default.
    \item \opt{edgecolor} : is a string that defines the edge color. This is the current line color by default.
    \item \opt{hiddenstyle} : is a string that defines the line style of hidden edges. By default, this is the value contained in the global variable \emph{Hiddenlinestyle} (which itself is "dotted" by default).
    \item \opt{hiddencolor} : is a string that defines the color of hidden edges. This is the current line color by default.
    \item \opt{edgewidth} : line thickness of edges in tenths of a point. This is the current thickness by default.
    \item \opt{opacity} : a number between 0 and 1 that allows you to set transparency or not on the facets. The default value is 1, which means no transparency.
    \item \opt{backcull}: Boolean that defaults to false. When true, facets considered invisible (normal vectors not directed towards the observer) are not displayed. This option is useful for convex polyhedra because it reduces the number of facets to draw.
    \item \opt{twoside}: Boolean that defaults to true, meaning that both sides of the facets (inner and outer) are distinguished; the two sides will not have exactly the same color.
    \item \opt{color}: String defining the fill color of the facets; it is "white" by default.     \item \opt{usepalette} (\emph{nil} by default), this option allows you to specify a color palette for painting the facets as well as a calculation mode, the syntax is: \emph{usepalette = \{palette,mode\}}, where \emph{palette} designates a table of colors which are themselves tables of the form \emph{\{r,g,b\}} where r, g and b are numbers between $0$ and $1$, and \emph{mode} which is a string that can be either \emph{"x"}, or \emph{"y"}, or \emph{"z"}. In the first case for example, the facets with the center of gravity of minimum abscissa have the first color of the palette, the facets with the center of gravity of maximum abscissa have the last color of the palette, for the others, the color is calculated according to the abscissa of the center of gravity by linear interpolation. \end{itemize}

\begin{demo}{Section of a tetrahedron by a plane}
\begin{luadraw}{name=tetra_coupe}
local g = graph3d:new{viewdir={10,60},bbox=false, size={10,10}, bg="gray!30"}
local A,B,C,D = M(-2,-4,-2),M(4,0,-2),M(-2,4,-2),M(0,0,2)
local T = tetra(A,B-A,C-A,D-A) -- tetrahedron with vertices A, B, C, D
local plan = {Origin, -vecK}  -- sectional plan
local T1, T2, section = cutpoly(T,plan) -- we cut the tetrahedron
-- T1 is the resulting polyhedron in the half-space containing -vecK
-- T2 is the resulting polyhedron in the other half-space
-- section is a facet (it's the cut)
g:Dpoly(T1,{color="Crimson", edgecolor="white", opacity=0.8, edgewidth=8})
g:Filloptions("bdiag","Navy"); g:Dpolyline3d(section,true,"draw=none")
g:Dpoly(shift3d(T2,2*vecK), {color="Crimson", edgecolor="white", opacity=0.8, edgewidth=8})
g:Dballdots3d({A,B,C,D+2*vecK}) -- we drew T2 translated with the vector 2*vecK
g:Show()
\end{luadraw}
\end{demo}

\subsection{Polyhedron Construction Functions}

The following functions return a polyhedron, that is, a table with two fields: a first field called \emph{vertices}, which is the list of the polyhedron's vertices (3D points), and a second field called \emph{facets}, which is the list of facets. However, in the definition of facets, the vertices are replaced by their index in the \emph{vertices} list.

\begin{itemize}
    \item \textbf{tetra(S,v1,v2,v3)} returns the tetrahedron with vertices $S$ (3D point), $S+v1$, $S+v2$, $S+v3$. The three vectors $v1$, $v2$, $v3$ (3D points) are assumed to be forward-directed.

    \item \textbf{parallelep(A,v1,v2,v3)} returns the parallelepiped constructed from vertex $A$ (3d point) and three vectors $v1$, $v2$, $v3$ (3d points) assumed to be in the forward direction.

    \item \textbf{prism(base,vector,open)} returns a prism. The argument \emph{base} is a list of 3d points (one of the two bases of the prism). \emph{vector} is the translation vector (3d point) used to obtain the second base. The optional argument \emph{open} is a Boolean indicating whether the prism is open or not (false by default). If it is open, only the lateral facets are returned. The \emph{base} must be oriented by the \emph{vector}.

    \item \textbf{pyramid(base,vertex,open)} returns a pyramid. The argument \emph{base} is a list of 3D points, and \emph{vertex} is the apex of the pyramid (3D point). The optional argument \emph{open} is a Boolean indicating whether the pyramid is open or not (false by default). If it is open, only the side facets are returned. The \emph{base} must be vertex-oriented.

    \item \textbf{regular\_pyramid(n,side,height,open,center,axis)} returns a regular pyramid. $n$ is the number of sides of the base, the argument \emph{side} is the length of a side, and \emph{height} is the height of the pyramid. The optional argument \emph{open} is a Boolean indicating whether the pyramid is open or not (false by default). If it is open, only the lateral facets are returned. The optional argument \emph{center} is the center of the base (\emph{Origin} by default), and the optional argument \emph{axis} is a direction vector of the pyramid axis (\emph{vecK} by default).

    \item \textbf{truncated\_pyramid(base,vertex,height,open)} returns a truncated pyramid; the argument \emph{base} is a list of 3D points; \emph{vertex} is the apex of the pyramid (3D point). The argument \emph{height} is a number indicating the height from the base where the truncation occurs; this is parallel to the plane of the base. The optional argument \emph{open} is a boolean indicating whether the pyramid is open or not (false by default). If it is open, only the lateral facets are returned. The base must be oriented by the vertex.

    \item \textbf{cylinder(A,V,R,nbfacet,open)} returns a cylinder of radius $R$, with axis \{A,V\}, where $A$ is a 3D point, the center of one of the circular bases, and $V$ is a non-zero 3D vector such that the center of the second base is the point $A+V$. The optional argument \emph{nbfacet} is 35 by default (number of lateral facets). The optional argument \emph{open} is a Boolean indicating whether the cylinder is open or not (false by default). If it is open, only the lateral facets are returned.

    \item \textbf{cylinder(A,R,B,nbfacet,open)} returns a cylinder of radius $R$, axis $(AB)$ where $A$ is a 3D point, the center of one of the circular bases and $B$ the center of the second base. The cylinder is right. The optional argument \emph{nbfacet} is 35 by default (number of lateral facets). The optional argument \emph{open} is a Boolean indicating whether the cylinder is open or not (false by default). If it is open, only the lateral facets are returned.

    \item \textbf{cylinder(A,R,V,B,nbfacet,open)} returns a cylinder of radius $R$, axis $(A)$ where $A$ is a 3d point, center of one of the circular bases, $B$ is the center of the second base, and \emph{V} is a 3d vector normal to the plane of the circular bases (the cylinder can therefore be tilted). The optional argument \emph{nbfacet} is 35 by default (number of lateral facets). The optional argument \emph{open} is a boolean indicating whether the cylinder is open or not (false by default). If it is open, only the lateral facets are returned.
    
    \item \textbf{cone(A,V,R,nbfacet,open)} returns a cone with vertex $A$ (3d point), axis \{A,V\}, and circular base, the circle with center $A+V$ and radius $R$ (in a plane orthogonal to $V$). The optional argument \emph{nbfacet} is 35 by default (number of lateral facets). The optional argument \emph{open} is a boolean indicating whether the cone is open or not (false by default). If it is open, only the lateral facets are returned.

    \item \textbf{cone(C,R,A,nbfacet,open)} returns a cone with vertex $A$ (3d point), \emph{C} is the circular base center, and \emph{R} is its radius (in a plane orthogonal to the $(AC)$ axis). The optional argument \emph{nbfacet} is 35 by default (number of lateral facets). The optional argument \emph{open} is a Boolean indicating whether the cone is open or not (false by default). If it is open, only the lateral facets are returned.

    \item \textbf{cone(C,R,V,A,nbfacet,open)} returns a cone with vertex $A$ (3d point), \emph{C} is the circular base center, \emph{R} is its radius, and the base is in a plane orthogonal to \emph{V} (3d vector). The $(AC)$ axis is therefore not necessarily orthogonal to the circular face (tilted cone). The optional argument \emph{nbfacet} is 35 by default (number of lateral facets). The optional argument \emph{open} is a Boolean indicating whether the cone is open or not (false by default). If it is open, only the lateral facets are returned.

    \item \textbf{frustum(C,R,r,V,nbfacet,open)} returns a right frustum. Point $C$ (point 3d) is the center of the circular base of radius $R$, and vector $V$ directs the axis of the frustum. The center of the other circular base is point $C+V$, and its radius is $r$ (the bases are orthogonal to $V$). The optional argument \emph{nbfacet} is 35 by default (number of lateral facets). The optional argument \emph{open} is a Boolean indicating whether the frustum is open or not (false by default). If it is open, only the lateral facets are returned.

    \item \textbf{frustum(C,R,r,V,A,nbfacet,open)} returns a right frustum of a cone. Point $C$ (3d point) is the center of the circular base of radius $R$, the center of the other circular base is point $A$, and its radius is $r$. The bases are orthogonal to vector $V$, but not necessarily orthogonal to axis $(AC)$. The optional argument \emph{nbfacet} is 35 by default (number of lateral facets). The optional argument \emph{open} is a Boolean indicating whether the frustum is open or not (false by default). If it is open, only the lateral facets are returned.

    \item \textbf{sphere(A,R,nbu,nbv)} returns the sphere with center $A$ (3d point) and radius $R$. The optional argument \emph{nbu} represents the number of spindles (36 by default) and the optional argument \emph{nbv} the number of parallels (20 by default).
\end{itemize}

\begin{demo}{Truncated cone, truncated pyramid, oblique cylinder}
\begin{luadraw}{name=frustum_pyramid}
local g = graph3d:new{adjust2d=true,bbox=false, size={10,10} }
g:Dfrustum(M(-1,-4,0),3,1,5*vecK, {color="cyan"})
g:Dcylinder(M(-4,4,0),2,vecK,M(-4,2,5), {color="orange"})
local base = map(toPoint3d,polyreg(0,3,5))
g:Dpoly(truncated_pyramid( shift3d(base,8*vecI-vecJ-2*vecK), M(5,0,5),4), {mode=4,color="Crimson"})
g:Dcone(M(6,7,-2),3,vecK,M(6,8,5),{color="Pink"})
g:Show()            
\end{luadraw}
\end{demo}

\paragraph{Note}: We already have primitives for drawing cylinders, cones, and spheres without using facets. One of the advantages of defining these objects as polyhedra is that we can perform certain calculations on them, such as plane sections.

\begin{demo}{Hyperbola: cone-plane intersection}
\begin{luadraw}{name=hyperbole}
local g = graph3d:new{window={-8,6,-9,9},bbox=false, viewdir=perspective("central",45,65), size={10,10}}
Hiddenlinestyle = "dashed"; Hiddenlines = true
local C1 = cone(Origin,4*vecK,3,35,true)
local C2 = cone(Origin, -4*vecK,3,35,true)
local P = {M(1,-1,-2),vecI} -- sectional plan
local I1 = g:Intersection3d(C1,P) -- intersection between cone C1 and plane P
local I2 = g:Intersection3d(C2,P) -- intersection between cone C2 and plane P
-- I1 et I2 sont de type Edges (arêtes)
g:Dcone(Origin,4*vecK,3,{color="orange"}); g:Dcone(Origin,-4*vecK,3,{color="orange"})
g:Lineoptions("solid","Navy",8)
g:Dedges(I1); g:Dedges(I2) -- drawing of edges I1 and I2
g:Dplane(P, vecK,14,9)
g:Show()
\end{luadraw}
\end{demo}

In this example, cones $C_1$ and $C_2$ are defined as polyhedra to determine their intersection with plane $P$, but not to draw them. The method \textbf{g:Intersection3d(C1,P)} returns the intersection of polyhedron $C_1$ with plane $P$ as a two-field table: one field named \emph{visible} that contains a 3D polygonal line representing the visible "edges" (segments) of the intersection (i.e., those that are on a visible facet of $C_1$), and another field named \emph{hidden} that contains a 3D polygonal line representing the hidden "edges" of the intersection (i.e., those that are on a non-visible facet of $C_1$). The method \textbf{g:Dedges} can be used to draw these types of objects.

\begin{demo}{Cone section with multiple views}
\begin{luadraw}{name=several_views}
local g = graph3d:new{window3d={-3,3,-3,3,-3,3}, size={10,10}, margin={0,0,0,0}}
g:Labelsize("footnotesize")
local y0, R = 1, 2.5
local C = cone(M(0,0,3),-6*vecK,R,35,true) -- cone ouvert
local P1 = {M(0,0,0),vecK+vecJ} -- 1er plan de coupe
local P2 = {M(0,y0,0),vecJ} -- 2ieme plan de coupe
local I, I2
local dessin = function() -- un dessin par vue
g:Dboxaxes3d({grid=true,gridcolor="gray",fillcolor="LightGray"})
I1 = g:Intersection3d(C,P1) -- intersection entre le cône C et les plans P1 et P2
I2 = g:Intersection3d(C,P2) -- I1 et I2 sont de type Edges
g:Dpolyline3d( {{M(0,-3,3),M(0,0,3),M(0,0,-3),M(3,0,-3)}, {M(0,0,-3),M(0,3,-3)}},"red,line width=0.4pt" )
g:Dcone( M(0,0,3),-6*vecK,R, {color="cyan"})
g:Dedges(I1, {hidden=true,color="Navy", width=8})
g:Dedges(I2, {hidden=true,color="DarkGreen", width=8})
end
-- en haut à gauche, vue dans l'espace, on ajoute les plans au dessin
g:Saveattr(); g:Viewport(-5,0,0,5); g:Coordsystem(-7,6,-6,5,1); g:Setviewdir(perspective("central")); dessin()
g:Dpolyline3d( {M(-3,-3,3),M(3,-3,3),M(3,3,-3),M(-3,3,-3)},"Navy,line width=0.8pt")
g:Dpolyline3d( {M(-3,y0,3),M(3,y0,3),M(3,y0,-3)},"DarkGreen,line width=0.8pt")
g:Dlabel3d( "$P_1$",M(3,-3,3),{pos="SE",dir={-vecI,-vecJ+vecK},node_options="Navy, draw"})
g:Dlabel3d( "$P_2$",M(-3,y0,3),{pos="SW",dir={-vecI,vecK},node_options="DarkGreen,draw"})
g:Restoreattr()
-- en haut à droite, projection sur le plan xOy
g:Saveattr(); g:Viewport(0,5,0,5); g:Coordsystem(-6,6,-6,5,1); g:Setviewdir("xOy"); dessin()
g:Restoreattr()
-- en bas à gauche, projection sur le plan xOz
g:Saveattr(); g:Viewport(-5,0,-5,0); g:Coordsystem(-6,6,-6,5,1); g:Setviewdir("xOz"); dessin()
g:Restoreattr()
-- en bas à droite, projection sur le plan yOz
g:Saveattr(); g:Viewport(0,5,-5,0); g:Coordsystem(-6,6,-6,5,1); g:Setviewdir("yOz"); dessin()
g:Restoreattr()
g:Show()
\end{luadraw}
\end{demo}

\subsection{Reading from an obj file}

The function \textbf{red\_obj\_file(file)}\footnote{This function is a contribution by Christophe BAL.} allows you to read the contents of the file \emph{obj} designated by the string \emph{file}. The function reads the vertex definitions (lines beginning with \verb|v |), and the lines defining the facets (lines beginning with \verb|f |). The other lines are ignored. The function returns a sequence consisting of the polyhedron, followed by a list of four real numbers \emph{\{x1,x2,y1,y2,z1,z2\}} representing the 3D bounding box of the polyhedron.

\begin{demo}{Mask of Nefertiti}
\begin{luadraw}{name=lecture_obj}
local P,bbox = read_obj_file("obj/nefertiti.obj")
local g = graph3d:new{window3d=bbox,window={-6,5,-7,7},viewdir=perspective("central",35,65,20),
    margin={0,0,0,0}, size={10,10}, bg="LightGray"}
g:Dpoly(P, {usepalette={palAutumn,"z"},mode=mShadedOnly})
g:Show() 
\end{luadraw}
\end{demo}


\subsection{Drawing a List of Facets: Dfacet and Dmixfacet}

There are two possible methods:
\begin{enumerate}
    \item For a solid $S$ in the form of a list of facets (with 3D points), the method is:
\par\hfil\textbf{g:Dfacet(S,options)}\hfil\par
where $S$ is the list of facets and \emph{options} is a table defining the options. These are:
    \begin{itemize}
        \item \opt{mode=}: Sets the representation mode.
            \begin{itemize}
                \item \emph{mode=mWireframe}: Wireframe mode, draws only the edges.
                \item \emph{mode=mFlat or mFlatHidden}: Draws the faces in a solid color, as well as the edges.     \item \emph{mode=mShaded or mShadedHidden}: The faces are drawn in shaded color based on their inclination, as well as the edges. The default mode is 3.
                \item \emph{mode=mShadedOnly}: The faces are drawn in shaded color based on their inclination, but not the edges.
            \end{itemize}
        \item \opt{contrast}: This is a number that defaults to 1. This number allows you to accentuate or diminish the shade of the facet colors in the \emph{mShaded}, \emph{mShadedHidden}, and \emph{mShadedOnly} modes.
        \item \opt{edgestyle}: This is a string that defines the line style of the edges. This is the current style by default.
        \item \opt{edgecolor}: This is a string that defines the color of the edges. This is the current default line color.
        \item \opt{hiddenstyle} : is a string that defines the line style of the hidden edges. By default, this is the value contained in the global variable \emph{Hiddenlinestyle} (which itself is "dotted" by default).
        \item \opt{hiddencolor} : is a string that defines the color of the hidden edges. This is the current default line color.
        \item \opt{edgewidth} : line thickness of the edges in tenths of a point. This is the current default thickness.
        \item \opt{opacity} : a number between 0 and 1 that allows you to set transparency or not on the facets. The default value is 1, which means no transparency.
        \item \opt{backcull} : a boolean that defaults to false. When set to true, facets considered invisible (normal vector not directed towards the observer) are not displayed. This option is useful for convex polyhedra because it reduces the number of facets to be drawn.
        \item \opt{clip}: Boolean that defaults to false. When set to true, the facets are clipped by the 3D window.
        \item \opt{twoside}: Boolean that defaults to true, meaning that both sides of the facets (inner and outer) are distinguished; the two sides will not have exactly the same color.
        \item \opt{color}: String defining the fill color of the facets; it is "white" by default.     \item \opt{usepalette} (\emph{nil} by default), this option allows you to specify a color palette for painting the facets as well as a calculation mode, the syntax is: \emph{usepalette = \{palette,mode\}}, where \emph{palette} designates a table of colors which are themselves tables of the form \emph{\{r,g,b\}} where r, g and b are numbers between $0$ and $1$, and \emph{mode} which is a string that can be either \emph{"x"}, or \emph{"y"}, or \emph{"z"}. In the first case for example, the facets with the center of gravity of minimum abscissa have the first color of the palette, the facets with the center of gravity of maximum abscissa have the last color of the palette, for the others, the color is calculated according to the abscissa of the center of gravity by linear interpolation. 
    \end{itemize}
    \item For multiple facet lists in the same drawing, the method is:
\par\hfil\textbf{g:Dmixfacet(S1,options1, S2,options2, ...)}\hfil\par
where $S1$, $S2$, ... are facet lists, and \emph{options1}, \emph{options2}, ... are the corresponding options. The options in one facet list also apply to the following ones if they are not changed. These options are identical to the previous method.

This method is useful for drawing multiple solids together, provided there are no intersections between the objects, as these are not handled here.
\end{enumerate}

\begin{demo}[courbeniv]{Example of contour lines on a surface}
\begin{luadraw}{name=courbes_niv}
local cos, sin = math.cos, math.sin, math.pi
local g = graph3d:new{window3d={0,5,0,10,0,11}, adjust2d=true, size={10,10}, viewdir=perspective("central",220,60,15,M(2.5,5,5.5))}
g:Labelsize("footnotesize")
local S = cartesian3d(function(u,v) return (u+v)/(2+cos(u)*sin(v)) end,0,5,0,10,{30,30})
local n = 10 -- nombre de niveaux
local Colors = getpalette(palGasFlame,n,true) -- liste de 10 couleurs au format table
local niv, S1 = {}
for k = 1, n do
    S1, S = cutfacet(S,{M(0,0,k),-vecK}) -- section de S avec le plan z=k
    insert(niv,{S1, {color=Colors[k],mode=mShaded,edgewidth=0.5}}) -- S1 est la partie sous le plan et S au dessus
end
insert(niv,{S, {color=Colors[n+1]}}) -- insertion du dernier niveau
-- niv est une liste du type {facettes1, options1, facettes2, options2, ...}
g:Dboxaxes3d({grid=true, gridcolor="gray",fillcolor="lightgray"})
g:Dmixfacet(table.unpack(niv))
for k = 1, n do
    g:Dballdots3d( M(5,0,k), rgb(Colors[k]))
end
g:Dlabel("$z=\\frac{x+y}{2+\\cos(x)\\sin(y)}$", Z((g:Xinf()+g:Xsup())/2, g:Yinf()), {pos="N"})
g:Show()
\end{luadraw}
\end{demo}

\subsection{Functions for Constructing Facet Lists}

The following functions return a solid as a list of facets (with 3D points).
\subsubsection{surface()}

The function \textbf{surface(f,u1,u2,v1,v2,grid)} returns the surface parameterized by the function $f\colon(u,v) \mapsto f(u,v)\in \mathbf R^3$. The range for parameter $u$ is given by \emph{u1} and \emph{u2}. The range for parameter $v$ is given by \emph{v1} and \emph{v2}. The optional parameter \emph{grid} is $\{25,25\}$ by default; it defines the number of points to calculate for parameter $u$ followed by the number of points to calculate for parameter $v$.

There are two variants for surfaces:

\subsubsection{cartesian3d()}

The function \textbf{cartesian3d(f,x1,x2,y1,y2,grid,addWall)} returns the Cartesian surface with equation $z=f(x,y)$ where $f\colon(x,y)\mapsto f(x,y)\in\mathbb R$. The interval for $x$ is given by \emph{x1} and \emph{x2}. The interval for $y$ is given by \emph{y1} and \emph{y2}. The optional parameter \emph{grid} is $\{25,25\}$ by default; it defines the number of points to calculate for $x$ followed by the number of points to calculate for $y$. The parameter \emph{addWall} is 0 or "x", or "y", or "xy" (0 by default). When this option is set to "x" (or "xy"), the function returns, after the list of facets composing the surface, a list of separating facets (walls or partitions) between each "layer" of facets. A layer corresponds to two consecutive values ​​of the parameter $x$. With the value "y" (or "xy"), it is a list of separating facets (walls) between each "layer" corresponding to two consecutive values ​​of the parameter "y". This option can be useful with the \textbf{g:Dscene3d} method (only), because the separating partitions form a partition of space isolating the facets of the surface, which avoids unnecessary intersection calculations between them. This is particularly the case with non-convex surfaces.

For example, here is the code for figure \ref{pointcol}:
\begin{Luacode}
\begin{luadraw}{name=point_col}
local g = graph3d:new{window3d={-2,2,-2,2,-4,4}, window={-3.5,3,-5,5}, size={8,9,0}, viewdir={120,60}}
local S = cartesian3d(function(u,v) return u^2-v^2 end, -2,2,-2,2,{20,20}) -- surface of equation z=x^2-y^2
local Tx = g:Intersection3d(S, {Origin,vecI}) --intersection of S with the yOz plane
local Ty = g:Intersection3d(S, {Origin,vecJ}) --intersection of S with the xOz plane
g:Dboxaxes3d({grid=true,gridcolor="gray",fillcolor="LightGray",drawbox=true})
g:Dfacet(S,{mode=mShadedOnly,color="ForestGreen"}) -- surface drawing
g:Dedges(Tx, {color="Crimson", hidden=true, width=8}) -- intersection with yOz
g:Dedges(Ty, {color="Navy",hidden=true, width=8}) -- intersection with xOz
g:Dpolyline3d( {M(2,0,4),M(-2,0,4),M(-2,0,-4)}, "Navy,line width=.8pt")
g:Dpolyline3d( {M(0,-2,4),M(0,2,4),M(0,2,-4)}, "Crimson,line width=.8pt")
g:Show()
\end{luadraw}
\end{Luacode}

\subsubsection{cylindrical\_surface()}

The function \textbf{cylindrical\_surface(r,z,u1,u2,theta1,theta2,grid,addWall)} returns the surface parameterized as cylindrical by \emph{r(u,theta), theta, z(u,theta)}. The arguments $r$ and $z$ are therefore two real-valued functions of $u$ and $\theta$. The interval for $u$ is given by \emph{u1} and \emph{u2}. The interval for $\theta$ is given by \emph{theta1} and \emph{theta2} (in radians). The optional parameter \emph{grid} is $\{25,25\}$ by default; it defines the number of points to calculate for $u$ followed by the number of points to calculate for $v$. The parameter \emph{addWall} is 0 or "v" or "z" or "vz" (0 by default). When this option is "v" or "vz", the function returns, after the list of facets composing the surface, a list of separating facets (walls or partitions) between each "layer" of facets, a layer corresponds to two consecutive values ​​of the angle $\theta$\footnote{These partitions are in fact planes of equation $\theta=$constant}. When this option is set to "z" or "vz", the function returns, after the list of facets composing the surface, a list of separating facets (walls or partitions) between each "layer" of facets. A layer corresponds to two consecutive values ​​of the dimension $z$\footnote{These partitions are actually planes with the equation $z=$constant}, the values ​​of $z$ are calculated from the values ​​of the parameter $u$ and with the value \emph{theta1}. This is useful when $z$ only depends on $u$ (and therefore not on \emph{theta}). This option can be useful with the \textbf{g:Dscene3d} method (only), because the separating partitions form a partition of space isolating the surface facets, which avoids unnecessary intersection calculations between them. This is particularly the case with non-convex surfaces.

\begin{demo}{Surfaces using the \emph{addWall} option}
\begin{luadraw}{name=surface_with_addWall}
local pi, ch, sh = math.pi, math.cosh, math.sinh
local g = graph3d:new{window3d={-4,4,-4,4,-5,5}, window={-10,10,-4,4}, size={10,10}, viewdir={60,60}}
g:Labelsize("footnotesize")
local S,wall = cartesian3d(function(x,y) return x^2-y^2 end,-2,2,-2,2,nil,"xy")
g:Saveattr(); g:Viewport(-10,0,-4,4); g:Coordsystem(-4.5,4.5,-4.5,4.75)
g:Dscene3d( 
    g:addWall(wall), -- 2 facet cutouts with this instruction, and 529 facet cutouts without it
    g:addFacet(S,{color="SteelBlue"}),
    g:addAxes(Origin,{arrows=1}) )
g:Restoreattr() 
g:Saveattr(); g:Viewport(0,10,-4,4); g:Coordsystem(-5,5,-5,5)
local r = function(u,v) return ch(u) end
local z = function(u,v) return sh(u) end
S,wall = cylindrical_surface(r,z,2,-2,-pi,pi,{25,51},"zv")
g:Dscene3d( 
    g:addWall(wall), -- 13 facet cutouts with this instruction, and more than 17000 facet cutouts without it ...
    g:addFacet(S,{color="Crimson"}),
    g:addAxes(Origin,{arrows=1})  )
g:Restoreattr()     
g:Show()
\end{luadraw}
\end{demo}

\subsubsection{curve2cone()}
The function \textbf{curve2cone(f,t1,t2,S,args)} constructs a cone with vertex S (3d point) and the base curve parametrized by $f\colon t\mapsto f(t)\in\mathbf R^3$ on the interval defined by \emph{t1} et \emph{t2}. The argument \emph{args} is an optional table to define the options, which are:
\begin{itemize}
    \item \opt{nbdots} which represents the minimum number of points on the curve to calculate (15 by default).
    \item \opt{ratio} which is a number representing the homothety ratio (centered at vertex S) to construct the other part of the cone. By default, \emph{ratio} is 0 (no second part).     
    \item \opt{nbdiv} which is a positive integer indicating the number of times the interval between two consecutive values ​​of the parameter $t$ can be bisected (dichotomized) when the corresponding points are too far apart. By default, \emph{nbdiv} is 0.
\end{itemize}

This function returns a list of facets, followed by a 3D polygonal line representing the edges of the cone.

\begin{demo}{Elliptical Cone Example}
\begin{luadraw}{name=curve2cone}
local cos, sin, pi = math.cos, math.sin, math.pi
local g = graph3d:new{ window3d={-2,2,-4,4,-3,3},window={-5.5,5.5,-5.5,5},size={10,10},viewdir=perspective("central")}
local f = function(t) return M(2*cos(t),4*sin(t),-3) end -- ellipse dans le plan z=-3
local C, bord = curve2cone(f,-pi,pi,Origin,{nbdiv=2, ratio=-1})
g:Dboxaxes3d({grid=true,gridcolor="gray",fillcolor="LightGray"})
g:Dpolyline3d(bord[1],"red,line width=2.4pt") -- bord inférieur
g:Dfacet(C, {mode=mShadedOnly,color="LightBlue"})  -- cône
g:Dpolyline3d(bord[2],"red,line width=0.8pt") -- bord supérieur
g:Show()
\end{luadraw}
\end{demo}

\subsubsection{curve2cylinder()}
The function \textbf{curve2cylinder(f,t1,t2,V,args)} constructs a cylinder with axis directed by the vector $V$ (3d point) and with a base parameterized by $f\colon t\mapsto f(t)\in\mathbf R^3$ on the interval defined by \emph{t1} and \emph{t2}. The second base is the translation of the first with the vector $V$. The argument \emph{args} is an optional table to define the options, which are:
\begin{itemize}
    \item \opt{nbdots} which represents the minimum number of points on the curve to calculate (15 by default).     \item \opt{nbdiv} which is a positive integer indicating the number of times the interval between two consecutive values ​​of the parameter $t$ can be cut in two (dichotomized) when the corresponding points are too far apart. By default, \emph{nbdiv} is 0.
\end{itemize}
This function returns a list of facets, followed by a 3D polygonal line representing the edges of the cylinder.

\begin{demo}{Section of a non-circular cylinder}
\begin{luadraw}{name=curve2cylinder}
local cos, sin, pi = math.cos, math.sin, math.pi
local g = graph3d:new{ window3d={-5,5,-5,5,-4,4},window={-9,8,-10.5,5.5},viewdir=perspective("central",39,64),size={10,10}}
local f = function(t) return M(4*cos(t)-cos(4*t),4*sin(t)-sin(4*t),-4) end -- courbe dans le plan z=-3
local V = 8*vecK
local C = curve2cylinder(f,-pi,pi,V,{nbdots=25,nbdiv=2})
local plan = {M(0,0,2), -vecK} -- plan de coupe z=2
local C1, C2, section = cutfacet(C,plan)
g:Dboxaxes3d({grid=true,gridcolor="gray",fillcolor="LightGray"})
g:Dfacet(C1, {mode=mShaded,color="LightBlue"})  -- partie sous le plan
g:Dfacet(g:Plane2facet(plan), {opacity=0.3,color="Chocolate"}) -- dessin du plan sous forme d'une facette
g:Filloptions("fdiag","red"); g:Dpolyline3d(section) -- dessin de la section
g:Dfacet(C2, {mode=3,color="LightBlue"})  -- partie du cylindre au dessus du plan
g:Show()
\end{luadraw}
\end{demo}

\subsubsection{line2tube()}
The function \textbf{line2tube(L,r,args)} constructs (as a list of facets) a tube centered on \emph{L}, which must be a 3D polygonal line (list of 3D points or list of lists of 3D points). The argument \emph{r} represents the radius of this tube. The argument \emph{args} is a table for defining the options, which are:
\begin{itemize}
    \item \opt{nbfacet}: number indicating the number of lateral facets of the tube (3 by default).
    \item \opt{close}: boolean indicating whether the polygonal line $L$ should be closed (false by default).
    \item \opt{hollow}: boolean indicating whether both ends of the tube should be open or not (false by default). When the \opt{close} option is set to true, the \opt{hollow} option is automatically set to true.     \item \opt{addwall}: A number that is 0 or 1 (0 by default). When this option is 1, the function returns, after the list of facets composing the tube, a list of separating facets (walls) between each "section" of the tube, which can be useful with the \textbf{g:Dscene3d} method (only).
\end{itemize}

The function \textbf{section2tube(section,L,args)} also constructs a tube centered on \emph{L}, which must be a list of 3D points. The argument \emph{section} must be a facet centered on the first point of \emph{L}; it represents a section of the tube to be constructed. The argument \emph{args} is an array for defining the options, which are:

\begin{itemize}
    \item \opt{close}: Boolean indicating whether the polygonal line $L$ should be closed (false by default).
    \item \opt{hollow}: Boolean indicating whether both ends of the tube should be open or not (false by default). When the \opt{close} option is true, the \opt{hollow} option automatically takes the value true.
    \item \opt{addwall}: A number that is either 0 or 1 (0 by default). When this option is set to 1, the function returns, after the list of facets composing the tube, a list of separating facets (walls) between each "section" of the tube, which can be useful with the \textbf{g:Dscene3d} method (only).
\end{itemize}

\begin{demo}{Example with line2tube and section2tube}
\begin{luadraw}{name=line2tube_section2tube}
local g = graph3d:new{window={-5,6,-4.5,8}, viewdir={45,60}, margin={0,0,0,0}, size={10,10}}
local L1 = map(toPoint3d,polyreg(0,3,6)) -- hexagone régulier dans le plan xOy, centre O de sommet M(3,0,0)
local L2 = shift3d(rotate3d(L1,90,{Origin,vecJ}),3*vecJ)
local L3 = shift3d(reverse(L1),6*vecK)
L3[6] = L3[5]-2*vecK -- modification of the last point
local section = shift3d({M(2,0,0.5),M(4,0,0.5),M(4,0,-0.5),M(2,0,-0.5)},6*vecK)
local T1 = line2tube(L1,1,{nbfacet=8,close=true}) -- tube 1 refermé
local T2 = line2tube(L2,1,{nbfacet=8})  -- tube 2 non refermé
local T3 = section2tube(section, L3,{hollow=true})
g:Dmixfacet( T1, {color="Crimson",opacity=0.8}, T2, {color="SteelBlue"}, T3, {color="ForestGreen"} )
g:Show()
\end{luadraw}
\end{demo}

\subsubsection{rotcurve()}
The function \textbf{rotcurve(p,t1,t2,axis,angle1,angle2,args)} constructs, as a list of facets, the surface swept by the curve parameterized by $p\colon t\mapsto p(t)\in \mathbf R^3$ over the interval defined by \emph{t1} and \emph{t2}, by rotating it around \emph{axis} (which is a table of the form \{point3d, 3d vector\} representing an oriented line in space), by an angle ranging from \emph{angle1} (in degrees) to \emph{angle2}. The \emph{args} argument is a table for defining the options, which are:
\begin{itemize}
    \item \opt{grid}: table consisting of two numbers, the first being the number of points calculated for the t parameter, and the second being the number of points calculated for the angular parameter. By default, the value of \opt{grid} is \{25,25\}.

    \item \opt{addwall}: number equal to 0, 1, or 2 (0 by default). When this option is set to 1, the function returns, after the list of facets composing the surface, a list of separating facets (walls) between each "layer" of facets (a layer corresponds to two consecutive values ​​of the t parameter), and with a value of 2, it is a list of separating facets (walls) between each rotation "slice" (a layer corresponds to two consecutive values ​​of the angular parameter; this is useful when the curve is in the same plane as the rotation axis). This option can be useful with the \textbf{g:Dscene3d} method (only).
\end{itemize}

\begin{demo}{Example with rotcurve}
\begin{luadraw}{name=rotcurve}
local cos, sin, pi, i = math.cos, math.sin, math.pi, cpx.I
local g = graph3d:new{viewdir={30,60},size={10,10}}
local p = function(t) return M(0,sin(t)+2,t) end -- curve in the plane yOz
local axe = {Origin,vecK}
local S = rotcurve(p,pi,-pi,axe,0,360,{grid={25,35}})
local  visible, hidden = g:Classifyfacet(S)
g:Dfacet(hidden, {mode=mShadedOnly,color="cyan"})
g:Dline3d(axe,"red,line width=1.2pt")
g:Dfacet(visible, {mode=5,color="cyan"})
g:Dline3d(axe,"red,line width=1.2pt,dashed")
g:Dparametric3d(p,{t={-pi,pi},draw_options="red,line width=1.2pt"})
g:Show()
\end{luadraw}
\end{demo}

\paragraph{Note}: If the surface orientation does not seem correct, simply swap the parameters \emph{t1} and \emph{t2}, or \emph{angle1} and \emph{angle2}.

\subsubsection{rotline()}

The function \textbf{rotline(L,axis,angle1,angle2,args)} constructs, as a list of facets, the surface swept by the list of 3d points $L$ by rotating it around \emph{axis} (which is a table of the form \{point3d, 3d vector\} representing an oriented line in space), through an angle ranging from \emph{angle1} (in degrees) to \emph{angle2}. The \emph{args} argument is a table for defining the options, which are:
\begin{itemize}
    \item \opt{nbdots}: This is the number of points calculated for the angular parameter. By default, the value of \opt{nbdots} is 25.

    \item \opt{close}: A boolean indicating whether $L$ should be closed (false by default).

    \item \opt{addwall}: A number equal to 0, 1, or 2 (0 by default). When this option is set to 1, the function returns, after the list of facets composing the surface, a list of separating facets (walls) between each "layer" of facets (a layer corresponds to two consecutive points in the list $L$), and with a value of 2, it is a list of separating facets (walls) between each rotation "slice" (a layer corresponds to two consecutive values ​​of the angular parameter; this is useful when the curve is in the same plane as the rotation axis). This option can be useful with the \textbf{g:Dscene3d} method (only).
\end{itemize}

\begin{demo}{Example with rotline}
\begin{luadraw}{name=rotline}
local g = graph3d:new{window={-4,4,-4,4},size={10,10}}
local L = {M(0,0,4),M(0,4,0),M(0,0,-4)} -- list of points in the yOz plane
local axe = {Origin,vecK}
local S = rotline(L,axe,0,360,{nbdots=5}) -- point 1 and point 5 are confused
g:Dfacet(S,{color="Crimson",edgecolor="Gold",opacity=0.8})
g:Show()
\end{luadraw}
\end{demo}      


\subsection{Edges of a solid}

An "edge" object is a table with two fields: one field named \emph{visible} that contains a 3D polygonal line corresponding to the visible edges, and another field named \emph{hidden} that contains a 3D polygonal line corresponding to the hidden edges.

\begin{itemize}
    \item The method \textbf{g:Edges(P)}, where $P$ is a polyhedron, returns the edges of $P$ as an "edge" object. An edge of $P$ is visible when it belongs to at least one visible face.     \item The method \textbf{g:Intersection3d(P,plane)}, where $P$ is a polyhedron or a list of facets, returns as an "edge" object the intersection between $P$ and the plane represented by \emph{plane} (it is a table of the form \{A,u\} where $A$ is a point on the plane and $u$ is a normal vector, so they are two 3d points).
    \item The method \textbf{g:Dedges(edges,options)} allows you to draw \emph{edges}, which must be an "edge" object. The argument \emph{options} is a table defining the options, these are:
\begin{itemize}
    \item \opt{hidden}: Boolean indicating whether hidden edges should be drawn (false by default).
    \item \opt{visible}: Boolean indicating whether visible edges should be drawn (true by default).
    \item \opt{clip}: Boolean indicating whether edges should be clipped by the 3D window (false by default).
    \item \opt{hiddenstyle}: String defining the line style of hidden edges. By default, this option contains the value of the global variable \emph{Hiddenlinestyle} (which defaults to "dotted").
    \item \opt{hiddencolor}: String defining the color of hidden edges. By default, this option contains the same color as the \opt{color} option.
    \item \opt{style}: String defining the line style of visible edges. By default, this option contains the current line drawing style.     \item \opt{color}: String defining the color of the visible edges. By default, this option contains the current line drawing color.
    \item \opt{width}: Number representing the line thickness of the edges (in tenths of a point). By default, this variable contains the current line drawing thickness.
\end{itemize}

    \item \textbf{Complement}:
\begin{itemize}
    \item The function \textbf{facetedges(F)}, where $F$ is a list of facets or a polyhedron, returns a list of 3D segments representing all the edges of $F$. The result is not an "edge" object, and is drawn with the \textbf{g:Dpolyline3d} method.     \item The function \textbf{facetvertices(F)}, where $F$ is a list of facets or a polyhedron, returns the list of all vertices of $F$ (3d points).
\end{itemize}
\end{itemize}

\subsection{Methods and functions applying to facets or polyhedra}

\begin{itemize}
    \item The method \textbf{g:Isvisible(F)}, where $F$ denotes \textbf{a} facet (list of at least 3 coplanar and non-aligned 3D points), returns true if facet $F$ is visible (normal vector directed towards the observer). If $A$, $B$, and $C$ are the first three points of $F$, the normal vector is calculated by performing the vector product $\vec{AB}\wedge\vec{AC}$.

    \item The method \textbf{g:Classifyfacet(F)}, where $F$ is a list of facets or a polyhedron, returns \textbf{two} lists of facets: the first is the list of visible facets, and the second, the list of invisible facets.

    \item The method \textbf{g:Sortfacet(F,backcull)}, where $F$ is a list of facets, returns this list of facets sorted from furthest to closest to the observer. The optional argument \emph{backcull} is a boolean that defaults to false; when it is true, non-visible facets are excluded from the result (only visible facets are then returned after being sorted). The calculation of a facet's distance is based on its center of gravity. The so-called "painter" technique consists of displaying the facets from furthest to closest, therefore in the order of the list returned by this function (the displayed result, however, is not always correct depending on the size and shape of the facets).

    \item The method \textbf{g:Sortpolyfacet(P,backcull)}, where $P$ is a polyhedron, returns the list of facets of $P$ (facets with 3D points) sorted from furthest to closest to the observer. The optional argument \emph{backcull} is a boolean that defaults to false; when it is true, invisible facets are excluded from the result, as in the previous method. These two sorting methods are used by the methods for drawing polyhedrons or facets (\emph{Dpoly}, \emph{Dfacet}, and \emph{Dmixfacet}).

    \item The method \textbf{g:Outline(P)}, where $P$ is a polyhedron, returns the "outline" of $P$ as a two-field table. One field, named \emph{visible}, contains a 3D polygonal line representing the "edges" (segments) belonging to a single facet, the latter being visible, or to two facets, one visible and one hidden; the other field, named \emph{hidden}, contains a 3D polygonal line representing the "edges" belonging to a single facet, the latter being hidden.

    \item The function \textbf{border(P)}, where $P$ is a polyhedron or a list of facets, returns a 3D polygonal line corresponding to the edges belonging to a single facet of $P$ (these edges are placed "end to end" to form a polygonal line).

    \item The function \textbf{getfacet(P,list)}, where $P$ is a polyhedron, returns the list of facets of $P$ (with 3D points) whose number appears in the table \emph{list}. If the argument \emph{list} is not specified, the list of all facets of $P$ is returned (in this case, it is the same as \textbf{poly2facet(P)}).

    \item The function \textbf{facet2plane(L)}, where $L$ is either a facet or a list of facets, returns either the plane containing the facet or the list of planes containing each of the facets of $L$. A plane is a table of the type \{A,u\} where $A$ is a point on the plane and $u$ is a normal vector to the plane (i.e., two 3D points).

    \item The function \textbf{reverse\_face\_orientation(F)} where $F$ is either a facet, a list of facets, or a polyhedron, returns a result of the same nature as $F$ but in which the order of the vertices of each facet has been reversed. This can be useful when the orientation of space has been changed.

\begin{demo}{Sphere inscribed in an octahedron with the center projected onto the faces}
\begin{luadraw}{name=sphere_octaedre}
require "luadraw_polyhedrons"
local g = graph3d:new{ window3d={-3,3,-3,3,-3,3}, size={10,10}}
local P = octahedron(Origin,M(0,0,3)) -- polyhedron defined in the luadraw_polyhedrons module
P = rotate3d(P,-10,{Origin,vecK}) -- rotate3d on a polyhedron returns a polyhedron
local V, H = g:Classifyfacet(P) -- V for visible facets, H for hidden
local S = map(function(p) return {proj3d(Origin,p),p[2]} end, facet2plane(V) )
-- S contains the list of: {projected, normal vector} (projected from Origin onto the visible faces)
local R = pt3d.abs(S[1][1]) -- sphere radisu
g:Dboxaxes3d({grid=true, gridcolor="gray", fillcolor="LightGray"})
g:Dfacet(H, {color="blue",opacity=0.9}) -- drawing of non-visible facets
g:Dsphere(Origin,R,{mode=mBorder,color="orange"}) -- drawing of the sphere
g:Dballdots3d(Origin,"gray",0.75) -- center of the sphere
for _,D in ipairs(S) do -- segments connecting the origin to the projected
    g:Dpolyline3d( {Origin,D[1]},"dashed,gray")
end
g:Dfacet(V,{opacity=0.4, color="LightBlue"}) -- visible facets of the octahedron
g:Dcrossdots3d(S,nil,0.75) -- drawing of the projections on the faces
g:Dpolyline3d( {M(0,-3,3), M(0,0,3), M(-3,0,3)},"gray")
g:Show()            
\end{luadraw}
\end{demo}
\end{itemize}

\subsection{Cutting a solid: cutpoly and cutfacet}

\begin{itemize}
    \item The function \textbf{cutpoly(P,plane,close)} cuts the polyhedron $P$ with the plane \emph{plane} (a ​​table of type \{A,n\} where $A$ is a point on the plane and $n$ is a vector normal to the plane). The function returns three things: the part located in the half-space containing the vector $n$ (in the form of a polyhedron), followed by the part located in the other half-space (still in the form of a polyhedron), followed by the section in the form of a facet oriented by $-n$. When the optional argument \emph{close} is true, the section is added to both resulting polyhedra, which closes them (false by default).\par
Note: When the polyhedron $P$ is not convex, the section result is not always correct.

\begin{demo}{Cube cut by a plane (cutpoly), with \emph{close}=false and with \emph{close}=true}
\begin{luadraw}{name=cutpoly}
local g = graph3d:new{window3d={-3,3,-3,3,-3,3}, window={-4,4,-3,3},size={10,10}}
local P = parallelep(M(-1,-1,-1),2*vecI,2*vecJ,2*vecK)
local A, B, C = M(0,-1,1), M(0,1,1), M(1,-1,0)
local plane = {A, pt3d.prod(B-A,C-A)}
local P1 = cutpoly(P,plane)
local P2 = cutpoly(P,plane,true)
g:Lineoptions(nil,"Gold",8)
g:Dpoly( shift3d(P1,-2*vecJ), {color="Crimson",mode=mShadedHidden} )
g:Dpoly( shift3d(P2,2*vecJ), {color="Crimson",mode=mShadedHidden} )
g:Dlabel3d(
    "close=false", M(2,-2,-1), {dir={vecJ,vecK}},
    "close=true", M(2,2,-1), {}
    )
g:Show()            
\end{luadraw}
\end{demo}

    \item The function \textbf{cutfacet(F,plane,close)}, where $F$ is a facet, a list of facets, or a polyhedron, does the same thing as the previous function except that this function returns lists of facets and not polyhedra. This function was used in the contour line example in Figure \ref{courbeniv}.
\end{itemize}

\subsection{Clipping Facets with a Convex Polyhedron: clip3d}

The function \textbf{clip3d(S,P,exterior)} clips the solid $S$ (a list of facets or a polyhedron) with the convex solid $P$ (a list of facets or a polyhedron) and returns the resulting list of facets. The optional argument \emph{exterior} is a boolean that defaults to false. In this case, the part of $S$ that is interior to $P$ is returned; otherwise, the part of $S$ that is exterior to $P$ is returned.
\textbf{Note}: The result is not always satisfactory for the exterior part.

\paragraph{Special case}: Clipping a list of facets $S$ (or polyhedron) with the current 3D window can be done with this function as follows:

\begin{center}
\textbf{S = clip3d(S, g:Box3d())}
\end{center}

Indeed, the \textbf{g:Box3d()} method returns the current 3D window as a parallelepiped.

\begin{demo}[clip3d]{Example with clip3d: constructing a die from a cube and a sphere}
\begin{luadraw}{name=clip3d}
local g = graph3d:new{window={-3,3,-3,3},size={10,10},viewdir=perspective("central")}
local S = sphere(Origin,3)
local C = parallelep(M(-2,-2,-2),4*vecI,4*vecJ,4*vecK)
local C1 = clip3d(S,C) -- sphere clipped by the cube
local C2 = clip3d(C,S) -- cube clipped by the sphere
local V = g:Classifyfacet(C2) -- visible facets of C2
g:Dfacet( concat(C1,C2), {color="Beige",mode=mShadedOnly,backcull=true} ) -- only visible faces
g:Dpolyline3d(V,true,"line width=0.8pt") -- outline of the visible faces of C2
local A, B, C, D = M(2,-2,-2), M(2,2,2), M(-2,2,-2), M(0,0,2) -- drawing black dots
g:Filloptions("full","black")
g:Dcircle3d( D,0.25,vecK); g:Dcircle3d( (2*A+B)/3,0.25,vecI)
g:Dcircle3d( (A+2*B)/3,0.25,vecI); g:Dcircle3d( (3*B+C)/4,0.25,vecJ)
g:Dcircle3d( (B+C)/2,0.25,vecJ); g:Dcircle3d( (B+3*C)/4,0.25,vecJ)
g:Show()            
\end{luadraw}
\end{demo}

\subsection{Clip a plane with a convex polyhedron: clipplane}

The function \textbf{clipplane(plane,P)}, where the argument \emph{plane} is a table of the form \emph{\{A,n\}} representing the plane passing through $A$ (3d point) and normal vector $n$ (non-zero 3d point), and \emph{P} is a convex polyhedron, returns the section of the polyhedron through the plane, if it exists, in the form of a facet (list of 3d points) oriented by $n$.
%
\section{The Dscene3d Method}

\subsection{The Principle, the Limitations}

The major flaw of the \textbf{g:Dpoly}, \textbf{g:Dfacet}, and \textbf{g:Dmixfacet} methods is that they do not handle possible intersections between facets of different solids. Not to mention that sometimes, even for a simple convex polyhedron, the painter's algorithm does not always produce the correct result (because the facets are sorted only by their center of gravity). Furthermore, these methods only allow you to draw facets.

The principle of the \textbf{g:Dscene3d()} method is to classify the 3D objects to be drawn (facets, polygonal lines, points, labels, etc.) in a tree (which represents the scene). At each node of the tree, there is a 3D object, let's call it $A$, and two descendants. One of the descendants will contain the 3D objects that are in front of object A (i.e., closer to the observer than $A$), and the other descendant will contain the 3D objects that are behind object A (i.e., further from the observer than $A$).

In particular, to classify a facet $B$ with respect to a facet $A$ that is already in the tree, we proceed as follows: we split facet $B$ with the plane containing facet $A$, which generally results in two "half" facets, one that will be in front of $A$ (the one in the half-space "containing" the observer), and the other that will therefore be behind $A$.

This method is effective but has limitations because it can cause the number of facets in the tree to explode, thus increasing its size exponentially. This can make it prohibitive to use this method when there are many facets (long computation time\footnote{Lua is an interpreted language, so execution is generally longer than with a compiled language.}, excessively large tkz file size, excessively long drawing time per tikz). However, it is very effective when there are few facets, and therefore few facet intersections (convex objects with few facets). Furthermore, it is possible to draw below and above the 3D scene, i.e., before using the \textbf{g:Dscene3d} method, and after its use.

This method should therefore be reserved for very simple scenes. For complex 3D scenes, the vector format is not suitable, so it is better to turn to other tools like Povray, Blender, or WebGL.


\subsection{Construction of a 3D scene}

The \textbf{g:Dscene3d(...)} method allows this construction. It takes as arguments the 3D objects that will constitute this scene one after the other. These 3D objects are themselves created using dedicated methods that will be detailed later. In the current version, these 3D objects can be:
\begin{itemize}
    \item polyhedra,
    \item lists of facets (with point3d),
    \item 3D polygonal lines,
    \item 3D points,
    \item labels,
    \item axes,
    \item planes, lines,
    \item angles,
    \item circles, and arcs.
\end{itemize}

\begin{demo}[planes]{First example with Dscene3d: intersection of two planes}
\begin{luadraw}{name=intersection_plans}
local g = graph3d:new{viewdir={-10,60}, window={-5,5.5,-5.5,5.5},bg="gray", size={10,10}}
local P1 = planeEq(1,1,1,-2) -- plane of equation x+y+z-2=0
local P2 = {Origin, vecK-vecJ} -- plane passing through O and normal to (1,1,1)
local D = interPP(P1,P2) -- line of intersection between P1 and P2 (D = {A,u})
local posD = D[1]+1.85*D[2] -- to place the label
Hiddenlines = true; Hiddenlinestyle = "dotted" -- display hidden dotted lines
g:Dscene3d(
    g:addPlane(P1, {color="Crimson",edge=true,edgecolor="Pink",edgewidth=8}), -- plane P1
    g:addPlane(P2, {color="ForestGreen",edge=true,edgecolor="Pink",edgewidth=8}),  -- plane P2
    g:addLine(D, {color="Navy",edgewidth=12}),  -- line D
    g:addAxes(Origin, {arrows=1, color="Gold",width=8}),  -- added arrow axes
    g:addLabel( -- added labels, these could have been added over the scene
        "$D=P_1\\cap P_2$",posD,{color="Navy"},
        "$P_2$", M(3,0,0)+3.5*M(0,1,1),{color="white",dir={vecI,vecJ+vecK}},
        "$P_1$",M(2,0,0)+1.8*M(-1,-1,2), {dir={M(-1,1,0),M(-1,-1,2),1.125*M(1,-1,0)}}
        )
    )
g:Show()
\end{luadraw}
\end{demo}

\subsection{Methods for adding an object to the 3D scene}

These methods are to be used as arguments to the \textbf{g:Dscene3d(...)} method, as in the example above.

\subsubsection{Adding facets: g:addFacet and g:addPoly}

The \textbf{g:addFacet(list,options)} method, where \emph{list} is a facet or a list of facets (with 3D points), allows you to add these facets to the scene.

The \textbf{g:addPoly(list,options)} method allows you to add the polyhedron $P$ to the scene.

In both cases, the optional \emph{options} argument is a 12-field table. These options (with their default values) are:

\begin{itemize}
    \item \opt{color="white"}: Sets the fill color of the facets. This color will be shaded depending on their inclination. By default, the edges of the facets are not drawn (only the fill).     \item \opt{usepalette} (\emph{nil} by default), this option allows you to specify a color palette for painting the facets as well as a calculation mode, the syntax is: \emph{usepalette = \{palette,mode\}}, where \emph{palette} designates a table of colors which are themselves tables of the form \emph{\{r,g,b\}} where r, g and b are numbers between $0$ and $1$, and \emph{mode} which is a string which can be either \emph{"x"}, or \emph{"y"}, or \emph{"z"}. In the first case for example, the facets at the center of gravity of minimum abscissa have the first color of the palette, the facets at the center of gravity of maximum abscissa have the last color of the palette, for the others, the color is calculated according to the abscissa of the center of gravity by linear interpolation.     \item \opt{opacity=1}: Number between 0 and 1 to define the opacity of the facets (1 means no transparency).
    \item \opt{backcull=false}: Boolean indicating whether invisible facets should be excluded from the scene. By default, they are present.
    \item \opt{clip=false}: Boolean indicating whether the facets should be clipped by the 3D window.
    \item \opt{contrast=1}: Numerical value to increase or decrease the color contrast between the facets. With a value of 0, all facets have the same color.
    \item \opt{twoside=true}: Boolean indicating whether the inner and outer sides of the facets are distinguished. The color of the inner side is slightly lighter than that of the outer side.

    \item \opt{edge=false}: Boolean indicating whether edges should be added to the scene.
    \item \opt{edgecolor=}: Indicates the color of the edges when they are drawn; this is the current color by default.
    \item \opt{edgewidth=}: Indicates the line thickness (in tenths of a point) of the edges; this is the current thickness by default.
    \item \opt{hidden=Hiddenlines}: Boolean indicating whether hidden edges should be drawn. \emph{Hiddenlines} is a global variable that defaults to false.
    \item \opt{hiddenstyle=Hiddenlinestyle}: String defining the line style of the hidden edges. \emph{Hiddenlinestyle} is a global variable that defaults to "dotted."     \item \opt{matrix=ID3d}: 3D facet transformation matrix, by default this is the 3D identity matrix, i.e. the table \{M(0,0,0),vecI,vecJ,vecK\}.
\end{itemize}


\subsubsection{Adding a plane: g:addPlane and g:addPlaneEq}

The \textbf{g:addPlane(P,options)} method adds the plane $P$ to the 3D scene. This plane is defined as a table \{A,u\} where $A$ is a point on the plane (a 3D point) and $u$ is a normal vector to the plane (a non-zero 3D point). This function determines the intersection between this plane and the parallelepiped given by the \emph{window3d} argument (itself defined when the graph is created), which results in a facet, which is added to the scene. This method uses \textbf{g:addFacet}.

The method \textbf{g:addPlaneEq(coef,options)}, where \emph{coef} is a table consisting of four real numbers \{a,b,c,d\}, allows you to add the plane with the equation $ax+by+cz+d=0$ to the scene (this method uses the previous one).

In both cases, the optional argument \emph{options} is a table with 12 fields. These options are those of the \textbf{g:addFacet} method, plus the option \opt{scale=1}: this number is a homothety ratio. The homothety ratio is applied to the facet with the center of gravity of the facet and the ratio \emph{scale}. This allows you to adjust the size of the plane in its representation.

\subsubsection{Add a polygonal line: g:addPolyline}

The method \textbf{g:addPolyline(L,options)}, where $L$ is a list of 3D points, or a list of lists of 3D points, adds $L$ to the scene. The optional argument \emph{options} is a 10-field table. These options (with their default values) are:
\begin{itemize}
    \item \opt{style="solid"}: to set the line style; this is the current style by default.
    \item \opt{color=}: line color; this is the current color by default.
    \item \opt{close=false}: indicates whether the line $L$ (or each component of $L$) should be closed.     \item \opt{clip=false}: Indicates whether the line $L$ (or each component of $L$) should be clipped by the 3D window.
    \item \opt{width=}: Line thickness in tenths of a point; this is the current thickness by default.
    \item \opt{opacity=1}: Opacity of the line drawing (1 means no transparency).
    \item \opt{hidden=Hiddenlines}: Boolean indicating whether the hidden parts of the line should be represented. \emph{Hiddenlines} is a global variable that defaults to false.
    \item \opt{hiddenstyle=Hiddenlinestyle}: String defining the line style of the hidden parts. \emph{Hiddenlinestyle} is a global variable that defaults to "dotted."     \item \opt{arrows=0}: This option can be 0 (no arrow added to the line), 1 (an arrow added at the end of the line), or 2 (an arrow at the beginning and end of the line). The arrows are small cones.
    \item \opt{arrowscale=1}: Allows you to reduce or increase the size of the arrows.
    \item \opt{matrix=ID3d}: 3D transformation matrix (of the line). By default, this is the 3D identity matrix, i.e., the table \{M(0,0,0),vecI,vecJ,vecK\}.
\end{itemize}

\subsubsection{Add Axes: g:addAxes}

The \textbf{g:addAxes(O,options)} method adds the axes ($O$,\emph{vecI}), ($O$,\emph{vecJ}), and ($O$,\emph{vecK}) to the 3D scene, where the argument $O$ is a 3D point. The options are those of the \textbf{g:addPolyline} method, plus the \opt{legend=true} option, which automatically adds the name of each axis ($x$, $y$, and $z$) to the endpoint. These axes are not graduated.

    
\subsubsection{Add a line: g:addLine}

The \textbf{g:addLine(d,options)} method adds the line $d$ to the scene. This line $d$ is a table of the form \{A,u\} where $A$ is a point on the line (3D point) and $u$ is a direction vector (non-zero 3D point). The optional argument \emph{options} is a 10-field table. These options are those of the \textbf{g:addPolyline} method, plus the \opt{scale=1} option: this number is a homothety ratio. The homothety ratio is applied to the facet, with the center being the midpoint of the segment representing the line, and the ratio \emph{scale}. This allows you to adjust the size of the segment in its representation. This segment is the line clipped by the polyhedron given by the \emph{window3d} argument (itself defined when the graph is created), which results in a segment (possibly empty).

\subsubsection{Adding a "right" angle: g:addAngle}

The \textbf{g:addAngle(B,A,C,r,options)} method allows you to add the angle $\widehat{BAC}$ in the form of a parallelogram with side $r$ ($r$ is 0.25 by default). Only two sides are represented. The arguments $B$, $A$, and $C$ are 3D points. The options are those of the \textbf{g:addPolyline} method.

\subsubsection{Add a circular arc: g:addArc}

The \textbf{g:addArc(B,A,C,r,direction,normal,options)} method adds the arc of a circle centered at $A$ (3D point), with radius $r$, extending from $B$ to $C$ (3D points) in the direct direction if \emph{direction} is 1 (indirect otherwise). The arc is drawn in the plane passing through $A$ and orthogonal to the \emph{normal} vector (non-zero 3D point); this same vector orients the plane. The options are those of the \textbf{g:addPolyline} method.

\subsubsection{Add a circle: g:addCircle}

The \textbf{g:addCircle(A,r,normal,options)} method adds the circle with center $A$ (3D point) and radius $r$ in the plane passing through $A$ and orthogonal to the \emph{normal} vector (non-zero 3D point). The options are those of the \textbf{g:addPolyline} method.

\begin{demo}{Solid cylinder immersed in water}
\begin{luadraw}{name=cylindres_imbriques}
local g = graph3d:new{window={-5,5,-6,5}, viewdir={30,75},size={10,10},margin={0,0,0,0}}
Hiddenlines = false
local R, r, A, B = 3, 1.5
local C1 = cylinder(M(0,0,-5),5*vecK,R)  -- to model water
local C2 = cylinder(Origin,2*vecK,R,35,true) -- part of the container above the water (open cylinder)
local C3 = cylinder(M(0,0,-3),7*vecK,r) -- small cylinder immersed in water
-- sous la scène 3d
g:Lineoptions(nil,"gray",12)
g:Dcylinder(M(0,0,-5),7*vecK,R,{hiddenstyle="noline"}) -- container outline (large cylinder)
-- scène 3d
g:Dscene3d(
        g:addPoly(C1,{contrast=0.125,color="cyan",opacity=0.5}), 
        g:addPoly(C2,{contrast=0.125,color="WhiteSmoke", opacity=0.5}), 
        g:addPoly(C3,{contrast=0.25,color="Salmon",backcull=true}), 
        g:addCircle(M(0,0,2),R,vecK,{color="gray"}), -- upper edge of the container
        g:addCircle(M(0,0,-5),R,vecK,{color="gray"}), -- lower edge of the container
        g:addCircle(Origin,R-0.025,vecK, {width=2,color="cyan"}) -- upper edge water
        )
-- over the 3d scene
g:Lineoptions(nil,"black",8); A = 4*vecK; B = A+r*g:ScreenX()
g:Dpolyline3d( {A,B}, "<->"); g:Dlabel3d("$3\\,$cm",(A+B)/2,{pos="N",dist=0.25})
A = Origin+(r+1)*g:ScreenX(); B = A-3*vecK
g:Dpolyline3d( {A,B}, "<->"); g:Dlabel3d("h",(A+B)/2,{pos="E"})
g:Lineoptions("dashed")
g:Dpolyline3d({{A,A-g:ScreenX()},{B,B-g:ScreenX()}})
A = Origin-(R+1)*g:ScreenX(); B = A-vecK
g:Dpolyline3d({{A,A+g:ScreenX()},{B,B+g:ScreenX()}})
g:Linestyle("solid")
g:Dpolyline3d( {A,B}, "<->"); g:Dlabel3d("$2$\\,cm",(A+B)/2,{pos="W"})
g:Show()
\end{luadraw}
\end{demo}

\paragraph{Notes}:
\begin{itemize}
    \item The \textbf{g:ScreenX()} method returns the space vector (3D point) corresponding to the vector with affix 1 in the screen plane, and the \textbf{g:ScreenY()} method returns the space vector (3D point) corresponding to the vector with affix i in the screen plane.
    \item For the small cylinder (C3), we use the \opt{backcull=true} option to reduce the number of facets; however, we do not do this for the other two cylinders (C1 and C2) because they are transparent.
\end{itemize}

\subsubsection{Adding points: g:addDots}

The \textbf{g:addDots(dots,options)} method allows you to add 3D points to the scene. The argument \emph{dots} is either a 3D point or a list of 3D points. The optional argument \opt{options} is a four-field table. These options are:
\begin{itemize}
    \item \opt{style="ball"}: String defining the dot style. These are all 2D point styles, plus the "ball" (sphere) style, which is the default.
    \item \opt{color="black"}: String defining the dot color.
    \item \opt{scale=1}: Number allowing you to adjust the size of the points.
    \item \opt{matrix=ID3d}: 3D transformation matrix. By default, this is the 3D identity matrix, i.e., the table \{M(0,0,0),vecI,vecJ,vecK\}.
\end{itemize}

\subsubsection{Adding Labels: g:addLabels}

The \textbf{g:addLabel(text1, anchor1, options1, text2, anchor2, options2, ...)} method allows you to add the labels \emph{text1}, \emph{text2}, etc. The (required) arguments \emph{anchor1}, \emph{anchor2}, etc., are 3D points representing the anchor points of the labels. The (required) arguments \emph{options1}, \emph{options2}, etc., are 7-field tables. These options are:
\begin{itemize}
    \item \opt{color}: String defining the label color, initialized to the current label color.     \item \opt{style}: String defining the label style (as in 2D: "N", "NW", "W", ...), initialized to the current label style.
    \item \opt{dist=0}: Expresses the distance between the label and its anchor point (in the screen plane).
    \item \opt{size}: String defining the label size, initialized to the current label size.
    \item \opt{dir=\{\}}: Table defining the writing direction in space (usual direction by default).
In general, \emph{dir=\{dirX,dirY,dep\}}, and the three values ​​\emph{dirX}, \emph{dirY}, and \emph{dep} are three 3D points representing three vectors: the first two indicate the writing direction, the third a displacement (translation) of the label relative to the anchor point.
    \item \opt{showdot=false}: Boolean indicating whether a (2D) point should be drawn at the anchor point.
    \item \opt{matrix=ID3d}: 3D transformation matrix; by default, this is the 3D identity matrix, i.e., the table \{M(0,0,0),vecI,vecJ,vecK\}.
\end{itemize}

\begin{demo}{Construction of an icosahedron}
\begin{luadraw}{name=icosaedre}
local g = graph3d:new{window={-2.25,2.25,-2,2}, viewdir={40,60},bg="gray",size={10,10},margin={0,0,0,0}}
Hiddenlines = false
local phi = (1+math.sqrt(5))/2 -- nombre d'or
local A1, B1, C1, D1 = M(phi,-1,0), M(phi,1,0), M(-phi,1,0), M(-phi,-1,0) -- in the plane z=0
local A2, B2, C2, D2 = M(0,phi,1), M(0,phi,-1), M(0,-phi,-1), M(0,-phi,1) -- in the plane x=0
local A3, B3, C3, D3 = M(1,0,phi), M(-1,0,phi), M(-1,0,-phi), M(1,0,-phi) -- in the plane y=0
local ico = {   {A1,B1,A3}, {B1,A1,D3}, {D1,C1,C3}, {C1,D1,B3},
                {B2,A2,B1}, {A2,B2,C1}, {D2,C2,A1}, {C2,D2,D1},
                {B3,A3,A2}, {A3,B3,D2}, {D3,C3,B2}, {C3,D3,C2},
                {A1,A3,D2}, {B1,A2,A3}, {A2,C1,B3}, {D1,D2,B3},
                {B2,B1,D3}, {A1,C2,D3}, {B2,C3,C1}, {C2,D1,C3}  }
g:Dscene3d(
    g:addFacet({A2,B2,C2,D2},{color="Navy",twoside=false,opacity=0.8}),
    g:addFacet({A1,B1,C1,D1},{color="Crimson",twoside=false,opacity=0.8}),
    g:addFacet({A3,B3,C3,D3},{color="Chocolate",twoside=false,opacity=0.8}),
    g:addPolyline(facetedges(ico), {color="Gold",width=12}), -- drawing edges only
    g:addDots({A1,B1,C1,D1,A2,B2,C2,D2,A3,B3,C3,D3}, {color="black",scale=1.2}),
    g:addLabel("A1",A1,{style="W",dist=0.1}, "B1",B1,{style="S"}, "C2",C2,{}, "C3",C3,{}, "A3",A3,{style="N"}, "D1",D1,{},  "A2",A2,{},  "D2",D2,{}, "B3",B3,{style="E"}, "C1",C1,{}, "B2",B2,{}, "D3",D3,{style="W"} )
)
g:Show()
\end{luadraw}
\end{demo}

\subsubsection{Adding dividing walls: g:addWall}

Dividing walls are 3D objects that are inserted first into the scene tree. These objects are not drawn (therefore invisible); their role is to partition the space because a facet on one side of a dividing wall cannot be cut by the plane of a facet on the other side of the wall. This allows, in some cases, to significantly reduce the number of facet (or polygonal line) cuts during scene construction. A dividing wall can be an entire plane (i.e., a table of two 3D points of the form \{A,n\}, i.e., a point and a normal vector), or just a facet.

The syntax is: \textbf{g:addWall(C,options)} where $C$ is either a plane, a list of planes, a facet, or a list of facets. The \emph{options} argument is a table. The only available option is
\begin{itemize}
    \item \opt{matrix=ID3d}: 3D transformation matrix. By default, this is the 3D identity matrix, i.e., the table \{M(0,0,0),vecI,vecJ,vecK\}.
\end{itemize}

In the following example, the two dividing walls have been drawn for visualization, but they are normally invisible:
\begin{demo}{Example with addWall (the two transparent pink facets are normally invisible)}
\begin{luadraw}{name=addWall}
local g = graph3d:new{size={10,10},window={-8,8,-4,8}, margin={0,0,0,0}}
local C = cylinder(M(0,0,-1),5*vecK,2)
g:Dscene3d(
    g:addWall( {{Origin,vecI}, {Origin,vecJ}}),
    g:addPlane({Origin,vecI}, {color="Pink",opacity=0.3,scale=1.125,edge=true}), -- to show the first wall
    g:addPlane({Origin,vecJ}, {color="Pink",opacity=0.3,scale=1.125,edge=true}), -- to show the second wall
    g:addPoly( shift3d(C,M(-3,-3,1)), {color="Cyan"} ),
    g:addPoly( shift3d(C,M(-3,3,0.5)), {color="ForestGreen"} ),
    g:addPoly( shift3d(C,M(3,-3,-0.5)), {color="Crimson"} )
)
g:Show()
\end{luadraw}
\end{demo}

\paragraph{Notes on this example}:
\begin{itemize}
    \item with the two dividing walls, there are no cut facets, and the scene contains exactly 111 (37 per cylinder).
    \item without the dividing walls, there are 117 (useless) facet cuts, bringing their number to 228 in the scene.
    \item with the two dividing walls, and the \opt{backcull=true} option for each cylinder, there are no cut facets, and the scene contains only 57.
\end{itemize}

Here is another, much more convincing example where the use of dividing walls is essential to have a drawing of reasonable size. It involves obtaining a lemniscate as the intersection of a torus with a certain plane. Since the torus is non-convex, the number of unnecessary facet cuts can be very high.

\begin{demo}{Torus and lemniscate}
\begin{luadraw}{name=torus}
local g = graph3d:new{size={10,10}, margin={0,0,0,0}}
local cos, sin, pi = math.cos, math.sin, math.pi
local R, r = 2.5, 1
local x0 = R-r
local f = function(t) return M(0,R+r*cos(t),r*sin(t)) end
local plan = {M(x0,0,0),-vecI} -- plane whose section with the torus gives the lemniscate
local C, wall = rotcurve(f,-pi,pi,{Origin,vecK},360,0,{grid={25,37},addwall=2})
local C1 = cutfacet(C,plan)  -- part of the torus in the half-space containing -vecI
g:Dscene3d(
    g:addWall(plan), g:addWall(wall), -- addition of partition walls
    g:addFacet( C1, {color="Crimson", backcull=false}),
    g:addPlane(plan, {color="Pink",opacity=0.4,edge=true}), -- sectional plan
    g:addAxes( Origin, {arrows=1})
)
-- Cartesian equation of the torus : (x^2+y^2+z^2+R^2-r^2)^2-4*R^2*(x^2+y^2) = 0
-- the lemniscate therefore has the equation (x0^2+y^2+z^2+R^2-r^2)^2-4*R^2*(x0^2+y^2)=0 (implicit curve)
local h = function(y,z) return (x0^2+y^2+z^2+R^2-r^2)^2-4*R^2*(x0^2+y^2) end
local I = implicit(h,-4,4,-3,3,{50,50}) -- 2d polygonal line (list of lists of complex numbers)
local lemniscate = map(function(z) return M(x0,z.re,z.im) end, I[1]) -- conversion to 3d coordinates
g:Dpolyline3d(lemniscate,"Navy,line width=1.2pt")
g:Show()
\end{luadraw}
\end{demo}

\paragraph{Notes on this example}:
\begin{itemize}
    \item With the dividing walls, we have 30 facets that are cut and a tkz file of approximately 140 KB.
    \item Without the dividing walls, we have 2068 facet cuts (!) and a tkz file of approximately 550 KB.
    \item We could have used the cut section returned by the \emph{cutfacet} function, but the result is not very satisfactory (this is because the torus is non-convex). 
    \item If we hadn't wanted the axes passing through the torus and the cutting plane, we could have drawn the drawing with the \textbf{g:Dfacet} method, replacing the \emph{g:Dscene3d(...)} instruction with:
\begin{Luacode}
g:Dfacet(C1, {mode=mShadedOnly,color="Crimson"} )
g:Dfacet( g:Plane2facet(plan,0.75), {color="Pink",opacity=0.4})
\end{Luacode}
We get exactly the same thing but without the axes (and without facet cutting, of course).
\end{itemize}

\paragraph{To conclude this section}: we use the \textbf{g:Dscene3d()} method when it is not possible to do otherwise, for example when there are intersections (few) that cannot be handled "by hand". But this isn't the case for all intersections! In the following example, we represent a section of a sphere using a plane, but without using the \textbf{g:Dscene3d()} method, as this would require drawing a faceted sphere, which isn't very attractive. The trick here is to draw the sphere using the \textbf{g:Dsphere()} method, then draw a previously perforated facet over the plane, the hole corresponding to the outline (3D path) of the part of the sphere located above the plane:

\begin{demo}{Section of a sphere without Dscene3d()}
\begin{luadraw}{name=section_sphere}
local g = graph3d:new{ window3d={-4,4,-4,4,-4,4}, window={-5.5,5.5,-4,5}, viewdir={30,75}, size={10,10}}
local O, R = Origin, 2.5 -- center and radius
local S, P = sphere(O,R), {M(0,0,1.5),vecK+vecJ/2} -- the sphere and the section plane
local w, n = pt3d.normalize(P[2]), g.Normal -- unit vectors normal to P for w and to the screen for n
local I, r = interPS(P,{O,R}) -- center and radius of the small circle (intersection between the plane and the sphere)
local C = g:Intersection3d(S,P) -- It is a list of edges
local N = I-O
local J = I+r*pt3d.normalize(vecJ-vecK/2) -- a point on the small circle
local a = R/pt3d.abs(N)
local A, B = O+a*N, O-a*N -- points of intersection of the axis (O,I) with the sphere
local c1, alpha = Orange, 0.4
local coul = {c1[1]*alpha, c1[2]*alpha,c1[3]*alpha} -- to simulate transparency
g:Dhline( g:Proj3d({B,-N})) -- half-line (point B is not visible)
g:Dsphere(O,R,{mode=mBorder,color="orange"})
g:Dline3d(A,B,"dotted") -- dotted line (A,B)
g:Dedges(C, {hidden=true,hiddenstyle="dashed"}) -- drawing of the intersection
g:Dpolyline3d({I,J,O},"dashed") 
g:Dangle3d(O,I,J)  -- right angle
g:Dcrossdots3d({{B,N},{I,N},{O,N}},rgb(coul),0.75) -- points in the sphere
g:Dlabel3d("$O$", O, {pos="NW"})
local L = C.visible[1] -- visible part of the intersection (arc of a circle)
A1 = L[1]; A2 = L[#L] -- ends of L
local F = g:Plane2facet(P) -- plan converted to facet
-- hole plane as 3d path, the hole is the outline of the part of the sphere above the plane
insert(F,{"l","cl",A1,"m",I,A2,r,-1,w,"ca",Origin,A1,R,-1,n,"ca"})
g:Dpath3d( F,"fill=Beige,fill opacity=0.6") -- drawing of the perforated plan
g:Dhline( g:Proj3d({A,N})) -- half-line, upper part of the axis (AB)
g:Dcrossdots3d({A,N},"black",0.75); g:Dballdots3d(J,"black",0.75)
g:Dlabel3d("$A$", A, {pos="NW"}, "$I$", I, {}, "$B$", B, {pos="E"}, "$J$", J, {pos="S"})
g:Show()            
\end{luadraw}
\end{demo}
%
\section{Geometric Constructions}

This section groups together functions that construct geometric figures without dedicated graphics methods.

\subsection{Circumscribed circle, incircle: circumcircle3d(), incircle3d()}

\begin{itemize}
    \item The function \textbf{circumcircle3d(A,B,C)}, where $A$, $B$, and $C$ are three non-aligned 3D points, returns the circumcircle of the triangle formed by these three points, in the form of a sequence: $A,R,n$, where $A$ is the center of the circle, $R$ its radius, and $n$ a normal vector to the plane of the circle.     \item The function \textbf{incircle3d(A,B,C)}, where $A$, $B$, and $C$ are three non-aligned 3D points, returns the circle inscribed in the triangle formed by these three points, as a sequence: $A,R,n$, where $A$ is the center of the circle, $R$ its radius, and $n$ a normal vector to the plane of the circle.
\end{itemize}

\subsection{Convex Hull: cvx\_hull3d()}

The function \textbf{cvx\_hull3d(L)}, where $L$ is a list of \textbf{distinct} 3D points, calculates and returns the convex hull of $L$ as a list of facets.

\begin{demo}{Using cvx\_hull3d()}
\begin{luadraw}{name=cvx_hull3d}
local g = graph3d:new{window={-2,4,-6,1},bbox=false,size={10,10}}
local L = {Origin, 4*vecI, M(4,4,0), 4*vecJ}
insert(L, shift3d(L,-3*vecK))
insert(L, {M(2,1,2), M(2,3,2)})
local V = cvx_hull3d(L)
local P = facet2poly(V)
g:Dpoly(P , {color="cyan",mode=mShadedHidden})
g:Show()
\end{luadraw}
\end{demo}

\subsection{Planes: plane(), planeEq(), orthoframe(), plane2ABC()}

A plane in space is a table of the form $\{A,n\}$ where $A$ is a point in the plane (3d point) and $n$ is a normal vector to the plane (non-zero 3d point).
\begin{itemize}
    \item The function \textbf{plane(A,B,C)} returns the plane passing through the three 3d points $A$, $B$, and $C$ (if they are not aligned, otherwise the result is \emph{nil}).
    \item The function \textbf{planeEq(a,b,c,d)} returns the plane whose Cartesian equation is $ax+by+cz+d=0$ (if the coefficients $a$, $b$, and $c$ are not all zero, otherwise the result is \emph{nil}).
    \item The function \textbf{plane2ABC(P)}, where $P=\{A,n\}$ denotes a plane, returns a sequence of three 3d points $A,B,C$, belonging to the plane, and such that $(A,\vec{AB},\vec{AC})$ is a direct orthonormal frame of this plane.
    \item The function \textbf{orthoframe(P)}, where $P=\{A,n\}$ denotes a plane, returns a sequence of three 3d points $A,u,v$, such that $(A,u,v)$ is a direct orthonormal frame of this plane.
\end{itemize}

\begin{demo}{Faces of a cube with holes in it and a regular hexagon}
\begin{luadraw}{name=plans}
local g = graph3d:new{window={-3,3,-3.25,3.25},margin={0,0,0,0},viewdir={20,60},bg="LightGray",size={10,10}}
Hiddenlines = true; Hiddenlinestyle = "dashed"
local p = polyreg(0,1,6)
local P = parallelep(M(-2,-2,-2),4*vecI,4*vecJ,4*vecK)
local V = g:Sortpolyfacet(P)
local list = {}
g:Filloptions("full","Crimson",1,true); -- true pour le mode evenodd
g:Lineoptions("solid","Gold",8)
for _, F in  ipairs(V) do
    local P1 = plane(isobar3d(F),F[1],F[2]) -- plan de la facette F
    local A, u, v = orthoframe(P1)  -- repère orthonormé sur la facette avec centre de gravité comme origine
    local p1 = map(function(z) return A+z.re*u+z.im*v end,p) -- hexagone reproduit sur la facette
    table.insert(p1,2,"m")
    local color = "Crimson"
    if not g:Isvisible(F) then  color = "Crimson!60!black" end
    g:Dpath3d( concat(F,{"l"},p1,{"l","cl"}),"fill="..color ) -- dessin de la facette "trouée" avec l'hexagone
end
g:Show()
\end{luadraw}
\end{demo}

\subsection{Circumscribed Sphere, Inscribed Sphere: circumsphere(), insphere()}

\begin{itemize}
    \item The function \textbf{circumsphere(A,B,C,D)}, where $A$, $B$, $C$, and $D$ are four non-coplanar 3d points, returns the sphere circumscribed within the tetrahedron formed by these four points, as a sequence: $A,R$, where $A$ is the center of the sphere, and $R$ its radius.
    \item The function \textbf{insphere(A,B,C,D)}, where $A$, $B$, $C$, and $D$ are four non-coplanar 3d points, returns the sphere inscribed within the tetrahedron formed by these four points, as a sequence: $A,R$, where $A$ is the center of the sphere, and $R$ its radius.
\end{itemize}

\subsection{Fixed-length tetrahedron: tetra\_len()}

The function \textbf{tetra\_len(ab,ac,ad,bc,bd,cd)} calculates the vertices $A,B,C,D$ of a tetrahedron whose edge lengths are given, i.e., such that $AB=ab$, $AC=ac$, $AD=ad$, $BC=bc$, $BD=bd$, and $CD=cd$. The function returns the sequence of four points $A,B,C,D$. Vertex $A$ is always the point $M(0,0,0)$ (\emph{Origin}) and vertex $B$ is always the point \emph{ab*vecI} and vertex $C$ in the $xOy$ plane. The tetrahedron as a polyhedron can then be constructed with the function \textbf{tetra(A,B-A,C-A,D-A)}.

\begin{demo}{A tetrahedron with fixed edge lengths}
\begin{luadraw}{name=tetra_len}
local g = graph3d:new{window={-4,4,-4,4},margin={0,0,0,0},viewdir={25,65},size={10,10}}
Hiddenlines = true; Hiddenlinestyle = "dashed"
require 'luadraw_spherical'
local R = 4
local A,B,C,D = tetra_len(R,R,R,R,R,R)
local T = tetra(A,B-A,C-A,D-A)
g:Define_sphere({radius=R})
g:DSpolyline( facetedges(T), {color="DarkGreen"})
g:DSbigcircle( {B,C},{color="Blue"} )
g:DSbigcircle( {B,D},{color="Blue"} )
g:DSbigcircle( {C,D},{color="Blue"}  )
g:DSlabel("$R$",(2*A+C)/3,{pos="S"})
g:Dspherical()
g:Ddots3d({A,B,C,D})
g:Dlabel3d("$A$",A,{pos="S"},"$B$",B,{pos="SW"},"$C$",C,{},"$D$",D,{pos="N"} )
g:Show()
\end{luadraw}
\end{demo}

\subsection{Triangles: sss\_triangle3d(), sas\_triangle3d(), asa\_triangle3d()}

These functions are the 3D version of the sss\_triangle(), sas\_triangle(), and asa\_triangle() functions already described.
\begin{itemize}
    \item The function \textbf{sss\_triangle3d(ab,bc,ca)}, where \emph{ab}, \emph{bc}, and \emph{ca} are three lengths, computes and returns a list of three 3D points $\{A,B,C\}$ forming the vertices of a direct triangle in the $xOy$ plane, whose side lengths are the arguments, i.e., $AB=ab$, $BC=bc$, and $CA=ca$, when possible. Vertex $A$ is always point $M(0,0,0)$ (\emph{Origin}) and vertex $B$ is always point \emph{ab*vecI}. This triangle can be drawn with the method \textbf{g:Dpolyline3d}.
    \item The function \textbf{sas\_triangle3d(ab,alpha,ca)} where \emph{ab} and \emph{ca} are two lengths, \emph{alpha} an angle in degrees, computes and returns a list of three 3d points $\{A,B,C\}$ forming the vertices of a triangle in the plane $xOy$ such that $AB=ab$, $CA=ca$, and such that the angle $(\vec{AB},\vec{AC})$ has measure \emph{alpha}, when possible. Vertex $A$ is always point $M(0,0,0)$ (\emph{Origin}) and vertex $B$ is always point \emph{ab*vecI}. This triangle can be drawn with the method \textbf{g:Dpolyline3d}.
    \item The function \textbf{asa\_triangle3d(alpha,ab,beta)} where \emph{ab} is a length, \emph{alpha} and \emph{beta} are two angles in degrees, computes and returns a list of three 3d points $\{A,B,C\}$ forming the vertices of a triangle in the $xOy$ plane such that $AB=ab$, such that angle $(\vec{AB},\vec{AC})$ has measure \emph{alpha}, and such that angle $(\vec{BA},\vec{BC})$ has measure \emph{beta}, when possible. Vertex $A$ is always point $M(0,0,0)$ (\emph{Origin}) and vertex $B$ is always point \emph{ab*vecI}. This triangle can be drawn with the \textbf{g:Dpolyline3d} method.
\end{itemize}
%
\section{Matrix Calculus Transformations and Some Mathematical Functions}

\subsection{3D Transformations}

In the following functions:
\begin{itemize}
\item the argument \emph{L} is either a 3D point, a polyhedron, a list of 3D points (facet), or a list of lists of 3D points (facet list),
\item a line \emph{d} is a list of two 3D points \{A,u\}: a point on the line ($A$) and a direction vector ($u$),
\item a plane \emph{P} is a list of two 3D points \{A,n\}: a point on the plane ($A$) and a normal vector to the plane ($n$).
\end{itemize}
The returned result is of the same type as $L$.

\subsubsection{Apply a transformation function: ftransform3d}

The function \textbf{ftransform3d(L,f)} returns the image of \emph{L} by the function \emph{f}; this must be a function from $\mathbf R^3$ to $\mathbf R^3$.

\subsubsection{Projections: proj3d, proj3dO, dproj3d}

\begin{itemize}
\item The function \textbf{proj3d(L,P)} returns the image of $L$ by the orthogonal projection onto the plane $P$.
\item The function \textbf{proj3dO(L,P,v)} returns the image of $L$ by the projection onto the plane $P$ parallel to the direction of the vector $v$ (non-zero 3d point).
\item The function \textbf{dproj3d(L,d)} returns the image of $L$ by the projection onto the line $d$.
\end{itemize}

\subsubsection{Projections onto axes or planes related to axes}

\begin{itemize}
\item The function \textbf{pxy(L)} returns the image of $L$ by the orthogonal projection onto the $xOy$ plane.
\item The function \textbf{pyz(L)} returns the image of $L$ by the orthogonal projection onto the $yOz$ plane.
\item The function \textbf{pxz(L)} returns the image of $L$ by the orthogonal projection onto the $xOz$ plane.
\item The function \textbf{px(L)} returns the image of $L$ by the orthogonal projection onto the $Ox$ axis.
\item The function \textbf{py(L)} returns the image of $L$ by the orthogonal projection onto the $Oy$ axis.
\item The function \textbf{pz(L)} returns the image of $L$ by the orthogonal projection onto the $Oz$ axis.
\end{itemize}

\subsubsection{Symmetries: sym3d, sym3dO, dsym3d, psym3d}

\begin{itemize}
\item The function \textbf{sym3d(L,P)} returns the image of $L$ by the orthogonal symmetry about the $P$ plane.
\item The function \textbf{sym3dO(L,P,v)} returns the image of $L$ by the symmetry about the $P$ plane and parallel to the direction of the $v$ vector (non-zero 3d point).
\item The function \textbf{dsym3d(L,d)} returns the image of $L$ by the orthogonal symmetry with respect to the line $d$.
\item The function \textbf{psym3d(L,point)} returns the image of $L$ by the symmetry with respect to \emph{point} (3d point).
\end{itemize}

\subsubsection{Rotation: rotate3d, rotateaxe3d}

\begin{itemize}
\item The function \textbf{rotate3d(L,angle,d)} returns the image of $L$ rotated along axis $d$ (oriented by the direction vector, which is $d[2]$), and by \emph{angle} degrees.
\item The function \textbf{rotateaxe3d(L,v1,v2,center)} returns the image of $L$ rotated along axis passing through the 3d point \emph{center}, which transforms the vector \emph{v1} into the vector \emph{v2}. These vectors are normalized by the function. The argument \emph{center} is optional and defaults to the point \emph{Origin}.
\end{itemize}

\subsubsection{Scaling: scale3d}

The function \textbf{scale3d(L,k,center)} returns the image of $L$ by the scaling with center at the 3D point \emph{center}, and ratio \emph{k}. The argument \emph{center} is optional and is $M(0,0,0)$ by default (origin).

\subsubsection{Inversion: inv3d}

The function \textbf{inv3d(L,radius,center)} returns the image of $L$ by the inversion with respect to the sphere with center \emph{center}, and radius \emph{radius}. The argument \emph{center} is optional and is $M(0,0,0)$ by default (origin).

\subsubsection{Stereography: projstereo and inv\_projstereo}

Function \textbf{projstereo(L,S,N,h)}: the argument \emph{L} denotes a 3D point or a list of 3D points or a list of lists of 3D points, all belonging to the sphere \emph{S}, where \emph{S=\{C,r\}} ($C$ is the center of the sphere, and $r$ the radius). The argument \emph{N} denotes a point on the sphere that will be the pole of the projection. The argument \emph{h} is a real number that defines the projection plane. This plane is perpendicular to the axis $(CN)$, and passes through the point $I=C+h \frac{\vec{CN}}{CN}$ (with $h=0$ it is the equatorial plane, with $h=-r$ it is the plane tangent to the sphere at the opposite pole). The function returns the image of $L$ by the stereographic projection with respect to the sphere $S$ with $N$ as pole, and on the plane \emph{\{I,N-C\}}.

Inverse function \textbf{inv\_projstereo(L,S,N)}: \emph{S=\{C,r\}} is the sphere with center $C$ and radius $r$, \emph{N} is a point on the sphere $S$ (pole), and \emph{L} is a 3D point or a list of 3D points or a list of lists of 3D points all belonging to the same plane orthogonal to the $(CN)$ axis. The function returns the image of $L$ by the inverse of the stereographic projection with respect to $S$ and with pole $N$.

\subsubsection{Translation: shift3d}

The function \textbf{shift3d(L,v)} returns the image of $L$ by the translation of vector $v$ (3D point).


\subsection{Matrix Calculus}

If $f$ is an affine mapping of the space $\mathbf R^3$, we will call the list (table) of $f$ a matrix:
\begin{Luacode}
{ f(Origin), Lf(vecI), Lf(vecJ), Lf(vecK) }
\end{Luacode}
where $Lf$ denotes the linear part of $f$ (we have \emph{Lf(vecI) = f(vecI)-f(Origin)}, etc.). The identity matrix is ​​denoted \emph{ID3d} in the \emph{luadraw} package; it simply corresponds to the list \mintinline{Lua}{ {Origin,vecI,vecJ,vecK} }.

\subsubsection{applymatrix3d ​​and applyLmatrix3d}

\begin{itemize}
\item The function \textbf{applymatrix3d(A,M)} applies the matrix $M$ to the 3d point $A$ and returns the result (which is equivalent to calculating $f(A)$ if $M$ is the matrix of $f$). If $A$ is not a 3d point, the function returns $A$.

\item The function \textbf{applyLmatrix3d(A,M)} applies the linear part of the matrix $M$ to the 3d point $A$ and returns the result (which is equivalent to calculating $Lf(A)$ if $M$ is the matrix of $f$). If $A$ is not a 3d point, the function returns $A$.

\end{itemize}

\subsubsection{composematrix3d}
The function \textbf{composematrix3d(M1,M2)} performs the matrix product $M1\times M2$ and returns the result.

\subsubsection{invmatrix3d}
The function \textbf{invmatrix3d(M)} calculates and returns the inverse of the matrix $M$ when possible.

\subsubsection{matrix3dof}

The function \textbf{matrix3dof(f)} calculates and returns the matrix of $f$ (which must be an affine mapping of the space $\mathbf R^3$).

\subsubsection{mtransform3d and mLtransform3d}
\begin{itemize}
\item The function \textbf{mtransform3d(L,M)} applies the matrix $M$ to the list $L$ and returns the result. $L$ must be a list of 3D points (a facet) or a list of lists of 3D points (a list of facets).
\item The function \textbf{mLtransform3d(L,M)} applies the linear part of the matrix $M$ to the list $L$ and returns the result. $L$ must be a list of 3D points (a facet) or a list of lists of 3D points (a list of facets).
\end{itemize}

\subsection{Matrix associated with the 3D graph}

When creating a graph in the \emph{luadraw} environment, for example:
\begin{Luacode}
local g = graph3d:new{size={10,10}}
\end{Luacode}
The created \emph{g} object has a 3D transformation matrix that is initially the identity. All graphics methods automatically apply the graph's 3D transformation matrix. One caveat, however: the \emph{Dcylinder}, \emph{Dcone}, and \emph{Dsphere} methods only yield the correct result with the transformation matrix equal to the identity. To manipulate this matrix, the following methods are available.

\subsubsection{g:Composematrix3d()}
The \textbf{g:Composematrix3d(M)} method multiplies the 3d matrix of the graph \emph g by the matrix \emph{M} (with \emph{M} on the right), and the result is assigned to the graph's 3d matrix. The argument \emph{M} must therefore be a 3d matrix.

\subsubsection{g:Det3d()}
The \textbf{g:Det3d()} method returns $1$ when the 3d transformation matrix has a positive determinant, and $-1$ otherwise. This information is useful when we need to know whether the orientation of space has been changed or not.

\subsubsection{g:IDmatrix3d()}
The \textbf{g:IDmatrix3d()} method reassigns the identity to the 3d matrix of the graph \emph g.

\subsubsection{g:Mtransform3d()}
The \textbf{g:Mtransform3d(L)} method applies the 3d graph matrix of \emph g to \emph{L} and returns the result. The argument \emph L must be a list of 3d points (a facet) or a list of lists of 3d points (a list of facets).

\subsubsection{g:MLtransform3d()}
The \textbf{g:MLtransform3d(L)} method applies the linear part of the 3d matrix of the graph \emph g to \emph{L} and returns the result. The argument \emph L must be a list of 3D points (a facet) or a list of lists of 3D points (a list of facets).

\subsubsection{g:Rotate3d()}
The method \textbf{g:Rotate3d(angle,axis)} modifies the 3D transformation matrix of the graph \emph g by composing it with the rotation matrix of angle \emph{angle} (in degrees) and axis \emph{axis}.

\subsubsection{g:Scale3d()}
The method \textbf{g:Scale3d(factor, center)} modifies the 3D transformation matrix of the graph \emph g by composing it with the homothety matrix of ratio \emph{factor} and center \emph{center}. The argument \emph{center} is a 3D point that defaults to \emph{Origin}.

\subsubsection{g:Setmatrix3d()}
The \textbf{g:Setmatrix3d(M)} method assigns the matrix \emph M to the 3D transformation matrix of the graph \emph g.

\subsubsection{g:Shift3d()}
The \textbf{g:Shift3d(v)} method modifies the 3D transformation matrix of the graph \emph g by composing it with the translation matrix of vector \emph{v}, which must be a 3D point.


\subsection{Additional Mathematical Functions}

\subsubsection{clippolyline3d()}
The function \textbf{clippolyline3d(L, poly, exterior, close)} clips the 3D polygonal line \emph{L} to the \textbf{convex} polyhedron \emph{poly}. If the optional argument \emph{exterior} is true, then the part outside the polyhedron is returned (false by default). If the optional argument \emph{close} is true, then the polygonal line is closed (false by default). \emph{L} is a list of 3D points or a list of lists of 3D points.
\textbf{Note}: The result is not always satisfactory for the exterior part.

\paragraph{Special case}: Clipping a 3D polygonal line $L$ with the current 3D window can be done with this function as follows:

\begin{center}
\textbf{L = clippolyline3d(L, g:Box3d())}
\end{center}

Indeed, the \textbf{g:Box3d()} method returns the current 3D window as a parallelepiped.

\subsubsection{clipline3d()}
The function \textbf{clipline3d(line, poly)} clips the line \emph{line} with the \textbf{convex} polyhedron \emph{poly}; the function returns the part of the line inside the polyhedron. The argument \emph{line} is a table of the form \{A,u\} where $A$ is a point on the line and $u$ is a direction vector (two 3D points).

\paragraph{Special case}: Clipping a line $d$ with the current 3D window can be done with this function as follows:

\begin{center}
\textbf{d = clipline3d(d, g:Box3d())}
\end{center}

Indeed, the \textbf{g:Box3d()} method returns the current 3D window as a parallelepiped ($d$ then becomes a segment).

\subsubsection{cutpolyline3d()}
The function \textbf{cutpolyline3d(L,plane,close)} intersects the 3D polygonal line \emph{L} with the plane \emph{plane}. If the optional argument \emph{close} is true, then the line is closed (false by default). \emph{L} is a list of 3D points or a list of lists of 3D points, \emph{plane} is a table of the form \{A,n\} where $A$ is a point in the plane and $n$ is a normal vector (two 3D points).

The function returns three things:
\begin{itemize}
\item the part of \emph{L} that is in the half-space containing the vector $n$,
\item followed by the part of \emph{L} that is in the other half-space,
\item followed by the list of intersection points.
\end{itemize}

\subsubsection{getbounds3d()}
The function \textbf{getbounds3d(L)} returns the bounds xmin, xmax, ymin, ymax, zmin, zmax of the 3D polygonal line \emph{L} (list of 3D points or a list of lists of 3D points).

\subsubsection{interDP()}
The function \textbf{interDP(d,P)} calculates and returns (if it exists) the intersection between line $d$ and plane $P$.

\subsubsection{interPP()}
The function \textbf{interPP(P1,P2)} calculates and returns (if it exists) the intersection between planes $P_1$ and $P_2$.

\subsubsection{interDD()}
The function \textbf{interDD(D1,D2,epsilon)} calculates and returns (if it exists) the intersection between lines $D_1$ and $D_2$. The argument \emph{epsilon} is $10^{-10}$ by default (used to test whether a given float is zero).

\subsubsection{interDS()}
The function \textbf{interDS(d,S)} calculates and returns (if it exists) the intersection between the line $d$ and the sphere $S$ where $S$ is a table $S=\{C,r\}$ with $C$ as the center (3d point) and $r$ as the radius. The function returns either \emph{nil} (empty intersection), a single point, or two points.

\subsubsection{interPS()}
The function \textbf{interPS(P,S)} calculates and returns (if it exists) the intersection between the plane $P$ and the sphere $S$ where $S$ is a table $S=\{C,r\}$ with $C$ as the center (3d point) and $r$ as the radius. The function returns either \emph{nil} (empty intersection) or a sequence of the form $I,r,n$, where I is a 3D point representing the center of a circle, $r$ its radius, and $n$ a normal vector to the plane of the circle. This circle is the desired intersection.

\subsubsection{interSS()}
The function \textbf{interPS(S1,S2)} calculates and returns (if it exists) the intersection between the sphere $S1=\{C1,r1\}$ and $S2=\{C2,r2\}$. The function returns either \emph{nil} (empty intersection) or a sequence of the form $I,r,n$, where I is a 3D point representing the center of a circle, $r$ its radius, and $n$ a normal vector to the plane of the circle. This circle is the desired intersection.

\subsubsection{merge3d()}
The function \textbf{merge3d(L)} combines, if possible, the connected components of \emph{L}, which must be a list of lists of 3D points. The function returns the result.

\subsubsection{split\_points\_by\_visibility()}
The function \textbf{split\_points\_by\_visibility(L, visible\_function)}, where $L$ is a list of 3D points, or a list of lists of 3D points, and where \emph{visible\_function} is a function such that \emph{visible\_function(A)} returns \emph{true} if the 3D point $A$ is visible, \emph{false} otherwise, sorts the points of $L$ according to whether they are visible or not. The function returns a sequence of two tables: \emph{visible\_points}, \emph{hidden\_points}.

\begin{demo}{A curve on a cylinder}
\begin{luadraw}{name=curve_on_cylinder}
local g = graph3d:new{adjust2d=true,bbox=false,size={10,10}};
g:Labelsize("footnotesize")
Hiddenlines = true; Hiddenlinestyle = "dashed"

local curve_on_cylinder = function(curve,cylinder) 
-- curve is a 3d polyline on a cylinder, 
-- cylinder = {A,r,V,B}
    local  A,r,V,B = table.unpack(cylinder)
    if B == nil then B = V; V = B-A end
    local U = B-A
    local visible_function = function(N)
        local I = dproj3d(N,{A,U})
        return (pt3d.dot(N-I,g.Normal) >= 0)
    end
    return split_points_by_visibility(curve,visible_function)
end
-- test
local A, r, B = -5*vecJ, 4, 5*vecJ -- cylinder
local p = function(t) return Mc(r,t,t/3) end
local Curve = rotate3d( parametric3d(p,-4*math.pi,4*math.pi),90,{Origin,vecI})
local Vi, Hi = curve_on_cylinder(Curve,{A,r,B})
local curve_color = "DarkGreen"
g:Dboxaxes3d({grid=true,gridcolor="gray",fillcolor="LightGray"})
g:Dcylinder(A,r,B,{color="orange"})
g:Dpolyline3d(Vi,curve_color)
g:Dpolyline3d(Hi,curve_color..","..Hiddenlinestyle)
g:Show()
\end{luadraw}
\end{demo}
%
\section{More Advanced Examples}

\subsection{The Box of Sugars}

The problem\footnote{Problem posed in a forum, the objective being to use it as counting exercises for students.} is to draw sugars in a box. You need to be able to position the desired number of pieces, and wherever you want them in the box\footnote{A piece must rest either on the bottom of the box or on top of another piece} without having to rewrite the entire code. Another constraint: to keep the figure as light as possible, only the facets actually seen should be displayed. In the code below, we keep the default viewing angles, and:
\begin{itemize}
    \item the sugars are cubes of side 1 (we then modify the 3D matrix of the graph to "elongate" them),
    \item each piece is identified by the coordinates $(x,y,z)$ of the upper right corner of the front face, with $x$ an integer 1 and \emph{Lg}, $y$ an integer between 1 and \emph{lg}, and $z$ an integer between 1 and \emph{ht}.
    \item to store the positions of the pieces, we use a three-dimensional matrix \emph{positions}, one for $x$, one for $y$, and one for $z$, with the convention that \emph{positions[x][y][z]} is 1 if there is a sugar at position $(x,y,z)$, and 0 otherwise.     \item For each piece, there are at most three visible faces: the top one, the right one, and the front one.\footnote{Provided you don't change the viewing angles!}, but we only draw the top face if there isn't another sugar cube above it, we only draw the right face if there isn't another sugar cube to the right, and we only draw the front face if there isn't another sugar cube in front. This builds the list of facets actually seen.
    \item In the scene display, you must \textbf{put the box first}, otherwise its facets will be cut off by the planes of the sugar cube facets. The sugar cube facets cannot be cut off by the box because they are all inside.
\end{itemize}

\begin{demo}{Box of Sugar Cubes}
\begin{luadraw}{name=boite_sucres}
local g = graph3d:new{window={-9,8,-10,4},size={10,10}}
Hiddenlines = false
local Lg, lg, ht = 5, 4, 3 -- length, width, height (box size)
local positions = {} -- 3-dimensional matrix initialized with 0s
for L = 1, Lg do
    local X = {}
    for l = 1, lg do
        local Y = {}
        for h = 1, ht do table.insert(Y,0) end
        table.insert(X,Y)
    end
    table.insert(positions,X)
end
local facetList = function() -- returns the list of facets to draw (pay attention to the orientation)
    local facet = {}
    for x = 1, Lg do -- loop over the positions matrix
        for y = 1, lg do
            for z = 1, ht do
                if positions[x][y][z] == 1 then -- il y a un sucre en (x,y,z)
                    if (z == ht) or (positions[x][y][z+1] == 0) then -- no sugar on top so top side visible
                        table.insert(facet, {M(x,y,z),M(x-1,y,z),M(x-1,y-1,z),M(x,y-1,z)}) -- insert top face
                    end
                    if (y == lg) or (positions[x][y+1][z] == 0) then -- no sugar on the right so right side visible
                        table.insert(facet, {M(x,y,z),M(x,y,z-1),M(x-1,y,z-1),M(x-1,y,z)}) -- insert right face
                    end
                    if (x == Lg) or (positions[x+1][y][z] == 0) then -- no sugar in front so front side visible
                        table.insert(facet, {M(x,y,z),M(x,y-1,z),M(x,y-1,z-1),M(x,y,z-1)}) -- insert front face
                    end
                end
            end
        end
    end
    return facet
end
-- creation of the box (parallelepiped)
local O = Origin -0.1*M(1,1,1) -- so that the box does not stick to the sugars
local boite = parallelep(O, (Lg+0.2)*vecI, (lg+0.2)*vecJ, (ht+0.5)*vecK)
table.remove(boite.facets,2) -- we remove the top of the box, this is facet number 2
-- on positionne des sucres
for y = 1, 4 do for z = 1, 3 do  positions[1][y][z] = 1 end end
for x = 2, 5 do for z = 1, 2 do positions[x][1][z] = 1 end end
for z = 1, 3 do positions[5][3][z] = 1 end
for z = 1, 2 do positions[4][4][z] = 1 end
for z = 1, 2 do positions[3][4][z] = 1 end
positions[5][1][3] = 1; positions[3][1][3] = 1; positions[5][4][1] = 1; positions[2][3][1] = 1
g:Setmatrix3d({Origin,3*vecI,2*vecJ,vecK}) -- expansion on Ox and Oy to "lengthen" the cubes...
g:Dscene3d( -- dessin
    g:addPoly(boite,{color="brown",edge=true,opacity=0.9}),
    g:addFacet(facetList(), {backcull=true,contrast=0.25,edge=true})    )
g:Labelsize("huge"); g:Dlabel3d( "SUGAR", M(Lg/2+0.1,lg+0.1,ht/2+0.1), {dir={-vecI,vecK}})
g:Show()
\end{luadraw}
\end{demo}

\subsection{Stack of Cubes}

We can modify the previous example to draw a stack of randomly positioned cubes, with four views. We'll position the cubes by placing a random number per column, starting from the bottom. We'll create four views of the stack, adding axes to help us navigate between these different views. This slightly changes the search for potentially visible facets; there are five cases per cube, not just three (front, back, left, right, and top; we don't create bottom views). To make the stack more readable, we use three colors to paint the cube faces (two opposite faces have the same color).

\begin{demo}{Stack of Cubes}
\begin{luadraw}{name=cubes_empiles}
local g = graph3d:new{window3d={-6,6,-6,6,-6,6},size={10,10}}
Hiddenlines = false
local Lg, lg, ht, a = 5, 5, 5, 2 -- length, width, height of the space to fill, size of a cube
local positions = {} -- 3-dimensional matrix initialized with 0s
for L = 1, Lg do
    local X = {}
    for l = 1, lg do
        local Y = {}
        for h = 1, ht do table.insert(Y,0) end
        table.insert(X,Y)
    end
    table.insert(positions,X)
end
for x = 1, Lg do  -- random positioning of cubes
    for y = 1, lg do
        local nb = math.random(0,ht) -- we put number of cubes in the column (x,y,*) starting from the bottom
        for z = 1, nb do positions[x][y][z] = 1 end
    end
end
local dessus,gauche,devant = {},{},{} -- to memorize the facets
for x = 1, Lg do -- loop over the positions matrix to determine the facets to draw
    for y = 1, lg do
        for z = 1, ht do
            if positions[x][y][z] == 1 then -- il y a un cube en (x,y,z)
                if (z == ht) or (positions[x][y][z+1] == 0) then -- no cube above so face up
                    table.insert(dessus,{M(x,y,z),M(x-1,y,z),M(x-1,y-1,z),M(x,y-1,z)}) -- insert top face
                end
                if (y == lg) or (positions[x][y+1][z] == 0) then -- pas de cube à droite donc face  visible
                    table.insert(gauche,{M(x,y,z),M(x,y,z-1),M(x-1,y,z-1),M(x-1,y,z)}) -- insert right face
                end
                if (y == 1) or (positions[x][y-1][z] == 0) then -- no cube on the left so face up
                    table.insert(gauche,{M(x,y-1,z),M(x-1,y-1,z),M(x-1,y-1,z-1),M(x,y-1,z-1)}) -- insert left face
                end                    
                if (x == Lg) or (positions[x+1][y][z] == 0) then -- no cube in front so face up
                    table.insert(devant,{M(x,y,z),M(x,y-1,z),M(x,y-1,z-1),M(x,y,z-1)}) -- insert front face
                end
                if (x == 1) or (positions[x-1][y][z] == 0) then -- no cube behind so back face visible
                    table.insert(devant,{M(x-1,y,z),M(x-1,y,z-1),M(x-1,y-1,z-1),M(x-1,y-1,z)}) -- insert back face
                end
            end
        end
    end
end
g:Setmatrix3d({M(-a*Lg/2,-a*lg/2,-a*ht/2),a*vecI,a*vecJ,a*vecK}) -- to center the figure and have cubes of side a
local dessin = function()
    g:Dscene3d(
        g:addFacet(dessus, {backcull=true,color="Crimson"}), g:addFacet(gauche, {backcull=true,color="DarkGreen"}),
        g:addFacet(devant, {backcull=true,color="SteelBlue"}),
        g:addPolyline(facetedges(concat(dessus,gauche,devant))), -- dessin des arêtes
        g:addAxes(Origin,{arrows=1}))
end
g:Saveattr(); g:Viewport(-5,0,0,5); g:Coordsystem(-11,11,-11,11); g:Setviewdir(45,60) -- top left
 dessin(); g:Restoreattr()
g:Saveattr(); g:Viewport(0,5,0,5);g:Coordsystem(-11,11,-11,11); g:Setviewdir(-45,60) -- top right
dessin(); g:Restoreattr()
g:Saveattr(); g:Viewport(-5,0,-5,0);g:Coordsystem(-11,11,-11,11); g:Setviewdir(-135,60) -- bottom left
dessin(); g:Restoreattr()
g:Saveattr(); g:Viewport(0,5,-5,0);g:Coordsystem(-11,11,-11,11); g:Setviewdir(135,60) -- bottom right
dessin(); g:Restoreattr()
g:Show()
\end{luadraw}
\end{demo}


\subsection{Illustration of Dandelin's Theorem}

\begin{demo}{Illustration of Dandelin's Theorem}
\begin{luadraw}{name=Dandelin}
local g = graph3d:new{window3d={-5,5,-5,5,-5,5}, window={-5,5,-5,6}, bg="lightgray",viewdir={-10,85}}
g:Linewidth(8)
local sqrt = math.sqrt
local sqr = function(x) return x*x end
local L, a = 4.5, 2
local R = (a+5)*L/sqrt(100+L^2) --grosse sphère centre=M(0,0,a) rayon=R
local S2 = sphere(M(0,0,a),R,45,45)
local k = 0.35 --rapport d'homothetie
local b, r = (a+5)*k-5, k*R -- petite sphère centre=M(0,0,b) rayon=r
local S1 = sphere(M(0,0,b),r,45,45)
local c = (b+k*a)/(1+k)  --deuxieme centre d'homothetie
local z = a+sqr(R)/(c-a) --image de c par l'inversion par rapport à la grosse sphère
local M1 = M(0,sqrt(sqr(R)-sqr(z-a)),z)--point de la grosse sphère et du plan tangent
local N = M1-M(0,0,a) -- vecteur normal au plan tangent
local plan = {M(0,0,c),-N} -- plan tangent
local z2 = a+sqr(R)/(-5-a) --image du sommet par l'inversion par rapport à la grosse sphère
local z1 = b+sqr(r)/(-5-b) -- image du sommet par l'inversion par rapport à la petite sphère
local P2 = M(sqrt(R^2-(z2-a)^2),0,z2)
local P1= M(sqrt(r^2-(z1-b)^2),0,z1)
local S = M(0,0,-5)
local P = interDP({P1,P2-P1},plan)
local C = cone(M(0,0,-5),10*vecK,L,45,true)
local ellips = g:Intersection3d(C,plan)
local plan1 = {M(0,0,z1),vecK}
local plan2 = {M(0,0,z2),vecK}
local L1, L2 = g:Intersection3d(S1,plan1), g:Intersection3d(S2,plan2)
local F1, F2 = proj3d(M(0,0,b), plan), proj3d(M(0,0,a), plan)  --foyers
local s1, s2 = g:Proj3d(M(0,0,a)), g:Proj3d(M(0,0,b))
local V, H = g:Classifyfacet(C) -- on sépare facettes visibles et les autres
local V1, V2 = cutfacet(V,plan)
local H1, H2 = cutfacet(H,plan)
-- Dessin
g:Dpolyline3d( border(H2),"left color=white, right color=DarkSeaGreen, draw=none" ) -- faces non visibles sous le plan, remplissage seulement
g:Dsphere( M(0,0,b), r, {mode=mBorder,color="Orange"}) -- petite sphère
g:Dpolyline3d( border(V2),"left color=white, right color=DarkSeaGreen, fill opacity=0.4" ) -- faces visibles sous le plan
g:Dpolyline3d({S,P})  -- segment [S,P] qui est sous le plan en partie
g:Dfacet( g:Plane2facet(plan,0.75), {color="Chocolate", opacity=0.8}) -- le plan
g:Dpolyline3d( border(H1),"left color=white, right color=DarkSeaGreen,draw=none,fill opacity=0.7" ) -- contour faces non visibles au dessus du plan, remplissage seulement
g:Dsphere( M(0,0,a),R, {mode=2,color="SteelBlue"}) -- grosse sphère
g:Dpolyline3d( border(V1),"left color=white, right color=DarkSeaGreen, fill opacity=0.6" ) -- contour faces visibles au dessus du plan
g:Dcircle3d(M(0,0,5),L,vecK) -- ouverture du cône
g:Dpolyline3d({{P,F1},{F2,P,P2}})
g:Dedges(L1,{hidden=true,color="FireBrick"})
g:Dedges(L2,{hidden=true,color="FireBrick"})
g:Dedges(ellips,{hidden=true, color="blue"})
g:Dballdots3d({F1,F2,S,P1,P,P2},nil,0.75)
g:Dlabel3d(
  "$F_1$",F1,{pos="N"}, "$F_2$",F2,{}, "$N_2$",P2,{},"$S$",S,{pos="S"}, "$N_1$",P1,{pos="SE"}, "$P$",P,{pos="SE"} )
g:Show()
\end{luadraw}
\end{demo}

We want to draw a cone with a section through a plane and two spheres inside this cone (and tangent to the plane), but without drawing any spheres or faceted cones. The starting point, however, is the creation of these faceted solids, the spheres \emph{S1} and \emph{S2} (lines 11 and 8 of the listing) as well as the cone \emph{C} in line 23. The drawing principle is as follows:
\begin{enumerate}
    \item We separate the facets of the cone into two categories: the visible facets (facing the observer) and the others (variables \emph{V} and \emph{H} in line 30), which actually correspond to the front of the cone and the back of the cone.
    \item We divide the two lists of facets with the plane (lines 31 and 32). Thus, \emph{V1} corresponds to the front facets located above the plane and \emph{V2} corresponds to the front facets located below the plane (same thing with \emph{H1} and \emph{H2} for the back).
    \item We then draw the outline of \emph{H2} with a gradient fill (only) (line 34).
    \item We draw the small sphere (in orange, line 35).
    \item We draw the outline of \emph{V2} with a gradient fill and transparency to see the small sphere (line 36).
    \item We draw the segment $[S,P]$ (line 37) then the plane as a transparent facet (line 38).
    \item We draw the outline of \emph{H1} with a gradient fill (line 39). This is the back part above the plane.
    \item We draw the large sphere (line 40).
    \item Finally, we draw the outline of \emph{V1} with a gradient fill (line 41) and transparency to see the sphere (this is the front part of the cone above the plane), then the opening of the cone (line 42).
    \item We draw the intersections between the cone and the spheres (lines 44 and 45) as well as between the cone and the plane (line 46).
\end{enumerate}

\subsection{Volume defined by a double integral}
\begin{demo}{Volume correspondant à $\int_{x_1}^{x_2}\int_{y_1}^{y_2}f(x,y)dxdy$}
\begin{luadraw}{name=volume_integrale}
local i, pi, sin, cos = cpx.I, math.pi, math.sin, math.cos
local g = graph3d:new{window3d={-4,4,-4,4,0,6},adjust2d=true,margin={0,0,0,0},size={10,10}}
local x1, x2, y1, y2 = -3,3,-3,3 -- bornes
local f = function(x,y) return cos(x)+sin(y)+5 end -- fonction à intégrer
local p = function(u,v) return M(u,v,f(u,v)) end -- paramétrage surface z=f(x,y)
local Fx1 = concat({pxy(p(x1,y2)), pxy(p(x1,y1))}, parametric3d(function(t) return p(x1,t) end,y1,y2,25,false,0)[1])
local Fx2 = concat({pxy(p(x2,y1)), pxy(p(x2,y2))}, parametric3d(function(t) return p(x2,t) end,y2,y1,25,false,0)[1])
local Fy1 = concat({pxy(p(x1,y1)), pxy(p(x2,y1))}, parametric3d(function(t) return p(t,y1) end,x2,x1,25,false,0)[1])
local Fy2 = concat({pxy(p(x2,y2)), pxy(p(x1,y2))}, parametric3d(function(t) return p(t,y2) end,x1,x2,25,false,0)[1])
g:Dboxaxes3d({grid=true, gridcolor="gray",fillcolor="LightGray",labels=false})
g:Filloptions("fdiag","black"); g:Dpolyline3d( {M(x1,y1,0),M(x1,y2,0),M(x2,y2,0),M(x2,y1,0)}) -- dessous
g:Dfacet( {Fx1,Fx2,Fy1,Fy2},{mode=mShaded,opacity=0.7,color="Crimson"} )
g:Dfacet(surface(p,x1,x2,y1,y2), {mode=mShadedOnly,color="cyan"})
g:Dlabel3d("$x_1$", M(x1,4.75,0),{}, "$x_2$", M(x2,4.75,0),{}, "$y_1$", M(4.75,y1,0),{}, "$y_2$", M(4.75,y2,0),{}, "$0$",M(4,-4.75,0),{})  
g:Show()  
\end{luadraw}
\end{demo}

Here, the solid represented has lateral faces (Fx1, Fx2, Fy1, and Fy2) with one side being a parametric curve. We therefore take the points of this parametric curve (its first connected component) and add the projections of the two ends onto the $xOy$ plane. Care must be taken with the direction of travel so that the faces are correctly oriented (normal outward). This normal is calculated from the first three points of the face; it is best to start the face with the two projections onto the plane to be sure of the orientation.
We draw the bottom first, then the lateral faces, and finish with the surface.

\subsection{Volume defined on something other than a block}
\begin{demo}{Volume : $0\leqslant x\leqslant1;\ 0\leqslant y \leqslant x^2;\ 0\leqslant z\leqslant y^2$}
\begin{luadraw}{name=volume2}
local i = cpx.I
local g = graph3d:new{window3d={0,1,0,1,0,1}, margin={0,0,0,0},adjust2d=true,viewdir={170,40}, size={10,10}}
g:Labelsize("scriptsize")
local f = function(t) return M(t,t^2,0) end
local h = function(t) return M(1,t,t^2) end
local C = parametric3d(f,0,1,25,false,0)[1] -- courbe y=x^2 dans le plan z=0 (première composante connexe)
local D = parametric3d(h,1,0,25,false,0)[1] -- courbe z=y^2 dans le plan x=1, en sens inverse
local dessous = concat({M(1,0,0)},C) -- forme la face du dessous
local arriere = concat({M(1,1,0)},D) -- forme la face arrière
local  avant, dessus, A, B = {}, {}, nil, C[1]
for k = 2, #C do --on construit les faces avant et de dessus facette par facette, en partant des points de C
    A = B; B = C[k]
    table.insert(avant, {B,A,M(A.x,A.y,A.y^2),M(B.x,B.y,B.y^2)})
    table.insert(dessus, {M(B.x,B.y,B.y^2),M(A.x,A.y,A.y^2),M(1,A.y,A.y^2),M(1,B.y,B.y^2)})
end
g:Dboxaxes3d({grid=true, gridcolor="gray",fillcolor="LightGray", drawbox=false, 
    xyzstep=0.25, zlabelstyle="W",zlabelsep=0})
g:Lineoptions(nil,"Navy",8)  
g:Dpolyline3d(arriere,close,"fill=Crimson, fill opacity=0.6") -- face arrière (plane)
g:Filloptions("fdiag","black"); g:Dpolyline3d(dessous,close) -- dessous
g:Dmixfacet(avant,{color="Crimson",opacity=0.7,mode=mShadedOnly}, dessus,{color="cyan",opacity=1})
g:Filloptions("none"); g:Dpolyline3d(concat(border(avant),border(dessus)))
g:Show() 
\end{luadraw}
\end{demo}

In this example, the surface has the equation $z=y^2$ (parabolic cylinder), but we are no longer on a block. The front face is not flat; we construct it like a cylinder (line 14) with vertical facets resting on curve $C$ at the bottom, and on curve $t\mapsto M(t,t^2,t^4)$ at the top.

Similarly, the top face (the surface) is constructed like a horizontal cylinder resting on curves $D$ and $t\mapsto M(t,t^2,t^4)$.

We could not construct the surface by hand (called \emph{top} in the code), and instead draw the following surface (after the front face):
\begin{Luacode}
g:Dfacet( surface(function(u,v) return M(u,v*u^2,v^2*u^4) end, 0,1,0,1), {mode=mShadedOnly, color="cyan"})
\end{Luacode}
but it has many more facets (25*25) than the cylinder-shaped construction (21 facets), which is less interesting.
%
\section{Extensions}

\subsection{The \emph{luadraw\_polyhedrons} module}

This module is still in draft form and is expected to be expanded in the future. As its name suggests, it contains the definition of polyhedra. All numerical data comes from the \href{https://dmccooey.com/polyhedra/}{Visual Polyhedra} website.

All functions follow the same model: \textbf{<name>(C,S,all)} where $C$ is the center of the polyhedron (3D point) and $S$ is a vertex of the polyhedron (3D point). When $C$ or $S$ have the value \emph{nil}, the untransformed polyhedron (centered at the origin) is returned. The optional argument \emph{all} is a boolean. When it has the value \emph{true}, the function returns four things: \emph{P, V, E, F} where:
\begin{itemize}
    \item $P$ is the solid as a polyhedron,
    \item $V$ the list (table) of vertices,
    \item $E$ the list (table) of edges (with 3D points),
    \item $F$ the list of facets (with 3D points). Some polyhedra have multiple facet types; in this case, the returned result is of the form: \emph{P, V, E, F1, F2, ...}, where $F1$, $F2$, ... are lists of facets. This can allow them to be drawn with different colors, for example. \end{itemize}
The argument \emph{all} is set to \emph{false}, which is the default value; the function only returns the polyhedron.

Here are the solids currently contained in this module:

\begin{itemize}
    \item The Platonic solids, these solids have only one face type:
\begin{itemize}
    \item The function \textbf{tetrahedron(C,S,all)} allows the construction of a regular tetrahedron with center $C$ (3d point) and one vertex at $S$ (3d point).
    \item The function \textbf{octahedron(C,S,all)} allows the construction of an octahedron with center $C$ (3d point) and one vertex at $S$ (3d point).
    \item The function \textbf{cube(C,S,all)} allows the construction of a cube with center $C$ (3d point) and one vertex at $S$ (3d point).
    \item The function \textbf{icosahedron(C,S,all)} allows the construction of an icosahedron with center $C$ (3d point) and one vertex at $S$ (3d point).
    \item The function \textbf{dodecahedron(C,S,all)} allows the construction of a dodecahedron with center $C$ (3d point) and one vertex at $S$ (3d point).
\end{itemize}

    \item The Archimedean Solids:
\begin{itemize}
    \item The function \textbf{cuboctahedron(C,S,all)} allows the construction of a cuboctahedron with center $C$ (3d point) and one vertex at $S$ (3d point). This solid has two types of faces.
    \item The function \textbf{icosidodecahedron(C,S,all)} allows the construction of an icosidodecahedron with center $C$ (3d point) and one vertex at $S$ (3d point). This solid has two types of faces.
    \item The function \textbf{lsnubcube(C,S,all)} allows the construction of a snub cube (form 1) with center $C$ (3d point) and one vertex at $S$ (3d point). This solid has two types of faces.
    \item The function \textbf{lsnubdodecahedron(C,S,all)} allows the construction of a snub dodecahedron (form 1) with center $C$ (3d point) and one vertex at $S$ (3d point). This solid has two types of faces.
    \item The function \textbf{rhombicosidodecahedron(C,S,all)} allows the construction of a rhombicosidodecahedron with center $C$ (3d point) and one vertex at $S$ (3d point). This solid has three types of faces.
    \item The function \textbf{rhombicuboctahedron(C,S,all)} allows the construction of a rhombicuboctahedron with center $C$ (3d point) and one vertex at $S$ (3d point). This solid has two types of faces.
    \item The function \textbf{rsnubcube(C,S,all)} allows the construction of a snub cube (shape 2) with center $C$ (3d point) and one vertex at $S$ (3d point). This solid has two types of faces.
    \item The function \textbf{rsnubdodecahedron(C,S,all)} allows the construction of a snub dodecahedron (shape 2) with center $C$ (3d point) and one vertex at $S$ (3d point). This solid has two types of faces.
    \item The function \textbf{truncatedcube(C,S,all)} allows the construction of a truncated cube with center $C$ (3d point) and one vertex at $S$ (3d point). This solid has two types of faces.
    \item The function \textbf{truncatedcuboctahedron(C,S,all)} allows the construction of a truncated cuboctahedron with center $C$ (3d point) and one vertex at $S$ (3d point). This solid has three types of faces.
    \item The function \textbf{truncateddodecahedron(C,S,all)} allows the construction of a truncated dodecahedron with center $C$ (3d point) and one vertex at $S$ (3d point). This solid has two types of faces.
    \item The function \textbf{truncatedicosahedron(C,S,all)} allows the construction of a truncated icosahedron with center $C$ (3d point) and one vertex at $S$ (3d point). This solid has two types of faces.
    \item The function \textbf{truncatedicosidodecahedron(C,S,all)} allows the construction of a truncated icosidodecahedron with center $C$ (3d point) and one vertex at $S$ (3d point). This solid has two threes of faces.
    \item The function \textbf{truncatedoctahedron(C,S,all)} allows the construction of a truncated octahedron with center $C$ (3d point) and one vertex at $S$ (3d point). This solid has two types of faces.
    \item The function \textbf{truncatedtetrahedron(C,S,all)} allows the construction of a truncated tetrahedron with center $C$ (3d point) and one vertex at $S$ (3d point). This solid has two types of faces.
\end{itemize}

    \item Other solids:
\begin{itemize}
    \item The function \textbf{octahemioctahedron(C,S,all)} allows the construction of an octahemioctahedron with center $C$ (3d point) and one vertex at $S$ (3d point). This solid has two types of faces.
    \item The function \textbf{small\_stellated\_dodecahedron(C,S,all)} allows the construction of a small stellated dodecahedron with center $C$ (3d point) and one vertex at $S$ (3d point). This solid has only one type of face.

\end{itemize}
\end{itemize}

\begin{demo}{Polyhedra from the \emph{luadraw\_polyhedrons} module}
\begin{luadraw}{name=polyhedrons}
local i = cpx.I
require 'luadraw_polyhedrons' -- chargement du module
local g = graph3d:new{bg="LightGray", size={10,10}}
g:Labelsize("small"); Hiddenlines = false
-- en haut à gauche 
g:Saveattr(); g:Viewport(-5,0,0,5); g:Coordsystem(-5,5,-5,5,true)
local T,S,A,F = icosahedron(Origin,M(0,2,4.5),true) 
g:Dscene3d(
    g:addFacet(F, {color="Crimson",opacity=0.8}),
    g:addPolyline(A, {color="Pink", width=8}),
    g:addDots(S) )
g:Dlabel("Icosaèdre",5*i,{})
g:Restoreattr()
-- en haut à droite
g:Saveattr()
g:Viewport(0,5,0,5); g:Coordsystem(-5,5,-5,5,true)
local T,S,A,F1,F2 = truncatedtetrahedron(Origin,M(0,0,5),true) -- sortie complète, affichage dans une scène 3d
g:Dscene3d(
    g:addFacet(F1, {color="Crimson",opacity=0.8}),
    g:addFacet(F2, {color="Gold"}),
    g:addPolyline(A, {color="Pink", width=8}),
    g:addDots(S) )
g:Dlabel("Tétraèdre tronqué",5*i,{})
g:Restoreattr()
-- en bas à gauche
g:Saveattr(); g:Viewport(-5,0,-5,0); g:Coordsystem(-5,5,-5,5,true)
local T,S,A,F1,F2,F3 = rhombicosidodecahedron(Origin,M(0,0,4.5),true)
g:Dscene3d(
    g:addFacet(F1, {color="Crimson",opacity=0.8}),
    g:addFacet(F2, {color="Gold",opacity=0.8}), g:addFacet(F3, {color="ForestGreen"}),
    g:addPolyline(A, {color="Pink", width=8}), g:addDots(S) )
g:Dlabel("Rhombicosidodécaèdre",-5*i,{})
g:Restoreattr()
-- en bas à droite
g:Saveattr(); g:Viewport(0,5,-5,0); g:Coordsystem(-5,5,-5,5,true)
local T,S,A,F1 = small_stellated_dodecahedron(Origin,M(0,0,5),true)
g:Dscene3d(
    g:addFacet(F1, {color="Crimson",opacity=0.8}),
    g:addPolyline(A, {color="Pink", width=8}),
    g:addDots(S) )
g:Dlabel("Petit dodécaèdre étoilé",-5*i,{})
g:Restoreattr()
g:Show()
\end{luadraw}
\end{demo}

\subsection{The \emph{luadraw\_spherical}} Module

This module allows you to draw a number of objects on a sphere (such as circles, spherical triangles, etc.) without having to manually manage the visible or invisible parts. Drawing is done in three steps:
\begin{enumerate}
    \item We define the characteristics of the sphere (center, radius, color, etc.)
    \item We define the objects to be added to the scene using dedicated methods.
    \item We display everything with the \textbf{g:Dspherical()} method.
\end{enumerate}
Of course, all 2D and 3D drawing methods remain usable.

\subsubsection{Global Module Variables and Functions}

\begin{itemize}
    \item Variables with their default values:
\begin{itemize}
    \item \textbf{Insidelabelcolor} = "DarkGray": Defines the color of labels whose anchor point is inside the sphere.
    \item \textbf{arrowBstyle} = "->": Type of arrow at the end of the line
    \item \textbf{arrowAstyle} = "<-": Type of arrow at the beginning of the line
    \item \textbf{arrowABstyle} = "<->": Very rarely used because most of the time the lines drawn on the sphere must be cut. \end{itemize}
    \item Functions:
\begin{itemize}
    \item \textbf{sM(x,y,z)}: returns a point on the sphere, point $I$ of the sphere such that the half-line $[O,I)$ ($O$ being the center of the sphere) passes through point $A$ with Cartesian coordinates $(x,y,z)$. The numbers $x$, $y$, and $z$ must not be zero simultaneously.
    \item \textbf{sM(theta,phi)}: where \emph{theta} and \emph{phi} are angles in degrees, returns a point on the sphere, whose spherical coordinates are \emph{(R,theta,phi)} where $R$ is the radius of the sphere.     \item \textbf{toSphere(A)}: Returns the same point on the sphere as \emph{Ms(A.x,A.y,A.z)}.
    \item \textbf{clear\_spherical()}: Removes objects that have been added to the scene, and resets the values ​​to their default values.
\end{itemize}
\end{itemize}

If the global variable \textbf{Hiddenlines} is set to \emph{true}, then the hidden parts will be drawn in the style defined by the global variable \textbf{Hiddenlinestyle}. However, this behavior can be modified using the local option \emph{hidden=true/false}.

\subsubsection{Sphere Definition}
By default, the sphere is centered at the origin, has a radius of $3$, and is orange, but this can be modified with the \textbf{g:Define\_sphere( options )} method, where \emph{options} is a table allowing you to adjust each parameter. These are as follows (with their default values ​​in parentheses):
\begin{itemize}
    \item \opt{center =} (Origin),
    \item \opt{radius =} (3),
    \item \opt{color =} ("Orange"),
    \item \opt{opacity =} (1),
    \item \opt{mode =} (\emph{mBorder}), sphere display mode (\emph{mWireframe} or \emph{mGrid} or \emph{mBorder}, see \textbf{Dsphere}),
    \item \opt{edgecolor =} ("LightGray"),
    \item \opt{edgestyle =} ("solid"),
    \item \opt{hiddenstyle =} (Hiddenlinestyle),
    \item \opt{hiddencolor =} ("gray"),
    \item \opt{edgewidth =} (4),
    \item \opt{show =} (true), to show or hide the sphere.
\end{itemize}

\subsubsection{Add a circle: g:DScircle}

The \textbf{g:DScircle(P,options)} method allows you to add a circle to the sphere. The argument \emph{P} is a table of the form $\{A,n\}$ that represents a plane (passing through $A$ and normal to $n$, two 3D points). The circle is then defined as the intersection of this plane with the sphere. The \emph{options} argument is a table with 5 fields, which are:
\begin{itemize}
    \item \opt{style =} (current line style),
    \item \opt{color =} (current line color),
    \item \opt{width =} (current line thickness in tenths of a point),
    \item \opt{opacity =} (current line opacity),
    \item \opt{hidden =} (value of \emph{Hiddenlines}),
    \item \opt{out =} (nil), if we assign a list variable to this \emph{out} parameter, then the function adds to this list the two points corresponding to the ends of the hidden arc, if any, which allows us to retrieve them without having to calculate them. \end{itemize}

\subsubsection{Add a great circle: g:DSbigcircle}

The method \textbf{g:DSbigcircle(AB,options)} adds a great circle to the sphere. The argument \emph{AB} is a table of the form $\{A,B\}$ where $A$ and $B$ are two distinct points on the sphere. The great circle is then the circle centered at the center of the sphere, and passing through $A$ and $B$. The \emph{options} argument is a table with 5 fields, which are:
\begin{itemize}
    \item \opt{style =} (current line style),
    \item \opt{color =} (current line color),
    \item \opt{width =} (current line thickness in tenths of a point),
    \item \opt{opacity =} (current line opacity),
    \item \opt{hidden =} (value of \emph{Hiddenlines}),
    \item \opt{out =} (nil), if we assign a table-type variable to this \emph{out} parameter, then the function adds to this list the two points corresponding to the endpoints of the hidden arc, if any, which allows us to retrieve them without having to calculate them. \end{itemize}

\subsubsection{Add a great circle arc: g:DSarc}

The method \textbf{g:DSarc(AB,sens,options)} allows you to add a great circle arc to the sphere. The argument \emph{AB} is a table of the form $\{A,B\}$ where $A$ and $B$ are two distinct points on the sphere. The great circle arc is then drawn from $A$ to $B$. The argument \emph{sens} is equal to 1 or -1 to indicate the direction of the arc. When $A$ and $B$ are not diametrically opposed, the plane $OAB$ (where $O$ is the center of the sphere) is oriented with $\vec{OA}\wedge\vec{OB}$. The \emph{options} argument is a table with 6 fields, which are:
\begin{itemize}
    \item \opt{style =} (current line style),
    \item \opt{color =} (current line color),
    \item \opt{width =} (current line thickness in tenths of a point),
    \item \opt{opacity =} (current line opacity),
    \item \opt{hidden =} (value of \emph{Hiddenlines}),
    \item \opt{arrows =} (0), three possible values: 0 (no arrow), 1 (one arrow at $B$), 2 (arrow at $A$ and $B$).
    \item \opt{normal =} (nil), allows you to specify a normal vector to the $OAB$ plane when these three points are aligned.
\end{itemize}

\subsubsection{Add an angle: g:DSangle}

The method \textbf{g:DSangle(B,A,C,r,sens,options)}, where $A$, $B$, and $C$ are three points on the sphere, allows you to draw a great circle arc on the sphere to represent the angle $(\vec{AB},\vec{AC})$ with a radius of \emph{r}. The argument \emph{sens} is 1 or -1 to indicate the direction of the arc; the plane $ABC$ is oriented with $\vec{AB}\wedge\vec{AC}$. The \emph{options} argument is a table with 6 fields, which are:
\begin{itemize}
    \item \opt{style =} (current line style),
    \item \opt{color =} (current line color),
    \item \opt{width =} (current line thickness in tenths of a point),
    \item \opt{opacity =} (current line opacity),
    \item \opt{hidden =} (value of \emph{Hiddenlines}),
    \item \opt{arrows =} (0), three possible values: 0 (no arrow), 1 (one arrow at $B$), 2 (arrow at $A$ and $B$).
    \item \opt{normal =} (nil), allows you to specify a normal vector to the $ABC$ plane when these three points are "aligned" on the same great circle. \end{itemize}

\subsubsection{Add a spherical facet: g:DSfacet}

The method \textbf{g:DSfacet(F,options)}, where \emph{F} is a list of points on the sphere, allows you to draw the facet represented by $F$, the edges being great circle arcs. The \emph{options} argument is a table with 6 fields, which are:
\begin{itemize}
    \item \opt{style =} (current line style),
    \item \opt{color =} (current line color),
    \item \opt{width =} (current line thickness in tenths of a point),
    \item \opt{opacity =} (current line opacity),
    \item \opt{hidden =} (value of \emph{Hiddenlines}),
    \item \opt{fill =} (""), string representing the fill color (none by default),
    \item \opt{fillopacity =} (0.3), opacity of the fill color. \end{itemize}

\subsubsection{Add a spherical curve: g:DScurve}

The method \textbf{g:DScurve(L,options)}, where \emph{L} is a list of points on the sphere, allows you to draw the curve represented by $L$. The \emph{options} argument is a table with six fields, which are:
\begin{itemize}
    \item \opt{style =} (current line style),
    \item \opt{color =} (current line color),
    \item \opt{width =} (current line thickness in tenths of a point),
    \item \opt{opacity =} (current line opacity),
    \item \opt{hidden =} (value of \emph{Hiddenlines}),
    \item \opt{out =} (nil). If we assign a table-type variable to this \emph{out} parameter, then the function adds the points corresponding to the ends of the hidden parts to this list.
\end{itemize}

We will now deal with objects that are not necessarily on the sphere, but that may pass through it, or be inside it, or outside it.

\subsubsection{ Add a segment: g:DSseg}

The \textbf{g:DSseg(AB,options)} method allows you to add a segment. The argument \emph{AB} is a table of the form $\{A,B\}$ where $A$ and $B$ are two points in space. The function handles interactions with the sphere. The \emph{options} argument is a table with 5 fields, which are:
\begin{itemize}
    \item \opt{style =} (current line style),
    \item \opt{color =} (current line color),
    \item \opt{width =} (current line thickness in tenths of a point),
    \item \opt{opacity =} (current line opacity),
    \item \opt{hidden =} (value of \emph{Hiddenlines}),
    \item \opt{arrows =} (0), three possible values: 0 (no arrow), 1 (one arrow in $B$), 2 (arrow in $A$ and $B$).
\end{itemize}

\subsubsection{Add a line: g:DSline}

The \textbf{g:DSline(d,options)} method allows you to add a line. The argument \emph{d} is a table of the form $\{A,u\}$ where $A$ is a point on the line and $u$ is a direction vector (two 3D points). The function handles interactions with the sphere. The drawn segment is obtained by intersecting the line with the 3D window; it may be empty if the window is too narrow. The \emph{options} argument is a table with 6 fields, which are:
\begin{itemize}
    \item \opt{style =} (current line style),
    \item \opt{color =} (current line color),
    \item \opt{width =} (current line thickness in tenths of a point),
    \item \opt{opacity =} (current line opacity),
    \item \opt{hidden =} (value of \emph{Hiddenlines}),
    \item \opt{arrows =} (0), three possible values: 0 (no arrow), 1 (one arrow at $B$), 2 (arrow at $A$ and $B$),
    \item \opt{scale =} (1), allows you to change the size of the plotted segment.
\end{itemize}

    
\subsubsection{ Add a polygonal line: g:DSpolyline}

The \textbf{g:DSpolyline(L,options)} method allows you to add a polygonal line. The argument \emph{L} is a list of points in space, or a list of lists of points in space. The function handles interactions with the sphere. The \emph{options} argument is a table with 6 fields, which are:
\begin{itemize}
    \item \opt{style =} (current line style),
    \item \opt{color =} (current line color),
    \item \opt{width =} (current line thickness in tenths of a point),
    \item \opt{opacity =} (current line opacity),
    \item \opt{hidden =} (value of \emph{Hiddenlines}),
    \item \opt{arrows =} (0), three possible values: 0 (no arrow), 1 (one arrow at $B$), 2 (arrow at $A$ and $B$),
    \item \opt{close =} (false), indicates whether the line should be closed. \end{itemize}

\subsubsection{Add a plane: g:DSplane}

The \textbf{g:DSplane(P,options)} method allows you to add the contour of a plane. The argument \emph{P} is a table of the form \emph{\{A,n\}}, where $A$ is a point on the plane and $n$ is a normal vector. The function draws a parallelogram representing the plane $P$, processing the interactions with the sphere. The \emph{options} argument is a table with 7 fields, which are:
\begin{itemize}
    \item \opt{style =} (current line style),
    \item \opt{color =} (current line color),
    \item \opt{width =} (current line thickness in tenths of a point),
    \item \opt{opacity =} (current line opacity),
    \item \opt{hidden =} (value of \emph{Hiddenlines}),
    \item \opt{scale =} (1), allows you to change the size of the parallelogram,
    \item \opt{angle =} (0), angle in degrees, allows you to rotate the parallelogram around the perpendicular line passing through the center of the sphere.
    \item \opt{trace =} (true), allows you to draw, or not, the intersection of the plane with the sphere when it is not empty. \end{itemize}

\subsubsection{Add a label: g:DSlabel}

The \textbf{g:DSlabel(text1,anchor1,options1,text2,anchor2,options2,...)} method allows you to add one or more labels using the same principle as the \emph{g:Dlabel3d} method, except that here the function handles cases where the anchor point is inside the sphere, behind the sphere, or in front of the sphere. When it is inside, the label color is given by the global variable \textbf{Insidelabelcolor}, which defaults to \emph{"DrakGray"}.

\subsubsection{Add points: g:DSdots and g:DSstars}

The \textbf{g:DSdots(dots,options)} method allows you to add points to the scene. The \emph{dots} argument is a list of 3D points. The function draws points by managing interactions with the sphere. The \emph{options} argument is a two-field table, which are:
\begin{itemize}
    \item \opt{hidden =} (value of \emph{Hiddenlines}),
    \item \opt{mark\_options =} (""), a string that will be passed directly to the \emph{\textbackslash draw} instruction.
\end{itemize}
If a point is inside the sphere, or on the hidden face, the point's color is given by the global variable \textbf{Insidelabelcolor}, which defaults to \emph{"DrakGray"}.

The \textbf{g:DSstars(dots,options)} method allows you to add points to the sphere. The \emph{dots} argument is a list of 3D points that will be projected onto the sphere. The function draws these points as an asterisk. The \emph{options} argument is a two-field table, which are:
\begin{itemize}
    \item \opt{style =} (current line style),
    \item \opt{color =} (current line color),
    \item \opt{width =} (current line thickness in tenths of a point),
    \item \opt{opacity =} (current line opacity),
    \item \opt{hidden =} (value of \emph{Hiddenlines}),
    \item \opt{scale =} (1), allows you to change the size of the parallelogram,
    \item \opt{circled =} (false), allows you to add a circle around the star,
    \item \opt{fill =} (""), string representing a color. When not empty, the asterisk is replaced by a circled hexagonal facet and filled with the color specified by this option. \end{itemize}
The points on the hidden face of the sphere have the color given by the global variable \textbf{Insidelabelcolor}, which defaults to \emph{"DrakGray"}.

\subsubsection{Inverse Stereography: g:DSinvstereo\_curve and g:DSinvstereo\_polyline}

The method \textbf{g:DSinvstereo\_curve(L,options)}, where \emph{L} is a 3D polygonal line representing a curve drawn on a plane with equation $z =$cte, draws the image of $L$ on the sphere by inverse stereography, the pole being the point \emph{C+r*vecK}, where $C$ is the center of the sphere and $r$ is the radius.

The method \textbf{g:DSinvstereo\_polyline(L,options)}, where \emph{L} is a 3D polygonal line drawn on a plane with equation $z =$cte, draws the image of $L$ on the sphere by inverse stereography, the pole being the point \emph{C+r*vecK}, where $C$ is the center of the sphere and $r$ is the radius.

In both cases, the \emph{options} are the same as for the \textbf{g:DScurve} method.

\subsubsection{Examples}

\begin{demo}{Cube in a Sphere}
\begin{luadraw}{name=cube_in_sphere}
local g = graph3d:new{window={-9,9,-4,5},viewdir={25,70},size={16,8}}
require 'luadraw_spherical'
arrowBstyle = "-stealth"
g:Linewidth(6); Hiddenlinestyle = "dashed"
local a = 4
local O = Origin
local cube = parallelep(O,a*vecI,a*vecJ,a*vecK)
local G = isobar3d(cube.vertices)
cube = shift3d(cube,-G) -- to center the cube at the origin
local R = pt3d.abs(cube.vertices[1])

local dessin = function()
    g:DSpolyline({{O,5*vecI},{O,5*vecJ},{O,5*vecK}},{arrows=1, width=8}) -- axes
    g:DSplane({a/2*vecK,vecK},{color="blue",scale=0.9,angle=20}); 
    g:DScircle({-a/2*vecK,vecK},{color="blue"})
    g:DSpolyline( facetedges(cube) ); g:DSlabel("$O$",O,{pos="W"})
    g:Dspherical()
end

g:Saveattr(); g:Viewport(-9,0,-4,5); g:Coordsystem(-5,5,-5,5)
Hiddenlines = true; g:Define_sphere({radius=R})
dessin()
g:Dlabel3d("$x$",5*vecI,{pos="SW"},"$y$",5*vecJ,{pos="E"},"$z$",5*vecK,{pos="N"})
g:Dlabel("Hiddenlines=true",0.5-4.5*cpx.I,{})
g:Restoreattr()

clear_spherical() -- deletes previously created objects

g:Saveattr(); g:Viewport(0,9,-4,5); g:Coordsystem(-5,5,-5,5)
Hiddenlines = false; g:Define_sphere({radius=R,opacity=0.7} )
dessin()
g:Dlabel3d("$x$",5*vecI,{pos="SW"},"$y$",5*vecJ,{pos="E"},"$z$",5*vecK,{pos="N"})
g:Dlabel("Hiddenlines=false, opacity=0.7",0.5-4.5*cpx.I,{})
g:Restoreattr()
g:Show()
\end{luadraw}
\end{demo}

\paragraph{Spherical curve}

\begin{demo}{Viviani window}
\begin{luadraw}{name=courbe_spherique}
local g = graph3d:new{window={-4.5,4.5,-4.5,4.5},viewdir={30,60},margin={0,0,0,0},size={10,10}}
require 'luadraw_spherical'
arrowBstyle = "-stealth"
g:Linewidth(6); Hiddenlinestyle = "dotted"
Hiddenlines = false; 
local C = cylinder(M(1.5,0,-3.5),1.5,M(1.5,0,3.5),35,true)
local L = parametric3d( function(t) return Ms(3,t-math.pi/2,t) end, -math.pi,math.pi) -- la courbe
g:DSpolyline(facetedges(C),{color="gray"}) -- affichage cylindre
g:DSpolyline({{-5*vecI,5*vecI},{-5*vecJ,5*vecJ},{-5*vecK,5*vecK}},{arrows=1}) --axes
Hiddenlines=true; g:DScurve(L,{width=12,color="blue"}) -- courbe avec partie cachée
g:Dspherical()
g:Show()
\end{luadraw}
\end{demo}

To avoid compromising the readability of the drawing, the hidden parts have not been displayed except for the curve.

\paragraph{A spherical tiling}

\begin{demo}{A spherical tiling}
\begin{luadraw}{name=pavage_spherique}
local g = graph3d:new{window={-3,3,-3,3},viewdir={30,60},size={10,10}}
require 'luadraw_spherical'
require "luadraw_polyhedrons"
g:Linewidth(6); Hiddenlines = true; Hiddenlinestyle = "dotted"
local P = poly2facet( octahedron(Origin,sM(30,10)) )
local colors = {"Crimson","ForestGreen","Gold","SteelBlue","SlateGray","Brown","Orange","Navy"}
for k,F in ipairs(P) do
    g:DSfacet(F,{fill=colors[k],style="noline",fillopacity=0.7})  -- facettes sans les bords
end
for _, A in ipairs(facetedges(P)) do
    g:DSarc(A,1,{width=8}) -- each edge is an arc of a great circle
end
g:Dspherical()
g:Show()
\end{luadraw}
\end{demo}

For this spherical tiling, we chose a regular octahedron with a center identical to that of the sphere and with one vertex on the sphere (and therefore all vertices are on the sphere).

\paragraph{Tangents to the sphere from a point}

\begin{demo}{Tangents to the sphere from a point}
\begin{luadraw}{name=tangent_to_sphere}
local g = graph3d:new{window={-4,5.5,-4,4},viewdir={30,60},size={10,10}}
require 'luadraw_spherical'
Hiddenlines=true; g:Linewidth(6)
local O, I = Origin, M(0,6,0)
local S,S1 = {O, 3}, {(I+O)/2,pt3d.abs(I-O)/2}
-- the circle of tangency is the intersection between spheres S and S1
local C,r,n = interSS(S,S1) 
local L = circle3d(C,r,n)[1] -- list of 3d points on the circle
local dots, lines = {}, {}
-- draw
g:Define_sphere({opacity=1})
g:DScircle({C,n},{color="red"})
for k = 1, math.floor(#L/4) do
    local A = L[4*(k-1)+1]
    table.insert(dots,A)
    table.insert(lines,{I, 2*A-I})
end
g:DSpolyline(lines ,{color="gray"})
g:DSstars(dots) -- drawing points on the sphere
g:DSdots({O,I});  -- points in the scene
g:DSlabel("$I$",I,{pos="S",node_options="red"},"$O$",O,{})
g:Dspherical()
g:Dseg3d({O,dots[1]},"gray,dashed"); g:Dangle3d(O,dots[1],I,0.2,"gray")
g:Show() 
\end{luadraw}
\end{demo}

\paragraph{Inverse Stereography}

\begin{demo}{\emph{DSinvstereo\_curve} and \emph{DSinvstereo\_polyline} methods}
\begin{luadraw}{name=stereographic_curve}
local g = graph3d:new{window3d={-5,5,-2,2,-2,2},window={-4.25,4.25,-2.5,2},size={10,10}, viewdir={40,70}}
Hiddenlines = true; Hiddenlinestyle="dashed"; g:Linewidth(6)
require 'luadraw_spherical'
local C, R = Origin, 1
local a = -R
local P = planeEq(0,0,1,-a)
local L = {M(2,0,a), M(2,2.5,a), M(-1,2,a)}
local L2 = circle3d(M(2.25,-1,a),0.5,vecK)[1]
local A, B = (L[2]+L[3])/2, L2[20]
local a,b = table.unpack( inv_projstereo({A,B},{C,R},C+R*vecK) )
g:Dplane(P,vecJ,6,6,15,"draw=none,fill=Beige")
g:Define_sphere( {center=C,radius=R, color="SlateGray!30", show=true} )
g:DSpolyline(L,{color="blue",close=true}); g:DSinvstereo_polyline(L,{color="red",width=8,close=true})
g:DSpolyline(L2,{color="Navy"}); g:DSinvstereo_curve(L2,{color="Brown",width=6})
g:DSplane(P,{scale=1.5})
g:DSpolyline({{C+R*vecK,A},{C+R*vecK,B}}, {color="ForestGreen",width=8})
g:DSpolyline({{-vecK,2*vecK}}, {arrows=1})
g:DSstars({C+R*vecK,a,b}, {scale=0.75})
g:Dspherical()
g:Dballdots3d({A,B},"ForestGreen",0.75)
g:Show()
\end{luadraw}
\end{demo}


\subsection{The \emph{luadraw\_palettes} Module}

The \emph{luadraw\_palettes} Module\footnote{This module is a contribution of \href{https://github.com/projetmbc/for-writing/tree/main/palcol}{Christphe BAL}.} defines $88$ color palettes, each with a name. A palette is a list (table) of colors, which are themselves lists of three numerical values ​​between $0$ and $1$ (red, green, and blue components). The names of these palettes, as well as their rendering, can be viewed in this \href{luadraw_palettes_doc.pdf}{document}.
%
\section{History}

\subsection{Version 2.2}
Non-exhaustive list:
\begin{itemize}
    \item Added the \emph{clip} option for the methods: \emph{Dfacet()}, \emph{Dmixfacet()}, \emph{addFacet()}, \emph{addPoly()} and \emph{addPolyline()}, as well as for point cloud drawing methods, and line drawing methods such as \emph{Dpolyline3d()}, \emph{Dparametric3d()}, \emph{Dpath3d()}, etc.
    \item Added the \emph{xyzstep} option for the \emph{Dboxaxes3d()} method. This option defines a common step for all three axes ($1$ by default).     \item Added the \emph{DSdots()}, \emph{DSstars()}, \emph{DSinvstereo\_curve()}, and \emph{DSinvstereo\_polyline()} methods to the \emph{luadraw\_spherical} module.
    \item Added the \emph{luadraw\_palettes} module.
    \item Added the \emph{interDC()} function (intersection between a line and a circle in 2D) and the \emph{interCC()} function (intersection between two circles in 2D).
    \item Added the \emph{curvilinear\_param()} and \emph{curvilinear\_param3d()} functions, which allow you to parameterize a list of points (one in 2D and the other in 3D) with a function of a variable $t$ between $0$ and $1$.
Added the function \emph{cvx\_hull2d()}, which returns the convex hull (polygonal line) of a list of points in 2D, and the function \emph{cvx\_hull3d()}, which returns the convex hull (list of facets) of a list of points in 3D.
Added the methods \emph{g:Beginclip(<path>)} and \emph{g:Endclip()}, which make it easier to set up clipping using tikz.
Added the functions \emph{normal()}, \emph{normalC()}, and \emph{normalI()}, which return the normal to a 2D curve at a given point. The corresponding graphics methods have also been added.
Added the function \emph{isobar()}, which returns the isobarycenter of a list of complexes. Added the \emph{usepalette=\{palette,mode\}} option for the \emph{Dpoly}, \emph{Dfacet}, \emph{Dmixfacet}, and \emph{addFacet} methods.
Added the \emph{clipplane()} function, which allows you to clip a plane with a convex polyhedron. The function returns the section, if it exists, as a facet.
Added the \emph{cartesian3d()} and \emph{cylindrical\_surface()} functions, which calculate and return surfaces, with the option to add dividing walls for the \emph{Dscene3d()} method.     \item Added the function \emph{evalf(f,...)} which allows a protected evaluation of $f(...)$. It returns the result of the evaluation if there is no runtime error from Lua, otherwise it returns \emph{nil} but without causing the script execution to terminate.
    \item Added the function \emph{split\_points\_by\_visibility()} (3d) to separate a curve into two parts: visible part, hidden part.
    \item In the methods \emph{g:Dfacet}, \emph{g:Dmixfacet}, \emph{g:Dpoly}, \emph{g:Dedges}, \emph{g:addFacet}, \emph{g:addPolyline}, \emph{g:addPoly}, the default values ​​for the line drawing options (thickness, color, and style) are the current values.
Bug fix...
...     \item Graduated axes (2d, 3d) use the \emph{siunitx} package to format labels when the global variable \emph{siunitx} is set to \emph{true}.
    \item Added upright and slanted truncated cones (\textbf{frustum} and \textbf{Dfrustum}).
    \item Added regular pyramids (\textbf{regular\_pyramid} and truncated pyramids \textbf{truncated\_pyramid}).
    \item Cylinders and cones are no longer necessarily upright; they can now be slanted.
    \item Added the \textbf{cutpolyline(L,D,close)} function.
    \item (Elementary) drawing of sets (\emph{set} function) and operations on sets (\emph{cap}, \emph{cup}, \emph{setminus}).
    \item Modification of the \emph{mode} argument of the \textbf{g:Dplane} method.
    \item Addition of the \emph{close} option for the \textbf{g:addPolyline} method.
    \item Bug fix...
\end{itemize}

\subsection{Version 2.0}

\begin{itemize}
    \item Introduction of the \emph{luadraw\_graph3d.lua} module for 3D drawings.
    \item Introduction of the \emph{dir} option for the \textbf{g:Dlabel} method.
    \item Minor changes in color management.
\end{itemize}

\subsection{Version 1.0}
First version.
%


\end{document}
