\documentclass[%
10pt,%
oneside,
a4paper,%
french,%
]%
{book}%


\usepackage{luatextra}
\usepackage{amsmath,amssymb,marvosym,stmaryrd,calrsfs,minted,setspace,array,caption,fvextra}%
\usepackage[math-style=french]{unicode-math}
\usepackage[math-style=french]{fourier-otf}%
\usepackage[svgnames]{xcolor}\usepackage[margin=1.5cm,top=1cm,bottom=1.5cm,headheight=5mm,headsep=2mm]{geometry}
\usepackage[unicode,pdfstartview = FitH, colorlinks, linkcolor=blue]{hyperref}
\usepackage{babel,microtype,multicol}%
%\usepackage{minitoc}
\usepackage[Lenny]{fncychap}

\usepackage[noexec,3d,cachedir=tkz]{luadraw}
\def\version{2.2}

\hypersetup{
  pdfauthor={P. Fradin},
  pdflang={fr},
  %hidelinks,
  pdfcreator={LuaLatex}
}

\title{\textbf{Le paquet} \emph{luadraw} \\ version \version\\ Dessin 2d et 3d avec lua (et tikz).\\
{\small \url{https://github.com/pfradin/luadraw}} \\

\vspace{1cm}

\begin{minipage}{16cm}
\small
\selectlanguage{english}

\hfil\textbf{Abstract}\hfil

The \emph{luadraw} package defines the environment of the same name, which lets you create mathematical graphs using the Lua language. These graphs are ultimately drawn by tikz (and automatically saved), so why make them in Lua? Because Lua brings all the power of a simple, efficient programming language with good computational capabilities.

\vspace{1cm}

\hfil\textbf{Résumé}\hfil

\selectlanguage{french}
Le paquet \emph{luadraw} définit l'environnement du même nom, celui-ci permet de créer des graphiques mathématiques en utilisant le langage Lua. Ces graphiques sont dessinés au final par tikz (et automatiquement sauvegardés), alors pourquoi les faire en Lua ? Parce que celui-ci apporte toute la puissance d'un langage de programmation simple, efficace, capable de faire des calculs, tout en utilisant les possibilités graphiques de tikz.
\end{minipage}
}
\author{Patrick Fradin}
%\date{}


\onehalfspacing
\newminted{Lua}{bgcolor=Beige,ignorelexererrors=true,linenos,numbersep=6pt,breaklines,fontsize=\footnotesize}%
\newminted{TeX}{bgcolor=Gray!30,breaklines,fontsize=\footnotesize}%

%\columnseprule 0.4pt
\columnsep 25pt

\definecolor{bgcol}{RGB}{238,240,252}% couleur de fond des listings

\newenvironment*{demo}[2][]{%
\gdef\legende{#2}%
\gdef\lab{#1}%
\bgroup
\VerbatimOut{\jobname.tmp}%
}%
{%
\endVerbatimOut%
\egroup%
%\begin{center}
%\begin{tabular}{|m{10cm}|m{7cm}|}
%\hline
%\begin{minipage}{10cm}
\inputminted[ignorelexererrors=true,breaklines,bgcolor=Beige,linenos,numbersep=6pt,frame=single,fontsize=\footnotesize]{Lua}{\jobname.tmp}%
%\end{minipage}
%&
\begin{minipage}{0.9\textwidth}
%\par\smallskip
\begin{center}
\captionof{figure}{\legende}\label{\lab}%
\input{\jobname.tmp}%
%\end{center}
%\par\smallskip
%\hline
%\end{tabular}
\end{center}
\end{minipage}
}

\begin{document}
\maketitle

\setcounter{tocdepth}{3}%
\setcounter{secnumdepth}{2}%

\begin{multicols}{2}
\tableofcontents
\end{multicols}

\listoffigures

\renewcommand{\labelitemi}{$\bullet$}
\renewcommand{\labelitemii}{--}
\renewcommand{\labelitemiii}{$*$}
%\renewcommand{\thechapter}{\Roman{chapter}}
\renewcommand{\thesection}{\Roman{section}~}
\renewcommand{\thesubsection}{\arabic{subsection})}
\renewcommand{\thesubsubsection}{\arabic{subsection}.\arabic{subsubsection}}
\renewcommand{\thefigure}{\arabic{figure}}

\chapter{Dessin 2d}

\begin{center}
\input{tkz/trigo.tkz}%
\par
\captionof{figure}{Un premier exemple : trois sous-figures dans un même graphique}
\end{center}

\section{Introduction}

\subsection{Prérequis}

\begin{itemize}
\item Dans le préambule, il faut déclarer le package \emph{luadraw} : \verb|\usepackage[options globales]{luadraw}|
\item La compilation se fait avec LuaLatex \textbf{exclusivement}.
\item Les couleurs dans l'environnement \emph{luadraw} sont des chaînes de caractères qui doivent correspondre à des couleurs connues de tikz. Il est fortement conseillé d'utiliser le package \emph{xcolor} avec l'option \emph{svgnames}.
\end{itemize}

Quelque soient les options globales choisies, ce paquet charge le module \emph{luadraw\_graph2d.lua} qui définit la classe \emph{graph}, et fournit l'environnement \emph{luadraw} qui permet de faire des graphiques en Lua.

\paragraph{Options globales du paquet} : \emph{noexec} et \emph{3d}.

\begin{itemize}
    \item \emph{noexec} : lorsque cette option globale est mentionnée la valeur par défaut de l'option \emph{exec} pour l'environnement \emph{luadraw} sera false (et non plus true).
    \item \emph{3d} : lorsque cette option globale est mentionnée, le module \emph{luadraw\_graph3d.lua} est également chargé. Celui-ci définit en plus la classe \emph{graph3d} (qui s'appuie sur la classe \emph{graph}) pour des dessins en 3d. 
\end{itemize}

\noindent\textbf{NB} : dans ce document nous ne parlerons pas de l'option \emph{3d}. Celle-ci fait l'objet du document \emph{luadraw\_graph3d.pdf}. Nous ne parlerons donc que de la version 2d.

Lorsqu'un graphique est terminé il est exporté au format tikz, donc ce paquet charge également le paquet \emph{tikz} ainsi que les librairies :
\begin{itemize}
\item\emph{patterns}
\item\emph{plotmarks}
\item\emph{arrows.meta}
\item\emph{decorations.markings}
\end{itemize}

Les graphiques sont créés dans un environnement \emph{luadraw}, celui-ci appelle \emph{luacode}, c'est donc du \textbf{langage Lua} qu'il faut utiliser dans cet environnement :

\begin{TeXcode}
\begin{luadraw}{ name=<filename>, exec=true/false, auto=true/false }
-- création d'un nouveau graphique en lui donnant un nom local
local g = graph:new{ window={x1,x2,y1,y2,xscale,yscale}, margin={left,right,top,bottom},
                     size={largeur,hauteur,ratio}, bg="color", border=true/false }
-- construction du graphique g
    instructions graphiques en langage Lua ...
-- affichage du graphique g et sauvegarde dans le fichier <filename>.tkz
g:Show()
-- ou bien sauvegarde uniquement dans le fichier <filename>.tkz
g:Save()
\end{luadraw}
\end{TeXcode}

\paragraph{Sauvegarde du fichier \emph{.tkz}} : le graphique est exporté au format tikz dans un fichier (avec l'extension \emph{tkz}), par défaut celui-ci est sauvegardé dans le dossier courant. Mais il est possible d'imposer un chemin spécifique en définissant dans le document, la commande \verb|\luadrawTkzDir|, par exemple : \verb|\def\luadrawTkzDir{tikz/}|, ce qui permettra d'enregistrer les fichiers \emph{*.tkz} dans le sous-dossier \emph{tikz} du dossier courant, à condition toutefois que ce sous-dossier existe !

\subsection{Options de l'environnement}

Celles-ci sont :
\begin{itemize}
\item \emph{name = \ldots{}} : permet de donner un nom au fichier tikz produit, on donne un nom sans extension (celle-ci sera automatiquement ajoutée, c'est \emph{.tkz}). Si cette option est omise, alors il y a un nom par défaut, qui est le nom du fichier maître suivi d'un numéro.
\item \emph{exec = true/false} : permet d'exécuter ou non le code Lua compris dans l'environnement. Par défaut cette option vaut true, \textbf{SAUF} si l'option globale \emph{noexec} a été mentionnée dans le préambule avec la déclaration du paquet. Lorsqu'un graphique complexe qui demande beaucoup de calculs est au point, il peut être intéressant de lui ajouter l'option \emph{exec=false}, cela évitera les recalculs de ce même graphique pour les compilations à venir.
\item \emph{auto = true/false} : permet d'inclure ou non automatiquement le fichier tikz en lieu et place de l'environnement \emph{luadraw} lorsque l'option \emph{exec} est à false. Par défaut l'option \emph{auto} vaut true.
\end{itemize}


\subsection{La classe cpx (complexes)}

Elle est automatiquement chargée par le module \emph{luadraw\_graph2d} et donc au chargement du paquet \emph{luadraw}. Cette classe permet de manipuler les nombres complexes et de faire les calculs habituels. On crée un complexe avec la fonction \textbf{Z(a,b)} pour \(a+i\times b\), ou bien avec la fonction \textbf{Zp(r,theta)} pour \(r\times e^{i\theta}\) en coordonnées polaires.

\begin{itemize}
\item Exemple : \emph{local z = Z(a,b)} va créer le complexe correspondant à \(a+i\times b\) dans la variable \emph{z}. On accède alors aux parties réelle et imaginaire de \emph{z} comme ceci : \emph{z.re} et \emph{z.im}.
\item \textbf{Attention} : un nombre réel \emph{x} n'est pas considéré comme complexe par Lua. Cependant, les fonctions proposées pour les constructions graphiques font la vérification et la conversion réel vers complexe. On peut néanmoins, utiliser \emph{Z(x,0)} à la place de \emph{x}.
\item Les opérateurs habituels ont été surchargés ce qui permet l'utilisation des symboles habituels, à savoir : +, x, -, /, ainsi que le test d'égalité avec =. Lorsqu'un calcul échoue le résultat renvoyé en principe doit être égal à \emph{nil}.
\item À cela s'ajoutent les fonctions suivantes (il faut utiliser la notation pointée en Lua) :
  \begin{itemize}
  \item le module : \textbf{cpx.abs(z)},
  \item la norme 1 : \textbf{cpx.N1(z)},
  \item l'argument principal : \textbf{cpx.arg(z)},
  \item le conjugué : \textbf{cpx.bar(z)},
  \item l'exponentielle complexe : \textbf{cpx.exp(z)},
  \item le produit scalaire : \textbf{cpx.dot(z1,z2)}, où les complexes représentent des affixes de vecteurs,
  \item le déterminant : \textbf{cpx.det(z1,z2)},
  \item l'arrondi : \textbf{cpx.round(z, nb decimales)},
  \item la fonction : \textbf{cpx.isNul(z)} teste si les parties réelle et imaginaire de \emph{z} sont en valeur absolue inférieures à une variable \emph{epsilon} qui vaut \emph{1e-16} par défaut.
  \end{itemize}
\end{itemize}

La dernière fonction renvoie un booléen, les fonctions bar, exponentielle et round renvoient un complexe, et les autres renvoient un réel.

On dispose également de la constante \emph{cpx.I} qui représente l'imaginaire pur \emph{i}. 

Exemple :

\begin{Luacode}
local i = cpx.I
local A = 2+3*i
\end{Luacode}

Le symbole de multiplication est obligatoire.

\subsection{Création d'un graphe}

Comme cela a été vu plus haut, la création se fait dans un environnement \emph{luadraw}, c'est à la première ligne à l'intérieur de l'environnement qu'est faite cette création en nommant le graphique :

\begin{Luacode}
local g = graph:new{ window={x1,x2,y1,y2,xscale,yscale}, margin={left,right,top,bottom}, 
                     size={largeur,hauteur,ratio}, bg="color", border=true/false }
\end{Luacode}

La classe \emph{graph} est définie dans le paquet \emph{luadraw}. On instancie cette classe en invoquant son constructeur et en donnant un nom (ici c'est \emph{g}), on le fait en local de sorte que le graphique \emph{g} ainsi créé, n'existera plus une fois sorti de l'environnement (sinon \emph{g} resterait en mémoire jusqu'à la fin du document).

\begin{itemize}
 \item Le paramètre (facultatif) \emph{window} définit le pavé de $\mathbf R^2$ correspondant au graphique : c'est $[x_1,x_2]\times[y_1,y_2]$. Les paramètres \emph{xscale} et \emph{yscale} sont facultatifs et valent $1$ par défaut, ils représentent l'échelle (cm par unité) sur les axes. Par défaut on a \emph{window = \{-5,5,-5,5,1,1\}}.
\item Le paramètre (facultatif) \emph{margin} définit des marges autour du graphique en cm. Par défaut on a \emph{margin = \{0.5,0.5,0.5,0.5\}}.
\item Le paramètre (facultatif) \emph{size} permet d'imposer une taille (en cm, marges incluses) pour le graphique, l'argument \emph{ratio} correspond au rapport d'échelle souhaité (\emph{xscale}/\emph{yscale}), un ratio de $1$ donnera un repère orthonormé, et si le ratio n'est pas précisé alors le ratio par défaut est conservé. L'utilisation de ce paramètre va modifier les valeurs de \emph{xscale} et \emph{yscale} pour avoir les bonnes tailles. Par défaut la taille est de $11\times11$ (en cm) avec les marges ($10\times10$ sans les marges).
\item Le paramètre (facultatif) \emph{bg} permet de définir une couleur de fond pour le graphique, cette couleur est une chaîne de caractères représentant une couleur pour tikz. Par défaut cette chaîne est vide ce qui signifie que le fond ne sera pas peint.
\item Le paramètre (facultatif) \emph{border} indique si un cadre doit être dessiné ou non autour du graphique (en incluant les marges). Par défaut ce paramètre vaut false.
\end{itemize}

\paragraph{Construction du graphique.}

\begin{itemize}
    \item L'objet instancié (\emph{g} ici dans l'exemple) possède un certain nombre de méthodes permettant de faire du dessin (segments, droites, courbes,...). Les instructions de dessins ne sont pas directement envoyées à \TeX, elles sont enregistrées sous forme de chaînes dans une table qui est une propriété de l'objet \emph{g}. C'est la méthode \textbf{g:Show()} qui va envoyer ces instructions à \TeX\ tout en les sauvegardant dans un fichier texte\footnote{Ce fichier contiendra un environnement \emph{tikzpicture}.}. La méthode \textbf{g:Save()} enregistre le graphique dans un fichier mais sans envoyer les instructions à \TeX.
    \item Le paquet \emph{luadraw} fournit aussi un certain nombre de fonctions mathématiques, ainsi que des fonctions permettant des calculs sur les listes (tables) de complexes, des transformations géométriques, ...etc.
\end{itemize}


\paragraph{Système de coordonnées. Repérage}

\begin{itemize}
\item L'objet instancié (\emph{g} ici dans l'exemple) possède :
    \begin{enumerate}
        \item Une vue originelle : c'est le pavé de $\mathbf R^2$ défini par l'option \emph{window} à la création. Celui-ci \textbf{ne doit pas être modifié} par la suite.
        \item Une vue courante : c'est un pavé de $\mathbf R^2$ qui doit être inclus dans la vue originelle, ce qui sort de ce pavé sera clippé. Par défaut la vue courante est la vue originelle. Pour retrouver la vue courante on peut utiliser la méthode \textbf{g:Getview()} qui renvoie une table \verb|{x1,x2,y1,y2}|, celle-ci représente la pavé $[x1,x2]\times [y1,y2]$.
        \item Une matrice de transformation : celle-ci est initialisée à la matrice identité. Lors d'une instruction de dessin les points sont automatiquement transformés par cette matrice avant d'être envoyés à tikz.
        \item Un système de coordonnées (repère cartésien) lié à la vue courante, c'est le repère de l'utilisateur. Par défaut c'est le repère canonique de $\mathbf R^2$, mais il est possible d'en changer. Admettons que la vue courante soit le pavé $[-5,5]\times[-5,5]$, il est possible par exemple, de décider que ce pavé représente l'intervalle $[-1,12]$ pour les abscisses et l'intervalle $[0,8]$ pour les ordonnées, la méthode qui fait ce changement va modifier la matrice de transformation du graphe, de telle sorte que pour l'utilisateur tout se passe comme s'il était dans le pavé $[-1,12]\times [0,8]$. On peut retrouver les intervalles du repère de l'utilisateur avec les méthodes : \textbf{g:Xinf(), g:Xsup(), g:Yinf() et g:Ysup()}.
    \end{enumerate}
\item On utilise les nombres complexes pour représenter les points ou les vecteurs dans le repère cartésien de l'utilisateur.
\item Dans l'export tikz les coordonnées seront différentes car le coin inférieur gauche (hors marges) aura pour coordonnées $(0,0)$, et le coin supérieur droit (hors marges) aura des coordonnées correspondant à la taille (hors marges) du graphique, et avec $1$ cm par unité sur les deux axes. Ce qui fait que normalement, tikz ne devrait manipuler que de \og petits\fg\ nombres.
\item La conversion se fait automatiquement avec la méthode \textbf{g:strCoord(x,y)} qui renvoie une chaîne de la forme \emph{(a,b)}, où \emph{a} et \emph{b} sont les coordonnées pour tikz, ou bien avec la méthode \textbf{g:Coord(z)} qui renvoie aussi une chaîne de la forme \emph{(a,b)} représentant les coordonnées tikz du point d'affixe \emph{z} dans le repère de l'utilisateur.
\end{itemize}

\subsection{Peut-on utiliser directement du tikz dans l'environnement \emph{luadraw} ?}

Supposons que l'on soit en train de créer un graphique nommé \emph{g} dans un environnement \emph{luadraw}. Il est possible d'écrire une instruction tikz lors de cette création, mais pas en utilisant \verb|tex.sprint("<instruction tikz>")|, car alors cette instruction ne ferait pas partie du graphique \emph{g}. Il faut pour cela utiliser la méthode \textbf{g:Writeln("<instruction tikz>")}, avec la contrainte que \textbf{les antislash doivent être doublés}, et sans oublier que les coordonnées graphiques d'un point dans \emph{g} ne sont pas les mêmes pour tikz. Par exemple : 
\begin{Luacode}
g:Writeln("\\draw"..g:Coord(Z(1,-1)).." node[red] {Texte};")
\end{Luacode}

Ou encore pour changer des styles :
\begin{Luacode}
g:Writeln("\\tikzset{every node/.style={fill=white}}")
\end{Luacode}

Dans une présentation beamer, cela peut aussi être utilisé pour insérer des pauses dans un graphique :
\begin{Luacode}
g:Writeln("\\pause")
\end{Luacode}  

\section{Méthodes graphiques}

On peut créer des lignes polygonales, des courbes, des chemins, des points, des labels.


\subsection{Lignes polygonales}

\textbf{Une ligne polygonale est une liste (table) de composantes connexes, et une composante connexe est une liste (table) de complexes qui représentent les affixes des points}. Par exemple l'instruction :
\begin{Luacode}
local L = { {Z(-4,0), Z(0,2), Z(1,3)}, {Z(0,0), Z(4,-2), Z(1,-1)} }
\end{Luacode}
crée une ligne polygonale à deux composantes dans une variable \emph{L}.

\paragraph{Dessin d'une ligne polygonale.}

C'est la méthode \textbf{g:Dpolyline(L,close, draw\_options)} (où \emph{g} désigne le graphique en cours de création), \emph{L} est une ligne polygonale, \emph{close} un argument facultatif qui vaut \emph{true} ou \emph{false} indiquant si la ligne doit être refermée ou non (\emph{false} par défaut), et \emph{draw\_options} est une chaîne de caractères qui sera passée directement à l'instruction \emph{\textbackslash draw} dans l'export.

\paragraph{Choix des options de dessin d'une ligne polygonale.}

On peut passer des options de dessin directement à l'instruction \emph{\textbackslash draw} dans l'export, mais elles auront un effet local uniquement. Il est possible de modifier ces options de manière globale avec la méthode \textbf{g:Lineoptions(style,color,width,arrows)} (lorsqu'un des arguments vaut \emph{nil}, c'est sa valeur par défaut qui est utilisée) :

\begin{itemize}
\item \emph{color} est une chaîne de caractères représentant une couleur connue de tikz ("black" par défaut),
\item \emph{style} est une chaîne de caractères représentant le type de ligne à dessiner, ce style peut être :
 \begin{itemize}
  \item \emph{"noline"} : trait non dessiné,
  \item \emph{"solid"} : trait plein, valeur par défaut,
  \item \emph{"dotted"} : trait pointillé,
  \item \emph{"dashed"} : tirets,
  \item style personnalisé : l'argument \emph{style} peut être une chaîne de la forme (exemple) \emph{"\{2.5pt\}\{2pt\}"} ce qui signifie : un trait de 2.5pt suivi d'un espace de 2pt, le nombre de valeurs peut être supérieur à 2, ex : \emph{"\{2.5pt\}\{2pt\}\{1pt\}\{2pt\}"} (succession de on, off).
  \end{itemize}
\item \emph{width} est un nombre représentant l'épaisseur de ligne exprimée en dixième de points, par exemple $8$ pour une épaisseur réelle de 0.8pt (valeur de $4$ par défaut),
\item \emph{arrows} est une chaîne qui précise le type de flèche qui sera dessiné, cela peut être :
 \begin{itemize}
  \item \emph{"-"} qui signifie pas de flèche (valeur par défaut),
  \item \emph{"->"} qui signifie une flèche à la fin,
  \item \emph{"<-"} qui signifie une flèche au début,
  \item \emph{"<->"}qui signifie une flèche à chaque bout.\\
  \textbf{ATTENTION} : la flèche n'est pas dessinée lorsque l'argument \emph{close} est true.
  \end{itemize}
\end{itemize}


On peut modifier les options individuellement avec les méthodes : 
\begin{itemize}
    \item \textbf{g:Linecolor(color)}, 
    \item \textbf{g:Linestyle(style)}, 
    \item \textbf{g:Linewidth(width)}, 
    \item \textbf{g:Arrows(arrows)}, 
    \item plus les méthodes suivantes : 
        \begin{itemize}
            \item \textbf{g:Lineopacity(opacity)} qui règle l'opacité du tracé de la ligne, l'argument \emph{opacity} doit être une
valeur entre $0$ (totalement transparent) et $1$ (totalement opaque), par défaut la valeur est de $1$.
            \item \textbf{g:Linecap(style)}, pour jouer sur les extrémités de la ligne, l'argument \emph{style} est une chaîne qui peut valoir :
            \begin{itemize}
                \item \emph{"butt"} (bout droit au point d'arrêt, valeur par défaut),
                \item \emph{"round"} (bout arrondi en demi-cercle),
                \item \emph{"square"} (bout \og arrondi\fg\ en carré).
            \end{itemize}
          
          \item \textbf{g:Linejoin(style)}, pour jouer sur la jointure entre segments, l'argument style est une chaîne qui peut valoir :
            \begin{itemize}
                \item \emph{"miter"} (coin pointu, valeur par défaut),
                \item \emph{"round"} (ou coin arrondi),
                \item \emph{"bevel"} (coin coupé).
            \end{itemize}
        \end{itemize}
\end{itemize}


\paragraph{Options de remplissage d'une ligne polygonale.}

C'est la méthode \textbf{g:Filloptions(style,color,opacity,evenodd)} (qui utilise la librairie \emph{patterns} de tikz, celle-ci est chargée avec le paquet). Lorsqu'un des arguments vaut \emph{nil}, c'est sa valeur par défaut qui est utilisée :

\begin{itemize}
  \item \emph{color} est une chaîne de caractères représentant une couleur connue de tikz ("black" par défaut).
  \item \emph{style} est une chaîne de caractères représentant le type de remplissage, ce style peut être :
      \begin{itemize}
      \item \emph{"none"} : pas de remplissage, c'est la valeur par défaut,
      \item \emph{"full"} : remplissage plein,
      \item \emph{"bdiag"} : hachures descendantes de gauche à droite,
      \item \emph{"fdiag"} : hachures montantes de gauche à droite,
      \item \emph{"horizontal"} : hachures horizontales,
      \item \emph{"`vertical"} : hachures verticales,
      \item \emph{"hvcross"} : hachures horizontales et verticales,
      \item \emph{"diagcross"} : diagonales descendantes et montantes,
      \item \emph{"gradient"} : dans ce cas le remplissage se fait avec un gradient défini avec la méthode \textbf{g:Gradstyle(chaîne)}, ce style est passé tel quel à l'instruction \emph{\textbackslash draw}. Par défaut la chaîne définissant le style de gradient est \emph{"left color = white, right color = red"},
      \item tout autre style connu de la librairie \emph{patterns} est également possible.
      \end{itemize}
\end{itemize}

On peut modifier certaines options individuellement avec les méthodes :
\begin{itemize}
    \item \textbf{g:Fillopacity(opacity)}, 
    \item \textbf{g:Filleo(evenodd)}.
\end{itemize}

\begin{demo}{Champ de vecteurs, courbe intégrale de $y'= 1-xy^2$}
\begin{luadraw}{name=champ}
local g = graph:new{window={-5,5,-5,5},bg="Cyan!30",size={10,10}}
local f = function(x,y) -- éq. diff. y'= 1-x*y^2=f(x,y)
    return 1-x*y^2     
end
local A = Z(-1,1) -- A = -1+i
local deltaX, deltaY, long = 0.5, 0.5, 0.4
local champ = function(f)
    local vecteurs, v = {}
    for y = g:Yinf(), g:Ysup(), deltaY do
        for x = g:Xinf(), g:Xsup(), deltaX do
            v = Z(1,f(x,y)) -- coordonnées 1 et f(x,y)
            v = v/cpx.abs(v)*long -- normalisation de v
            table.insert(vecteurs, {Z(x,y), Z(x,y)+v} )
        end
    end 
    return vecteurs -- vecteurs est une ligne polygonale
end
g:Daxes( {0,1,1}, {labelpos={"none","none"}, arrows="->"} )
g:Dpolyline( champ(f), "->,blue")
g:Dodesolve(f, A.re, A.im, {t={-2.75,5},draw_options="red,line width=0.8pt"})
g:Dlabeldot("$A$", A, {pos="S"})
g:Show()
\end{luadraw}
\end{demo}
\label{champ}


\subsection{Segments et droites}

\textbf{Un segment est une liste (table) de deux complexes représentant les extrémités. Une droite est une liste (table) de deux complexes, le premier représente un point de la droite, et le second un vecteur directeur de la droite (celui-ci doit être non nul).}

\subsubsection{Dangle}
\begin{itemize}
    \item La méthode \textbf{g:Dangle(B,A,C,r,draw\_options)} dessine l'angle \(BAC\) avec un parallélogramme (deux côtés seulement sont dessinés), l'argument facultatif \emph{r} précise la longueur d'un côté (0.25 par défaut). L'argument \emph{draw\_options} est une chaîne (vide par défaut) qui sera passée telle quelle à l'instruction  \emph{\textbackslash draw}.
    \item La fonction \textbf{angleD(B,A,C,r)} renvoie la liste des points de cet angle.
\end{itemize}

\subsubsection{Dbissec}
\begin{itemize}
    \item La méthode \textbf{g:Dbissec(B,A,C,interior,draw\_options)} dessine une bissectrice de l'angle géométrique BAC, intérieure si l'argument facultatif \emph{interior} vaut \emph{true} (valeur par défaut), extérieure sinon. L'argument \emph{draw\_options} est une chaîne (vide par défaut) qui sera passée telle quelle à l'instruction
  \emph{\textbackslash draw}.
  \item La fonction \textbf{bissec(B,A,C,interior)} renvoie cette bissectrice sous forme d'une liste \emph{\{A,u\}} où \emph{u} est un vecteur directeur de la droite.
\end{itemize}

\subsubsection{Dhline}
La méthode \textbf{g:Dhline(d,draw\_options)} dessine une demi-droite, l'argument \emph{d} doit être une liste de de complexes \emph{\{A,B\}}, c'est la demi-droite $[A,B)$ qui est dessinée.

Variante : \textbf{g:Dhline(A,B,draw\_options)}. L'argument \emph{draw\_options} est une chaîne (vide par défaut) qui sera passée telle quelle à l'instruction \emph{\textbackslash draw}.

\subsubsection{Dline}
La méthode \textbf{g:Dline(d,draw\_options)} trace la droite \emph{d}, celle-ci est une liste du type \emph{\{A,u\}} où \emph{A} représente un point de la droite (un complexe) et \emph{u} un vecteur directeur (un complexe non nul). 

Variante : la méthode \textbf{g:Dline(A,B,draw\_options)} trace la droite passant par les points \emph{A} et \emph{B} (deux complexes). L'argument \emph{draw\_options} est une chaîne (vide par défaut) qui sera passée telle quelle à l'instruction \emph{\textbackslash draw}.

\subsubsection{DlineEq}
\begin{itemize}
    \item La méthode \textbf{g:DlineEq(a,b,c,draw\_options)} dessine la droite d'équation \(ax+by+c=0\). L'argument \emph{draw\_options} est une chaîne (vide par défaut) qui sera passée telle quelle à l'instruction \emph{\textbackslash draw}.
    \item La fonction \textbf{lineEq(a,b,c)} renvoie la droite d'équation \(ax+by+c=0\) sous la forme d'une liste \emph{\{A,u\}} où \emph{A} est un point de la droite et \emph{u} un vecteur directeur.
\end{itemize}

\subsubsection{Dmarkarc}
La méthode \textbf{g:Dmarkarc(b,a,c,r,n,long,espace)} permet de marquer l'arc de cercle de centre \emph{a}, de rayon \emph{r}, allant de \emph{b} à \emph{c}, avec \emph{n} petits segments. Par défaut la longueur (argument \emph{long}) est de 0.25, et l'espacement (argument \emph{espace}) et de 0.0625.

\subsubsection{Dmarkseg}
La méthode \textbf{g:Dmarkseg(a,b,n,long,espace,angle)} permet de marquer le segment défini par \emph{a} et \emph{b} avec \emph{n} petits segments penchés de \emph{angle} degrés (45° par défaut). Par défaut la longueur (argument \emph{long}) est de 0.25, et l'espacement (argument \emph{espace}) et de 0.125.

\subsubsection{Dmed}
\begin{itemize}
    \item La méthode \textbf{g:Dmed(A,B, draw\_options)} trace la médiatrice du segment $[A;B]$.

  Variante : \textbf{g:Dmed(seg,draw\_options)} où segment est une liste de deux points représentant le segment. L'argument \emph{draw\_options} est une chaîne (vide par défaut) qui sera passée telle quelle à l'instruction \emph{\textbackslash draw}.
  \item La fonction \textbf{med(A,B)} (ou \textbf{med(seg)}) renvoie la médiatrice du segment \emph{{[}A,B{]}} sous la forme d'une liste \emph{\{C,u\}} où \emph{C} est un point de la droite et \emph{u} un vecteur directeur.
\end{itemize}

\subsubsection{Dparallel}
\begin{itemize}
    \item La méthode \textbf{g:Dparallel(d,A,draw\_options)} trace la parallèle à \emph{d} passant par \emph{A}. L'argument \emph{d} peut-être soit une droite (une liste constituée d'un point et un vecteur directeur) soit un vecteur non nul. L'argument \emph{draw\_options} est une chaîne (vide par défaut) qui sera passée telle quelle à l'instruction \emph{\textbackslash draw}.
    \item La fonction \textbf{parallel(d,A)} renvoie la parallèle à \emph{d} passant par \emph{A} sous la forme d'une liste \emph{\{B,u\}} où \emph{B} est un point de la droite et \emph{u} un vecteur directeur.
\end{itemize}

\subsubsection{Dperp}
\begin{itemize}
    \item La méthode \textbf{g:Dperp(d,A,draw\_options)} trace la perpendiculaire à \emph{d} passant par \emph{A}. L'argument \emph{d} peut-être soit une droite (une liste constituée d'un point et un vecteur directeur) soit un vecteur non nul. L'argument \emph{draw\_options} est une chaîne (vide par défaut) qui sera passée  telle quelle à l'instruction \emph{\textbackslash draw}.
    \item La fonction \textbf{perp(d,A)} renvoie la perpendiculaire à \emph{d} passant par \emph{A} sous la forme d'une liste \emph{\{B,u\}} où \emph{B} est un point de la droite et \emph{u} un vecteur directeur.
\end{itemize}

\subsubsection{Dseg}
La méthode \textbf{g:Dseg(seg,scale,draw\_options)} dessine le segment défini par l'argument \emph{seg} qui doit être une liste de deux complexes. L'argument facultatif \emph{scale} (1 par défaut) est un nombre qui permet d'augmenter ou réduire la longueur du segment (la longueur naturelle est multipliée par \emph{scale}). L'argument \emph{draw\_options} est une chaîne (vide par défaut) qui sera passée telle quelle à l'instruction \emph{\textbackslash draw}.

\subsubsection{Dtangent}
\begin{itemize}
    \item La méthode \textbf{g:Dtangent(p,t0,long,draw\_options)} dessine la tangente à la courbe paramétrée par \(p: t \mapsto p(t)\) (à valeurs complexes), au point de paramètre \(t0\). Si l'argument \emph{long} vaut \emph{nil} (valeur par défaut) alors c'est la droite entière qui est dessinée, sinon c'est un segment de longueur \emph{long}. L'argument \emph{draw\_options} est une chaîne (vide par défaut) qui sera passée telle quelle à l'instruction \emph{\textbackslash draw}.
    \item La fonction \textbf{tangent(p,t0,long)} renvoie soit la droite, soit un segment (suivant que \emph{long} vaut \emph{nil} ou pas).
\end{itemize}

\subsubsection{ DtangentC}
\begin{itemize}
    \item La méthode \textbf{g:DtangentC(f,x0,long,draw\_options)} dessine la tangente à la courbe cartésienne d'équation \(y=f(x)\), au point d'abscisse \emph{x0}. Si l'argument \emph{long} vaut \emph{nil} (valeur par défaut) alors c'est la droite entière qui est dessinée, sinon c'est un segment de longueur \emph{long}. L'argument \emph{draw\_options} est une chaîne (vide par défaut) qui sera passée telle quelle à l'instruction \emph{\textbackslash draw}.
    \item La fonction \textbf{tangentC(f,x0,long)} renvoie soit la droite, soit un segment (suivant que \emph{long} vaut \emph{nil} ou pas).
\end{itemize}
  

\begin{demo}{Symétrique de l'orthocentre}
\begin{luadraw}{name=orthocentre}
local g = graph:new{window={-5,5,-5,5},bg="cyan!30",size={10,10}}
local i = cpx.I
local A, B, C = 4*i, -2-2*i, 3.5
local h1, h2 = perp({B,C-B},A), perp({A,B-A},C) -- hauteurs
local A1, F = proj(A,{B,C-B}), proj(C,{A,B-A}) -- projetés
local H = interD(h1,h2) -- orthocentre
local A2 = 2*A1-H -- symétrique de H par rapport à BC
g:Dpolyline({A,B,C},true, "draw=none,fill=Maroon,fill opacity=0.3") -- fond du triangle
g:Linewidth(6); g:Filloptions("full", "blue", 0.2)
g:Dangle(C,A1,A,0.25); g:Dangle(B,F,C,0.25) -- angles droits
g:Linecolor("black"); g:Filloptions("full","cyan",0.5)
g:Darc(H,C,A2,1); g:Darc(B,A,A1,1) -- arcs
g:Filloptions("none","black",1) -- on rétablit l'opacité à 1
g:Dmarkarc(H,C,A1,1,2); g:Dmarkarc(A1,C,A2,1,2) -- marques
g:Dmarkarc(B,A,H,1,2)
g:Linewidth(8); g:Linecolor("black")
g:Dseg({A,B},1.25); g:Dseg({C,B},1.25); g:Dseg({A,C},1.25) -- côtés
g:Linecolor("red"); g:Dcircle(A,B,C) -- cercle
g:Linecolor("blue"); g:Dline(h1); g:Dline(h2) -- hauteurs
g:Dseg({A2,C}); g:Linecolor("red"); g:Dseg({H,A2}) -- segments
g:Dmarkseg(H,A1,2); g:Dmarkseg(A1,A2,2) -- marques
g:Labelcolor("blue") -- pour les labels
g:Dlabel("$A$",A, {pos="NW",dist=0.1}, "$B$",B, {pos="SW"}, "$A_2$",A2,{pos="E"}, "$C$", C, {pos="S" }, "$H$", H, {pos="NE"}, "$A_1$", A1, {pos="SW"})
g:Linecolor("black"); g:Filloptions("full"); g:Ddots({A,B,C,H,A1,A2}) -- dessin des points
g:Show(true)
\end{luadraw}
\end{demo}

\subsection{Figures géométriques}

\subsubsection{Darc}
\begin{itemize}
    \item La méthode \textbf{g:Darc(B,A,C,r,sens,draw\_options)} dessine un arc de cercle de centre \emph{A} (complexe), de rayon \emph{r}, allant de \emph{B} (complexe) vers \emph{C} (complexe) dans le sens trigonométrique si l'argument \emph{sens} vaut 1, le sens inverse sinon. L'argument \emph{draw\_options} est une chaîne (vide par défaut) qui sera passée telle quelle à l'instruction \emph{\textbackslash draw}.
    \item La fonction \textbf{arc(B,A,C,r,sens)} renvoie la liste des points de cet arc (ligne polygonale). 
    \item La fonction \textbf{arcb(B,A,C,r,sens)} renvoie cet arc sous forme d'un chemin (voir Dpath) utilisant des courbes de Bézier.
\end{itemize}

\subsubsection{Dcircle}
\begin{itemize}
    \item La méthode \textbf{g:Dcircle(c,r,d,draw\_options)} trace un cercle. Lorsque l'argument \emph{d} est nil, c'est le cercle de centre \emph{c} (complexe) et de rayon \emph{r}, lorsque \emph{d} est précisé (complexe) alors c'est le cercle passant par les points d'affixe \emph{c},\emph{r} et \emph{d}. L'argument \emph{draw\_options} est une chaîne (vide par défaut) qui sera passée telle quelle à l'instruction \emph{\textbackslash draw}.
  \item La fonction \textbf{circle(c,r,d)} renvoie la liste des points de ce cercle (ligne polygonale). 
  \item La fonction \textbf{circleb(c,r,d)} renvoie ce cercle sous forme d'un chemin (voir Dpath) utilisant des courbes de Bézier.
\end{itemize}

\subsubsection{Dellipse}
\begin{itemize}
    \item  La méthode \textbf{g:Dellipse(c,rx,ry,inclin,draw\_options)} dessine l'ellipse de centre \emph{c} (complexe), les arguments \emph{rx} et \emph{ry} précisent les deux rayons (sur x et sur y), l'argument facultatif \emph{inclin} est un angle en degrés qui indique l'inclinaison de l'ellipse par rapport à l'axe \(Ox\) (angle nul par défaut). L'argument \emph{draw\_options} est une chaîne (vide par défaut) qui sera passée telle quelle à l'instruction \emph{\textbackslash draw}.
  \item La fonction \textbf{ellipse(c,rx,ry,inclin)} renvoie la liste des points de cette ellipse (ligne polygonale). 
  \item La fonction \textbf{ellipseb(c,rx,ry,inclin)} renvoie cette ellipse sous forme d'un chemin (voir Dpath) utilisant des courbes de Bézier.
\end{itemize}

\subsubsection{Dellipticarc}
\begin{itemize}
    \item La méthode \textbf{g:Dellipticarc(B,A,C,rx,ry,sens,inclin,draw\_options)} dessine un arc d'ellipse de centre \emph{A} (complexe) de rayons \emph{rx} et \emph{ry}, faisant un angle égal à \emph{inclin} par rapport à l'axe \(Ox\) (angle nul par défaut), allant de \emph{B} (complexe) vers \emph{A} (complexe) dans le sens trigonométrique si l'argument \emph{sens} vaut 1, le sens inverse sinon. L'argument \emph{draw\_options} est une chaîne (vide par défaut) qui sera passée telle quelle à l'instruction \emph{\textbackslash draw}.
    \item La fonction \textbf{ellipticarc(B,A,C,rx,ry,sens,inclin)} renvoie la liste des points de cet arc (ligne polygonale). 
    \item La fonction \textbf{ellipticarcb(B,A,C,rx,ry,sens,inclin)} renvoie cet arc sous forme d'un chemin (voir Dpath) utilisant des courbes de Bézier.
\end{itemize}

\subsubsection{Dpolyreg}
\begin{itemize}
    \item La méthode \textbf{g:Dpolyreg(sommet1,sommet2,nbcotes,sens,draw\_options)} ou \par \textbf{g:Dpolyreg(centre,sommet,nbcotes,draw\_options)} dessine un polygone régulier. L'argument \emph{draw\_options} est une chaîne (vide par défaut) qui sera passée telle quelle à l'instruction \emph{\textbackslash draw}.
    \item La fonction \textbf{polyreg(sommet1,sommet2,nbcotes,sens)} et la fonction \textbf{polyreg(centre,sommet,nbcotes)}, renvoient la liste des sommets de ce polygone régulier.
\end{itemize}

\subsubsection{Drectangle}
\begin{itemize}
    \item La méthode \textbf{g:Drectangle(a,b,c,draw\_options)} dessine le rectangle ayant comme sommets consécutif \emph{a} et \emph{b} et dont le côté opposé passe par \emph{c}. L'argument \emph{draw\_options} est une chaîne (vide par défaut) qui sera passée telle quelle à l'instruction \emph{\textbackslash draw}.
    \item La fonction \textbf{rectangle(a,b,c)} renvoie la liste des sommets de ce rectangle.
\end{itemize}

\subsubsection{Dsequence}
\begin{itemize}
    \item La méthode \textbf{g:Dsequence(f,u0,n,draw\_options)} fait le dessin des "escaliers" de la suite récurrente définie par son premier terme \emph{u0} et la relation \(u_{k+1}=f(u_k)\). L'argument \emph{f} doit être une fonction d'une variable réelle et à valeurs réelles, l'argument \emph{n} est le nombre de termes calculés. L'argument \emph{draw\_options} est une chaîne (vide par défaut) qui sera passée telle quelle à l'instruction \emph{\textbackslash draw}.
    \item La fonction \textbf{sequence(f,u0,n)} renvoie la liste des points constituant ces "escaliers".
\end{itemize}

\begin{demo}{Suite $u_{n+1}=\cos(u_n)$}
\begin{luadraw}{name=sequence}
local g = graph:new{window={-0.1,1.7,-0.1,1.1},size={10,10,0}}
local i, pi, cos = cpx.I, math.pi, math.cos
local f = function(x) return cos(x)-x end
local ell = solve(f,0,pi/2)[1]
local L = sequence(cos,0.2,5) -- u_{n+1}=cos(u_n), u_0=0.2
local seg, z = {}, L[1]
for k = 2, #L do 
    table.insert(seg,{z,L[k]})
    z = L[k]
end -- seg est la liste des segments de l'escalier
g:Writeln("\\tikzset{->-/.style={decoration={markings, mark=at position #1 with {\\arrow{Stealth}}}, postaction={decorate}}}")
g:Daxes({0,1,1}, {arrows="-Stealth"})
g:DlineEq(1,-1,0,"line width=0.8pt,ForestGreen")
g:Dcartesian(cos, {x={0,pi/2},draw_options="line width=1.2pt,Crimson"})
g:Dpolyline(seg,false,"->-=0.65,blue")
g:Dlabel("$u_0$",0.2,{pos="S",node_options="blue"})
g:Dseg({ell, ell*(1+i)},1,"dashed,gray")
g:Dlabel("$\\ell\\approx"..round(ell,3).."$", ell,{pos="S"})
g:Ddots(ell*(1+i)); g:Labelcolor("Crimson")
g:Dlabel("${\\mathcal C}_{\\cos}$",Z(1,cos(1)),{pos="E"})
g:Labelcolor("ForestGreen"); g:Labelangle(g:Arg(1+i)*180/pi)
g:Dlabel("$y=x$",Z(0.4,0.4),{pos="S",dist=0.1}) 
g:Show()
\end{luadraw}
\end{demo}

La méthode \textbf{g:Arg(z)} calcule et renvoie l'argument \textit{réel} du complexe $z$, c'est à dire son argument (en radians) à l'export dans le repère de tikz (il faut pour cela appliquer la matrice de transformation du graphe à $z$, puis faire le changement de repère vers celui de tikz). Si le repère du graphe est orthonormé et si la matrice de transformation est l'identité alors le résultat est identique à celui de \textbf{cpx.arg(z)} (ce n'est pas le cas dans l'exemple ci-dessus).

De même, la méthode \textbf{g:Abs(z)} calcule et renvoie le module \textit{réel} du complexe $z$, c'est à dire son module à l'export dans le repère de tikz, c'est donc une longueur en centimètres. Si le repère du graphe est orthonormé avec 1cm par unité sur chaque axe, et si la matrice de transformation est une isométrie alors le résultat est identique à celui de \textbf{cpx.abs(z)}.


\subsubsection{Dsquare}
\begin{itemize}
    \item La méthode \textbf{g:Dsquare(a,b,sens,draw\_options)} dessine le carré ayant comme sommets consécutifs \emph{a} et \emph{b}, dans le sens trigonométrique lorsque \emph{sens} vaut 1 (valeur par défaut). L'argument \emph{draw\_options} est une chaîne (vide par défaut) qui sera passée telle quelle à l'instruction \emph{\textbackslash draw}.
  \item La fonction \textbf{square(a,b,sens)} renvoie la liste des sommets de ce carré.
\end{itemize}

\subsubsection{Dwedge}
La méthode \textbf{g:Dwedge(B,A,C,r,sens,draw\_options)} dessine un secteur angulaire de centre \emph{A} (complexe), de rayon \emph{r}, allant de \emph{B} (complexe) vers \emph{C} (complexe) dans le sens trigonométrique si l'argument \emph{sens} vaut 1, le sens inverse sinon. L'argument \emph{draw\_options} est une chaîne (vide par défaut) qui sera passée telle quelle à l'instruction \emph{\textbackslash draw}.


\subsection{Courbes}

\subsubsection{Paramétriques : Dparametric}

\begin{itemize}
\item La fonction \textbf{parametric(p,t1,t2,nbdots,discont,nbdiv)} fait le calcul des points et renvoie une ligne polygonale (pas de dessin).
  \begin{itemize}
    \item L'argument \emph{p} est le paramétrage, ce doit être une fonction d'une variable réelle \emph{t} et à valeurs complexes, par exemple :
    \mintinline{Lua}{local p = function(t) return cpx.exp(t*cpx.I) end}
    \item  Les arguments \emph{t1} et \emph{t2} sont obligatoires avec \(t1 < t2\), ils forment les bornes de l'intervalle pour le paramètre.
    \item L'argument \emph{nbdots} est facultatif, c'est le nombre de points (minimal) à calculer, il vaut 50 par défaut.
    \item L'argument \emph{discont} est un booléen facultatif qui indique s'il y a des discontinuités ou non, c'est \emph{false} par défaut.
    \item L'argument \emph{nbdiv} est un entier positif qui vaut 5 par défaut et indique le nombre de fois que l'intervalle entre deux valeurs consécutives du paramètre peut être coupé en deux (dichotomie) lorsque les points correspondants sont trop éloignés.
  \end{itemize}
  
\item La méthode \textbf{g:Dparametric(p,args)} fait le calcul des points et le dessin de la courbe paramétrée par \emph{p}. Le paramètre \emph{args} est une table à 5 champs :

\begin{TeXcode}
 { t={t1,t2}, nbdots=50, discont=true/false, nbdiv=5, draw_options="" }
\end{TeXcode}

  \begin{itemize}
      \item Par défaut, le champ \emph{t} est égal à \emph{\{g:Xinf(),g:Xsup()\}},
      \item le champ \emph{nbdots} vaut 50, 
      \item le champ \emph{discont} vaut \emph{false},
      \item le champ \emph{nbdiv} vaut 5,
      \item le champ \emph{draw\_options} est une chaîne vide (celle-ci sera transmise telle quelle à l'instruction \emph{\textbackslash draw}).
  \end{itemize}
\end{itemize}  

\subsubsection{Polaires : Dpolar}

\begin{itemize}
\item La fonction \textbf{polar(rho,t1,t2,nbdots,discont,nbdiv)} fait le calcul des points et renvoie une ligne polygonale (pas de dessin). L'argument \emph{rho} est le paramétrage polaire de la courbe, ce doit être une fonction d'une variable réelle \emph{t} et à valeurs
  réelles, par exemple :

    \mintinline{Lua}{local rho = function(t) return 4*math.cos(2*t) end}

    Les autres arguments sont identiques aux courbes paramétrées.
\item La méthode \textbf{g:Dpolar(rho,args)} fait le calcul des points et le  dessin de la courbe polaire paramétrée par \emph{rho}.  Le paramètre \emph{args} est une table à 5 champs :

\begin{TeXcode}
  { t={t1,t2}, nbdots=50, discont=true/false, nbdiv=5, draw_options="" }
\end{TeXcode}

  \begin{itemize}
      \item Par défaut, le champ \emph{t} est égal à $\{-\pi,\pi\}$,
      \item le champ \emph{nbdots} vaut 50,
      \item le champ \emph{discont} vaut \emph{false}, 
      \item le champ \emph{nbdiv} vaut 5, 
      \item le champ \emph{draw\_options} est une chaîne vide (celle-ci sera transmise telle quelle à l'instruction \emph{\textbackslash draw}).
  \end{itemize}
\end{itemize}

\subsubsection{Cartésiennes : Dcartesian}

\begin{itemize}
\item La fonction \textbf{cartesian(f,x1,x2,nbdots,discont,nbdiv)} fait le calcul des points et renvoie une ligne polygonale (pas de dessin). L'argument \emph{f} est la fonction dont on veut la courbe, ce doit être une fonction d'une variable réelle \emph{x} et à valeurs réelles, par exemple :

    \mintinline{Lua}{local f = function(x) return 1+3*math.sin(x)*x end}

    Les arguments \emph{x1} et \emph{x2} sont obligatoires et forment les bornes de l'intervalle pour la variable. Les autres arguments sont identiques aux courbes paramétrées.

\item La méthode \textbf{g:Dcartesian(f,args)} fait le calcul des points et le dessin de la courbe de \emph{f}. Le paramètre \emph{args} est une table à 5 champs :

\begin{TeXcode}
  { x={x1,x2}, nbdots=50, discont=true/false, nbdiv=5, draw_options="" }
\end{TeXcode}
  
  \begin{itemize}
      \item   Par défaut, le champ \emph{x} est égal à \emph{\{g:Xinf(),g:Xsup()\}}, 
      \item le champ \emph{nbdots} vaut 50, 
      \item le champ \emph{discont} vaut \emph{false}, 
      \item le champ \emph{nbdiv} vaut 5, 
      \item le champ \emph{draw\_options} est une chaîne vide (celle-ci sera transmise telle quelle à l'instruction \emph{\textbackslash draw}).
  \end{itemize}
\end{itemize}

\subsubsection{Fonctions périodiques : Dperiodic}

\begin{itemize}
\item La fonction \textbf{periodic(f,period,x1,x2,nbdots,discont,nbdiv)} fait le calcul des points et renvoie une ligne polygonale (pas de dessin).

  \begin{itemize}
    \item L'argument \emph{f} est la fonction dont on veut la courbe, ce doit être une fonction d'une variable réelle \emph{x} et à valeurs réelles.
    \item L'argument \emph{period} est une table du type \emph{\{a,b\}} avec \(a<b\) représentant une période de la fonction \emph{f}.
    \item Les arguments \emph{x1} et \emph{x2} sont obligatoires et forment les bornes de l'intervalle pour la variable.
    \item Les autres arguments sont identiques aux courbes paramétrées.
  \end{itemize}
\item La méthode \textbf{g:Dperiodic(f,period,args)} fait le calcul des points et le dessin de la courbe de \emph{f}. Le paramètre \emph{args} est une table à 5 champs :

\begin{TeXcode}
  { x={x1,x2}, nbdots=50, discont=true/false, nbdiv=5, draw_options="" }
\end{TeXcode}

  \begin{itemize}
      \item Par défaut, le champ \emph{x} est égal à \emph{\{g:Xinf(),g:Xsup()\}},
      \item le champ \emph{nbdots} vaut 50, 
      \item le champ \emph{discont} vaut \emph{false}, 
      \item le champ \emph{nbdiv} vaut 5, 
      \item le champ \emph{draw\_options} est une chaîne vide (celle-ci sera transmise telle quelle à l'instruction \emph{\textbackslash draw}).
  \end{itemize}

\end{itemize}

\subsubsection{Fonctions en escaliers : Dstepfunction}

\begin{itemize}
\item La fonction \textbf{stepfunction(def,discont)} fait le calcul des points et renvoie une ligne polygonale (pas de dessin).

  \begin{itemize}
  \item  L'argument \emph{def} permet de définir la fonction en escaliers, c'est une table à deux champs :

\begin{TeXcode}
  { {x1,x2,x3,...,xn}, {c1,c2,...} }
\end{TeXcode}

  Le premier élément \emph{\{x1,x2,x3,\ldots,xn\}} doit être une subdivision du segment \([x1,xn]\).
  
  Le deuxième élément \emph{\{c1,c2,\ldots\}} est la liste des constantes avec \emph{c1} pour le morceau \emph{{[}x1,x2{]}}, \emph{c2} pour le morceau \emph{{[}x2,x3{]}}, etc.
  
  \item   L'argument \emph{discont} est un booléen qui vaut \emph{true} par défaut.
  \end{itemize}
  
\item La méthode \textbf{g:Dstepfunction(def,args)} fait le calcul des points et le dessin de la courbe de la fonction en escalier.

  \begin{itemize}
  \item L'argument \emph{def} est le même que celui décrit au-dessus (définition de la fonction en escalier).
  \item L'argument \emph{args} est une table à 2 champs : 
  
\begin{TeXcode}
  { discont=true/false, draw_options="" }
\end{TeXcode}

  Par défaut, le champ \emph{discont} vaut true, et le champ \emph{draw\_options} est une chaîne vide (celle-ci sera transmise telle quelle à l'instruction \emph{\textbackslash draw}).
  \end{itemize}
\end{itemize}

\subsubsection{Fonctions affines par morceaux : Daffinebypiece}

\begin{itemize}
\item La fonction \textbf{affinebypiece(def,discont)} fait le calcul des points et renvoie une ligne polygonale (pas de dessin).

  \begin{itemize}
  \item  L'argument \emph{def} permet de définir la fonction en escaliers, c'est une table à deux champs :

\begin{TeXcode}
 { {x1,x2,x3,...,xn}, { {a1,b1}, {a2,b2},...} }
\end{TeXcode}

  Le premier élément \emph{\{x1,x2,x3,\ldots,xn\}} doit être une subdivision du segment \([x1,xn]\).
  
  Le deuxième élément \emph{\{ \{a1,b1\}, \{a2,b2\}, \ldots\}} signifie que sur \emph{{[}x1,x2{]}} la fonction est \(x\mapsto a_1x+b_1\), sur \emph{{[}x2,x3{]}} la fonction est
  \(x\mapsto a_2x+b_2\), etc.
  
  \item L'argument \emph{discont} est un booléen qui vaut \emph{true} par défaut.
  \end{itemize}
  
\item La méthode \textbf{g:Daffinebypiece(def,args)} fait le calcul des points et le dessin de la courbe de la fonction affine par morceaux.

  \begin{itemize}
  \item L'argument \emph{def} est le même que celui décrit au-dessus (définition de la fonction affine par morceaux).
  \item L'argument \emph{args} est une table à 2 champs :
  
\begin{TeXcode}
  { discont=true/false, draw_options="" }
\end{TeXcode}

  Par défaut, le champ \emph{discont} vaut \emph{true}, et le champ \emph{draw\_options} est une chaîne vide (celle-ci sera transmise telle quelle à l'instruction \emph{\textbackslash draw}).
  \end{itemize}
\end{itemize}

\subsubsection{Équations différentielles : Dodesolve}

\begin{itemize}
\item La fonction \textbf{odesolve(f,t0,Y0,tmin,tmax,nbdots,method)} permet une résolution approchée de l'équation différentielle \(Y'(t)=f(t,Y(t))\) dans l'intervalle {[}tmin,tmax{]} qui doit contenir \emph{t0}, avec la condition initiale $Y(t0)=Y0$.

\begin{itemize}
  \item L'argument \emph{f} est une fonction \(f: (t,Y) -> f(t,Y)\) à valeurs dans \(R^n\) et où \emph{Y} est également dans \(R^n\) : \emph{Y=\{y1, y2,\ldots, yn\}} (lorsque $n=1$, \emph{Y} est un réel).
  \item Les arguments \emph{t0} et \emph{Y0} donnent les conditions initiales avec \emph{Y0=\{y1(t0), \ldots, yn(t0)\}} (les yi sont réels), ou \emph{Y0=y1(t0)} lorsque $n=1$.
  \item Les arguments \emph{tmin} et \emph{tmax} définissent l'intervalle de résolution, celui-ci doit contenir \(t0\).
  \item L'argument \emph{nbdots} indique le nombre de points calculés de part et d'autre de \(t0\).
  \item L'argument optionnel \emph{method} est une chaîne qui peut valoir \emph{"rkf45"} (valeur par défaut), ou \emph{"rk4"}. Dans le premier cas, on utilise la méthode de Runge Kutta-Fehlberg (à pas variable), dans le second cas c'est la méthode classique de Runge-Kutta d'ordre 4.
  \item En sortie, la fonction renvoie la matrice suivante (liste de listes de réels) :

\begin{TeXcode}
{ {tmin,...,tmax}, {y1(tmin),...,y1(tmax)}, {y2(tmin),...,y2(tmax)},...}
\end{TeXcode}

  La première composante est la liste des valeurs de \emph{t} (dans l'ordre croissant), la deuxième est la liste des valeurs (approchées) de la composante \emph{y1} correspondant à ces valeurs de \emph{t}, ... etc.
    \end{itemize}
    
\item La méthode \textbf{g:DplotXY(X,Y,draw\_options)}, où les arguments \emph{X} et \emph{Y} sont deux listes de réels de même longueur, dessine la ligne polygonale constituée des points $(X[k],Y[k])$. L'argument \emph{draw\_options} est une chaîne (vide par défaut) qui sera passée telle quelle à l'instruction \emph{\textbackslash draw}.

\begin{demo}{Un système différentiel de Lokta-Volterra}
\begin{luadraw}{name=lokta_volterra}
local g = graph:new{window={-5,50,-0.5,5},size={10,10,0}, border=true}
local i = cpx.I
local f = function(t,y) return {y[1]-y[1]*y[2],-y[2]+y[1]*y[2]} end
g:Labelsize("footnotesize")
g:Daxes({0,10,1},{limits={{0,50},{0,4}}, nbsubdiv={4,0}, legendsep={0.1,0}, originpos={"center","center"}, legend={"$t$",""}})
local y0 = {2,2}
local M = odesolve(f,0,y0,0,50,250) -- résolution approchée
-- M est une table à 3 éléments: t, x et y
g:Lineoptions("solid","blue",8)
g:Dseg({5+3.5*i,10+3.5*i}); g:Dlabel("$x$",10+3.5*i,{pos="E"})
g:DplotXY(M[1],M[2]) -- points (t,x(t))
g:Linecolor("red"); g:Dseg({5+3*i,10+3*i}); g:Dlabel("$y$",10+3*i,{pos="E"})
g:DplotXY(M[1],M[3])  -- points (t,y(t))
g:Lineoptions(nil,"black",4)
g:Saveattr(); g:Viewport(20,50,3,5) -- changement de vue
g:Coordsystem(-0.5,3.25,-0.5,3.25) -- nouveau repère associé
g:Daxes({0,1,1},{legend={"$x$","$y$"},arrows="->"})
g:Lineoptions(nil,"ForestGreen",8); g:DplotXY(M[2],M[3]) -- points (x(t),y(t))
g:Restoreattr() -- retour à l'ancienne vue
g:Dlabel("$\\begin{cases}x'=x-xy\\\\y'=-y+xy\\end{cases}$", 5+4.75*i,{})
g:Show()
\end{luadraw}
\end{demo}
    
\item La méthode \textbf{g:Dodesolve(f,t0,Y0,args)} permet le dessin d'une solution à l'équation \(Y'(t)=f(t,Y(t))\).
  \begin{itemize}
  \item L'argument obligatoire \emph{f} est une fonction \(f: (t,Y) -> f(t,Y)\) à valeurs dans \(R^n\) et où \emph{Y} est également dans \(R^n\) : \emph{Y=\{y1, y2,\ldots, yn\}} (lorsque $n=1$, \emph{Y} est un réel).
  \item Les arguments \emph{t0} et \emph{Y0} donnent les conditions initiales avec \emph{Y0=\{y1(t0), \ldots, yn(t0)\}} (les yi sont réels), ou \emph{Y0=y1(t0)} lorsque $n=1$.
  \item L'argument \emph{args} (facultatif) permet de définir les paramètres pour la courbe, c'est une table à 5 champs : 
  
\begin{TeXcode}
  { t={tmin,tmax}, out={i1,i2}, nbdots=50, method="rkf45"/"rk4", draw_options="" }
\end{TeXcode}  

      \begin{itemize}
        \item Le champ \emph{t} détermine l'intervalle pour la variable \(t\), par défaut il vaut \emph{\{g:Xinf(), g:Xsup()\}}.
        \item Le champ \emph{out} est une table de deux entiers \{i1, i2\}, si \emph{M} désigne la matrice renvoyée par la fonction \emph{odesolve}, les points dessinés auront pour abscisses les M{[}i1{]} et pour ordonnées les M{[}i2{]}. Par défaut on a \emph{i1=1} et \emph{i2=2}, ce qui correspond à la fonction \emph{y1} en fonction de \emph{t}.
        \item Le champ \emph{nbdots} détermine le nombre de points à calculer pour la fonction (50 par défaut).
        \item Le champ \emph{method} détermine la méthode à utiliser, les valeurs possibles sont \emph{"rkf45"} (valeur par défaut), ou \emph{"rk4"}.
        \item Le champ \emph{draw\_options} est une chaîne (vide par défaut) qui sera passée telle quelle à l'instruction \emph{\textbackslash draw}.
      \end{itemize}
  \end{itemize}
\end{itemize}

\subsubsection{Courbes implicites : Dimplicit}

\begin{itemize}
\item La fonction \textbf{implicit(f,x1,x2,y1,y2,grid)} calcule et renvoie une une ligne polygonale constituant la courbe implicite d'équation $f(x,y)=0$ dans le pavé $[x_1,x_2]\times[y_1,y_2]$. Ce pavé est découpé en fonction du paramètre \emph{grid}.

      \begin{itemize}
      \item L'argument obligatoire \emph{f} est une fonction \(f: (x,y) -> f(x,y)\) à valeurs dans \(R\).
      \item Les arguments \emph{x1}, \emph{x2} ,\emph{y1}, \emph{y2} définissent la fenêtre du tracé, qui sera le pavé $[x_1,x_2]\times[y_1,y_2]$, on doit avoir \(x1<x2\) et \(y1<y2\).
      \item L'argument \emph{grid} est une table contenant deux entiers positifs : \{n1,n2\}, le premier entier indique le nombre de subdivisions suivant $x$, et le second le nombre de subdivisions suivant $y$.
      \end{itemize}
  
\item La méthode \textbf{g:Dimplicit(f,args)} fait le dessin de la courbe implicite d'équations $f(x,y)=0$.

      \begin{itemize}
      \item L'argument obligatoire \emph{f} est une fonction \(f: (x,y) -> f(x,y)\) à valeurs dans \(R\).
      \item   L'argument \emph{args} permet de définir les paramètres du tracé, c'est une table à 3 champs :
      
      \begin{TeXcode}
      { view={x1,x2,y1,y2}, grid={n1,n2}, draw_options="" }
      \end{TeXcode}
            \begin{itemize}
                \item Le champ \emph{view} détermine la zone de dessin $[x_1,x_2]\times[y_1,y_2]$.  Par défaut on a \emph{view=\{g:Xinf(), g:Xsup(), g:Yinf(), g:Ysup()\}},
                \item le champ \emph{grid} détermine la grille, ce champ vaut par défaut \emph{\{50,50\}},
                \item le champ \emph{draw\_options} est une chaîne (vide par défaut) qui sera passée telle quelle à l'instruction \emph{\textbackslash draw}.
            \end{itemize}
      \end{itemize}
\end{itemize}

\subsubsection{Courbes de niveau : Dcontour}

La méthode \textbf{g:Dcontour(f,z,args)} fait le dessin de \textbf{lignes de niveau} de la fonction \(f: (x,y) -> f(x,y)\) à valeurs réelles.

\begin{itemize}
  \item L'argument \emph{z} (obligatoire) est la liste des différents niveaux à tracer.
  \item L'argument \emph{args} (facultatif) permet de définir les paramètres du tracé, c'est une table à 4 champs :
  
  \begin{TeXcode}
    { view={x1,x2,y1,y2}, grid={n1,n2}, colors={"color1","color2",...}, draw_options="" }
  \end{TeXcode}

    \begin{itemize}
        \item Le champ \emph{view} détermine la zone de dessin {[}x1,x2{]}x{[}y1,y2{]}, par défaut on a \emph{view=\{g:Xinf(),g:Xsup(), g:Yinf(), g:Ysup()\}}.
        \item Le champ \emph{grid} détermine la grille, par défaut on a \emph{grid=\{50,50\}}.
        \item Le champ \emph{colors} est la liste des couleurs par niveau, par défaut cette liste est vide et c'est la couleur courante de tracé qui est utilisée.
        \item Le champ \emph{draw\_options} est une chaîne (vide par défaut) qui sera passée telle quelle à l'instruction \emph{\textbackslash draw}.
    \end{itemize}
\end{itemize}

\begin{demo}{Exemple avec Dcontour}
\begin{luadraw}{name=Dcontour}
local g = graph:new{window={-1,6.5,-1.5,11},size={10,10,0}}
local i, sin, cos = cpx.I, math.sin, math.cos
local f = function(x,y) return (x+y)/(2+cos(x)*sin(y)) end
local rainbow = {Purple,Indigo,Blue,Green,Yellow,Orange,Red}
local Lz = range(1,10) -- niveaux à tracer
local Colors = getpalette(rainbow,10)
g:Dgradbox({0,5+10*i,1,1},{legend={"$x$","$y$"}, grid=true, title="$z=\\frac{x+y}{2+\\cos(x)\\sin(y)}$"})
g:Linewidth(12); g:Dcontour(f,Lz,{view={0,5,0,10}, colors=Colors})
for k = 1, 10 do
    local y = (2*k+4)/3*i
    g:Dseg({5.25+y,5.5+y},1,"color="..Colors[k])
    g:Labelcolor(Colors[k])
    g:Dlabel("$z="..k.."$",5.5+y,{pos="E"})
end
g:Show()
\end{luadraw}
\end{demo}

\subsection{Domaines liés à des courbes cartésiennes}

\subsubsection{Ddomain1}

La méthode \textbf{g:Ddomain1(f,args)} dessine le contour délimité par la courbe de la fonction \emph{f} sur un intervalle \([a;b]\), l'axe \emph{Ox}, et les droites \(x=a\), \(x=b\).

L'argument \emph{args} (facultatif) permet de définir les paramètres pour la courbe, c'est une table à 5 champs : 

  \begin{TeXcode}
    { x={a,b},  nbdots=50, discont=false, nbdiv=5, draw_options="" }
  \end{TeXcode}
  
    \begin{itemize}
        \item Le champ \emph{x} détermine l'intervalle d'étude, par défaut il vaut \emph{\{g:Xinf(), g:Xsup()\}}.
        \item Le champ \emph{nbdots} détermine le nombre de points à calculer pour la fonction (50 par défaut).
        \item Le champ \emph{discont} indique s'il y a ou non des discontinuité pour la fonction (\emph{false} par défaut).
        \item Le champ \emph{nbdiv} est utilisé dans la méthode de calcul des points de la courbe (5 par défaut).
        \item Le champ \emph{draw\_options} est une chaîne (vide par défaut) qui sera passée telle quelle à l'instruction \emph{\textbackslash draw}.
  
    \end{itemize}
    
\subsubsection{Ddomain2}

La méthode \textbf{g:Ddomain2(f,g,args)} dessine le contour délimité par la courbe de la fonction \emph{f} et la courbe de la fonction \emph{g} sur un intervalle \([a;b]\).

L'argument \emph{args} (facultatif) permet de définir les paramètres pour la courbe, c'est une table à 6 champs : 

\begin{TeXcode}
    { x={a,b}, nbdots=50, discont=false, nbdiv=5, draw_options="" }
\end{TeXcode}

\begin{itemize}
  \item Le champ \emph{x} détermine l'intervalle d'étude, par défaut il vaut \emph{\{g:Xinf(), g:Xsup()\}}.
  \item Le champ \emph{nbdots} détermine le nombre de points à calculer pour la fonction (50 par défaut).
  \item Le champ \emph{discont} indique s'il y a ou non des discontinuité pour la fonction (false par défaut).
  \item Le champ \emph{nbdiv} est utilisé dans la méthode de calcul des points de la courbe (5 par défaut).
  \item Le champ \emph{draw\_options} est une chaîne (vide par défaut) qui sera passée telle quelle à l'instruction \emph{\textbackslash draw}.
\end{itemize}
  
\subsubsection{Ddomain3}

La méthode \textbf{g:Ddomain3(f,g,args)} dessine le contour délimité par la courbe de la fonction \emph{f} et celle de la fonction \emph{g}.

L'argument \emph{args} (facultatif) permet de définir les paramètres pour la courbe, c'est une table à 5 champs : 

\begin{TeXcode}
    { x={a,b}, nbdots=50, discont=false, nbdiv=5, draw_options="" }
\end{TeXcode}

\begin{itemize}
  \item Le champ \emph{x} détermine l'intervalle d'étude, par défaut il vaut \emph{\{g:Xinf(), g:Xsup()\}}.
  \item Le champ \emph{nbdots} détermine le nombre de points à calculer pour la fonction (50 par défaut).
  \item Le champ \emph{discont} indique s'il y a ou non des discontinuité pour la fonction (false par défaut).
  \item Le champ \emph{nbdiv} est utilisé dans la méthode de calcul des points de la courbe (5 par défaut).
  \item Le champ \emph{draw\_options} est une chaîne (vide par défaut) qui sera passée telle quelle à l'instruction \emph{\textbackslash draw}.
\end{itemize}

\begin{demo}{Partie entière, fonctions Ddomain1 et Ddomain3}
\begin{luadraw}{name=courbe}
local g = graph:new{ window={-5,5,-5,5}, bg="", size={10,10} }
local f = function(x) return (x-2)^2-2 end
local h = function(x) return 2*math.cos(x-2.5)-2.25 end
g:Daxes( {0,1,1},{grid=true,gridstyle="dashed", arrows="->"})
g:Filloptions("full","brown",0.3)
g:Ddomain1( math.floor, { x={-2.5,3.5} })
g:Filloptions("none","white",1); g:Lineoptions("solid","red",12)
g:Dstepfunction( {range(-5,5), range(-5,4)},{draw_options="arrows={Bracket-Bracket[reversed]},shorten >=-2pt"})
g:Labelcolor("red")
g:Dlabel("Partie entière",Z(-3,3),{node_options="fill=white"})
g:Ddomain3(f,h,{draw_options="fill=blue,fill opacity=0.6"})
g:Dcartesian(f, {x={0,5}, draw_options="blue"})
g:Dcartesian(h, {x={0,5}, draw_options="green"})
g:Show()
\end{luadraw}
\end{demo}

\subsection{Points (Ddots) et labels (Dlabel)}

\begin{itemize}
\item La méthode pour dessiner un ou plusieurs points est : \textbf{g:Ddots(dots, mark\_options)}.

    \begin{itemize}
    \item L'argument \emph{dots} peut être soit un seul point (donc un complexe), soit une liste (une table) de complexes, soit une liste  de liste de complexes. Les points sont dessinés dans la couleur courante du tracé de lignes.
    \item L'argument \emph{mark\_options} est une chaîne de caractères facultative qui sera passée telle quelle à l'instruction \emph{\textbackslash draw} (modifications locales), exemple :
\begin{TeXcode}
    "color=green, line width=1.2, scale=0.25"
\end{TeXcode}

    \item  Deux méthodes pour modifier globalement l'apparence des points :
        \begin{itemize}
        \item La méthode \textbf{g:Dotstyle(style)} qui définit le style de point, l'argument \emph{style} est une chaîne de caractères qui vaut par défaut \emph{"*"}. Les styles possibles sont ceux de la librairie \emph{plotmarks}.
        \item La méthode \textbf{g:Dotscale(scale)} permet de jouer sur la taille du point, l'argument \emph{scale} est un entier positif qui vaut $1$ par défaut, il sert à multiplier la taille par défaut du point. La largeur courante de tracé de ligne intervient également dans la taille du point. Pour les style de points "pleins" (par exemple le style \emph{triangle*}), le style et la couleur de remplissage courants sont utilisés par la librairie.
        \end{itemize}
    \end{itemize}
    
\item La méthode pour placer un label est : 

\hfil\textbf{g:Dlabel(text1, anchor1, args1, text2, anchor2, args2, ...)}.\hfil

    \begin{itemize}
    \item  Les arguments \emph{text1, text2,...} sont des chaînes de caractères, ce sont les labels.
    \item  Les arguments \emph{anchor1, anchor2,...} sont des complexes représentant les points d'ancrage des labels.
    \item  Les arguments \emph{args1,arg2,...}  permettent de définir localement les paramètres des labels, ce sont des tables à 4 champs :
\begin{TeXcode}
    { pos=nil, dist=0, dir={dirX,dirY,dep}, node_options="" }
\end{TeXcode}
        \begin{itemize}
            \item Le champ \emph{pos} indique la position du label par rapport au point d'ancrage, il peut valoir \emph{"N"} pour nord, \emph{"NE"} pour nord-est, \emph{"NW"} pour nord-ouest, ou encore \emph{"S"}, \emph{"SE"}, \emph{"SW"}. Par défaut, il vaut \emph{center}, et dans ce cas le label est centré sur le point  d'ancrage.
            \item Le champ \emph{dist} est une distance en cm qui vaut $0$ par défaut, c'est la distance entre le label et son point d'ancrage lorsque \emph{pos} n'est pas égal a \emph{center}.
            \item \emph{dir=\{dirX,dirY,dep\}} est la direction de l'écriture (\emph{nil}, valeur par défaut, pour le sens par défaut). Les 3 valeurs \emph{dirX}, \emph{dirY} et \emph{dep} sont trois complexes représentant 3 vecteurs, les deux premiers indiquent le sens de l'écriture, le troisième un déplacement (translation) du label par rapport au point d'ancrage.
            \item L'argument \emph{node\_options} est une chaîne (vide par défaut) destinée à recevoir des options qui seront directement passées à tikz dans l'instruction \emph{node{[}{]}}.
            \item Les labels sont dessinés dans la couleur courante du texte du document, mais on peut changer de couleur avec l'argument \emph{node\_options} en mettant par exemple : \emph{node\_options="color=blue"}.
            
            \textbf{Attention} : les options choisies pour un label s'appliquent aussi aux labels suivants si elles sont inchangées.
        \end{itemize}
  \end{itemize}

Options globales pour les labels :

    \begin{itemize}
        \item la méthode \textbf{g:Labelstyle(position)} permet de préciser la position des labels par rapport aux points d'ancrage. L'argument \emph{position} est une chaîne qui peut valoir : \emph{"N"} pour  nord, \emph{"NE"} pour nord-est, \emph{"NW"} pour nord-ouest, ou  encore \emph{"S"}, \emph{"SE"}, \emph{"SW"}. Par défaut, il
  vaut \emph{center}, et dans ce cas le label est centré sur le point  d'ancrage.
    \item La méthode \textbf{g:Labelcolor(color)} permet de définir la couleur des labels. L'argument \emph{color} est une chaîne représentant une couleur pour tikz. Par défaut l'argument est une chaîne vide ce qui représente la couleur courante du document.
    \item La méthode \textbf{g:Labelangle(angle)} permet de préciser un angle (en degrés) de rotation des labels autour du point d'ancrage, cet angle est nul par défaut.
    \item La méthode \textbf{g:Labelsize(size)} permet de gérer la taille des labels. L'argument \emph{size} est une chaîne qui peut valoir : \emph{"tiny"}, ou \emph{"scriptsize"} ou \emph{"footnotesize"}, etc. Par défaut l'argument est une chaîne vide, ce qui représente la taille \emph{"normalsize"}.
  \end{itemize}
  
  
\item La méthode \textbf{g:Dlabeldot(texte,anchor,args)} permet de placer un label et de dessiner le point d'ancrage en même temps.

    \begin{itemize}
    \item L'argument \emph{texte} est une chaîne de caractères, c'est le label.
    \item L'argument \emph{anchor} est un complexe représentant le point d'ancrage du label.
    \item L'argument \emph{args} (facultatif) permet de définir les paramètres du label et du point, c'est une table à 4 champs :
    
\begin{TeXcode}
    { pos=nil, dist=0, node_options="", mark_options="" }
\end{TeXcode}

    On retrouve les champs identiques à ceux de la méthode \emph{Dlabel}, plus le champ \emph{mark\_options} qui est une chaîne de caractères qui sera passée telle quelle à l'instruction \emph{\textbackslash draw} lors du dessin du point d'ancrage.
    \end{itemize}
\end{itemize}

\subsection{Chemins : Dpath, Dspline et Dtcurve}

\begin{itemize}
\item La fonction \textbf{path( chemin )} renvoie une ligne polygonale contenant les points constituant le \emph{chemin}. Celui-ci est une table de complexes et d'instructions (sous forme de chaînes) par exemple:

\begin{TeXcode}
  { Z(-3,2),-3,-2,"l",0,2,2,-1,"ca",3,Z(3,3),0.5,"la",1,Z(-1,5),Z(-3,2),"b" } 
\end{TeXcode}  
avec :
      \begin{itemize}
      \item \emph{"m"} pour moveto,
      \item \emph{"l"} pour lineto,
      \item \emph{"b"} pour bézier (il faut deux points de contrôles),
      \item \emph{"s"} pour une spline cubique naturelle passant par les points cités,
      \item \emph{"c"} pour cercle (il faut un point et le centre, ou alors trois points),
      \item \emph{"ca"} pour arc de cercle (il faut 3 points, un rayon et un sens),
      \item \emph{"ea"} arc d'ellipse (il faut 3 points, un rayon rx, un rayon ry, un sens, et éventuellement une inclinaison en degrés),
      \item \emph{"e"} pour ellipse (il faut un point, le centre, un rayon rx, un rayon ry, et éventuellement une inclinaison en degrés),
      \item \emph{"cl"} pour close (ferme la composante courante),
      \item \emph{"la"} pour line arc, c'est à dire une ligne aux angles arrondis, (il faut indiquer le rayon juste avant l'instruction \emph{"la"}),
      \item \emph{"cla"} ligne fermée aux angles arrondis (il faut indiquer le rayon juste avant l'instruction \emph{"cla"}).
      \end{itemize}
  
\item La méthode \textbf{g:Dpath(chemin,draw\_options)} fait le dessin du \emph{chemin} (en utilisant au maximum les courbes de Bézier, y compris pour les arcs, les ellipses, etc). L'argument \emph{draw\_options} est une chaîne de caractères qui sera passée directement à l'instruction \emph{\textbackslash draw}.
      \begin{itemize}
      \item L'argument \emph{chemin} a été décrit ci-dessus.
      \item L'argument \emph{draw\_options} est une chaîne (vide par défaut) qui sera passée telle quelle à l'instruction \emph{\textbackslash draw}.
      \end{itemize}
  
\item La fonction \textbf{spline(points,v1,v2)} renvoie sous forme de chemin (à dessiner avec Dpath) la spline cubique passant par les points de l'argument \emph{points} (qui doit être une liste de complexes). Les arguments \emph{v1} et \emph{v2} sont vecteurs tangents imposés aux extrémités (contraintes), lorsque ceux-ci sont égaux à \emph{nil}, c'est une spline cubique naturelle (c'est à dire sans contrainte) qui est calculée.

\item La méthode \textbf{g:Dspline(points,v1,v2,draw\_options)} fait le dessin de la spline décrite ci-dessus. L'argument \emph{draw\_options} est une chaîne de caractères qui sera passée directement à l'instruction \emph{\textbackslash draw}.

\begin{demo}{Path et Spline}
\begin{luadraw}{name=path_spline}
local g = graph:new{window={-5,5,-5,5},size={10,10},bg="Beige"}
local i = cpx.I
local p = {-3+2*i,-3,-2,"l",0,2,2,1,"ca",3,3+3*i,0.5,"la",1,-1+5*i,-3+2*i,"b",-1,"m",0,"c"}
g:Daxes( {0,1,1} )
g:Filloptions("full","blue!30",1,true); g:Dpath(p,"line width=0.8pt")
g:Filloptions("none")
local A,B,C,D,E = -4-i,-3*i,4,3+4*i,-4+2*i
g:Lineoptions(nil,"ForestGreen",12); g:Dspline({A,B,C,D,E},nil,-5*i) -- contrainte en E
g:Ddots({A,B,C,D,E},"fill=white,scale=1.25")
g:Show()
\end{luadraw}
\end{demo}

\item La fonction \textbf{tcurve(L} renvoie sous forme de chemin une courbe passant par des points donnés avec des vecteurs tangents (à gauche et à droite) imposés à chaque point. \emph{L} est une table de la forme :
\begin{Luacode}
L = {point1,{t1,a1,t2,a2}, point2,{t1,a1,t2,a2}, ..., pointN,{t1,a1,t2,a2}}
\end{Luacode}
\emph{point1}, ..., \emph{pointN} sont les points d'interpolation de la courbe (affixes), et chacun d'eux est suivi d'une table de la forme \verb|{t1,a1,t2,a2}| qui précise les vecteurs tangents à la courbe à gauche du point (avec \emph{t1} et \emph{a1}) et à droite du point (avec \emph{t2} et \emph{a2}). Le vecteur tangent à gauche est donné par la formule $V_g = t_1\times e^{ia_1\pi/180}$, donc $t1$ représente le module et $a1$ est un argument \textbf{en degrés} de ce vecteur. C'est la même chose avec \emph{t2} et \emph{a2} pour le vecteur tangent à droite, \textbf{mais ceux-ci sont facultatifs}, et s'ils ne sont pas précisés alors ils prennent les mêmes valeurs que \emph{t1} et \emph{a1}.

Deux points consécutifs seront reliés par une courbe de Bézier, la fonction calcule les points de contrôle pour avoir les vecteurs tangents souhaités.

\item La méthode \textbf{g:Dtcurve(L,options)} fait le dessin du chemin obtenu par \emph{tcurve} décrit ci-dessus. L'argument \emph{options} est une table à deux champs:
    \begin{itemize}
        \item \emph{showdots=true/false} (false par défaut), cette option permet de dessiner les points d'interpolation donnés ainsi que les points de contrôles calculés, ce qui permet une visualisation des contraintes.
        \item \emph{draw\_options=""}, c'est une chaîne de caractères qui sera passée directement à l'instruction \emph{\textbackslash draw}.
    \end{itemize}
\end{itemize}

\begin{demo}{Courbe d'interpolation avec vecteurs tangents imposés}
\begin{luadraw}{name=tcurve}
local g = graph:new{window={-0.5,10.5,-0.5,6.5},size={10,10,0}}
local i = cpx.I
local L = {
    1+4*i,{2,-20},
    2+3*i,{2,-70},
    4+i/2,{3,0},
    6+3*i,{4,15},
    8+6*i,{4,0,4,-90}, -- point anguleux
    10+i,{3,-15}}
g:Dgrid({0,10+6*i},{gridstyle="dashed"})
g:Daxes(nil,{limits={{0,10},{0,6}},originpos={"center","center"}, arrows="->"})
g:Dtcurve(L,{showdots=true,draw_options="line width=0.8pt,red"})
g:Show()
\end{luadraw}
\end{demo}


\subsection{Axes et grilles}

Varaibles globales utilisées pour les axes et les grilles :
\begin{itemize}
    \item \emph{maxGrad = 100} : nombre max de graduations sur un axe.
    \item \emph{defaultlabelshift = 0.125} : lorsqu'une grille est dessinée avec les axes (option \emph{grid=true}) les labels sont automatiquement décalés le long de l'axe avec cette variable.
    \item \emph{defaultxylabelsep = 0} : définit la distance par défaut entre les labels et les graduations.
    \item \emph{defaultlegendsep = 0.2} : définit la distance par défaut entre la légende et l'axe.
    \item \emph{dollar = true} :  pour ajouter des dollars autour des labels des graduations.
\end{itemize}

\subsubsection{Daxes}
\def\opt#1{\textcolor{blue}{\texttt{#1}}}%
Le tracé des axes s'obtient avec la méthode \textbf{g:Daxes( \{A,xpas,ypas\}, options)}.
\begin{itemize}
    \item Le premier argument précise le point d'intersection des deux axes (c'est le complexe \emph{A}), le pas des graduations sur l'axe $Ox$ (c'est \emph{xpas}) et le pas des graduations sur $Oy$ (c'est \emph{ypas}). Par défaut le point \emph{A} est l'origine $Z(0,0)$, et les deux pas sont égaux à $1$.
    \item L'argument \emph{options} est une table précisant les options possibles. Voici ces options avec leur valeur par défaut :
        \begin{itemize}
            \item \opt{showaxe=\{1,1\}}. Cette option précise si les axes doivent être tracés ou pas ($1$ ou $0$). La première valeur est pour l'axe $Ox$ et la seconde pour l'axe $Oy$.
            \item \opt{arrows="-"}. Cette option permet d'ajouter ou non une flèche aux axes (pas de flèche par défaut, mettre "->" pour ajouter une flèche).
            \item \opt{limits=\{"auto","auto"\}}. Cette option permet de préciser l'étendue des deux axes (première valeur pour $Ox$, seconde valeur pour $Oy$). La valeur "auto" signifie que c'est la droite en entier, mais on peut préciser les abscisses extrêmes, par exemple : \opt{limits=\{\{-4,4\},"auto"\}}.
            \item \opt{gradlimits=\{"auto","auto"\}}. Cette option permet de préciser l'étendue des graduations sur les deux axes (première valeur pour $Ox$, seconde valeur pour $Oy$). La valeur "auto" signifie que c'est la droite en entier, mais on peut préciser les graduations extrêmes, par exemple : \opt{gradlimits=\{\{-4,4\},\{-2,3\}\}}.
            \item \opt{unit=\{"",""\}}. Cette option permet de préciser de combien en combien vont les graduations sur les axes. La valeur par défaut ("") signifie qu'il faut prendre la valeur du pas (\emph{xpas} sur $Ox$, ou \emph{ypas} sur $Oy$), SAUF lorsque l'option \opt{labeltext} n'est pas la chaîne vide, dans ce cas \emph{unit} prend la valeur $1$.
            \item \opt{nbsubdiv=\{0,0\}}. Cette option permet de préciser le nombre de subdivisions entre deux graduations principales sur l'axe.
            \item \opt{tickpos=\{0.5,0.5\}}. Cette option précise la position des graduations par rapport à chaque axe, ce sont deux nombres entre $0$ et $1$, la valeur par défaut de $0.5$ signifie qu'ils sont centrés sur l'axe. ($0$ et $1$ représentent les extrémités).
            \item \opt{tickdir=\{"auto","auto"\}}. Cette option indique la direction des graduations sur l'axe. Cette direction est un vecteur (complexe) non nul. La valeur par défaut "auto" signifie que les graduations sont orthogonales à l'axe.
            \item \opt{xyticks=\{0.2,0.2\}}. Cette option précise la longueur des graduations sur l'axe.
            \item \opt{xylabelsep=\{0,0\}}. Cette option précise la distance entre les labels et les graduations sur l'axe.
            \item \opt{originpos=\{"right","top"\}}. Cette option précise la position du label à l'origine sur l'axe, les valeurs possibles sont : "none", "center", "left", "right" pour $Ox$, et "none", "center", "bottom", "top" pour $Oy$.
            \item \opt{originnum=\{A.re,A.im\}}. Cette option précise la valeur de la graduation au croisement des axes (graduation numéro $0$). 
            
            La formule qui définit le label à la graduation numéro $n$ est : \textbf{(originnum + unit*n)"labeltext"/labelden}.
            \item \opt{originloc=A}. Cette option précise le point de croisement des axes.
            \item \opt{legend=\{"",""\}}. Cette option permet de préciser une légende pour l'axe.
            \item \opt{legendpos=\{0.975,0.975\}}. Cette option précise la position (entre $0$ et $1$) de la légende par rapport à chaque axe.
            \item \opt{legendsep=\{0.2,0.2\} }. Cette option précise la distance entre la légende et l'axe. La légende est de l'autre côté de l'axe par rapport aux graduations.
            \item \opt{legendangle=\{"auto","auto"\}}. Cette option précise l'angle (en degrés) que doit faire la légende pour l'axe. La valeur "auto" par défaut signifie que la légende doit être parallèle à l'axe si l'option \emph{labelstyle} est aussi à "auto", sinon la légende est horizontale.
            \item \opt{labelpos=\{"bottom","left"\}}.Cette option précise la position des labels par rapport à l'axe. Pour l'axe $Ox$, les valeurs possibles sont : "none", "bottom" ou "top", pour l'axe $Oy$ c'est :  "none", "right" ou "left".
            \item \opt{labelden=\{1,1\}}. Cette option précise le dénominateur des labels (entier) pour l'axe. La formule qui définit le label à la graduation numéro $n$ est : \textbf{(originnum + unit*n)"labeltext"/labelden}.
            \item \opt{labeltext=\{"",""\}}. Cette option définit le texte qui sera ajouté au numérateur des labels pour l'axe.
            \item \opt{labelstyle=\{"S","W"\}}. Cette option définit le style des labels pour chaque axe. Les valeurs possibles sont "auto","N", "NW", "W", "SW", "S", "SE", "E".
            \item \opt{labelangle=\{0,0\}}. Cette option définit pour chaque axe l'angle des labels en degrés par rapport à l'horizontale.
            \item \opt{labelcolor=\{"",""\}}. Cette option permet de choisir une couleur pour les labels sur chaque axe. La chaîne vide représente la couleur par défaut.
            \item \opt{labelshift=\{0,0\}}. Cette option permet de définir un décalage systématique pour les labels sur l'axe (décalage de long de l'axe).
            \item \opt{nbdeci=\{2,2\}}. Cette option précise le nombre de décimales pour les valeurs numériques sur l'axe.
            \item \opt{numericFormat=\{0,0\}}. Cette option précise le type d'affiche numérique (non encore implémenté).
            \item \opt{myxlabels=""}. Cette option permet d'imposer des labels personnels sur l'axe $Ox$. Lorsqu'il y en a, la valeur passée à l'option doit être une liste du type : \verb|{pos1,"text1", pos2,"text2",...}|. Le nombre \emph{pos1} représente une abscisse dans le repère (A,xpas), ce qui correspond au point d'affixe $A+$pos1$*$xpas.
            \item \opt{myylabels=""}. Cette option permet d'imposer des labels personnels sur l'axe $Oy$. Lorsqu'il y en a, la valeur passée à l'option doit être une liste du type : \verb|{pos1,"text1", pos2,"text2",...}|. Le nombre \emph{pos1} représente une abscisse dans le repère (A,i*ypas), ce qui correspond au point d'affixe $A+$pos1$*$ypas$*i$.
            \item \opt{grid=false}. Cette option permet d'ajouter ou non une grille.
            \item \opt{drawbox=false}. Cette option de dessiner les axes sous la forme d'une boite, dans ce cas, les graduations sont sur le côté gauche et le côté bas.
            \item \opt{gridstyle="solid"}. Cette option définit le style ligne pour la grille principale.
            \item \opt{subgridstyle="solid"}. Cette option définit le style ligne pour la grille secondaire. Une grille secondaire apparaît lorsqu'il y a des subdivisions sur un des axes.
            \item  \opt{gridcolor="gray"}. Ceci définit la couleur de la grille principale.
            \item \opt{subgridcolor="lightgray"}. Ceci définit la couleur de la grille secondaire.
            \item \opt{gridwidth=4}. Épaisseur de trait de la grille principale (ce qui fait 0.4pt).
            \item \opt{subgridwidth=2}. Épaisseur de trait de la grille secondaire (ce qui fait 0.2pt).
        \end{itemize}
\end{itemize}

\begin{demo}{Exemple avec axes avec grille}
\begin{luadraw}{name=axes_grid}
local g = graph:new{window={-6.5,6.5,-3.5,3.5}, size={10,10,0}}
local i, pi, a = cpx.I, math.pi, math.sqrt(2)
local f = function(x) return 2*a*math.sin(x) end
g:Labelsize("footnotesize"); g:Linewidth(8)
g:Daxes({0,pi/2,a},{labeltext={"\\pi","\\sqrt{2}"}, labelden={2,1},nbsubdiv={1,1},grid=true,arrows="->"})
g:Lineoptions("solid","Crimson",12); g:Dcartesian(f, {x={-2*pi,2*pi}})
g:Show()
\end{luadraw}
\end{demo}

\subsubsection{DaxeX et DaxeY}

Les méthodes \textbf{g:DaxeX(\{A,xpas\}, options)} et \textbf{g:DaxeY(\{A,ypas\}, options)} permettent de tracer les axes séparément.
\begin{itemize}
    \item Le premier argument précise le point servant d'origine (c'est le complexe \emph{A}) et le pas des graduations sur l'axe. Par défaut le point \emph{A} est l'origine $Z(0,0)$, et le pas est égal à $1$.
    \item L'argument \emph{options} est une table précisant les options possibles. Voici ces options avec leur valeur par défaut :
        \begin{itemize}
            \item \opt{showaxe=1}. Cette option précise si l'axe doit être tracé ou non ($1$ ou $0$).
            \item \opt{arrows="-"}. Cette option permet d'ajouter ou non une flèche à l'axe (pas de flèche par défaut, mettre "->" pour ajouter une flèche).
            \item \opt{limits="auto"}. Cette option permet de préciser l'étendue des deux axes. La valeur "auto" signifie que c'est la droite en entier, mais on peut préciser les abscisses extrêmes, par exemple : \opt{limits=\{-4,4\}}.
            \item \opt{gradlimits="auto"}. Cette option permet de préciser l'étendue des graduations sur les deux axes. La valeur "auto" signifie que c'est la droite en entier, mais on peut préciser les graduations extrêmes, par exemple : \opt{gradlimits=\{-2,3\}}.
            \item \opt{unit=""}. Cette option permet de préciser de combien en combien vont les graduations sur l'axe. La valeur par défaut ("") signifie qu'il faut prendre la valeur du pas, SAUF lorsque l'option \opt{labeltext} n'est pas la chaîne vide, dans ce cas \emph{unit} prend la valeur $1$.
            \item \opt{nbsubdiv=0}. Cette option permet de préciser le nombre de subdivisions entre deux graduations principales.
            \item \opt{tickpos=0.5}. Cette option précise la position des graduations par rapport à l'axe, ce sont deux nombres entre $0$ et $1$, la valeur par défaut de $0.5$ signifie qu'ils sont centrés sur l'axe. ($0$ et $1$ représentent les extrémités).
            \item \opt{tickdir="auto"}. Cette option indique la direction des graduations sur l'axe. Cette direction est un vecteur (complexe) non nul. La valeur par défaut "auto" signifie que les graduations sont orthogonales à l'axe.
            \item \opt{xyticks=0.2}. Cette option précise la longueur des graduations.
            \item \opt{xylabelsep=0}. Cette option précise la distance entre les labels et les graduations.
            \item \opt{originpos="center"}. Cette option précise la position du label à l'origine sur l'axe, les valeurs possibles sont : "none", "center", "left", "right" pour $Ox$, et "none", "center", "bottom", "top" pour $Oy$.
            \item \opt{originnum=A.re} pour $Ox$ et \opt{originnum=A.im} pour $Oy$. Cette option précise la valeur de la graduation à l'origine (graduation numéro $0$). 
            
            La formule qui définit le label à la graduation numéro $n$ est : \textbf{(originnum + unit*n)"labeltext"/labelden}.

            \item \opt{legend=""}. Cette option permet de préciser une légende pour l'axe.
            \item \opt{legendpos=0.975}. Cette option précise la position (entre $0$ et $1$) de la légende par rapport à l'axe.
            \item \opt{legendsep=0.2}. Cette option précise la distance entre la légende et l'axe. La légende est de l'autre côté de l'axe par rapport aux graduations.
            \item \opt{legendangle="auto"}. Cette option précise l'angle (en degrés) que doit faire la légende pour l'axe. La valeur "auto" par défaut signifie que la légende doit être parallèle à l'axe si l'option \emph{labelstyle} est aussi à "auto", sinon la légende est horizontale.
            \item \opt{labelpos="bottom"} pour $Ox$ et \opt{labelpos="left"} pour $Oy$. Cette option précise la position des labels par rapport à l'axe. Pour l'axe $Ox$, les valeurs possibles sont : "none", "bottom" ou "top", pour l'axe $Oy$ c'est :  "none", "right" ou "left".
            \item \opt{labelden=1}. Cette option précise le dénominateur des labels (entier) pour l'axe. La formule qui définit le label à la graduation numéro $n$ est : \textbf{(originnum + unit*n)"labeltext"/labelden}.
            \item \opt{labeltext=""}. Cette option définit le texte qui sera ajouté au numérateur des labels.
            \item \opt{labelstyle="S"} pour $Ox$ et \opt{labelstyle="W"} pour $Oy$. Cette option définit le style des labels. Les valeurs possibles sont "auto","N", "NW", "W", "SW", "S", "SE", "E".
            \item \opt{labelangle=0}. Cette option définit l'angle des labels en degrés par rapport à l'horizontale.
            \item \opt{labelcolor=""}. Cette option permet de choisir une couleur pour les labels. La chaîne vide représente la couleur courante du texte.
            \item \opt{labelshift=0}. Cette option permet de définir un décalage systématique pour les labels sur l'axe (décalage de long de l'axe).
            \item \opt{nbdeci=2}. Cette option précise le nombre de décimales pour les labels numériques.
            \item \opt{numericFormat=0}. Cette option précise le type d'affiche numérique (non encore implémenté).
            \item \opt{mylabels=""}. Cette option permet d'imposer des labels personnels. Lorsqu'il y en a, la valeur passée à l'option doit être une liste du type : \verb|{pos1,"text1", pos2,"text2",...}|. Le nombre \emph{pos1} représente une abscisse dans le repère (A,xpas) pour $Ox$, ou (A,ypas$*$i) pour $Oy$, ce qui correspond au point d'affixe $A+$pos1$*$xpas pour $Ox$, et $A+$pos1$*$ypas$*i$ pour $Oy$.
        \end{itemize}
\end{itemize}

\subsubsection{Dgrid}

La méthode \textbf{g:Dgrid(\{A,B\},options} permet le dessin d'une grille.
\begin{itemize}
    \item Le premier argument est obligatoire, il précise le coin inférieur gauche (c'est le complexe \emph{A}), le coin supérieur droit (c'est le complexe $B$) de la grille.
    \item L'argument \emph{options} est une table précisant les options possibles. Voici ces options avec leur valeur par défaut :
        \begin{itemize}
            \item \opt{unit=\{1,1\}}. Cette option définit les unités sur les axes pour la grille principale.
            \item \opt{gridwidth=4}. Cette option définit l'épaisseur du trait de la grille principale (0.4pt par défaut).
            \item \opt{gridcolor="gray"}. Couleur grille de la grille principale.
            \item \opt{gridstyle="solid"}. Style de trait pour la grille principale.
            \item \opt{nbsubdiv=\{0,0\}}. Nombre de subdivisions (pour chaque axe) entre deux traits de la grille principale. Ces subdivisions déterminent la grille secondaire.
            \item \opt{subgridcolor="lightgray"}. Couleur de la grille secondaire.
            \item \opt{subgridwidth=2}. Épaisseur du trait de la grille secondaire (0.2pt par défaut).
            \item \opt{subgridstyle="solid"}. Style de trait pour la grille secondaire.
            \item \opt{originloc=A}. Localisation de l'origine de la grille.
        \end{itemize}
\end{itemize}

\paragraph{Exemple : } il est possible de travailler dans un repère non orthogonal. Voici un exemple où l'axe $Ox$ est conservé, mais la première bissectrice devient le nouvel axe $Oy$, on modifie pour cela la matrice de transformation du graphe. À partir de cette modification les affixes représentent les coordonnées dans le nouveau repère.
\begin{demo}{Exemple de repère non orthogonal}
\begin{luadraw}{name=axes_non_ortho}
local g = graph:new{window={-5.25,5.25,-4,4},size={10,10}}
local i, pi = cpx.I, math.pi
local f = function(x) return 2*math.sin(x) end
g:Setmatrix({0,1,1+i}); g:Labelsize("small")
g:Dgrid({-5-4*i,5+4*i},{gridstyle="dashed"})
g:Daxes({0,1,1}, {arrows="-Stealth"})
g:Lineoptions("solid","ForestGreen",12); g:Dcartesian(f,{x={-5,5}})
g:Dcircle(0,3,"Crimson")
g:DlineEq(1,0,3,"Navy") -- droite d'équation x=-3
g:Lineoptions("solid","black",8); g:DtangentC(f,pi/2,1.5,"<->")
g:Dpolyline({pi/2,pi/2+2*i,2*i},"dotted")
g:Ddots(Z(pi/2,2))
g:Dlabeldot("$\\frac{\\pi}2$",pi/2,{pos="SW"})
g:Show()
\end{luadraw}
\end{demo}

\subsubsection{Dgradbox}

La méthode \textbf{g:Dgradbox(\{A,B,xpas,ypas\},options} permet le dessin d'une boîte graduée.
\begin{itemize}
    \item Le premier argument est obligatoire, il précise le coin inférieur gauche (c'est le complexe \emph{A}) et le coin supérieur droit (c'est le complexe $B$) de la boîte, ainsi que le pas sur chaque axe.
    \item L'argument \emph{options} est une table précisant les options possibles. Ce sont les mêmes que pour les axes, mises à part certaines valeurs par défaut. À celles-ci s'ajoute l'option suivante : \opt{title=""} qui permet d'ajouter un titre en haut de la boite, attention cependant à laisser suffisamment de place pour cela.
\end{itemize}

\begin{demo}{Utilisation de Dgradbox}
\begin{luadraw}{name=gradbox}
local g = graph:new{window={-5,4,-5.5,5},size={10,10}}
local i, pi = cpx.I, math.pi
local h = function(x) return x^2/2-2 end
local f = function(x) return math.sin(3*x)+h(x) end
g:Dgradbox({-pi-4*i,pi+4*i,pi/3,1},{grid=true,originloc=0, originnum={0,0},labeltext={"\\pi",""},labelden={3,1}, title="\\textbf{Title}",legend={"Legend $x$","Legend $y$"}})
g:Saveattr(); g:Viewport(-pi,pi,-4,4) -- on limite la vue (clip)
g:Filloptions("full","blue",0.6); g:Linestyle("noline"); g:Ddomain2(f,h,{x={-pi/2,2*pi/3}})
g:Filloptions("none",nil,1); g:Lineoptions("solid",nil,8); g:Dcartesian(h,{x={-pi,pi}, draw_options="DarkBlue"})
g:Dcartesian(f,{x={-pi,pi},draw_options="Crimson"})
g:Restoreattr()
g:Show()
\end{luadraw}
\end{demo}

\subsection{Calculs sur les couleurs}

Dans l'environnement \emph{luadraw} les couleurs sont des chaînes de caractères qui doivent correspondre à des couleurs connues de tikz. Le package \emph{xcolor} est fortement conseillé pour ne pas être limité aux couleurs de bases.

Afin de pouvoir faire des manipulations sur les couleurs, celles-ci ont été définies (dans le module \emph{luadraw\_colors.lua}) sous la forme de tables de trois composantes : rouge, vert, bleu, chaque composante étant un nombre entre $0$ et $1$, et avec leur nom au format \emph{svgnames} du package \emph{xcolor}, par exemple on y trouve (entre autres) les déclarations :
\begin{Luacode}
AliceBlue = {0.9412, 0.9725, 1}
AntiqueWhite = {0.9804, 0.9216, 0.8431}
Aqua = {0.0, 1.0, 1.0}
Aquamarine = {0.498, 1.0, 0.8314}
\end{Luacode}
On pourra se référer à la documentation de \emph{xcolor} pour avoir la liste de ces couleurs.

Pour utiliser celles-ci dans l'environnement \emph{luadraw}, on peut :
\begin{itemize}
    \item soit les utiliser avec leur nom si on a déclaré dans le préambule : \verb|\usepackage[svgnames]{xcolor}|, par exemple : \mintinline{Lua}{g:Linecolor("AliceBlue")},
    \item soit les utiliser avec la fonction \textbf{rgb()} de \emph{luadraw}, par exemple : \mintinline{Lua}{g:Linecolor(rgb(AliceBlue))}. Par contre, avec cette fonction \emph{rgb()}, pour changer localement de couleur il faut faire comme ceci (exemple) : \par
    \mintinline{Lua}{g:Dpolyline(L,"color="..rgb(AliceBlue))}, ou \mintinline{Lua}{g:Dpolyline(L,"fill="..rgb(AliceBlue))}. Car la fonction \emph{rgb()} ne renvoie pas un nom de couleur, mais une définition de couleur.
\end{itemize}

\paragraph{Fonctions pour la gestion des couleurs :}
\begin{itemize}
    \item La fonction \textbf{rgb(r,g,b)} ou \textbf{rgb(\{r,g,b\})}, renvoie la couleur sous forme d'une chaîne de caractères compréhensible par tikz dans les options \verb|color=...| et \verb|fill=...|. Les valeurs de $r$, $g$ et $b$ doivent être entre $0$ et $1$.
    
    \item La fonction \textbf{hsb(h,s,b,table)} renvoie la couleur sous forme d'une chaîne de caractères compréhensible par tikz. L'argument $h$ (hue) doit être un nombre enter $0$ et $360$, l'argument $s$ (saturation) doit être entre $0$ et $1$, et l'argument $b$ (brightness) doit être aussi entre $0$ et $1$.
    L'argument (facultatif) \emph{table} est un booléen (false par défaut) qui indique si le résultat doit être renvoyé sous forme de table \verb|{r,g,b}| ou non (par défaut c'est sous forme d'une chaîne).
    
    \item La fonction \textbf{mixcolor(color1,proportion1 color2,proportion1,...,colorN,proportionN)} mélange les couleurs \emph{color1}, ...,\emph{colorN} dans les proportions demandées et renvoie la couleur qui en résulte sous forme d'une chaîne de caractères compréhensible par tikz, suivie de cette même couleur sous forme de table \verb|{r,g,b}| . Chacune des couleurs doit être une table de trois composantes \verb|{r,g,b}|.
    
    \item La fonction \textbf{palette(colors,pos,table)} : l'argument \emph{colors} est une liste (table) de couleurs au format \verb|{r,b,g}|, l'argument \emph{pos} est un nombre entre $0$ et $1$, la valeur $0$ correspond à la première couleur de la liste et la valeur $1$ à la dernière. La fonction calcule et renvoie la couleur correspondant à la position \emph{pos} dans la liste par interpolation linéaire. L'argument (facultatif) \emph{table} est un booléen (false par défaut) qui indique si le résultat doit être renvoyé sous forme de table \verb|{r,g,b}| ou non (par défaut c'est sous forme d'une chaîne).
    
    \item La fonction \textbf{getpalette(colors,nb,table)} : l'argument \emph{colors} est une liste (table) de couleurs au format \verb|{r,b,g}|, l'argument \emph{nb} indique le nombre de couleurs souhaité. La fonction renvoie une liste de \emph{nb} couleurs régulièrement réparties dans \emph{colors}. L'argument (facultatif) \emph{table} est un booléen (false par défaut) qui indique si les couleurs sont renvoyées sous forme de tables \verb|{r,g,b}| ou non (par défaut c'est sous forme de chaînes).  
    
    \item La méthode \textbf{g:Newcolor(name,rgbtable)} permet de définir dans l'export tikz au format rgb une nouvelle couleur dont le nom sera \emph{name} (chaîne), \emph{rgbtable} est une table de trois composantes : rouge, vert, bleu (entre 0 et 1) définissant cette couleur.
    
\end{itemize}
On peut également utiliser toutes les possibilités habituelles de tikz pour la gestion des couleurs.

\begin{demo}{Utilisation de la fonction palette()}
\begin{luadraw}{name=palette}
local g = graph:new{window={-5,5,-1,1},size={10,10},
      margin={0.1,0.1,0.1,0.1},border=true}
local i = cpx.I
local colors = {Purple,Indigo,Blue,Green,Yellow,Orange,Red}
local N = 200
g:Linewidth(18)
for k = 1, N do
    local pos = (k-1)/(N-1)
    local x = -5+10*pos
    g:Dpolyline({x-i,x+i},"color="..palette(colors,pos))
end
g:Show()
\end{luadraw}
\end{demo}

\section{Calculs sur les listes}

\subsection{concat}
La fonction \textbf{concat\{table1, table2, \ldots{} \}} concatène toutes les tables passées en argument, et renvoie la table qui en résulte.

\begin{itemize}
 \item Chaque argument peut être un réel un complexe ou une table.
\item Exemple : l'instruction \mintinline{Lua}{concat( 1,2,3,{4,5,6},7 )} renvoie la table \emph{\{1,2,3,4,5,6,7\}}.
\end{itemize}

\subsection{cut}
La fonction \textbf{cut(L,A,before)} permet de couper la ligne polygonale \emph{L} au point \emph{A} qui est sensé être situé sur la ligne \emph{L}. Si l'argument \emph{before} vaut \emph{true}, c'est la partie située avant \emph{A} qui sera coupée, et la partie située après \emph{A} qui sera renvoyée, sinon c'est l'inverse (\emph{before} vaut \emph{false} par défaut). Le résultat est une ligne polygonale (liste de listes de complexes).

\subsection{getbounds}
\begin{itemize}
    \item La fonction \textbf{getbounds(L)} renvoie les bornes xmin,xmax,ymin,ymax de la ligne polygonale \emph{L}.
    \item Exemple : \mintinline{Lua}{ local xmin, xmax, ymin, ymax = getbounds(L)} (où \emph{L} désigne une ligne polygonale).
\end{itemize}

\subsection{getdot}
La fonction \textbf{getdot(x,L)} renvoie le point d'abscisse \emph{x} (réel entre $0$ et $1$) le long de la composante connexe \emph{L} (liste de complexes). L'abscisse $0$ correspond au premier point et l'abscisse $1$ au dernier, plus généralement, \emph{x} correspond à un pourcentage de la longueur de \emph{L}.

\subsection{insert}
La fonction \textbf{insert(table1, table2, pos)} insère les éléments de \emph{table2} dans \emph{table1} à la position \emph{pos}.

\begin{itemize}
    \item L'argument \emph{table2} peut être un réel, un complexe ou une table.
    \item L'argument \emph{table1} doit être une variable qui désigne une table, celle-ci sera modifiée par la fonction.
    \item Si l'argument \emph{pos} vaut \emph{nil}, l'insertion se fait à la fin de \emph{table1}.
    \item Exemple : si une variable \emph{L} vaut \emph{\{1,2,6\}}, alors après l'instruction \mintinline{Lua}{insert(L, {3,4,5},3)}, la variable \emph{L} sera égale à \emph{\{1,2,3,4,5,6\}}.
\end{itemize}

\subsection{interD}
La fonction \textbf{interD(d1,d2)} renvoie le point d'intersection des droites \emph{d1} et \emph{d2}, une droite est une liste de deux complexes : un point de la droite et un vecteur directeur.

\subsection{interDL}
La fonction \textbf{interDL(d,L)} renvoie la liste des points d'intersection entre la droite \emph{d} et la ligne polygonale \emph{L}.

\subsection{interL}
La fonction \textbf{interL(L1,L2)} renvoie la liste des points d'intersection des lignes polygonales définies par \emph{L1} et \emph{L2}, ces deux arguments sont deux listes de complexes ou deux listes de listes de complexes).

\subsection{interP}
La fonction \textbf{interP(P1,P2)} renvoie la liste des points d'intersection des chemins définis par \emph{P1} et \emph{P2}, ces deux arguments sont deux listes de complexes et d'instructions (voir \emph{Dpath}).

\subsection{linspace}
La fonction \textbf{linspace(a,b,nbdots)} renvoie une liste de \emph{nbdots} nombres équirépartis de \emph{a} jusqu'à \emph{b}. Par défaut \emph{nbdots} vaut 50.

\subsection{map}
La fonction \textbf{map(f,list)} applique la fonction \emph{f} à chaque élément de la \emph{list} et renvoie la table des résultats. Lorsqu'un résultat vaut \emph{nil}, c'est le complexe \emph{cpx.Jump} qui est inséré dans la liste.

\subsection{merge}
La fonction \textbf{merge(L)} recolle si c'est possible, les composantes connexes de \emph{L} qui doit être une liste de listes de complexes, la fonction renvoie le résultat.

\subsection{range}
La fonction \textbf{range(a,b,step)} renvoie la liste des nombres de \emph{a} jusqu'à \emph{b} avec un pas égal à \emph{step}, celui-ci vaut 1 par défaut.

\subsection{Fonctions de clipping}

\begin{itemize}
    \item La fonction \textbf{clipseg(A,B,xmin,xmax,ymin,ymax)} clippe le segment \emph{{[}A,B{]}} avec la fenêtre \emph{{[}xmin,xmax{]}x{[}ymin,ymax{]}}et renvoie le résultat.
    \item La fonction \textbf{clipline(d,xmin,xmax,ymin,ymax)} clippe la droite \emph{d} avec la fenêtre \emph{{[}xmin,xmax{]}x{[}ymin,ymax{]}} et renvoie le résultat. La droite \emph{d} est une liste de deux complexes : un point et un vecteur directeur.
    \item La fonction \textbf{clippolyline(L,xmin,xmax,ymin,ymax,close)} clippe ligne polygonale \emph{L} avec \emph{{[}xmin,xmax{]}x{[}ymin,ymax{]}} et renvoie le résultat. L'argument \emph{L} est une liste de complexes ou une liste de listes de complexes. L'argument facultatif \emph{close} (false par défaut) indique si la ligne polygonale doit être refermée.
    \item La fonction \textbf{clipdots(L,xmin,xmax,ymin,ymax)} clippe la liste de points \emph{L} avec la fenêtre \emph{{[}xmin,xmax{]}x{[}ymin,ymax{]}} et renvoie le résultat (les points extérieurs sont simplement exclus). L'argument \emph{L} est une liste de complexes ou une liste de listes de complexes.
\end{itemize}

\subsection{Ajout de fonctions mathématiques}
Outre les fonctions associées aux méthodes graphiques qui font des calculs et renvoient une ligne polygonale (comme \emph{cartesian}, \emph{periodic}, \emph{implicit}, \emph{odesolve}, etc), le paquet \emph{luadraw} ajoute quelques fonctions mathématiques qui ne sont pas proposées nativement dans le module \emph{math}.

\subsubsection{int}
La fonction \textbf{int(f,a,b)} renvoie une valeur approchée de l'intégrale de la fonction \emph{f} sur l'intervalle $[a;b]$. La fonction \emph{f} est à variable réelle et à valeurs réelles ou complexes. La méthode utilisée est la méthode de Simpson accélérée deux fois avec la méthode Romberg.

\paragraph{Exemple :}
\begin{TeXcode}
$\int_0^1 e^{t^2}\mathrm d t \approx \directlua{tex.sprint(int(function(t) return math.exp(t^2) end, 0, 1))}$
\end{TeXcode}
\paragraph{Résultat :} $\int_0^1 e^{t^2}\mathrm d t \approx \directlua{tex.sprint(int(function(t) return math.exp(t^2) end, 0, 1))}$.

\subsubsection{gcd}
La fonction \textbf{gcd(a,b)} renvoie le plus grand diviseur commun entre $a$ et $b$.

\subsubsection{lcm}
La fonction \textbf{lcm(a,b)} renvoie le plus petit diviseur commun strictement positif entre $a$ et $b$.

\subsubsection{solve}
La fonction \textbf{solve(f,a,b,n)} fait une résolution numérique de l'équation $f(x)=0$ dans l'intervalle $[a;b]$, celui-ci est subdivisé en $n$ morceaux ($n$ vaut $25$ par défaut). La fonction renvoie une liste de résultats ou bien \emph{nil}. La méthode utilisée est une variante de Newton.

\paragraph{Exemple 1 :}
\begin{TeXcode}
\begin{luacode}
resol = function(f,a,b)
    local y = solve(f,a,b)
    if y == nil then tex.sprint("\\emptyset")
    else
        local str = y[1]
        for k = 2, #y do
            str = str..", ".. y[k]
        end
        tex.sprint(str)
    end
end
\end{luacode}
\def\solve#1#2#3{\directlua{resol(#1,#2,#3)}}%
\begin{luacode}
f1 = function(x) return math.cos(x)-x end
f2 = function(x) return x^3-2*x^2+1/2 end
\end{luacode}
La résolution de l'équation $\cos(x)=x$ dans $[0;\frac{\pi}2]$ donne $\solve{f1}{0}{math.pi/2}$.\par
La résolution de l'équation $\cos(x)=x$ dans $[\frac{\pi}2;\pi]$ donne $\solve{f1}{math.pi/2}{math.pi}$.\par
La résolution de l'équation $x^3-2x^2+\frac12=0$ dans $[-1;2]$ donne : $\{\solve{f2}{-1}{2}\}$.
\end{TeXcode}
\paragraph{Résultat :}\ \par

\begin{luacode}
resol = function(f,a,b)
    local y = solve(f,a,b)
    if y == nil then tex.sprint("\\emptyset")
    else
        local str = y[1]
        for k = 2, #y do
            str = str..", ".. y[k]
        end
        tex.sprint(str)
    end
end
\end{luacode}
\def\solve#1#2#3{\directlua{resol(#1,#2,#3)}}%
\begin{luacode}
f1 = function(x) return math.cos(x)-x end
f2 = function(x) return x^3-2*x^2+1/2 end
\end{luacode}

La résolution de l'équation $\cos(x)=x$ dans $[0;\frac{\pi}2]$ donne $\solve{f1}{0}{math.pi/2}$.\par
La résolution de l'équation $\cos(x)=x$ dans $[\frac{\pi}2;\pi]$ donne $\solve{f1}{math.pi/2}{math.pi}$.\par
La résolution de l'équation $x^3-2x^2+\frac 12=0$ dans $[-1;2]$ donne : $\{\solve{f2}{-1}{2}\}$.

\paragraph{Exemple 2 :} on souhaite tracer la courbe de la fonction $f$ définie par la condition :
\[\forall x\in \mathbf R,\ \int_x^{f(x)} \exp(t^2)\mathrm d t = 1.\]
On a deux méthodes possibles :
\begin{enumerate}
    \item On considère la fonction $G\colon (x,y) \mapsto \int_x^y \exp(t^2)\mathrm d t-1$, et on dessine la courbe implicite d'équation $G(x,y)=0$.
    \item On détermine un réel $y_0$ tel que $\int_0^{y_0}\exp(t^2)\mathrm d t = 1$ et on dessine la solution de l'équation différentielle $y'=e^{x^2-y^2}$ vérifiant la  condition initiale $y(0)=y_0$.
\end{enumerate}
Dessinons les deux :
\begin{demo}{Fonction $f$ définie par $\int_x^{f(x)} \exp(t^2)\mathrm d t = 1$.}
\begin{luadraw}{name=int_solve}
local g = graph:new{window={-3,3,-3,3},size={10,10}}
local h = function(t) return math.exp(t^2) end
local G = function(x,y) return int(h,x,y)-1 end
local H = function(y) return G(0,y) end
local F = function(x,y) return math.exp(x^2-y^2) end
local y0 = solve(H,0,1)[1] -- solution de H(x)=0
g:Daxes({0,1,1}, {arrows="->"})
g:Dimplicit(G, {draw_options="line width=4.8pt,Pink"})
g:Dodesolve(F,0,y0,{draw_options="line width=0.8pt"}) 
g:Lineoptions("dashed","gray",4); g:DlineEq(1,-1,0); g:DlineEq(1,1,0) -- bissectrices
g:Dlabel("${\\mathcal C}_f$",Z(2.15,2),{pos="S"})
g:Show()
\end{luadraw}
\end{demo}

On voit que les deux courbes se superposent bien, cependant la première méthode (courbe implicite) est beaucoup plus gourmande en calculs, la méthode 2 est donc préférable.


\section{Transformations}
Dans ce qui suit :
\begin{itemize}
    \item l'argument \emph{L} est soit un complexe, soit une liste de complexes soit une liste de listes de complexes,
    \item la droite \emph{d} est une liste de deux complexes : un point de la droite et un vecteur directeur.
  \end{itemize}
  
\subsection{affin}
La fonction \textbf{affin(L,d,v,k)} renvoie l'image de \emph{L} par l'affinité de base la droite \emph{d}, parallèlement au vecteur \emph{v} et de rapport \emph{k}.

\subsection{ftransform}
La fonction \textbf{ftransform(L,f)} renvoie l'image de \emph{L} par la fonction \emph{f} qui doit être une fonction de la variable complexe. Si un des éléments de \emph{L} est le complexe \emph{cpx.Jump} alors celui-ci est renvoyé tel quel dans le résultat.

\subsection{hom}
La fonction \textbf{hom(L,factor,center)} renvoie l'image de \emph{L} par l'homothétie de centre \emph{center} et de rapport \emph{factor}. Par défaut, l'argument \emph{center} vaut 0.

\subsection{inv}
La fonction \textbf{inv(L, center, r)} renvoie l'image de \emph{L} par l'inversion par rapport au cercle de centre \emph{center} et de rayon \emph{r}.

\subsection{proj}
La fonction \textbf{proj(L,d)} renvoie l'image de \emph{L} par la projection orthogonale sur la droite \emph{d}.

\subsection{projO}
La fonction \textbf{projO(L,d,v)} renvoie l'image de \emph{L} par la projection sur la droite \emph{d} parallèlement au vecteur \emph{v}.

\subsection{rotate}
La fonction \textbf{rotate(L,angle,center)} renvoie l'image de \emph{L} par la rotation de centre \emph{center} et d'angle \emph{angle} (en degrés). Par défaut, l'argument \emph{center} vaut 0.  

\subsection{shift}
La fonction \textbf{shift(L,u)} renvoie l'image de \emph{L} par la translation de vecteur \(u\).

\subsection{simil}
La fonction \textbf{simil(L,factor,angle,center)} renvoie l'image de \emph{L} par la similitude de centre \emph{center}, de rapport \emph{factor} et d'angle \emph{angle} (en degrés). Par défaut, l'argument \emph{center} vaut 0.

\subsection{sym}
La fonction \textbf{sym(L,d)} renvoie l'image de \emph{L} par la symétrie orthogonale d'axe la droite \emph{d}.

\subsection{symG}
La fonction \textbf{symG(L,d,v)} renvoie l'image de \emph{L} par la symétrie par rapport à la droite \emph{d} suivie de la translation de vecteur \emph{v} (symétrie glissée).

\subsection{symO}
La fonction \textbf{symO(L,d)} renvoie l'image de \emph{L} par la symétrie par rapport à la droite \emph{d} et parallèlement au vecteur \emph{v} (symétrie oblique).

\begin{demo}{Utilisation de transformations}
\begin{luadraw}{name=Sierpinski}
local g = graph:new{window={-5,5,-5,5},size={10,10}}
local i = cpx.I
local rand = math.random
local A, B, C = 5*i, -5-5*i, 5-5*i -- triangle initial
local T, niv = {{A,B,C}}, 5
for k = 1, niv do
    T = concat( hom(T,0.5,A), hom(T,0.5,B), hom(T,0.5,C) )
end
for _,cp in ipairs(T) do
    g:Filloptions("full", rgb(rand(),rand(),rand()))
    g:Dpolyline(cp,true)
end
g:Show()
\end{luadraw}
\end{demo}

\section{Calcul matriciel}

Si $f$ est une application affine du plan complexe, on appellera matrice de $f$ la liste (table) :
\begin{Luacode}
{ f(0), Lf(1), Lf(i) }
\end{Luacode}
où $Lf$ désigne la partie linéaire de $f$ (on a $Lf(1) = f(1)-f(0)$ et $Lf(i)=f(i)-f(0)$). La matrice identité est notée \emph{ID} dans le paquet \emph{luadraw}, elle correspond simplement à la liste \mintinline{Lua}{ {0,1,i} }.

\subsection{Calculs sur les matrices}

\subsubsection{applymatrix et applyLmatrix}
\begin{itemize}
    \item La fonction \textbf{applymatrix(z,M)} applique la matrice $M$ au complexe $z$ et renvoie le résultat (ce qui revient à calculer $f(z)$ si $M$ est la matrice de $f$). Lorsque $z$ est le complexe \emph{cpx.Jump} alors le résultat est \emph{cpx.Jump}. Lorsque $z$ est une chaîne de caractères alors la fonction renvoie $z$.
    \item La fonction \textbf{applyLmatrix(z,M)} applique la partie linéaire la matrice $M$ au complexe $z$ et renvoie le résultat (ce qui revient à calculer $Lf(z)$ si $M$ est la matrice de $f$). Lorsque $z$ est le complexe \emph{cpx.Jump} alors le résultat est \emph{cpx.Jump}.
\end{itemize}

\subsubsection{composematrix}
La fonction \textbf{composematrix(M1,M2)} effectue le produit matriciel $M1\times M2$ et renvoie le résultat.

\subsubsection{invmatrix}
La fonction \textbf{invmatrix(M)} calcule et renvoie l'inverse de la matrice $M$ lorsque cela est possible.

\subsubsection{matrixof}
\begin{itemize}
    \item La fonction \textbf{matrixof(f)} calcule et renvoie la matrice de $f$ (qui doit être une application affine du plan complexe.
    \item Exemple : \mintinline{Lua}{ matrixof( function(z) return proj(z,{0,Z(1,-1)}) end )} renvoie \par
     \mintinline{Lua}{{0,Z(0.5,-0.5),Z(-0.5,0.5)}} (matrice de la projection orthogonale sur la deuxième bissectrice).
\end{itemize}

\subsubsection{mtransform et mLtransform}
\begin{itemize}
    \item La fonction \textbf{mtransform(L,M)} applique la matrice $M$ à la liste $L$ et renvoie le résultat. $L$ doit être une liste de complexes ou une liste de listes de complexes, si l'un d'eux est le complexe \emph{cpx.Jump} ou une chaîne de caractères alors il est inchangé (donc renvoyé tel quel).
    \item La fonction \textbf{mLtransform(L,M)} applique la partie linéaire la matrice $M$ à la liste $L$ et renvoie le résultat. $L$ doit être une liste de complexes, si l'un d'eux est le complexe \emph{cpx.Jump} alors il est inchangé.
\end{itemize}

\subsection{Matrice associée au graphe}

Lorsque l'on crée un graphe dans l'environnement \emph{luadraw}, par exemple :
\begin{Luacode}
local g = graph:new{window={-5,5,-5,5},size={10,10}}
\end{Luacode}
l'objet \emph{g} créé possède une matrice de transformation qui est initialement l'identité. Toutes les méthodes graphiques utilisées appliquent automatiquement la matrice de transformation du graphe. Cette matrice est désignée par \mintinline{Lua}{g.matrix}, mais pour manipuler celle-ci, on dispose des méthodes qui suivent.

\subsubsection{g:IDmatrix()}
La méthode \textbf{g:IDmatrix()} réaffecte l'identité à la matrice du graphe \emph g.

\subsubsection{g:Composematrix()}
La méthode \textbf{g:Composematrix(M)} multiplie la matrice du graphe \emph g par la matrice \emph{M} (avec \emph{M} à droite) et le résultat est affecté à la matrice du graphe. L'argument \emph{M} doit donc être une matrice.

\subsubsection{g:Mtransform()}
La méthode \textbf{g:Mtransform(L)} applique la matrice du graphe \emph g à \emph{L} et renvoie le résultat, l'argument \emph L doit être une liste de complexes, ou une liste de listes de complexes.

\subsubsection{g:MLtransform()}
La méthode \textbf{g:MLtransform(L)} applique la partie linéaire de la matrice du graphe \emph g à \emph{L} et renvoie le résultat, l'argument \emph L doit être une liste de complexes, ou une liste de listes de complexes.

\begin{demo}{Utilisation de la matrice du graphe}
\begin{luadraw}{name=Pythagore}
local g = graph:new{window={-15,15,0,22},size={10,10}}
local a, b, c = 3, 4, 5 -- un triplet de Pythagore
local i, arccos, exp = cpx.I, math.acos, cpx.exp
local f1 = function(z)
        return (z-c)*a/c*exp(-i*arccos(a/c))+c+i*c end
local M1 = matrixof(f1)
local f2 = function(z)
        return z*b/c*exp(i*arccos(b/c))+i*c end
local M2 = matrixof(f2)
local arbre
arbre = function(n)
    local color = mixcolor(ForestGreen,1,Brown,n)
    g:Linecolor(color); g:Dsquare(0,c,1,"fill="..color)
    if n > 0 then
        g:Savematrix(); g:Composematrix(M1); arbre(n-1)
        g:Restorematrix(); g:Savematrix(); g:Composematrix(M2)
        arbre(n-1); g:Restorematrix()
    end
end
arbre(8)
g:Show()
\end{luadraw}
\end{demo}


\subsubsection{g:Rotate()}
La méthode \textbf{g:Rotate(angle, center)} modifie la matrice de transformation du graphe \emph g en la composant avec la matrice de la rotation d'angle \emph{angle} (en degrés) et de centre \emph{center}. L'argument \emph{center} est un complexe qui vaut $0$ par défaut.

\subsubsection{g:Scale()}
La méthode \textbf{g:Scale(factor, center)} modifie la matrice de transformation du graphe \emph g en la composant avec la matrice de l'homothétie de rapport \emph{factor} et de centre \emph{center}. L'argument \emph{center} est un complexe qui vaut $0$ par défaut.

\subsubsection{g:Savematrix() et g:Restorematrix()}
\begin{itemize}
    \item La méthode \textbf{g:Savematrix()} permet de sauvegarder dans une pile la matrice de transformation du graphe \emph g.
    \item La méthode \textbf{g:Restorematrix()} permet de restaurer la matrice de transformation du graphe \emph g à sa dernière valeur sauvegardée.
\end{itemize}

\subsubsection{g:Setmatrix()}
La méthode \textbf{g:Setmatrix(M)} permet d'affecter la matrice \emph M à la matrice de transformation du graphe \emph g.

\subsubsection{g:Shift()}
La méthode \textbf{g:Shift(v)} modifie la matrice de transformation du graphe \emph g en la composant avec la matrice de la translation de vecteur \emph{v} qui doit être un complexe.

\begin{demo}{Utilisation de Shift, Rotate et Scale}
\begin{luadraw}{name=free_art}
local du = math.sqrt(2)/2
local g = graph:new{window={1-du,4+du,1-du,4+du},
            margin={0,0,0,0},size={7,7}}
local i = cpx.I
g:Linestyle("noline")
g:Filloptions("full","Navy",0.1)
for X = 1, 4 do
    for Y = 1, 4 do
        g:Savematrix()
        g:Shift(X+i*Y); g:Rotate(45)
        for k = 1, 25 do
            g:Dsquare((1-i)/2,(1+i)/2,1)
            g:Rotate(7); g:Scale(0.9)
        end
        g:Restorematrix()
    end
end
g:Show()
\end{luadraw}
\end{demo}

\subsection{Changement de vue. Changement de repère}

\paragraph{Changement de vue : } lors de la création d'un nouveau graphique, par exemple :
\begin{Luacode}
local g = graph:new{window={-5,5,-5,5},size={10,10}}
\end{Luacode}
L'option \emph{window=\{xmin,xmax,ymin,ymax\}} fixe la vue pour le graphique \emph{g}, ce sera le pavé \emph{[xmin, xmax]x [ymin, ymax]} de $\mathbf R^2$, et tous les tracés vont être clippés par cette fenêtre (sauf les labels qui peuvent débordés dans les marges, mais pas au-delà).
Il est possible, à l'intérieur de ce pavé, de définir un autre pavé pour faire une nouvelle vue, avec la méthode \textbf{g:Viewport(x1,x2,y1,y2)}. Les valeurs de \emph{x1}, \emph{x2}, \emph{y1}, \emph{y2} se réfèrent la fenêtre initiale définie par l'option \emph{window}. À partir de là, tout ce qui sort de cette nouvelle zone va être clippé, et la matrice du graphe est réinitialisée à l'identité, par conséquent il faut sauvegarder auparavant les paramètres graphiques courants :
\begin{Luacode}
g:Saveattr()
g:Viewport(x1,x2,y1,y2)
\end{Luacode}
Pour revenir à la vue précédente avec la matrice précédente, il suffit d'effectuer une restauration des paramètres graphiques avec la méthode \textbf{g:Restoreattr()}.

\paragraph{Attention : } à chaque instruction \emph{Saveattr()} doit correspondre une instruction \emph{Restoreattr()}, sinon il y aura une erreur à la compilation.

\paragraph{Changement de repère : } on peut changer le système de coordonnées de la vue courante avec la méthode \textbf{g:Coordsystem(x1,x2,y1,y2,ortho)}. Cette méthode va modifier la matrice du graphe de sorte que tout se passe comme si la vue courante correspondait au pavé $[x1,x2]\times[y1,y2]$, l'argument booléen facultatif \emph{ortho} indique si le nouveau repère doit être orthonormé ou non (false par défaut). Comme la matrice du graphe est modifiée il est préférable de sauvegarder les paramètres graphiques avant, et de les restaurer ensuite. Cela peut servir par exemple à faire plusieurs figures dans le graphique en cours.

\begin{demo}{Classification des points d'une courbe paramétrée}
\begin{luadraw}{name=viewport_changewin}
local g = graph:new{window={-5,5,-5,5},size={10,10}}
local i = cpx.I
g:Labelsize("tiny") 
g:Writeln("\\tikzset{->-/.style={decoration={markings, mark=at position #1 with {\\arrow{>}}}, postaction={decorate}}}")
g:Dline({0,1},"dashed,gray"); g:Dline({0,i},"dashed,gray")
local legende = {"Point ordinaire", "Point d'inflexion", "Rebroussement 1ère espèce", "Rebroussement 2ème espèce"}
local A, B, C =(1+i)*0.75, 0.75, 0
local A2, B2 ={-1.25+i*0.5,-0.75-i*0.5,1.25-0.5*i, 0.5+i}, {-0.75,-0.75,0.75,0.75}
local u = {Z(-5,0),Z(0,0),-5-5*i,-5*i}
for k = 1, 4 do
    g:Saveattr(); g:Viewport(u[k].re,u[k].re+5,u[k].im,u[k].im+5)
    g:Coordsystem(-1.4,2.25,-1,1.25)
    g:Composematrix({0,1,1+i}) -- pour pencher l'axe Oy
    g:Dpolyline({{-1,1},{-i*0.5,i}}) -- axes
    g:Lineoptions(nil,"blue",8)
    g:Dpath({A2[k],(B2[k]+2*A2[k])/3,(C+5*B2[k])/6, C,"b"},"->-=0.5")
    g:Dpath({C,(C+5*B)/6,(B+2*A)/3,A,"b"},"->-=0.75")
    g:Dpolyline({{0,0.75},{0,0.75*i}},false,"->,red")
    g:Dlabel(
        legende[k],0.75-0.5*i, {pos="S"},
        "$f^{(p)}(t_0)$",1,{pos="E",node_options="red"},
        "$f^{(q)}(t_0)$",0.75*i,{pos="W",dist=0.05})
    g:Restoreattr()
end
g:Show()
\end{luadraw}
\end{demo}

\section{Ajouter ses propres méthodes à la classe \emph{graph}}

Sans avoir à modifier les fichiers sources Lua associés au paquet \emph{luadraw}, on peut ajouter ses propres méthodes à la classe \emph{graph}, ou modifier une méthode existante. Ceci n'a d'intérêt que si ces modifications doivent être utilisées dans différents graphiques et/ou différents documents (sinon il suffit d'écrire localement une fonction dans le graphique où on en a besoin).

\subsection{Un exemple}
Dans le graphique de la page \pageref{champ}, nous avons dessiné un champ de vecteurs, pour cela on a écrit une fonction qui calcule les vecteurs avant de faire le dessin, mais cette fonction est locale. On pourrait en faire une fonction globale (en enlevant le mot clé \emph{local}), elle serait alors utilisable dans tout le document, mais pas dans un autre document !

Pour généraliser cette fonction, on va devoir créer un fichier Lua qui pourra ensuite être importé dans des documents en cas de besoin. Pour rendre l'exemple un peu consistant, on va créer un fichier qui va définir une fonction qui calcule les vecteurs d'un champ, et qui va ajouter à la classe \emph{graph} deux nouvelles méthodes : une pour dessiner un champ de vecteurs d'une fonction $f\colon(x,y)\to(x,y)\in \mathbf R^2$, on la nommera \emph{graph:Dvectorfield}, et une autre pour dessiner un champ de gradient d'une fonction $f\colon(x,y)\to\mathbf R$, on la nommera \emph{graph:Dgradientfield}. Du coup nous appellerons ce fichier : \emph{luadraw\_fields.lua}.

\paragraph{Contenu du fichier :}
\begin{Luacode}
-- luadraw_fields.lua
-- ajout de méthodes à la classe graph du paquet luadraw
-- pour dessiner des champs de vecteurs ou de gradient
function field(f,x1,x2,y1,y2,grid,long)  -- fonction mathématique, indépendante du graphique
-- calcule un champ de vecteurs dans le pavé [x1,x2]x[y1,y2]
-- f fonction de deux variables à valeurs dans R^2
-- grid = {nbx, nby} : nombre de vecteurs suivant x et suivant y
-- long = longueur d'un vecteur
    if grid == nil then grid = {25,25} end
    local deltax, deltay = (x2-x1)/(grid[1]-1), (y2-y1)/(grid[2]-1) -- pas suivant x et y
    if long == nil then long = math.min(deltax,deltay) end -- longueur par défaut
    local vectors = {} -- contiendra la liste des vecteurs
    local x, y, v = x1 
    for _ = 1, grid[1] do -- parcours suivant x
        y = y1
        for _ = 1, grid[2] do -- parcours suivant y
            v = f(x,y) -- on suppose que v est bien défini
            v = Z(v[1],v[2]) -- passage en complexe
            if not cpx.isNul(v) then
                v = v/cpx.abs(v)*long -- normalisation de v
                table.insert(vectors, {Z(x,y), Z(x,y)+v} ) -- on ajoute le vecteur
            end
            y = y+deltay
        end
        x = x+deltax
    end
    return vectors -- on renvoie le résultat (ligne polygonale)
end

function graph:Dvectorfield(f,args) -- ajout d'une méthode à la classe graph
-- dessine un champ de vecteurs
-- f fonction de deux variables à valeurs dans R^2
-- args table à 4 champs :
-- { view={x1,x2,y1,y2}, grid={nbx,nby}, long=, draw_options=""}
    args = args or {}
    local view = args.view or {self:Xinf(),self:Xsup(),self:Yinf(),self:Ysup()} -- repère utilisateur par défaut
    local vectors = field(f,view[1],view[2],view[3],view[4],args.grid,args.long) -- calcul du champ
    self:Dpolyline(vectors,false,args.draw_options) -- le dessin (ligne polygonale non fermée)
end

function graph:Dgradientfield(f,args) -- ajout d'une autre méthode à la classe graph
-- dessine un champ de gradient
-- f fonction de deux variables à valeurs dans R
-- args table à 4 champs :
-- { view={x1,x2,y1,y2}, grid={nbx,nby}, long=, draw_options=""}
    local h = 1e-6
    local grad_f = function(x,y) -- fonction gradient de f
        return { (f(x+h,y)-f(x-h,y))/(2*h), (f(x,y+h)-f(x,y-h))/(2*h) }
    end
    self:Dvectorfield(grad_f,args) -- on utilise la méthode précédente
end
\end{Luacode}

\subsection{Comment importer le fichier}

Il y a deux méthodes pour cela :

\begin{enumerate}
    \item Avec l'instruction Lua \emph{dofile}. On peut l'écrire par exemple dans le préambule après la déclaration du paquet :
    \begin{TeXcode}
    \usepackage[]{luadraw}
    \directlua{dofile("<chemin>/luadraw_fields.lua")}
    \end{TeXcode}
    Bien entendu, il faudra remplacer \verb|<chemin>| par le chemin d'accès à ce fichier. 
    
    L'instruction \verb|\directlua{dofile("<chemin>/luadraw_fields.lua")}| peut être placée ailleurs dans le document pourvu que ce soit après le chargement du paquet (sinon la classe \emph{graph} ne sera pas reconnue lors de la lecture du fichier). On peut aussi placer l'instruction \verb|dofile("<chemin>/luadraw_fields.lua")| dans un environnement \emph{luacode}, et donc en particulier dans un environnement \emph{luadraw}.
    
    Dès que le fichier est importé, les nouvelles méthodes sont disponibles pour la suite du document.
    
    Cette façon de procéder a au moins deux inconvénients : il faut se souvenir à chaque utilisation de \verb|<chemin>|, et d'autre part l'instruction \emph{dofile} ne vérifie par si le fichier a déjà été lu. Pour ces raisons, on préférera la méthode suivante.
    
    \item Avec l'instruction Lua \emph{require}. On peut l'écrire par exemple dans le préambule après la déclaration du paquet :
    \begin{TeXcode}
    \usepackage[]{luadraw}
    \directlua{require "luadraw_fields"}
    \end{TeXcode}
    On remarquera l'absence du chemin (et l'extension lua est inutile).
    
    L'instruction \verb|\directlua{require "luadraw_fields"}| peut être placée ailleurs dans le document pourvu que ce soit après le chargement du paquet (sinon la classe \emph{graph} ne sera pas reconnue lors de la lecture du fichier). On peut aussi placer l'instruction \verb|require "luadraw_fields"| dans un environnement \emph{luacode}, et donc en particulier dans un environnement \emph{luadraw}.
    
    L'instruction \emph{require} vérifie si le fichier a déjà été chargé ou non, ce qui est préférable. Mais il faut cependant que Lua soit capable de trouver ce fichier, et le plus simple pour cela est qu'il soit quelque part dans une arborescence connue de TeX. On peut par exemple créer dans son \emph{texmf} local le chemin suivant :
    \begin{TeXcode}
    texmf/tex/lualatex/myluafiles/
    \end{TeXcode}
    puis copier le fichier \emph{luadraw\_fields.lua} dans le dossier \emph{myluafiles}.
\end{enumerate}

\begin{demo}{Utilisation des nouvelles méthodes}
\begin{luadraw}{name=fields}
require "luadraw_fields" -- import des nouvelles méthodes
local g = graph:new{window={0,21,0,10},size={16,10}}
local i = cpx.I
g:Labelsize("footnotesize")
local f = function(x,y) return {x-x*y,-y+x*y} end -- Volterra
local F = function(x,y) return x^2+y^2+x*y-6 end
local H = function(t,Y) return f(Y[1],Y[2]) end
-- graphique du haut
g:Saveattr();g:Viewport(0,10,0,10);g:Coordsystem(-5,5,-5,5)
g:Dgradbox({-4.5-4.5*i,4.5+4.5*i,1,1}, {originloc=0,originnum={0,0},grid=true,title="gradient field, $f(x,y)=x^2+y^2+xy-6$"}) 
g:Arrows("->"); g:Lineoptions(nil,"blue",6)
g:Dgradientfield(F,{view={-4,4,-4,4},grid={15,15},long=0.5})
g:Arrows("-"); g:Lineoptions(nil,"Crimson",12); g:Dimplicit(F, {view={-4,4,-4,4}})
g:Restoreattr()
-- graphique du bas
g:Saveattr();g:Viewport(11,21,0,10);g:Coordsystem(-5,5,-5,5)
g:Dgradbox({-4.5-4.5*i,4.5+4.5*i,1,1}, {originloc=0,originnum={0,0},grid=true,title="vector field, $f(x,y)=(x-xy,-y+xy)$"}) 
g:Arrows("->"); g:Lineoptions(nil,"blue",6); g:Dvectorfield(f,{view={-4,4,-4,4}})
g:Arrows("-");g:Lineoptions(nil,"Crimson",12)
g:Dodesolve(H,0,{2,3},{t={0,50},out={2,3},nbdots=250})
g:Restoreattr()
g:Show()
\end{luadraw}
\end{demo}

\subsection{Modifier une méthode existante}

Prenons par exemple la méthode \emph{DplotXY(X,Y,draw\_options)} qui prend comme arguments deux listes (tables) de réels et dessine la ligne polygonale formée par les points de coordonnées $(X[k],Y[k])$. Nous allons la modifier afin qu'elle prenne en compte le cas où \emph{X} est une liste de noms (chaînes), dans ce cas, on affichera les noms sous l'axe des abscisses (avec l'abscisse $k$ pour le k\ieme\ nom) et on dessinera la ligne polygonale formée par les points de coordonnées $(k,Y[k])$, sinon on fera comme l'ancienne méthode. Il suffit pour cela de réécrire la méthode (dans un fichier Lua pour pouvoir ensuite l'importer) :

\begin{Luacode}
function graph:DplotXY(X,Y,draw_options)
-- X est une liste de réels ou de chaînes
-- Y est une liste de réels de même longueur que X 
    local L = {} -- liste des points à dessiner
    if type(X[1]) == "number" then -- liste de réels
        for k,x in ipairs(X) do
            table.insert(L,Z(x,Y[k]))
        end
    else
        local noms = {} -- liste des labels à placer
        for k = 1, #X do
            table.insert(L,Z(k,Y[k]))
            insert(noms,{X[k],k,{pos="E",node_options="rotate=-90"}})
        end
        self:Dlabel(table.unpack(noms)) --dessin des labels
    end
    self:Dpolyline(L,draw_options) -- dessin de la courbe
end
\end{Luacode}

Dès que le fichier sera importé, cette nouvelle définition va écraser l'ancienne (pour toute la suite du document). Bien entendu on pourrait imaginer ajouter d'autres options sur le style de tracé par exemple (ligne, bâtons, points ...).

\begin{demo}{Modification d'une méthode existante}
\begin{luadraw}{name=newDplotXY}
require "luadraw_modified" -- import de la méthode modifiée
local g = graph:new{window={-0.5,11,-1,20}, margin={0.5,0.5,0.5,1}, size={10,10,0}}
g:Labelsize("scriptsize")
local X, Y = {}, {} -- on définit deux listes X et Y, on pourrait aussi les lire dans un fichier
for k = 1, 10 do
    table.insert(X,"nom"..k)
    table.insert(Y,math.random(1,20))
end
defaultlabelshift = 0
g:Daxes({0,1,2},{limits={{0,10},{0,20}}, labelpos={"none","left"},arrows="->", grid=true})
g:DplotXY(X,Y,"line width=0.8pt, blue")
g:Show()
\end{luadraw}
\end{demo}
%
\chapter{Dessin 3d}

\begin{center}
\captionof{figure}{Point col en $M(0,0,0)$ ($z=x^2-y^2$)}\label{pointcol}\par
\begin{luadraw}{name=point_col}
local g = graph3d:new{window3d={-2,2,-2,2,-4,4}, window={-4,3.5,-5,5}, size={8,9,0}, viewdir={120,60}}
local S = cartesian3d(function(u,v) return u^2-v^2 end, -2,2,-2,2,{20,20})
local P = facet2poly(S) -- conversion en polyèdre
local Tx = g:Intersection3d(P, {Origin,vecI}) --intersection de P avec le plan yOz
local Ty = g:Intersection3d(P, {Origin,vecJ}) --intersection de P avec le plan xOz
g:Dboxaxes3d({grid=true,gridcolor="gray",fillcolor="LightGray",drawbox=true})
g:Dfacet(S,{mode=mShadedOnly,color="ForestGreen"}) -- dessin de la surface
g:Dedges(Tx, {color="Crimson", hidden=true, width=8}) -- intersection avec yOz
g:Dedges(Ty, {color="Navy",hidden=true, width=8}) -- intersection avec xOz
g:Dpolyline3d( {M(2,0,4),M(-2,0,4),M(-2,0,-4)}, "Navy,line width=.8pt")
g:Dpolyline3d( {M(0,-2,4),M(0,2,4),M(0,2,-4)}, "Crimson,line width=.8pt")
g:Show()
\end{luadraw}
\end{center}

\section{Introduction}

\subsection{Prérequis}

\begin{itemize}
\item Ce document présente l'utilisation du package \emph{luadraw} avec l'option globale \emph{3d} :
\verb|\usepackage[3d]{luadraw}|.
\item Le paquet charge le module \emph{luadraw\_graph2d.lua} qui définit la classe \emph{graph}, et fournit l'environnement \emph{luadraw} qui permet de faire des graphiques en Lua. Tout ce qui est dit dans la documentation \emph{LuaDraw2d.pdf} s'applique donc, et est supposé connu ici.
\item L'option globale \emph{3d} permet en plus le chargement du module \emph{luadraw\_graph3d.lua}. Celui-ci définit en plus la classe \emph{graph3d} (qui s'appuie sur la classe \emph{graph}) pour des dessins en 3d. 
\end{itemize}

\subsection{Quelques rappels}

\begin{itemize}
    \item Autre option globale du paquet : \emph{noexec}. Lorsque cette option globale est mentionnée la valeur par défaut de l'option \emph{exec} pour l'environnement \emph{luadraw} sera false (et non plus true).

    \item Lorsqu'un graphique est terminé il est exporté au format tikz, donc ce paquet charge également le paquet \emph{tikz} ainsi que les librairies :
    \begin{itemize}
        \item\emph{patterns}
        \item\emph{plotmarks}
        \item\emph{arrows.meta}
        \item\emph{decorations.markings}
        \end{itemize}
    \item Les graphiques sont créés dans un environnement \emph{luadraw}, celui-ci appelle \emph{luacode}, c'est donc du \textbf{langage Lua} qu'il faut utiliser dans cet environnement.

    \item Sauvegarde du fichier \emph{.tkz} : le graphique est exporté au format tikz dans un fichier (avec l'extension \emph{tkz}), par défaut celui-ci est sauvegardé dans le dossier courant. Mais il est possible d'imposer un chemin spécifique en définissant dans le document, la commande \verb|\luadrawTkzDir|, par exemple : \verb|\def\luadrawTkzDir{tikz/}|, ce qui permettra d'enregistrer les fichiers \emph{*.tkz} dans le sous-dossier \emph{tikz} du dossier courant, à condition toutefois que ce sous-dossier existe !

    \item Les options de l'environnement sont :
    \begin{itemize}
    \item \emph{name = \ldots{}} : permet de donner un nom au fichier tikz produit, on donne un nom sans extension (celle-ci sera automatiquement ajoutée, c'est \emph{.tkz}). Si cette option est omise, alors il y a un nom par défaut, qui est le nom du fichier maître suivi d'un numéro.
    \item \emph{exec = true/false} : permet d'exécuter ou non le code Lua compris dans l'environnement. Par défaut cette option vaut true, \textbf{SAUF} si l'option globale \emph{noexec} a été mentionnée dans le préambule avec la déclaration du paquet. Lorsqu'un graphique complexe qui demande beaucoup de calculs est au point, il peut être intéressant de lui ajouter l'option \emph{exec=false}, cela évitera les recalculs de ce même graphique pour les compilations à venir.
    \item \emph{auto = true/false} : permet d'inclure ou non automatiquement le fichier tikz en lieu et place de l'environnement \emph{luadraw} lorsque l'option \emph{exec} est à false. Par défaut l'option \emph{auto} vaut true.
    \end{itemize}
\end{itemize}


\subsection{Création d'un graphe 3d}

\begin{TeXcode}
\begin{luadraw}{ name=<filename>, exec=true/false, auto=true/false }
-- création d'un nouveau graphique en lui donnant un nom local
local g = graph3d:new{ window3d={x1,x2,y1,y2,z1,z2}, adjust2d=true/false, viewdir={30,60}, window={x1,x2,y1,y2,xscale,yscale}, margin={left,right,top,bottom}, size={largeur,hauteur,ratio}, bg="color", border=true/false }
-- construction du graphique g
    instructions graphiques en langage Lua ...
-- affichage du graphique g et sauvegarde dans le fichier <filename>.tkz
g:Show()
-- ou bien sauvegarde uniquement dans le fichier <filename>.tkz
g:Save()
\end{luadraw}
\end{TeXcode}

La création se fait dans un environnement \emph{luadraw}, c'est à la première ligne à l'intérieur de l'environnement qu'est faite cette création en nommant le graphique :

\begin{Luacode}
local g = graph3d:new{ window3d={x1,x2,y1,y2,z1,z2}, adjust2d=true/false, viewdir={30,60}, window={x1,x2,y1,y2,xscale,yscale}, margin={left,right,top,bottom}, size={largeur,hauteur,ratio}, bg="color", border=true/false }
\end{Luacode}

La classe \emph{graph3d} est définie dans le paquet \emph{luadraw} grâce à l'option globale \emph{3d}. On instancie cette classe en invoquant son constructeur et en donnant un nom (ici c'est \emph{g}), on le fait en local de sorte que le graphique \emph{g} ainsi créé, n'existera plus une fois sorti de l'environnement (sinon \emph{g} resterait en mémoire jusqu'à la fin du document).

\begin{itemize}
 \item Le paramètre (facultatif) \emph{window3d} définit le pavé de $\mathbf R^3$ correspondant au graphique : c'est $[x_1,x_2]\times[y_1,y_2]\times[z_1,z_2]$. Par défaut c'est $[-5,5]\times[-5,5]\times[-5,5]$.
 \item Le paramètre (facultatif) \emph{adjust2d} indique si la fenêtre 2d qui va contenir la projection orthographique du dessin 3d, doit être déterminée automatiquement (false par défaut). Cette fenêtre 2d correspond à l'argument \emph{window}.
 
 \item Le paramètre (facultatif) \emph{viewdir} est une table qui définit les deux angles de vue (en degrés) pour la projection orthographique. Par défaut c'est la table \{30,60\}.
 
\begin{center}
\captionof{figure}{Angles de vue}\label{viewdir}
\begin{luadraw}{name=viewdir}
local g = graph3d:new{ size={8,8} }
local i = cpx.I
local O, A = Origin, M(4,4,4)
local B, C, D, E = pxy(A), px(A), py(A), pz(A)
g:Dpolyline3d( {{O,A},{-5*vecI,5*vecI},{-5*vecJ,5*vecJ},{-5*vecK,5*vecK}}, "->")
g:Dpolyline3d( {{E,A,B,O}, {C,B,D}}, "dashed")
g:Dpath3d( {C,O,B,2.5,1,"ca",O,"l","cl"}, "draw=none,fill=cyan,fill opacity=0.8")
g:Darc3d(C,O,B,2.5,1,"->")
g:Dpath3d( {E,O,A,2.5,1,"ca",O,"l","cl"}, "draw=none,fill=cyan,fill opacity=0.8")
g:Darc3d(E,O,A,2.5,1,"->")
g:Dballdots3d(O)
g:Labelsize("footnotesize")
g:Dlabel3d(
    "$x$", 5.25*vecI,{}, "$y$", 5.25*vecJ,{}, "$z$", 5.25*vecK,{},
    "vers observateur", A, {pos="E"},
    "$O$", O, {pos="NW"},
    "$\\theta$", (B+C)/2, {pos="N", dist=0.15},
    "$\\varphi$", (A+E)/2, {pos="S",dist=0.25}
)
g:Dlabel("viewdir=\\{$\\theta,\\varphi$\\} (en degrés)",-5*i,{pos="N"})
g:Show()            
\end{luadraw}
\end{center}

\item Les autres paramètres sont ceux de la classe \emph{graph}, ils ont été décrits dans le chapitre 1.
\end{itemize}

\paragraph{Construction du graphique.}

\begin{itemize}
    \item L'objet instancié (\emph{g} ici dans l'exemple) possède toutes les méthodes de la classe \emph{graph}, plus des méthodes spécifiques à la 3d.
    \item La classe \emph{graph3d} amène aussi un certain nombre de fonctions mathématiques propres à la 3d.
\end{itemize}


\section{La classe \emph{pt3d} (point 3d)}

\subsection{Représentation des points et vecteurs}

\begin{itemize}
    \item L'espace usuel est $\mathbf R^3$, les points et les vecteurs sont donc des triplets de réels (appelés points 3d). Quatre triplets portent un nom spécifique (variables prédéfinies), il s'agit de :
    \begin{itemize}
        \item \textbf{Origin}, qui représente le triplet $(0,0,0)$.
        \item \textbf{vecI}, qui représente le triplet $(1,0,0)$.
        \item \textbf{vecJ}, qui représente le triplet $(0,1,0)$.
        \item \textbf{vecK}, qui représente le triplet $(0,0,1)$.
    \end{itemize}
    À cela s'ajoute la variable \textbf{ID3d} qui est la table \emph{\{Origin, vecI, vecJ, vecK\}} représentant la matrice unité 3d. Par défaut c'est la matrice de transformation du graphe 3d.
    \item La classe \emph{pt3d} (qui est automatiquement chargée) définit les triplets de réels, les opérations possibles, et un certain nombre de méthodes. Pour créer un point 3d, il y a trois méthodes :
        \begin{itemize}
            \item Définition en cartésien : la fonction \textbf{M(x,y,z)} renvoie le triplet $(x,y,z)$. On peut également obtenir ce triplet en faisant : \emph{x*vecI+y*vecJ+z*vecK}.
            \item Définition en cylindrique : la fonction \textbf{Mc(r,$\theta$,z)} (angle exprimé en radians) renvoie le triplet $(r\cos(\theta),r\sin(\theta),z)$.
            \item Définition en sphérique : la fonction \textbf{Ms(r,$\theta$,$\varphi$)} renvoie le triplet $(r\cos(\theta)\sin(\varphi), r\sin(\theta)\sin(\varphi),r\cos(\varphi))$ (angles exprimés en radians).
        \end{itemize}
    Accès aux composantes d'un point 3d : si une variable $A$ désigne un point 3d, alors ses trois composantes sont $A.x$, $A.y$ et $A.z$.
    
    Pour tester si une variable $A$ désigne un point 3d, on dispose de la fonction \textbf{isPoint3d()} qui renvoie un booléen.
    
    Conversion : pour convertir un réel ou un complexe en point 3d, on dispose de la fonction \textbf{toPoint3d()}.
\end{itemize}

\subsection{Opérations sur les points 3d}

Ces opérations sont les opérations usuelles avec les symboles usuels :
\begin{itemize}
    \item L'addition (+), la différence (-), l'opposé (-).
    \item Le produit par un scalaire, si k et un réel, \emph{k*M(x,y,z)} renvoie \emph{M(ka,ky,kz)}.
    \item On peut diviser un point 3d par un scalaire, par exemple, si $A$ et $B$ sont deux points 3d, alors le milieu s'écrit simplement $(A+B)/2$.
    \item On peut tester l'égalité de deux points 3d avec le symbole =.
\end{itemize}

\subsection{Méthodes de la classe \emph{pt3d}}

Celles-ci sont :
\begin{itemize}
    \item \textbf{pt3d.abs(u)} : renvoie la norme euclidienne du point 3d $u$.
    \item \textbf{pt3d.abs2(u)} : renvoie la norme euclidienne au carré du point 3d $u$.
    \item \textbf{pt3d.N1(u)} : renvoie la norme 1 du point 3d $u$. Si $u=M(x,y,z)$, alors \emph{pt3d.N1(u)} renvoie $|x|+|y|+|z|$.
    \item \textbf{pt3d.dot(u,v)} : renvoie le produit scalaire entre les vecteurs (points 3d) $u$ et $v$.
    \item \textbf{pt3d.det(u,v,w)} : renvoie le déterminant entre les vecteurs (points 3d) $u$, $v$ et $w$.
    \item \textbf{pt3d.prod(u,v)} : renvoie le produit vectoriel entre les vecteurs (points 3d) $u$ et $v$.
    \item \textbf{pt3d.angle3d(u,v,epsilon)} : renvoie l'écart angulaire (en radians) entre les vecteurs (points 3d) $u$ et $v$ supposés non nuls. L'argument (facultatif) \emph{epsilon} vaut $0$ par défaut, il indique à combien près se fait un certain test d'égalité sur un flottant.

    \item \textbf{pt3d.normalize(u)} : renvoie le vecteur (point 3d) $u$ normalisé (renvoie \emph{nil} si $u$ est nul).
    \item \textbf{pt3d.round(u,nbDeci)} : renvoie un point 3d dont les composantes sont celles du point 3d $u$ arrondies avec \emph{nbDeci} décimales.
\end{itemize}

\subsection{Fonctions mathématiques}

Dans le fichier définissant la classe \emph{pt3d}, quelques fonctions mathématiques sont introduites :
\begin{itemize}
    \item \textbf{isobar3d(L)} : renvoie l'isobarycentre des points 3d de la liste (table) $L$ (les éléments de $L$ qui ne sont pas des points 3d sont ignorés).
    \item \textbf{insert3d(L,A,epsilon)} : cette fonction insère le point 3d $A$ dans la liste $L$ qui doit être une \textbf{variable} (et qui sera donc modifiée). Le point $A$ est inséré \textbf{sans doublon} et la fonction renvoie sa position (indice) dans la liste $L$ après insertion. L'argument (facultatif) \emph{epsilon} vaut $0$ par défaut, il indique à combien près se font les comparaisons.
\end{itemize}

\section{Méthodes graphiques élémentaires}

Toutes les méthodes graphiques 2d s'appliquent. À cela s'ajoute la possibilité de dessiner dans l'espace des lignes polygonales, des segments, droites, courbes, chemins, points, labels, plans, solides. Avec les solides vient également la notion de facettes que l'on ne trouvait pas en 2d.

Les méthodes graphiques 3d vont calculer automatiquement la projection sur le plan de l'écran, après avoir appliquer aux objets la matrice de transformation 3d associée au graphique (qui est l'identité par défaut), ce sont ensuite les méthodes graphiques 2d qui prendront le relai.

La méthode qui applique la matrice 3d et fait la projection sur l'écran (plan passant par l'origine et normal au vecteur unitaire dirigé vers l'observateur et défini par les angles de vue), est : \textbf{g:Proj3d(L)} où $L$ est soit un point 3d, soit une liste de points 3d, soit une liste de listes de points 3d. Cette fonction renvoie des complexes (affixes des projetés sur l'écran).

\subsection{Dessin aux traits}

\subsubsection{Ligne polygonale : Dpolyline3d}

La méthode \textbf{g:Dpolyline3d(L,close,draw\_options,clip)} (où \emph{g} désigne le graphique en cours de création), \emph{L} est une ligne polygonale 3d (liste de listes de points 3d), \emph{close} un argument facultatif qui vaut \emph{true} ou \emph{false} indiquant si la ligne doit être refermée ou non (\emph{false} par défaut), et \emph{draw\_options} est une chaîne de caractères qui sera passée directement à l'instruction \emph{\textbackslash draw} dans l'export. L'argument \emph{clip} vaut \emph{false} par défaut, il indique si la ligne \emph{L} doit être clippée avec la fenêtre 3d courante.
    
\subsubsection{Angle droit : Dangle3d}

La méthode \textbf{g:Dangle3d(B,A,C,r,draw\_options,clip)} dessine l'angle \(BAC\) avec un parallélogramme (deux côtés seulement sont dessinés), l'argument facultatif \emph{r} précise la longueur d'un côté (0.25 par défaut). Le parallélogramme est dans le plan défini par les points $A$, $B$ et $C$, ceux-ci ne doivent donc pas être alignés. L'argument \emph{draw\_options} est une chaîne (vide par défaut) qui sera passée telle quelle à l'instruction  \emph{\textbackslash draw}. L'argument \emph{clip} vaut \emph{false} par défaut, il indique si le tracé doit être clippé avec la fenêtre 3d courante.
    
\subsubsection{Segment : Dseg3d}

La méthode \textbf{g:Dseg3d(seg,scale,draw\_options,clip)} dessine le segment défini par l'argument \emph{seg} qui doit être une liste de deux points 3d. L'argument facultatif \emph{scale} (1 par défaut) est un nombre qui permet d'augmenter ou réduire la longueur du segment (la longueur naturelle est multipliée par \emph{scale}). L'argument \emph{draw\_options} est une chaîne (vide par défaut) qui sera passée telle quelle à l'instruction \emph{\textbackslash draw}. L'argument \emph{clip} vaut \emph{false} par défaut, il indique si le tracé doit être clippé avec la fenêtre 3d courante.
    
\subsubsection{Droite : Dline3d}

La méthode \textbf{g:Dline3d(d,draw\_options,clip)} trace la droite \emph{d}, celle-ci est une liste du type \emph{\{A,u\}} où \emph{A} représente un point de la droite (point 3d) et \emph{u} un vecteur directeur (un point 3d non nul). 

Variante : la méthode \textbf{g:Dline3d(A,B,draw\_options,clip)} trace la droite passant par les points \emph{A} et \emph{B} (deux points 3d). L'argument \emph{draw\_options} est une chaîne (vide par défaut) qui sera passée telle quelle à l'instruction \emph{\textbackslash draw}. L'argument \emph{clip} vaut \emph{false} par défaut, il indique si le tracé doit être clippé avec la fenêtre 3d courante.

La méthode \textbf{g:Line3d2seg(d,scale)} renvoie une table constituée de deux points 3d représentant un segment, ce segment est la partie de la droite \emph{d} à l'intérieur la fenêtre 3d courante. L'argument \emph{scale} (1 par défaut) permet de faire varier la taille de ce segment. Lorsque la fenêtre est trop petite l'intersection peut être vide.

 \subsubsection{Arc de cercle : Darc3d}
 
\begin{itemize}
    \item La méthode \textbf{g:Darc3d(B,A,C,r,sens,normal,draw\_options,clip)} dessine un arc de cercle de centre \emph{A} (point 3d), de rayon \emph{r}, allant de \emph{B} (point 3d) vers \emph{C} (point 3d) dans le sens direct si l'argument \emph{sens} vaut 1, le sens inverse sinon. Cet arc est tracé dans le plan contenant les trois points $A$, $B$ et $C$, lorsque ces trois points sont alignés il faut préciser l'argument \emph{normal} (point 3d non nul) qui représente un vecteur normal au plan. Ce plan est orienté par le produit vectoriel $\vec{AB}\wedge\vec{AC}$ ou bien par le vecteur \emph{normal} si celui-ci est précisé. L'argument \emph{draw\_options} est une chaîne (vide par défaut) qui sera passée telle quelle à l'instruction \emph{\textbackslash draw}. L'argument \emph{clip} vaut \emph{false} par défaut, il indique si le tracé doit être clippé avec la fenêtre 3d courante.
    
    \item La fonction \textbf{arc3d(B,A,C,r,sens,normal)} renvoie la liste des points de cet arc (ligne polygonale 3d). 
    
    \item La fonction \textbf{arc3db(B,A,C,r,sens,normal)} renvoie cet arc sous forme d'un chemin 3d (voir Dpath3d) utilisant des courbes de Bézier.
\end{itemize}

\subsubsection{Cercle : Dcircle3d}

\begin{itemize}
    \item La méthode \textbf{g:Dcircle3d(I,R,normal,draw\_options,clip)} trace le cercle de centre $I$ (point 3d) et de rayon $R$, dans le plan contenant $I$ et normal au vecteur défini par l'argument \emph{normal} (point 3d non nul). L'argument \emph{draw\_options} est une chaîne (vide par défaut) qui sera passée telle quelle à l'instruction \emph{\textbackslash draw}. L'argument \emph{clip} vaut \emph{false} par défaut, il indique si le tracé doit être clippé avec la fenêtre 3d courante. Autre syntaxe possible :  \textbf{g:Dcircle(C,draw\_options,clip)} où \emph{C=\{I,R,normal\}}.
    
    \item La fonction \textbf{circle3d(I,R,normal)} renvoie la liste des points de ce cercle (ligne polygonale 3d). 
    
    \item La fonction \textbf{circle3db(I,R,normal)} renvoie ce cercle sous forme d'un chemin 3d (voir Dpath3d) utilisant des courbes de Bézier.
\end{itemize}
    
\subsubsection{Chemin 3d : Dpath3d}

La méthode \textbf{g:Dpath3d(chemin,draw\_options,clip)} fait le dessin du \emph{chemin}. L'argument \emph{draw\_options} est une chaîne de caractères qui sera passée directement à l'instruction \emph{\textbackslash draw}. L'argument \emph{clip} vaut \emph{false} par défaut, il indique si le tracé doit être clippé avec la fenêtre 3d courante. L'argument \emph{chemin} est une liste de points 3d suivis d'instructions (chaînes) fonctionnant sur le même principe qu'en 2d. Les instructions sont :
    \begin{itemize}
      \item \emph{"m"} pour moveto,
      \item \emph{"l"} pour lineto,
      \item \emph{"b"} pour bézier (il faut deux points de contrôles),
      \item \emph{"c"} pour cercle (il faut un point du cercle, le centre et un vecteur normal),
      \item \emph{"ca"} pour arc de cercle (il faut 3 points, un rayon, un sens et éventuellement un vecteur normal),
      \item \emph{"cl"} pour close (ferme la composante courante).
      \end{itemize}

Voici par exemple le code de la figure \ref{viewdir}.

\begin{Luacode}
\begin{luadraw}{name=viewdir}
local g = graph3d:new{ size={8,8} }
local i = cpx.I
local O, A = Origin, M(4,4,4)
local B, C, D, E = pxy(A), px(A), py(A), pz(A) --projeté de A sur le plan xOy et sur les axes
g:Dpolyline3d( {{O,A},{-5*vecI,5*vecI},{-5*vecJ,5*vecJ},{-5*vecK,5*vecK}}, "->") -- axes
g:Dpolyline3d( {{E,A,B,O}, {C,B,D}}, "dashed")
g:Dpath3d( {C,O,B,2.5,1,"ca",O,"l","cl"}, "draw=none,fill=cyan,fill opacity=0.8") --secteur angulaire
g:Darc3d(C,O,B,2.5,1,"->") -- arc de cercle pour theta
g:Dpath3d( {E,O,A,2.5,1,"ca",O,"l","cl"}, "draw=none,fill=cyan,fill opacity=0.8") --secteur angulaire
g:Darc3d(E,O,A,2.5,1,"->") -- arc de cercle pour phi
g:Dballdots3d(O) -- le point origine sous forme d'une petite sphère
g:Labelsize("footnotesize")
g:Dlabel3d(
    "$x$", 5.25*vecI,{}, "$y$", 5.25*vecJ,{}, "$z$", 5.25*vecK,{},
    "vers observateur", A, {pos="E"},
    "$O$", O, {pos="NW"},
    "$\\theta$", (B+C)/2, {pos="N", dist=0.15},
    "$\\varphi$", (A+E)/2, {pos="S",dist=0.25}
)
g:Dlabel("viewdir=\\{$\\theta,\\varphi$\\} (en degrés)",-5*i,{pos="N"}) -- label 2d
g:Show()   
\end{luadraw}      
\end{Luacode}

\subsubsection{Plan : Dplane}

La méthode \textbf{g:Dplane(P,V,L1,L2,mode,draw\_options)} permet de dessiner les bords du plan $P=\{A,u\}$ où $A$ est un point du plan et $u$ un vecteur normal au plan ($P$ est donc une table de deux points 3d). L'argument $V$ doit être un vecteur non nul du plan $P$, $L_1$ et $L_2$ sont deux longueurs. La méthode construit un parallélogramme centré sur $A$, dont un côté est $L_1\frac{V}{\|V\|}$ et l'autre $L_2\frac{W}{\|W\|}$ où $W = u\wedge V$. L'argument \emph{mode} est un entier naturel qui indique les bords à tracer. Pour calculer cet entier on utilise les variables prédéfinies : \emph{top} (=8), \emph{right} (=4), \emph{bottom} (=2), \emph{left} (=1) et \emph{all} (=15), que l'on peut ajouter entre elles, par exemple :
    \begin{itemize}
        \item mode = bottom+left : pour les côtés bas et gauche
        \item mode = top+right+bottom : pour les côtés haut, droit et bas
        \item etc
    \end{itemize}
    Par défaut le mode vaut \emph{all} ce qui correspond à \emph{top+right+bottom+left}.

\begin{demo}{Dplane, exemple avec mode = left+bottom}
\begin{luadraw}{name=Dplane}
local g = graph3d:new{size={8,8},window={-5.25,3,-2.5,2.5},margin={0,0,0,0},border=true}
local i = cpx.I
g:Labelsize("footnotesize")
local A = Origin
local P = {A, vecK}
g:Dplane(P, vecJ, 6, 6, left+bottom)
g:Dcrossdots3d({A,vecK},nil,0.75)
g:Dseg3d({A,A+2*vecK},"->")
g:Dangle3d(-vecJ,A,vecK,0.25)
g:Dpolyline3d({{M(3.5,-3,0),M(3.5,3,0)},{M(3,-3.5,0), M(-3,-3.5,0)}}, "->,line width=0.8pt")
g:Dlabel3d("$A$",A,{pos="E"}, 
    "$u$",2*vecK,{},
    "$P$", M(3,-3,0),{pos="NE", dir={vecJ,-vecI}},
    "$L_1\\frac{V}{\\|V\\|}$ (bottom)", M(3.5,0,0), {pos="S"},
    "$L_2\\frac{W}{\\|W\\|}$ (left)", M(0,-3.5,0), {pos="N",dir={-vecI,-vecJ}}
)
g:Show()
\end{luadraw}
\end{demo}

\paragraph{Attention} : les notions de haut, droite, bas et gauche sont relatives ! Elles dépendent du sens des vecteurs $u$ (vecteur normal au plan) et $V$ (vecteur donné dans le plan). Le troisième vecteur $W$ est le produit vectoriel $u\wedge V$.

\subsubsection{Courbe paramétrique : Dparametric3d}

\begin{itemize}
\item La fonction \textbf{parametric3d(p,t1,t2,nbdots,discont,nbdiv)} fait le calcul des points dela courbe et renvoie une ligne polygonale 3d (pas de dessin).
  \begin{itemize}
    \item L'argument \emph{p} est le paramétrage, ce doit être une fonction d'une variable réelle \emph{t} et à valeurs dans $\mathbf R^3$ (les images sont des points 3d), par exemple :
    \mintinline{Lua}{local p = function(t) return Mc(3,t,t/3) end}
    
    \item  Les arguments \emph{t1} et \emph{t2} sont obligatoires avec \(t1 < t2\), ils forment les bornes de l'intervalle pour le paramètre.
    
    \item L'argument \emph{nbdots} est facultatif, c'est le nombre de points (minimal) à calculer, il vaut 40 par défaut.
    
    \item L'argument \emph{discont} est un booléen facultatif qui indique s'il y a des discontinuités ou non, c'est \emph{false} par défaut.
    
    \item L'argument \emph{nbdiv} est un entier positif qui vaut 5 par défaut et indique le nombre de fois que l'intervalle entre deux valeurs consécutives du paramètre peut être coupé en deux (dichotomie) lorsque les points correspondants sont trop éloignés.
  \end{itemize}
  
\item La méthode \textbf{g:Dparametric3d(p,args)} fait le calcul des points et le dessin de la courbe paramétrée par \emph{p}. Le paramètre \emph{args} est une table à 6 champs :

\begin{TeXcode}
 { t={t1,t2}, nbdots=40, discont=true/false, clip=true/false, nbdiv=5, draw_options="" }
\end{TeXcode}

  \begin{itemize}
      \item Par défaut, le champ \emph{t} est égal à \emph{\{g:Xinf(),g:Xsup()\}},
      \item le champ \emph{nbdots} vaut 40, 
      \item le champ \emph{discont} vaut \emph{false},
      \item le champ \emph{nbdiv} vaut 5,
      \item le champ \emph{clip} vaut \emph{false}, il indique si la courbe doit être clippée avec la fenêtre 3d courante.
      \item le champ \emph{draw\_options} est une chaîne vide (celle-ci sera transmise telle quelle à l'instruction \emph{\textbackslash draw}).
  \end{itemize}
\end{itemize} 

\begin{demo}{Une courbe et ses projections sur trois plans} 
\begin{luadraw}{name=Dparametric3d}
local g = graph3d:new{window3d={-4,4,-4,4,-3,3}, window={-7.5,6.5,-7,6}, size={8,8}}
local pi = math.pi
g:Labelsize("footnotesize")
local p = function(t) return Mc(3,t,t/3) end
local L = parametric3d(p,-2*pi,2*pi,25,false,2)
g:Dboxaxes3d({grid=true,gridcolor="gray",fillcolor="LightGray"})
g:Lineoptions("dashed","red",2)
-- projection sur le plan y=-4
g:Dpolyline3d(proj3d(L,{M(0,-4,0),vecJ}))
-- projection sur le plan x=-4
g:Dpolyline3d(proj3d(L,{M(-4,0,0),vecI}))
-- projection sur le plan z=-3
g:Dpolyline3d(proj3d(L,{M(0,0,-3),vecK}))
-- dessin de la courbe
g:Lineoptions("solid","Navy",8)
g:Dparametric3d(p,{t={-2*pi,2*pi}})
g:Show()
\end{luadraw}
\end{demo}

\subsubsection{Paramétrisation d'une ligne polygonale: \emph{curvilinear\_param3d}}
Soit $L$ une liste de points 3d représentant une ligne \og continue \fg, il est possible d'obtenir une paramétrisation de cette ligne en fonction d'un paramètre $t$ entre $0$ et $1$ ($t$ est l'abscisse curviligne divisée par la longueur totale de $L$).

La fonction \textbf{curvilinear\_param3d(L,close)} renvoie une fonction d'une variable $t\in[0;1]$ et à valeurs sur la ligne  $L$ (points 3d), la valeur en $t=0$ est le premier point de $L$, et la valeur en $t=1$ est le dernier point; cette fonction est suivie d'un nombre qui représente la longueur total de L. L'argument optionnel \emph{close} indique si la ligne $L$  doit être refermée (\emph{false} par défaut).


\subsubsection{Le repère : Dboxaxes3d}

La méthode \textbf{g:Dboxaxes3d( args )} permet de dessiner les trois axes, avec un certain nombre d'options définies dans la table \emph{args}. Ces options sont :
\def\opt#1{\textcolor{blue}{\texttt{#1}}}%
\begin{itemize}
    \item \opt{xaxe=true/false}, \opt{yaxe=true/false} et \opt{zaxe=true/false} : indique si les axes correspondant doivent être dessinés ou non (true par défaut).

    \item \opt{drawbox=true/false} : indique si une boite doit être dessinée avec les axes (false par défaut).

    \item \opt{grid=true/false} : indique si une grille doit être dessinée (une pour $x$, une pour $y$ et une pour $z$). Lorsque cette option vaut true, on peut utiliser aussi les options suivantes :
        \begin{itemize}
            \item \opt{gridwidth} (=1 par défaut) indique l'épaisseur de trait de la grille en dixième de point.
            \item \opt{gridcolor} ("black" par défaut) indique la couleur de la grille.
            \item \opt{fillcolor} ("" par défaut) permet de peindre ou non le fond des grilles.
        \end{itemize}
    
    \item \opt{xlimits=\{x1,x2\}}, \opt{ylimits=\{y1,y2\}}, \opt{zlimits=\{z1,z2\}} : permet de définir les trois intervalles utilisés pour les longueurs des axes. Par défaut ce sont les valeurs fournies à l'argument \opt{window3d} à la création du graphe.

    \item \opt{xgradlimits=\{x1,x2\}}, \opt{ygradlimits=\{y1,y2\}}, \opt{zgradlimits=\{z1,z2\}} : permet de définir les trois intervalles de graduation sur les axes. Par défaut ces options ont la valeur "auto", ce qui veut dre qu'elles prennent les mêmes valeurs que \opt{xlimits}, \opt{ylimits} et \opt{zlimits}.
    \item \opt{xyzstep} : indique le pas des graduations sur les trois axes (1 par défaut).
    \item \opt{xstep}, \opt{ystep}, \opt{zstep} : indique le pas des graduations sur chaque axe (valeur de \opt{xyzstep} par défaut).

    \item \opt{xyzticks} (0.2 par défaut) : indique la longueur des graduations.

    \item \opt{labels} (true par défaut) : indique si la valeur des graduations doit être affichée ou non.
    
    \item \opt{xlabelsep}, \opt{ylabelsep}, \opt{zlabelsep} : indique la distance entre les labels et les graduations (0.25 par défaut).
    
    \item \opt{xlabelstyle}, \opt{ylabelstyle}, \opt{zlabelstyle} : indique le style des labels, c'est à dire la position par rapport au point d'ancrage. Par défaut c'est le style en cours qui s'applique.

    \item \opt{xlegend} ("\$x\$" par défaut), \opt{ylegend} ("\$y\$" par défaut), \opt{zlegend} ("\$z\$" par défaut) : permet de définir une légende pour les axes.
    
    \item \opt{xlegendsep}, \opt{ylegendsep}, \opt{zlegendsep} : indique la distance entre les legendes et les graduations (0.5 par défaut).     
\end{itemize}

\subsection{Points et labels}

\subsubsection{Points 3d : Ddots3d, Dballdots3d, Dcrossdots3d}

Il y a trois possibilités de dessiner des points 3d. Pour les deux premières, l'argument \emph{L} peut être soit un seul point 3d, soit une liste (une table) de points 3d, soit une liste de listes de points 3d :

\begin{itemize}
    \item La méthode \textbf{g:Ddots3d(L, mark\_options,clip)}. Le principe est le même que dans la version 2d, les points sont dessinés dans la couleur courante du tracé de lignes avec le style courant. L'argument \emph{mark\_options} est une chaîne de caractères facultative qui sera passée telle quelle à l'instruction \emph{\textbackslash draw} (modifications locales). L'argument \emph{clip} vaut \emph{false} par défaut, il indique si le tracé doit être clippé avec la fenêtre 3d courante.
    
    \item La méthode \textbf{g:Dballdots3d(L,color,scale,clip)} dessine les points de \emph{L} sous forme d'une sphère. L'argument facultatif \emph{color} précise la couleur de la sphère ("black" par défaut), et l'argument facultatif \emph{scale} permet de jouer sur la taille de la sphère (1 par défaut).
    
    \item La méthode \textbf{g:Dcrossdots3d(L,color,scale,clip)} dessine les points de \emph{L} sous forme d'une croix plane. L'argument \emph{L} est une liste de la forme \{point 3d, vecteur normal\} ou \{ \{point3d, vecteur normal\}, \{point3d, vecteur normal\}, ...\}. Pour chaque point 3d, le vecteur normal associé permet de déterminer le plan contenant la croix. L'argument facultatif \emph{color} précise la couleur de la croix ("black" par défaut), et l'argument facultatif \emph{scale} permet de jouer sur la taille de la croix (1 par défaut).
\end{itemize}

\begin{demo}{Un tétraèdre et les centres de gravité de chaque face}
\begin{luadraw}{name=Ddots3d}
local g = graph3d:new{viewdir={15,60},bbox=false,size={8,8}}
local A, B, C, D = 4*M(1,0,-0.5), 4*M(-1/2,math.sqrt(3)/2,-0.5), 4*M(-1/2,-math.sqrt(3)/2,-0.5), 4*M(0,0,1)
local u, v, w = B-A, C-A, D-A
-- centres de gravité faces cachées
for _, F in ipairs({{A,B,C},{B,C,D}}) do
    local G, u = isobar3d(F), pt3d.prod(F[2]-F[1],F[3]-F[1])
    g:Dcrossdots3d({G,u}, "blue",0.75)
    g:Dpolyline3d({{F[1],G,F[2]},{G,F[3]}},"dotted")
end
-- dessin du tétraèdre construit sur A, B, C et D
g:Dpoly(tetra(A,u,v,w),{mode=mShaded,opacity=0.7,color="Crimson"})
-- centres de gravité faces visibles
for _, F in ipairs({{A,B,D},{A,C,D}}) do
    local G, u = isobar3d(F), pt3d.prod(F[2]-F[1],F[3]-F[1])
    g:Dcrossdots3d({G,u}, "blue",0.75)
    g:Dpolyline3d({{F[1],G,F[2]},{G,F[3]}},"dotted")
end
g:Dballdots3d({A,B,C,D}, "orange") --sommets
g:Show()
\end{luadraw}
\end{demo}

\subsubsection{Labels 3d : Dlabel3d}

La méthode pour placer un label dans l'espace est :

\hfil\textbf{g:Dlabel3d(text1, anchor1, args1, text2, anchor2, args2, ...)}.\hfil

    \begin{itemize}
    \item  Les arguments \emph{text1, text2,...} sont des chaînes de caractères, ce sont les labels.
    \item  Les arguments \emph{anchor1, anchor2,...} sont des points 3d représentant les points d'ancrage des labels.
    \item  Les arguments \emph{args1,arg2,...}  permettent de définir localement les paramètres des labels, ce sont des tables à 4 champs :
\begin{TeXcode}
    { pos=nil, dist=0, dir={dirX,dirY,dep}, node_options="" }
\end{TeXcode}
        \begin{itemize}
            \item Le champ \emph{pos} indique la position du label dans le plan de l'écran par rapport au point d'ancrage, il peut valoir \emph{"N"} pour nord, \emph{"NE"} pour nord-est, \emph{"NW"} pour nord-ouest, ou encore \emph{"S"}, \emph{"SE"}, \emph{"SW"}. Par défaut, il vaut \emph{center}, et dans ce cas le label est centré sur le point  d'ancrage.
            \item Le champ \emph{dist} est une distance en cm (dans le plan de l'écran) qui vaut $0$ par défaut, c'est la distance entre le label et son point d'ancrage lorsque \emph{pos} n'est pas égal a \emph{center}.
            \item \emph{dir=\{dirX,dirY,dep\}} est la direction de l'écriture dans l'espace (\emph{nil}, valeur par défaut, pour le sens par défaut). Les 3 valeurs \emph{dirX}, \emph{dirY} et \emph{dep} sont trois points 3d représentant 3 vecteurs, les deux premiers indiquent le sens de l'écriture, le troisième un déplacement (translation) du label par rapport au point d'ancrage.
            \item L'argument \emph{node\_options} est une chaîne (vide par défaut) destinée à recevoir des options qui seront directement passées à tikz dans l'instruction \emph{node{[}{]}}.
            \item Les labels sont dessinés dans la couleur courante du texte du document, mais on peut changer de couleur avec l'argument \emph{node\_options} en mettant par exemple : \emph{node\_options="color=blue"}.
            
            \textbf{Attention} : les options choisies pour un label s'appliquent aussi aux labels suivants si elles sont inchangées.
        \end{itemize}
  \end{itemize}

\subsection{Solides de base (sans facette)}

\subsubsection{Cylindre : Dcylinder}

Dessiner un cylindre à base circulaire (droit ou penché). Plusieurs syntaxes possibles :
\begin{itemize}
    \item Ancienne syntaxe : \textbf{g:Dcylinder(A,V,r,args)} dessine un cylindre droit, où \emph{A} est un point 3d représentant le centre d'une des faces circulaires, \emph{V} est un point 3d, c'est un vecteur représentant l'axe du cône, le centre de la face circulaire opposée est le point $A+V$ (cette face est orthogonale à $V$), et \emph{r} est le rayon de la base circulaire.
    \item La syntaxe : \textbf{g:Dcylinder(A,r,B,args)} dessine un cylindre droit, où \emph{A} est un point 3d représentant le centre d'une des faces circulaires, \emph{B} est le centre de la face opposée, et \emph{r} est le rayon. Le cylindre est droit, c'est à dire que les faces circulaires sont orthogonales à l'axe $(AB)$.
    \item Pour un cylindre penché :  \textbf{g:Dcylinder(A,r,V,B,args)}, où \emph{A} est un point 3d représentant le centre d'une des faces circulaires, \emph{B} est le centre de la face circulaire opposée, \emph{r} est le rayon, et \emph{V} est un vecteur 3d non nul orthogonal au plan des faces circulaires.
\end{itemize}
Pour les trois syntaxes, \emph{args} est une table à 5 champs pour définir les options de tracé. Ces options sont :
       \begin{itemize}
            \item \emph{mode=mWireframe ou mGrid} (\emph{mWireframe} par défaut). En mode \emph{mWireframe} c'est un dessin en fil de fer, en mode \emph{mGrid} c'est un dessin en grille (comme s'il y avait des facettes).
            \item \emph{hiddenstyle}, définit le style de ligne pour les parties cachées (mettre "noline" pour ne pas les afficher). Par défaut cette option a la valeur de la variable globale \emph{Hiddenlinestyle} qui est elle même initialisée avec la valeur \emph{"dotted"}.
            \item \emph{hiddencolor}, définit la couleur des lignes cachées (égale à edgecolor par défaut).
            \item \emph{edgecolor}, définit la couleur des lignes (couleur courante par défaut).
            \item \emph{color=""}, lorsque cette option est une chaîne vide (valeur par défaut) il n'y a pas de remplissage,  lorsque c'est une couleur (sous forme de chaîne) il y a un remplissage avec un gradient linéaire.
            \item \emph{opacity=1}, définit la transparence du dessin.
        \end{itemize}

\subsubsection{Cône : Dcone}

Dessiner un cône à base circulaire (droit ou penché). Plusieurs syntaxes possibles :
\begin{itemize}
    \item Ancienne syntaxe : \textbf{g:Dcone(A,V,r,args)} dessine un cône droit, où \emph{A} est un point 3d représentant le sommet du cône, \emph{V} est un point 3d, c'est un vecteur représentant l'axe du cône, le centre de la face circulaire est le point $A+V$ (cette face est orthogonale à $V$), et \emph{r} est le rayon de la base circulaire.
    \item La syntaxe : \textbf{g:Dcone(C,r,A,args)} dessine un cône droit, où \emph{A} est un point 3d représentant le sommet du cône, \emph{C} est le centre de la face circulaire, et \emph{r} est le rayon. Le cône est droit, c'est à dire que la face circulaire est orthogonale à l'axe $(AC)$.
    \item Pour un cône penché :  \textbf{g:Dcone(C,r,V,A,args)}, où \emph{A} est un point 3d représentant le sommet du cône, \emph{C} est le centre de la face circulaire, \emph{r} est le rayon, et \emph{V} est un vecteur 3d non nul orthogonal au plan de la face circulaire.
\end{itemize}
Pour les trois syntaxes, \emph{args} est une table à 5 champs pour définir les options de tracé. Ces options sont :
\begin{itemize}
    \item \emph{mode=mWireframe ou mGrid} (\emph{mWireframe} par défaut). En mode \emph{mWireframe} c'est un dessin en fil de fer, en mode \emph{mGrid} c'est un dessin en grille (comme s'il y avait des facettes).
    \item \emph{hiddenstyle}, définit le style de ligne pour les parties cachées (mettre "noline" pour ne pas les afficher). Par défaut cette option a la valeur de la variable globale \emph{Hiddenlinestyle} qui est elle même initialisée avec la valeur \emph{"dotted"}.
    \item \emph{hiddencolor}, définit la couleur des lignes cachées (égale à edgecolor par défaut).
    \item \emph{edgecolor}, définit la couleur des lignes (couleur courante par défaut).
    \item \emph{color=""}, lorsque cette option est une chaîne vide (valeur par défaut) il n'y a pas de remplissage,  lorsque c'est une couleur (sous forme de chaîne) il y a un remplissage avec un gradient linéaire.
    \item \emph{opacity=1}, définit la transparence du dessin.
\end{itemize}

\subsubsection{Tronc de cône : Dfrustum}

Dessiner un tronc de cône à base circulaire (droit ou penché). Deux syntaxes possibles :
La méthode \textbf{g:Dfrustum(A,V,R,r,args)} dessine un tronc de cône à bases circulaires.

\begin{itemize}
    \item La syntaxe : \textbf{g:Dfrustum(A,R,r,V,args)} pour un tronc de cône droit, \emph{A} est un point 3d représentant le centre de la face de rayon \emph{R}, \emph{V} est un vecteur 3d représentant l'axe du tronc de cône, le centre de la deuxième face circulaire est le point $A+V$, et son rayon est \emph{r},  (les faces sont orthogonales à $V$). Lorsque $R=r$ on a simplement un cylindre.
    \item La syntaxe : \textbf{g:Dfrustum(A,R,r,V,B,args)} pour un tronc de cône penché, \emph{A} est un point 3d représentant le centre de la face de rayon \emph{R}, \emph{V} est un vecteur 3d représentant un vecteur normal aux faces circulaires, le centre de la deuxième face circulaire est le point $B$, et son rayon est \emph{r}. Lorsque $R=r$ on a un cylindre penché.
\end{itemize}
Dans les deux cas, \emph{args} est une table à 5 champs pour définir les options de tracé. Ces options sont :
    \begin{itemize}
        \item \emph{mode=mWireframe ou mGrid} (\emph{mWireframe} par défaut). En mode \emph{mWireframe} c'est un dessin en fil de fer, en mode \emph{mGrid} c'est un dessin en grille (comme s'il y avait des facettes).
        \item \emph{hiddenstyle}, définit le style de ligne pour les parties cachées (mettre "noline" pour ne pas les afficher). Par défaut cette option a la valeur de la variable globale \emph{Hiddenlinestyle} qui est elle même initialisée avec la valeur \emph{"dotted"}.
        \item \emph{hiddencolor}, définit la couleur des lignes cachées (égale à edgecolor par défaut).
        \item \emph{edgecolor}, définit la couleur des lignes (couleur courante par défaut).
        \item \emph{color=""}, lorsque cette option est une chaîne vide (valeur par défaut) il n'y a pas de remplissage,  lorsque c'est une couleur (sous forme de chaîne) il y a un remplissage avec un gradient linéaire.
        \item \emph{opacity=1}, définit la transparence du dessin.
    \end{itemize}

\subsubsection{ Sphère : Dsphere}

La méthode \textbf{g:Dsphere(A,r,args)} dessine une sphère

\begin{itemize}
    \item \emph{A} est un point 3d représentant le centre de la sphère.
    \item \emph{r} est le rayon de la pshère
    \item \emph{args} est une table à 5 champs pour définir les options de tracé. Ces options sont :
        \begin{itemize}
            \item \emph{mode=mWireframe ou mGrid ou mBorder} (\emph{mWireframe} par défaut). En mode \emph{mWireframe} on dessine le contour (cercle) et l'équateur, en mode \emph{mGrid} c'est le contour avec méridiens et fuseaux (grille), et en mode \emph{mBorder} c'est le contour uniquement.
            \item \emph{hiddenstyle}, définit le style de ligne pour les parties cachées (mettre "noline" pour ne pas les afficher). Par défaut cette option a la valeur de la variable globale \emph{Hiddenlinestyle} qui est elle même initialisée avec la valeur \emph{"dotted"}.
            \item \emph{hiddencolor}, définit la couleur des lignes cachées (égale à edgecolor par défaut).
            \item \emph{edgecolor}, définit la couleur des lignes (couleur courante par défaut).
            \item \emph{color=""}, lorsque cette option est une chaîne vide (valeur par défaut) il n'y a pas de remplissage,  lorsque c'est une couleur (sous forme de chaîne) il y a un remplissage avec un gradient linéaire.
            \item \emph{opacity=1}, définit la transparence du dessin.
            \item \item \emph{edgestyle}, définit le style de ligne pour les arêtes visibles, par défaut c'est le style courant.
            \item \emph{edgecolor}, définit la couleur des arêtes visibles (couleur courante par défaut).
            \item \emph{edgewidth}, définit l'épaisseur des des arêtes visible en dixième de point (épaisseur courante par défaut).
        \end{itemize}
\end{itemize}

\begin{demo}{Cylindres, cônes et sphères}
\begin{luadraw}{name=cylindre_cone_sphere}
local g = graph3d:new{ size={10,10} }
local dessin = function(args)
    g:Dsphere(M(-1,-2.5,1),2.5, args)
    g:Dcone(M(-1,2.5,5),-5*vecK,2, args)
    g:Dcylinder(M(3,-2,0),6*vecJ,1.5, args)
end
-- en haut à gauche, options par défaut
g:Saveattr(); g:Viewport(-5,0,0,5); g:Coordsystem(-5,5,-5,5,true); dessin(); g:Restoreattr()
-- en haut à droite
g:Saveattr(); g:Viewport(0,5,0,5); g:Coordsystem(-5,5,-5,5,true)
dessin({mode=mGrid, hiddenstyle="solid", hiddencolor="LightGray"}); g:Restoreattr()
-- en bas à gauche
g:Saveattr(); g:Viewport(-5,0,-5,0); g:Coordsystem(-5,5,-5,5,true)
dessin({mode=Border, color="orange"}); g:Restoreattr()
-- en bas à droite
g:Saveattr(); g:Viewport(0,5,-5,0); g:Coordsystem(-5,5,-5,5,true)
dessin({mode=mGrid,opacity=0.8,hiddenstyle="noline",color="LightBlue"}); g:Restoreattr()
g:Show()            
\end{luadraw}
\end{demo}

\section{Solides à facettes}

\subsection{Définition d'un solide}

Il y a deux façons de définir un solide :
\begin{enumerate}
    \item Sous forme d'une liste (table) de facettes. Une facette est elle-même une liste de points 3d (au moins 3) coplanaires et non alignés, qui sont les sommets. Les facettes sont supposées convexes et elles sont orientées par l'ordre d'apparition des sommets. C'est à dire, si $A$, $B$ et $C$ sont les trois premiers sommets d'une facette $F$, alors la facette est orientée avec le vecteur normal $\vec{AB}\wedge\vec{AC}$. Si ce vecteur normal est dirigé vers l'observateur, alors la facette est considérée comme visible. Dans la définition d'un solide, les vecteurs normaux aux facettes doivent être dirigés vers \textbf{l'extérieur} du solide pour que l'orientation soit correcte.
    
    \item Sous forme de \textbf{polyèdre}, c'est à dire une table à deux champs, un premier champ appelé \emph{vertices} qui est la liste des sommets du polyèdre (points 3d), et un deuxième champ appelé \emph{facets} qui la liste des facettes, mais ici, dans la définition des facettes, les sommets sont remplacés par leur indice dans la liste \emph{vertices}. Les facettes sont orientées de la même façon que précédemment.
\end{enumerate}

Par exemple, considérons les quatre points $A=M(-2,-2,0)$, $B=M(3,0,0)$, $C=M(-2,2,0)$ et $D=M(0,0,4)$, alors on peut définir le tétraèdre construit sur ces quatre points :
\begin{itemize}
    \item soit sous forme d'une liste de facettes : \emph{T=\{\{A,B,D\},\{B,C,D\},\{C,A,D\},\{A,C,B\}\}} (attention à l'orientation),
    \item soit sous forme de polyèdre : 
    \emph{T=\{vertices=\{A,B,C,D\}, facets=\{\{1,2,4\},\{2,3,4\},\{3,1,4\},\{1,3,2\}\}\}}.
\end{itemize}

\paragraph{Fonctions de conversion entre les deux définitions}
\begin{itemize}
    \item La fonction \textbf{poly2facet(P)} où $P$ est un polyèdre, renvoie ce solide sous forme d'une liste de facettes.
    \item La fonction \textbf{facet2poly(L,epsilon)} renvoie la liste de facettes $L$ sous forme de polyèdre. L'argument facultatif \emph{epsilon} vaut $10^{-8}$ par défaut, il précise à combien près sont faites les comparaisons entre points 3d.
\end{itemize}

\subsection{Dessin d'un polyèdre : Dpoly}

La fonction \textbf{g:Dpoly(P,options)} permet de représenter le polyèdre $P$ (par l'algorithme naïf du peintre). L'argument \emph{options} est une table contenant les options :
\begin{itemize}
    \item \opt{mode=} : définit le mode de représentation.
        \begin{itemize}
            \item \emph{mode=mWireframe} : mode fil de fer, on dessine les arêtes visibles et cachées.
            \item \emph{mode=mFlat} : on dessine les faces de couleur unie, ainsi que les arêtes visibles.
            \item \emph{mode=mFlatHidden} : on dessine les faces de couleur unie, les arêtes visibles, et les arêtes cachées.
            \item \emph{mode=mShaded} : on dessine les faces de couleur nuancée en fonction de leur inclinaison, ainsi que les arêtes visibles. C'est le mode par défaut.
            \item \emph{mode=mShadedHidden} : on dessine les faces de couleur nuancée en fonction de leur inclinaison, les arêtes visibles et cachées.
            \item \emph{mode=mShadedOnly} :  on dessine les faces de couleur nuancée en fonction de leur inclinaison, mais pas les arêtes.
        \end{itemize}
        \item \opt{contrast} : c'est un nombre qui vaut 1 par défaut. Ce nombre permet d'accentuer ou diminuer la nuance des couleurs des facettes dans les modes \emph{mShaded}, \emph{mShadedHidden}, \emph{mShadedOnly}.
        \item \opt{edgestyle} : est une chaîne qui définit le style de ligne des arêtes. C'est le style courant par défaut.
        \item \opt{edgecolor} : est une chaîne qui définit la couleur des arêtes. C'est la couleur courante des lignes par défaut.
        \item \opt{hiddenstyle} : est une chaîne qui définit le style de ligne des arêtes cachées. Par défaut c'est la valeur contenue dans la variable globale \emph{Hiddenlinestyle} (qui vaut elle-même "dotted" par défaut).
        \item \opt{hiddencolor} : est une chaîne qui définit la couleur des arêtes cachées. C'est la couleur courante des lignes par défaut.
        \item \opt{edgewidth} : épaisseur de trait des arêtes en dixième de point. C'est l'épaisseur courante par défaut.
        \item \opt{opacity} : nombre entre 0 et 1 qui permet de mettre une transparence ou non sur les facettes. La valeur par défaut est 1, ce qui signifie pas de transparence.
        \item \opt{backcull} : booléen qui vaut false par défaut. Lorsqu'il a la valeur true, les facettes considérées comme non visibles (vecteur normal non dirigé vers l'observateur) ne sont pas affichées. Cette option est intéressante pour les polyèdres convexes car elle permet de diminuer le nombre de facettes à dessiner.
        \item \opt{twoside} : booléen qui vaut true par défaut, ce qui signifie qu'on distingue les deux côtés des facettes (intérieur et extérieur), les deux côtés n'auront pas exactement la même couleur.
        \item \opt{color} : chaîne définissant la couleur de remplissage des facettes, c'est "white" par défaut.
        \item \opt{usepalette} (\emph{nil} par défaut), cette option permet de préciser une palette de couleurs pour peindre les facettes ainsi qu'un mode de calcul, la syntaxe est : \emph{usepalette = \{palette,mode\}}, où \emph{palette} désigne une table de couleurs qui sont elles-mêmes des tables de la forme \emph{\{r,g,b\}} où r, g et b sont des nombres entre $0$ et $1$, et \emph{mode} qui est une chaîne qui peut être soit \emph{"x"}, soit \emph{"y"}, ou soit \emph{"z"}. Dans le premier cas par exemple, les facettes au centre de gravité d'abscisse minimale ont la première couleur de la palette, les facettes au centre de gravité d'abscisse maximale ont la dernière couleur de la palette, pour les autres, la couleur est calculée en fonction de l'abscisse du centre de gravité par interpolation linéaire.
\end{itemize}

\begin{demo}{Section d'un tétraèdre par un plan}
\begin{luadraw}{name=tetra_coupe}
local g = graph3d:new{viewdir={10,60},bbox=false, size={10,10}, bg="gray!30"}
local A,B,C,D = M(-2,-4,-2),M(4,0,-2),M(-2,4,-2),M(0,0,2)
local T = tetra(A,B-A,C-A,D-A) -- tétraèdre de sommets A, B, C, D
local plan = {Origin, -vecK}  -- plan de coupe
local T1, T2, section = cutpoly(T,plan) -- on coupe du tétraèdre
-- T1 est le polyèdre résultant dans le demi espace contenant -vecK
-- T2 est le polyèdre résultant dans l'autre demi espace
-- section est une facette (c'est la coupe)
g:Dpoly(T1,{color="Crimson", edgecolor="white", opacity=0.8, edgewidth=8})
g:Filloptions("bdiag","Navy"); g:Dpolyline3d(section,true,"draw=none")
g:Dpoly(shift3d(T2,2*vecK), {color="Crimson", edgecolor="white", opacity=0.8, edgewidth=8})
g:Dballdots3d({A,B,C,D+2*vecK}) -- on a dessiné T2 translaté avec le vecteur 2*vecK
g:Show()
\end{luadraw}
\end{demo}

\subsection{Fonctions de construction de polyèdres}

Les fonctions suivantes renvoient un polyèdre, c'est à dire une table à deux champs, un premier champ appelé \emph{vertices} qui est la liste des sommets du polyèdre (points 3d), et un deuxième champ appelé \emph{facets} qui la liste des facettes, mais dans la définition des facettes, les sommets sont remplacés par leur indice dans la liste \emph{vertices}.

\begin{itemize}
    \item \textbf{tetra(S,v1,v2,v3)} renvoie le tétraèdre de sommets $S$ (point 3d), $S+v1$, $S+v2$, $S+v3$. Les trois vecteurs $v1$, $v2$, $v3$ (points 3d) sont supposés dans le sens direct.
    
    \item \textbf{parallelep(A,v1,v2,v3)} renvoie le parallélépipède construit à partir du sommet $A$ (point 3d) et de 3 vecteurs $v1$, $v2$, $v3$ (points 3d) supposés dans le sens direct.
    
    \item \textbf{prism(base,vector,open)} renvoie un prisme, l'argument \emph{base} est une liste de points 3d (une des deux bases du prisme), \emph{vector} est le vecteur de translation (point 3d) qui permet d'obtenir la seconde base. L'argument facultatif \emph{open} est un booléen indiquant si le prisme est ouvert ou non (false par défaut). Dans le cas où il est ouvert, seules les facettes latérales sont renvoyées. La \emph{base} doit être orientée par le \emph{vector}.
    
    \item \textbf{pyramid(base,vertex,open)} renvoie une pyramide, l'argument \emph{base} est une liste de points 3d, \emph{vertex} est le sommet de la pyramide (point 3d). L'argument facultatif \emph{open} est un booléen indiquant si la pyramide est ouverte ou non (false par défaut). Dans le cas où elle est ouverte, seules les facettes latérales sont renvoyées. La \emph{base} doit être orientée par le sommet.
    
    \item \textbf{regular\_pyramid(n,side,height,open,center,axe)} renvoie une pyramide régulière, $n$ est le nombre de côtés de la base, l'argument \emph{side} est la longueur d'un côté, et \emph{height} est la hauteur de la pyramide. L'argument facultatif \emph{open} est un booléen indiquant si la pyramide est ouverte ou non (false par défaut). Dans le cas où elle est ouverte, seules les facettes latérales sont renvoyées. L'argument facultatif \emph{center} est le centre de la base (\emph{Origin} par défaut), et l'argument facultatif \emph{axe} est un vecteur directeur de l'axe de la pyramide (\emph{vecK} par défaut).
    
    \item \textbf{truncated\_pyramid(base,vertex,height,open)} renvoie une pyramide tronquée, l'argument \emph{base} est une liste de points 3d, \emph{vertex} est le sommet de la pyramide (point 3d). L'argument \emph{height} est un nombre indiquant la hauteur par rapport à la base, où s'effectue la troncature, celle-ci est parallèle au plan de la base. L'argument facultatif \emph{open} est un booléen indiquant si la pyramide est ouverte ou non (false par défaut). Dans le cas où elle est ouverte, seules les facettes latérales sont renvoyées. La \emph{base} doit être orientée par le sommet.
    
    \item \textbf{cylinder(A,V,R,nbfacet,open)} renvoie un cylindre de rayon $R$, d'axe \{A,V\} où $A$ est un point 3d, centre d'une des bases circulaires et $V$ vecteur 3d non nul tel que le centre de la seconde base est le point $A+V$. L'argument facultatif \emph{nbfacet} vaut 35 par défaut (nombre de facettes latérales). L'argument facultatif \emph{open} est un booléen indiquant si le cylindre est ouvert ou non (false par défaut). Dans le cas où il est ouvert, seules les facettes latérales sont renvoyées.
    
    \item \textbf{cylinder(A,R,B,nbfacet,open)} renvoie un cylindre de rayon $R$, d'axe $(AB)$ où $A$ est un point 3d, centre d'une des bases circulaires et $B$ le centre de la seconde base. Le cylindre est droit. L'argument facultatif \emph{nbfacet} vaut 35 par défaut (nombre de facettes latérales). L'argument facultatif \emph{open} est un booléen indiquant si le cylindre est ouvert ou non (false par défaut). Dans le cas où il est ouvert, seules les facettes latérales sont renvoyées.
    
   \item \textbf{cylinder(A,R,V,B,nbfacet,open)} renvoie un cylindre de rayon $R$, d'axe $(A)$ où $A$ est un point 3d, centre d'une des bases circulaires, $B$ est le centre de la seconde base, et \emph{V} est un vecteur 3d normal au plan des bases circulaires (le cylindre peut donc être penché). L'argument facultatif \emph{nbfacet} vaut 35 par défaut (nombre de facettes latérales). L'argument facultatif \emph{open} est un booléen indiquant si le cylindre est ouvert ou non (false par défaut). Dans le cas où il est ouvert, seules les facettes latérales sont renvoyées.
    
    \item \textbf{cone(A,V,R,nbfacet,open)} renvoie un cône de sommet $A$ (point 3d), d'axe \{A,V\}, de base circulaire le cercle de centre $A+V$ de rayon $R$ (dans un plan orthogonal à $V$). L'argument facultatif \emph{nbfacet} vaut 35 par défaut (nombre de facettes latérales). L'argument facultatif \emph{open} est un booléen indiquant si le cône est ouvert ou non (false par défaut). Dans le cas où il est ouvert, seules les facettes latérales sont renvoyées.

    \item \textbf{cone(C,R,A,nbfacet,open)} renvoie un cône de sommet $A$ (point 3d), \emph{C} est le centre de base circulaire et \emph{R} son rayon (dans un plan orthogonal à l'axe $(AC)$). L'argument facultatif \emph{nbfacet} vaut 35 par défaut (nombre de facettes latérales). L'argument facultatif \emph{open} est un booléen indiquant si le cône est ouvert ou non (false par défaut). Dans le cas où il est ouvert, seules les facettes latérales sont renvoyées.    
    
    \item \textbf{cone(C,R,V,A,nbfacet,open)} renvoie un cône de sommet $A$ (point 3d), \emph{C} est le centre de base circulaire, \emph{R} son rayon, la base est dans un plan orthogonal à \emph{V} (vecteur 3d). L'axe $(AC)$ n'est donc pas forcément orthogonal à la face circulaire (cône penché). L'argument facultatif \emph{nbfacet} vaut 35 par défaut (nombre de facettes latérales). L'argument facultatif \emph{open} est un booléen indiquant si le cône est ouvert ou non (false par défaut). Dans le cas où il est ouvert, seules les facettes latérales sont renvoyées.        

    \item \textbf{frustum(C,R,r,V,nbfacet,open)} renvoie un tronc de cône droit. Le point $C$ (point 3d) est le centre de la base circulaire de rayon $R$, le vecteur $V$ dirige l'axe du tronc de cône. Le centre de l'autre base circulaire est le point $C+V$, et son rayon est $r$ (les bases sont orthogonales à $V$). L'argument facultatif \emph{nbfacet} vaut 35 par défaut (nombre de facettes latérales). L'argument facultatif \emph{open} est un booléen indiquant si le tronc de cône est ouvert ou non (false par défaut). Dans le cas où il est ouvert, seules les facettes latérales sont renvoyées.    
    
    \item \textbf{frustum(C,R,r,V,A,nbfacet,open)} renvoie un tronc de cône droit. Le point $C$ (point 3d) est le centre de la base circulaire de rayon $R$, le centre de l'autre base circulaire est le point $A$, et son rayon est $r$, les bases sont orthogonales au vecteur $V$, mais pas forcément orthogonales à l'axe $(AC)$. L'argument facultatif \emph{nbfacet} vaut 35 par défaut (nombre de facettes latérales). L'argument facultatif \emph{open} est un booléen indiquant si le tronc de cône est ouvert ou non (false par défaut). Dans le cas où il est ouvert, seules les facettes latérales sont renvoyées.

    \item \textbf{sphere(A,R,nbu,nbv)} renvoie la sphère de centre $A$ (point 3d) et de rayon $R$. L'argument facultatif \emph{nbu} représente le nombre de fuseaux (36 par défaut) et l'argument facultatif \emph{nbv} le nombre de parallèles (20 par défaut).
\end{itemize}

\begin{demo}{Cône tronqué, pyramide tronquée, cylindre oblique}
\begin{luadraw}{name=frustum_pyramid}
local g = graph3d:new{adjust2d=true,bbox=false, size={10,10} }
g:Dfrustum(M(-1,-4,0),3,1,5*vecK, {color="cyan"})
g:Dcylinder(M(-4,4,0),2,vecK,M(-4,2,5), {color="orange"})
local base = map(toPoint3d,polyreg(0,3,5))
g:Dpoly(truncated_pyramid( shift3d(base,8*vecI-vecJ-2*vecK), M(5,0,5),4), {mode=4,color="Crimson"})
g:Dcone(M(6,7,-2),3,vecK,M(6,8,5),{color="Pink"})
g:Show()            
\end{luadraw}
\end{demo}

\paragraph{Remarque} : nous avons déjà des primitives pour dessiner des cylindres, cônes, et sphères sans passer par des facettes. Un des intérêts de donner une définition de ces objets sous forme de polyèdres est que l'on va pouvoir faire certains calculs sur ces objets comme par exemple des sections planes.

\begin{demo}{Hyperbole : intersection cône - plan}
\begin{luadraw}{name=hyperbole}
local g = graph3d:new{window={-6,6,-8,6}, viewdir={45,60}, size={10,10}}
Hiddenlinestyle = "dashed"
local C1 = cone(Origin,4*vecK,3,35,true)
local C2 = cone(Origin, -4*vecK,3,35,true)
local P = {M(1,0,-1),vecI} -- plan de coupe
local I1 = g:Intersection3d(C1,P) -- intersection entre le cône C1 et le plan P
local I2 = g:Intersection3d(C2,P) -- intersection entre le cône C2 et le plan P
-- I1 et I2 sont de type Edges (arêtes)
g:Dcone(Origin,4*vecK,3,{color="orange"}); g:Dcone(Origin,-4*vecK,3,{color="orange"})
g:Lineoptions("solid","Navy",8)
g:Dedges(I1,{hidden=true}); g:Dedges(I2,{hidden=true}) -- dessin des arêtes I1 et I2
g:Dplane(P, vecK,12,8)
g:Show()
\end{luadraw}
\end{demo}

Dans cet exemple, les cônes $C_1$ et $C_2$ sont définis sous forme de polyèdres pour déterminer leur intersection avec le plan $P$, mais pas pour les dessiner. La méthode \textbf{g:Intersection3d(C1,P)} renvoie l'intersection du polyèdre $C_1$ avec le plan $P$ sous la forme d'une table à deux champs, un champ nommé \emph{visible} qui contient une ligne polygonale 3d représentant les "arêtes" (segments) visibles de l'intersection (c'est à dire qui sont sur une facette visible de $C_1$), et un autre champ nommé \emph{hidden} qui contient une ligne polygonale 3d représentant les "arêtes" cachées de l'intersection (c'est à dire qui sont sur une facette non visible de $C_1$). La méthode \textbf{g:Dedges} permet de dessiner ce type d'objets.

\begin{demo}{Section de cône avec plusieurs vues}
\begin{luadraw}{name=several_views}
local g = graph3d:new{window3d={-3,3,-3,3,-3,3}, size={10,10}, margin={0,0,0,0}}
g:Labelsize("footnotesize")
local y0, R = 1, 2.5
local C = cone(M(0,0,3),-6*vecK,R,35,true) -- cone ouvert
local P1 = {M(0,0,0),vecK+vecJ} -- 1er plan de coupe
local P2 = {M(0,y0,0),vecJ} -- 2ieme plan de coupe
local I, I2 
local dessin = function() -- un dessin par vue
    g:Dboxaxes3d({grid=true,gridcolor="gray",fillcolor="LightGray"})
    I1 = g:Intersection3d(C,P1) -- intersection entre le cône C et les plans P1 et P2
    I2 = g:Intersection3d(C,P2) -- I1 et I2 sont de type Edges
    g:Dpolyline3d( {{M(0,-3,3),M(0,0,3),M(0,0,-3),M(3,0,-3)}, {M(0,0,-3),M(0,3,-3)}},"red,line width=0.4pt" )
    g:Dcone( M(0,0,3),-6*vecK,R, {color="cyan"})
    g:Dedges(I1, {hidden=true,color="Navy", width=8})
    g:Dedges(I2, {hidden=true,color="DarkGreen", width=8})
end
-- en haut à gauche, vue dans l'espace, on ajoute les plans au dessin
g:Saveattr(); g:Viewport(-5,0,0,5); g:Coordsystem(-7,6,-6,5,1); g:Setviewdir(30,60); dessin()
g:Dpolyline3d( {M(-3,-3,3),M(3,-3,3),M(3,3,-3),M(-3,3,-3)},"Navy,line width=0.8pt")
g:Dpolyline3d( {M(-3,y0,3),M(3,y0,3),M(3,y0,-3)},"DarkGreen,line width=0.8pt")
g:Dlabel3d( "$P_1$",M(3,-3,3),{pos="SE",dir={-vecI,-vecJ+vecK},node_options="Navy, draw"})
g:Dlabel3d( "$P_2$",M(-3,y0,3),{pos="SW",dir={-vecI,vecK},node_options="DarkGreen,draw"})
g:Restoreattr()
-- en haut à droite, projection sur le plan xOy
g:Saveattr(); g:Viewport(0,5,0,5); g:Coordsystem(-6,6,-6,5,1); g:Setviewdir("xOy"); dessin()
g:Restoreattr()
-- en bas à gauche, projection sur le plan xOz
g:Saveattr(); g:Viewport(-5,0,-5,0); g:Coordsystem(-6,6,-6,5,1); g:Setviewdir("xOz"); dessin()
g:Restoreattr()
-- en bas à droite, projection sur le plan yOz
g:Saveattr(); g:Viewport(0,5,-5,0); g:Coordsystem(-6,6,-6,5,1); g:Setviewdir("yOz"); dessin()
g:Restoreattr()
g:Show()
\end{luadraw}
\end{demo}

\subsection{Lecture dans un fichier obj}

La fonction \textbf{red\_obj\_file(file)}\footnote{Cette fonction est une contribution de Christophe BAL.} permet de lire le contenu du fichier \emph{obj} désigné par la chaîne de caractères \emph{file}. La fonction lit les définitions des sommets (lignes commençant par \verb|v |), et les lignes définissant les facettes (lignes commençant par \verb|f |). Les autres lignes sont ignorées. La fonction renvoie une séquence constituée du polyèdre, suivi d'une liste de quatre réels \emph{\{x1,x2,y1,y2,z1,z2\}} représentant la boîte 3d englobante (bounding box) du polyèdre.

\begin{demo}{Masque de Nefertiti}
\begin{luadraw}{name=lecture_obj}
local P,bbox = read_obj_file("obj/nefertiti.obj")
local g = graph3d:new{window3d=bbox,window={-7,5,-8,5},viewdir={35,60},
    margin={0,0,0,0}, size={10,10}, bg="LightGray"}
g:Dpoly(P, {usepalette={palAutumn,"z"},mode=mShadedOnly})
g:Show() 
\end{luadraw}
\end{demo}


\subsection{Dessin d'une liste de facettes : Dfacet et Dmixfacet}

Il y a deux méthodes possibles :
\begin{enumerate}
    \item Pour un solide $S$ sous forme d'une liste de facettes (avec points 3d), la méthode est :
    \par\hfil\textbf{g:Dfacet(S,options)}\hfil\par
    où $S$ est la liste de facettes et \emph{options} une table définissant les options. Celles-ci sont :
\begin{itemize}
    \item \opt{mode=} : définit le mode de représentation.
        \begin{itemize}
            \item \emph{mode=mWireframe} : mode fil de fer, on dessine les arêtes seulement.
            \item \emph{mode=mFlat ou mFlatHidden} : on dessine les faces de couleur unie, ainsi que les arêtes.
            \item \emph{mode=mShaded ou mShadedHidden} : on dessine les faces de couleur nuancée en fonction de leur inclinaison, ainsi que les arêtes. Le mode par défaut est 3.
            \item \emph{mode=mShadedOnly} :  on dessine les faces de couleur nuancée en fonction de leur inclinaison, mais pas les arêtes.
        \end{itemize}
        \item \opt{contrast} : c'est un nombre qui vaut 1 par défaut. Ce nombre permet d'accentuer ou diminuer la nuance des couleurs des facettes dans les modes \emph{mShaded}, \emph{mShadedHidden}, \emph{mShadedOnly}.
        \item \opt{edgestyle} : est une chaîne qui définit le style de ligne des arêtes. C'est le style courant par défaut.
        \item \opt{edgecolor} : est une chaîne qui définit la couleur des arêtes. C'est la couleur courante des lignes par défaut.
        \item \opt{hiddenstyle} : est une chaîne qui définit le style de ligne des arêtes cachées. Par défaut c'est la valeur contenue dans la variable globale \emph{Hiddenlinestyle} (qui vaut elle-même "dotted" par défaut).
        \item \opt{hiddencolor} : est une chaîne qui définit la couleur des arêtes cachées. C'est la couleur courante des lignes par défaut.
        \item \opt{edgewidth} : épaisseur de trait des arêtes en dixième de point. C'est l'épaisseur courante par défaut.
        \item \opt{opacity} : nombre entre 0 et 1 qui permet de mettre une transparence ou non sur les facettes. La valeur par défaut est 1, ce qui signifie pas de transparence.
        \item \opt{backcull} : booléen qui vaut false par défaut. Lorsqu'il a la valeur true, les facettes considérées comme non visibles (vecteur normal non dirigé vers l'observateur) ne sont pas affichées. Cette option est intéressante pour les polyèdres convexes car elle permet de diminuer le nombre de facettes à dessiner.
        \item \opt{clip} : booléen qui vaut false par défaut. Lorsqu'il a la valeur true, les facettes sont clippées par la fenêtre 3d.
        \item \opt{twoside} : booléen qui vaut true par défaut, ce qui signifie qu'on distingue les deux côtés des facettes (intérieur et extérieur), les deux côtés n'auront pas exactement la même couleur.
        \item \opt{color} : chaîne définissant la couleur de remplissage des facettes, c'est "white" par défaut.
        \item \opt{usepalette} (\emph{nil} par défaut), cette option permet de préciser une palette de couleurs pour peindre les facettes ainsi qu'un mode de calcul, la syntaxe est : \emph{usepalette = \{palette,mode\}}, où \emph{palette} désigne une table de couleurs qui sont elles-mêmes des tables de la forme \emph{\{r,g,b\}} où r, g et b sont des nombres entre $0$ et $1$, et \emph{mode} qui est une chaîne qui peut être soit \emph{"x"}, soit \emph{"y"}, ou soit \emph{"z"}. Dans le premier cas par exemple, les facettes au centre de gravité d'abscisse minimale ont la première couleur de la palette, les facettes au centre de gravité d'abscisse maximale ont la dernière couleur de la palette, pour les autres, la couleur est calculée en fonction de l'abscisse du centre de gravité par interpolation linéaire.
        \end{itemize}
    \item Pour plusieurs listes de facettes dans un même dessin, la méthode est :
        \par\hfil\textbf{g:Dmixfacet(S1,options1, S2,options2, ...)}\hfil\par
    où $S1$, $S2$, ... sont des listes de facettes, et \emph{options1}, \emph{options2}, ... sont les options correspondantes. Les options d'une liste de facettes s'appliquent aussi aux suivantes si elles ne sont pas changées. Ces options sont identiques à la méthode précédente.
    
    Cette méthode est utile pour dessiner plusieurs solides ensemble, à condition qu'il n'y ait pas d'intersections entre les objets, car celles-ci ne sont pas gérées ici.
\end{enumerate}

\begin{demo}[courbeniv]{Exemple de courbes de niveaux sur une surface}
\begin{luadraw}{name=courbes_niv}
local cos, sin = math.cos, math.sin, math.pi
local g = graph3d:new{window3d={0,5,0,10,0,11}, adjust2d=true, size={10,10}, viewdir={220,60}}
g:Labelsize("footnotesize")
local S = cartesian3d(function(u,v) return (u+v)/(2+cos(u)*sin(v)) end,0,5,0,10,{30,30})
local n = 10 -- nombre de niveaux
local Colors = getpalette(palGasFlame,n,true) -- liste de 10 couleurs au format table
local niv, S1 = {}
for k = 1, n do
    S1, S = cutfacet(S,{M(0,0,k),-vecK}) -- section de S avec le plan z=k
    insert(niv,{S1, {color=Colors[k],mode=mShaded,edgewidth=0.5}}) -- S1 est la partie sous le plan et S au dessus
end
insert(niv,{S, {color=Colors[n+1]}}) -- insertion du dernier niveau
-- niv est une liste du type {facettes1, options1, facettes2, options2, ...}
g:Dboxaxes3d({grid=true, gridcolor="gray",fillcolor="lightgray"})
g:Dmixfacet(table.unpack(niv))
for k = 1, n do
    g:Dballdots3d( M(5,0,k), rgb(Colors[k]))
end
g:Dlabel("$z=\\frac{x+y}{2+\\cos(x)\\sin(y)}$", Z((g:Xinf()+g:Xsup())/2, g:Yinf()), {pos="N"})
g:Show()
\end{luadraw}
\end{demo}

\subsection{Fonctions de construction de listes de facettes}

Les fonctions suivantes renvoient un solide sous forme d'une liste de facettes (avec points 3d).
\subsubsection{surface()}
    
La fonction \textbf{surface(f,u1,u2,v1,v2,grid)} renvoie la surface paramétrée par la fonction $f\colon(u,v) \mapsto f(u,v)\in \mathbf R^3$. L'intervalle pour le paramètre $u$ est donné par \emph{u1} et \emph{u2}. L'intervalle pour le paramètre $v$ est donné par \emph{v1} et \emph{v2}. Le paramètre facultatif \emph{grid} vaut $\{25,25\}$ par défaut, il définit le nombre de points à calculer pour le paramètre $u$ suivi du nombre de points à calculer pour le paramètre $v$. 
    
Il y a deux variantes pour les surfaces :

\subsubsection{cartesian3d()}

La fonction \textbf{cartesian3d(f,x1,x2,y1,y2,grid,addWall)} renvoie la surface cartésienne d'équation $z=f(x,y)$ où $f\colon(x,y)\mapsto f(x,y)\in\mathbb R$. L'intervalle pour $x$ est donné par \emph{x1} et \emph{x2}. L'intervalle pour $y$ est donné par \emph{y1} et \emph{y2}. Le paramètre facultatif \emph{grid} vaut $\{25,25\}$ par défaut, il définit le nombre de points à calculer pour $x$ suivi du nombre de points à calculer pour $y$. Le paramètre \emph{addWall} vaut 0 ou "x", ou "y", ou "xy" (0 par défaut). Lorsque cette option vaut "x" (ou "xy"), la fonction renvoie, après la liste des facettes composant la surface, une liste de facettes séparatrices (murs ou cloisons) entre chaque "couche" de facettes, une couche correspond à deux valeurs consécutives du paramètre $x$\footnote{Ces cloisons sont en fait des plans d'équation $x=$constante}. Avec la valeur "y" (ou "xy") c'est une liste de facettes séparatrices (murs) entre chaque "couche" correspond à deux valeurs consécutives du paramètre "y"\footnote{Ces cloisons sont en fait des plans d'équation $y=$constante}. Cette option peut être utile avec la méthode \textbf{g:Dscene3d} (uniquement), car les cloisons séparatrices forment une partition de l'espace isolant les facettes de la surface, ce qui permet d'éviter des calculs d'intersection inutiles entre elles. C'est notamment le cas avec des surfaces non convexes.

Par exemple, voici le code de la figure \ref{pointcol}:
\begin{Luacode}
\begin{luadraw}{name=point_col}
local g = graph3d:new{window3d={-2,2,-2,2,-4,4}, window={-3.5,3,-5,5}, size={8,9,0}, viewdir={120,60}}
local S = cartesian3d(function(u,v) return u^2-v^2 end, -2,2,-2,2,{20,20}) -- surface d'équation z=x^2-y^2
local P = facet2poly(S) -- conversion en polyèdre
local Tx = g:Intersection3d(P, {Origin,vecI}) --intersection de P avec le plan yOz
local Ty = g:Intersection3d(P, {Origin,vecJ}) --intersection de P avec le plan xOz
g:Dboxaxes3d({grid=true,gridcolor="gray",fillcolor="LightGray",drawbox=true})
g:Dfacet(S,{mode=mShadedOnly,color="ForestGreen"}) -- dessin de la surface
g:Dedges(Tx, {color="Crimson", hidden=true, width=8}) -- intersection avec yOz
g:Dedges(Ty, {color="Navy",hidden=true, width=8}) -- intersection avec xOz
g:Dpolyline3d( {M(2,0,4),M(-2,0,4),M(-2,0,-4)}, "Navy,line width=.8pt")
g:Dpolyline3d( {M(0,-2,4),M(0,2,4),M(0,2,-4)}, "Crimson,line width=.8pt")
g:Show()
\end{luadraw}
\end{Luacode}

\subsubsection{cylindrical\_surface()}

La fonction \textbf{cylindrical\_surface(r,z,u1,u2,theta1,theta2,grid,addWall)} renvoie la surface paramétrée en cylindrique par \emph{r(u,theta), theta, z(u,theta)}. Les arguments $r$ et $z$ sont donc deux fonctions de $u$ et $\theta$, à valeurs réelles. L'intervalle pour $u$ est donné par \emph{u1} et \emph{u2}. L'intervalle pour $\theta$ est donné par \emph{theta1} et \emph{theta2} (en radians). Le paramètre facultatif \emph{grid} vaut $\{25,25\}$ par défaut, il définit le nombre de points à calculer pour $u$ suivi du nombre de points à calculer pour $v$. Le paramètre \emph{addWall} vaut 0 ou "v"  ou "z" ou "vz" (0 par défaut). Lorsque cette option vaut "v" ou "vz", la fonction renvoie, après la liste des facettes composant la surface, une liste de facettes séparatrices (murs ou cloisons) entre chaque "couche" de facettes, une couche correspond à deux valeurs consécutives de l'angle $\theta$\footnote{Ces cloisons sont en fait des plans d'équation $\theta=$constante}. Lorsque cette option vaut "z" ou "vz", la fonction renvoie, après la liste des facettes composant la surface, une liste de facettes séparatrices (murs ou cloisons) entre chaque "couche" de facettes, une couche correspond à deux valeurs consécutives de la cote $z$\footnote{Ces cloisons sont en fait des plans d'équation $z=$constante}, les valeurs de $z$ sont calculées à  partir des valeurs du paramètres $u$ et avec la valeur \emph{theta1}, ceci est utile lorsque $z$ ne dépend que $u$ (et donc pas de \emph{theta}). Cette option peut être utile avec la méthode \textbf{g:Dscene3d} (uniquement), car les cloisons séparatrices forment une partition de l'espace isolant les facettes de la surface, ce qui permet d'éviter des calculs d'intersection inutiles entre elles. C'est notamment le cas avec des surfaces non convexes.

\begin{demo}{Surfaces utilisant l'option \emph{addWall}}
\begin{luadraw}{name=surface_with_addWall}
local pi, ch, sh = math.pi, math.cosh, math.sinh
local g = graph3d:new{window3d={-4,4,-4,4,-5,5}, window={-10,10,-4,4}, size={10,10}, viewdir={60,60}}
g:Labelsize("footnotesize")
local S,wall = cartesian3d(function(x,y) return x^2-y^2 end,-2,2,-2,2,nil,"xy")
g:Saveattr(); g:Viewport(-10,0,-4,4); g:Coordsystem(-4.5,4.5,-4.5,4.75)
g:Dscene3d( 
    g:addWall(wall), -- 2 facet cutouts with this instruction, and 529 facet cutouts without it
    g:addFacet(S,{color="SteelBlue"}),
    g:addAxes(Origin,{arrows=1}) )
g:Restoreattr() 
g:Saveattr(); g:Viewport(0,10,-4,4); g:Coordsystem(-5,5,-5,5)
local r = function(u,v) return ch(u) end
local z = function(u,v) return sh(u) end
S,wall = cylindrical_surface(r,z,2,-2,-pi,pi,{25,51},"zv")
g:Dscene3d( 
    g:addWall(wall), -- 13 facet cutouts with this instruction, and more than 17000 facet cutouts without it ...
    g:addFacet(S,{color="Crimson"}),
    g:addAxes(Origin,{arrows=1})  )
g:Restoreattr()     
g:Show()
\end{luadraw}
\end{demo}

\subsubsection{curve2cone()}
La fonction \textbf{curve2cone(f,t1,t2,S,args)} construit un cône de sommet $S$ (point 3d) et de base la courbe paramétrée par $f\colon t\mapsto f(t)\in\mathbf R^3$ sur l'intervalle défini par \emph{t1} et \emph{t2}. L'argument \emph{args} est une table facultative pour définir les options, qui sont :
    \begin{itemize}
        \item \opt{nbdots} qui représente le nombre minimal de points de la courbe à calculer (15 par défaut).
        \item \opt{ratio} qui est un nombre représentant le rapport d'homothétie (de centre le sommet $S$) pour construire l'autre partie du cône. Par défaut \emph{ratio} vaut 0 (pas de deuxième partie).
        \item \opt{nbdiv} qui est un entier positif indiquant le nombre de fois que l'intervalle entre deux valeurs consécutives du paramètre $t$ peut être coupé en deux (dichotomie) lorsque les points correspondants sont trop éloignés. Par défaut \emph{nbdiv} vaut 0.
    \end{itemize}
 Cette fonction renvoie une liste de facettes, suivie d'une ligne polygonale 3d qui représente les bords du cône.
 
\begin{demo}{Exemple de cône elliptique}
\begin{luadraw}{name=curve2cone}
local cos, sin, pi = math.cos, math.sin, math.pi
local g = graph3d:new{ window3d={-2,2,-4,4,-3,3},window={-5.5,5,-5,5},size={10,10}}
local f = function(t) return M(2*cos(t),4*sin(t),-3) end -- ellipse dans le plan z=-3
local C, bord = curve2cone(f,-pi,pi,Origin,{nbdiv=2, ratio=-1})
g:Dboxaxes3d({grid=true,gridcolor="gray",fillcolor="LightGray"})
g:Dpolyline3d(bord[1],"red,line width=2.4pt") -- bord inférieur
g:Dfacet(C, {mode=mShadedOnly,color="LightBlue"})  -- cône
g:Dpolyline3d(bord[2],"red,line width=0.8pt") -- bord supérieur
g:Show()
\end{luadraw}
\end{demo}

\subsubsection{curve2cylinder()}
La fonction \textbf{curve2cylinder(f,t1,t2,V,args)} construit un cylindre d'axe dirigé par le vecteur $V$ (point 3d) et de base la courbe paramétrée par $f\colon t\mapsto f(t)\in\mathbf R^3$ sur l'intervalle défini par \emph{t1} et \emph{t2}. La seconde base est la translatée de la première avec le vecteur $V$. L'argument \emph{args} est une table facultative pour définir les options, qui sont :
    \begin{itemize}
        \item \opt{nbdots} qui représente le nombre minimal de points de la courbe à calculer (15 par défaut).
        \item \opt{nbdiv} qui est un entier positif indiquant le nombre de fois que l'intervalle entre deux valeurs consécutives du paramètre $t$ peut être coupé en deux (dichotomie) lorsque les points correspondants sont trop éloignés. Par défaut \emph{nbdiv} vaut 0.
    \end{itemize}
 Cette fonction renvoie une liste de facettes, suivie d'une ligne polygonale 3d qui représente les bords du cylindre.
 
\begin{demo}{Section d'un cylindre non circulaire}
\begin{luadraw}{name=curve2cylinder}
local cos, sin, pi = math.cos, math.sin, math.pi
local g = graph3d:new{ window3d={-5,5,-5,5,-4,4},window={-9,8,-7,7},viewdir={39,70},size={10,10}}
local f = function(t) return M(4*cos(t)-cos(4*t),4*sin(t)-sin(4*t),-4) end -- courbe dans le plan z=-3
local V = 8*vecK
local C = curve2cylinder(f,-pi,pi,V,{nbdots=25,nbdiv=2})
local plan = {M(0,0,2), -vecK} -- plan de coupe z=2
local C1, C2, section = cutfacet(C,plan)
g:Dboxaxes3d({grid=true,gridcolor="gray",fillcolor="LightGray"})
g:Dfacet(C1, {mode=mShaded,color="LightBlue"})  -- partie sous le plan
g:Dfacet(g:Plane2facet(plan), {opacity=0.3,color="Chocolate"}) -- dessin du plan sous forme d'une facette
g:Filloptions("fdiag","red"); g:Dpolyline3d(section) -- dessin de la section
g:Dfacet(C2, {mode=3,color="LightBlue"})  -- partie du cylindre au dessus du plan
g:Show()
\end{luadraw}
\end{demo}

\subsubsection{line2tube()}
La fonction \textbf{line2tube(L,r,args)} construit (sous forme d'une liste de facettes) un tube centré sur $L$ qui doit être une ligne polygonale 3d, l'argument $r$ représente le rayon de ce tube. L'argument \emph{args} est une table pour définir les options, qui sont :
    \begin{itemize}
        \item \opt{nbfacet} : nombre indiquant le nombre de facettes latérales du tube (3 par défaut).
        \item \opt{close} : booléen indiquant si la ligne polygonale $L$ doit être refermée (false par défaut).
        \item \opt{hollow} : booléen indiquant si les deux extrémités du tube doivent être ouvertes ou non (false par défaut). Lorsque l'option \opt{close} vaut true, l'option \opt{hollow} prend automatiquement la valeur true.
        \item \opt{addwall} : nombre qui vaut 0 ou 1 (0 par défaut). Lorsque cette option vaut 1, la fonction renvoie, après la liste des facettes composant le tube, une liste de facettes séparatrices (murs) entre chaque "tronçon" du tube, ce qui peut être utile avec la méthode \textbf{g:Dscene3d} (uniquement).
    \end{itemize}
    
\begin{demo}{Exemple avec line2tube}
\begin{luadraw}{name=line2tube}
local cos, sin, pi, i = math.cos, math.sin, math.pi, cpx.I
local g = graph3d:new{window={-5,8,-4.5,4.5}, viewdir={45,60}, margin={0,0,0,0}, size={10,10}}
local L1 = map(toPoint3d,polyreg(0,3,6)) -- hexagone régulier dans le plan xOy, centre O de sommet M(3,0,0)
local L2 = shift3d(rotate3d(L1,90,{Origin,vecJ}),3*vecJ)
local T1 = line2tube(L1,1,{nbfacet=8,close=true}) -- tube 1 refermé
local T2 = line2tube(L2,1,{nbfacet=8})  -- tube 2 non refermé
g:Dmixfacet( T1, {color="Crimson",opacity=0.8}, T2, {color="SteelBlue"} )
g:Show()
\end{luadraw}
\end{demo}

\subsubsection{rotcurve()}
La fonction \textbf{rotcurve(p,t1,t2,axe,angle1,angle2,args)} construit sous forme d'une liste de facettes, la surface balayée par la courbe paramétrée par $p\colon t\mapsto p(t)\in \mathbf R^3$ sur l'intervalle défini par \emph{t1} et \emph{t2}, en la faisant tourner autour de \emph{axe} (qui est une table de la forme \{point3d, vecteur 3d\} représentant une droite orientée de l'espace), d'un angle allant de \emph{angle1} (en degrés) à \emph{angle2}. L'argument \emph{args} est une table pour définir les options, qui sont :
    \begin{itemize}
        \item \opt{grid} : table constituée de deux nombres, le premier est le nombre de points calculés pour le paramètre $t$, et le second le nombre de points calculés pour le paramètre angulaire. Par défaut la valeur de \opt{grid} est \{25,25\}.

        \item \opt{addwall} : nombre qui vaut 0 ou 1 ou 2 (0 par défaut). Lorsque cette option vaut 1, la fonction renvoie, après la liste des facettes composant la surface, une liste de facettes séparatrices (murs) entre chaque "couche" de facettes (une couche correspond à deux valeurs consécutives du paramètre $t$), et avec la valeur 2 c'est une liste de facettes séparatrices (murs) entre chaque "tranche" de rotation (une couche correspond à deux valeurs consécutives du paramètre angulaire, ceci est intéressant lorsque la courbe est dans un même plan que l'axe de rotation). Cette option peut être utile avec la méthode \textbf{g:Dscene3d} (uniquement).
        \end{itemize} 
        
\begin{demo}{Exemple avec rotcurve}
\begin{luadraw}{name=rotcurve}
local cos, sin, pi, i = math.cos, math.sin, math.pi, cpx.I
local g = graph3d:new{viewdir={30,60},size={10,10}}
local p = function(t) return M(0,sin(t)+2,t) end -- courbe dans le plan yOz
local axe = {Origin,vecK}
local S = rotcurve(p,pi,-pi,axe,0,360,{grid={25,35}})
local  visible, hidden = g:Classifyfacet(S)
g:Dfacet(hidden, {mode=mShadedOnly,color="cyan"})
g:Dline3d(axe,"red,line width=1.2pt")
g:Dfacet(visible, {mode=5,color="cyan"})
g:Dline3d(axe,"red,line width=1.2pt,dashed")
g:Dparametric3d(p,{t={-pi,pi},draw_options="red,line width=1.2pt"})
g:Show()
\end{luadraw}
\end{demo}

\paragraph{Remarque} : si l'orientation de la surface ne semble pas bonne, il suffit d'échanger les paramètres \emph{t1} et \emph{t2}, ou bien \emph{angle1} et \emph{angle2}.

\subsubsection{rotline()}

La fonction \textbf{rotline(L,axe,angle1,angle2,args)} construit sous forme d'une liste de facettes, la surface balayée par la liste de points 3d $L$ en la faisant tourner autour de \emph{axe} (qui est une table de la forme \{point3d, vecteur 3d\} représentant une droite orientée de l'espace), d'un angle allant de \emph{angle1} (en degrés) à \emph{angle2}. L'argument \emph{args} est une table pour définir les options, qui sont :
    \begin{itemize}
        \item \opt{nbdots} : qui est le nombre de points calculés pour le paramètre angulaire. Par défaut la valeur de \opt{nbdots} est 25.
        
        \item \opt{close} : booléen qui indique si $L$ doit être refermée (false par défaut).

        \item \opt{addwall} : nombre qui vaut 0 ou 1 ou 2 (0 par défaut). Lorsque cette option vaut 1, la fonction renvoie, après la liste des facettes composant la surface, une liste de facettes séparatrices (murs) entre chaque "couche" de facettes (une couche correspond à deux points consécutifs dans la liste $L$), et avec la valeur 2 c'est une liste de facettes séparatrices (murs) entre chaque "tranche" de rotation (une couche correspond à deux valeurs consécutives du paramètre angulaire, ceci est intéressant lorsque la courbe est dans un même plan que l'axe de rotation). Cette option peut être utile avec la méthode \textbf{g:Dscene3d} (uniquement).
        \end{itemize} 
\begin{demo}{Exemple avec rotline}
\begin{luadraw}{name=rotline}
local g = graph3d:new{window={-4,4,-4,4},size={10,10}}
local L = {M(0,0,4),M(0,4,0),M(0,0,-4)} -- liste de points dans le plan yOz
local axe = {Origin,vecK}
local S = rotline(L,axe,0,360,{nbdots=5}) -- le point 1 et le point 5 sont confondus
g:Dfacet(S,{color="Crimson",edgecolor="Gold",opacity=0.8})
g:Show()
\end{luadraw}
\end{demo}      


\subsection{Arêtes d'un solide}

Un objet de type "edge" est une table à deux champs, un champ nommé \emph{visible} qui contient une ligne polygonale 3d correspondant aux arêtes visibles, et un autre champ nommé \emph{hidden} qui contient une ligne polygonale 3d correspondant aux arêtes cachées.

\begin{itemize}
    \item La méthode \textbf{g:Edges(P)} où $P$ est un polyèdre, renvoie les arêtes de $P$ sous forme d'un objet de type "edge". Une arête de $P$ est visible lorsqu'elle appartient à au moins une face visible.
    \item La méthode \textbf{g:Intersection3d(P,plane)} où $P$ est un polyèdre ou bien une liste de facettes, renvoie sous forme d'objet de type "edge" l'intersection entre $P$ et le plan représenté par \emph{plane} (c'est une table de la forme \{A,u\} où $A$ est un point du plan et $u$ un vecteur normal, ce sont donc deux points 3d).
    \item La méthode \textbf{g:Dedges(edges,options)} permet de dessiner \emph{edges} qui doit être un objet de type "edge". L'argument \emph{options} est une table définissant les options, celles-ci sont :
    \begin{itemize}
        \item \opt{hidden} : booléen qui indique si les arêtes cachées doivent être dessinées (false par défaut).
        \item \opt{visible} : booléen qui indique si les arêtes visibles doivent être dessinées (true par défaut).
        \item \opt{clip} : booléen qui indique si les arêtes doivent être clippées par la fenêtre 3d (false par défaut).
        \item \opt{hiddenstyle} : chaîne de caractères définissant le style de ligne des arêtes cachées, par défaut cette option contient la valeur de la variable globale \emph{Hiddenlinestyle} (qui vaut "dotted" par défaut).
        \item \opt{hiddencolor} : chaîne de caractères définissant la couleur des arêtes cachées, par défaut cette option contient la même couleur que l'option \opt{color}.
        \item \opt{style} : chaîne de caractères définissant le style de ligne des arêtes visibles, par défaut cette option contient le style courant du dessin de lignes.
        \item \opt{color} : chaîne de caractères définissant la couleur des arêtes visibles, par défaut cette option contient la couleur courante de dessin de lignes.
        \item \opt{width} : nombre représentant l'épaisseur de trait des arêtes (en dixième de point), par défaut cette variable contient l'épaisseur courante du dessin de lignes.
    \end{itemize}

    \item \textbf{Complément} : 
        \begin{itemize}
            \item La fonction \textbf{facetedges(F)} où $F$ est une liste de facettes ou bien un polyèdre, renvoie une liste de segments 3d représentant toutes les arêtes de $F$. Le résultat n'est pas un objet de type "edge", et il se dessine avec la méthode \textbf{g:Dpolyline3d}.
            \item La fonction \textbf{facetvertices(F)} où $F$ est une liste de facettes ou bien un polyèdre, renvoie la liste de tous les sommets de $F$ (points 3d).
        \end{itemize}
\end{itemize}

\subsection{Méthodes et fonctions s'appliquant à des facettes ou polyèdres}

\begin{itemize}
    \item La méthode \textbf{g:Isvisible(F)} où $F$ désigne \textbf{une} facette (liste d'au moins 3 points 3d coplanaires et non alignés), renvoie true si la facette $F$ est visible (vecteur normal dirigé vers l'observateur). Si $A$, $B$ et $C$ sont les trois premiers points de $F$, le vecteur normal est calculé en faisant le produit vectoriel $\vec{AB}\wedge\vec{AC}$.
    
    \item La méthode \textbf{g:Classifyfacet(F)} où $F$ est une liste de facettes ou bien un polyèdre, renvoie \textbf{deux} listes de facettes, la première est la liste des facettes visibles, et la suivante, la liste des facettes non visibles.
    
    \item La méthode \textbf{g:Sortfacet(F,backcull)} où $F$ est une liste de facettes, renvoie cette liste de facettes triées de la plus éloignée à la plus proche de l'observateur. L'argument facultatif \emph{backcull} est un booléen qui vaut false par défaut, lorsqu'il a la valeur true, les facettes non visibles sont exclues du résultat (seules les facettes visibles sont alors renvoyées après avoir été triées). Le calcul de l'éloignement d'un facette se fait sur son centre de gravité. La technique dite du "peintre" consiste à afficher les facettes de la plus éloignée à la plus proche, donc dans l'ordre de la liste renvoyée par cette fonction (le résultat affiché n'est cependant pas toujours correct en fonction de la taille et de la forme des facettes).
    
    \item La méthode \textbf{g:Sortpolyfacet(P,backcull)} où $P$ est un polyèdre, renvoie la liste des facettes de $P$ (facettes avec points 3d) triées de la plus éloignée à la plus proche de l'observateur. L'argument facultatif \emph{backcull} est un booléen qui vaut false par défaut, lorsqu'il a la valeur true, les facettes non visibles sont exclues du résultat comme pour la méthode précédente. Ces deux méthodes de tris sont utilisées par les méthodes de dessin de polyèdres ou facettes (\emph{Dpoly}, \emph{Dfacet} et \emph{Dmixfacet}).
    
    \item La méthode \textbf{g:Outline(P)} où $P$ est un polyèdre, renvoie le "contour" de $P$ sous la forme d'une table à deux champs, un champ nommé \emph{visible} qui contient une ligne polygonale 3d représentant les "arêtes" (segments) appartenant à une seule facette, celle-ci étant visible, ou bien à deux facettes, une visible et une cachée; l'autre champ est nommé \emph{hidden} et contient une ligne polygonale 3d représentant les "arêtes" appartenant à une seule facette, celle-ci étant cachée.
    
    \item La fonction \textbf{border(P)} où $P$ est un polyèdre ou une liste de facette, renvoie une ligne polygonale 3d qui correspond aux arêtes appartenant à une seule facette de $P$ (ces arêtes sont mises "bout à bout" pour former une ligne polygonale).
    
    \item La fonction \textbf{getfacet(P,list)} où $P$ est un polyèdre, renvoie la liste des facettes de $P$ (avec points 3d) dont le numéro figure dans la table \emph{list}. Si l'argument \emph{list} n'est pas précisé, c'est la liste de toutes les facettes de $P$ qui est renvoyée (dans ce cas c'est la même chose que \textbf{poly2facet(P)}).
    
    \item La fonction \textbf{facet2plane(L)} où $L$ est soit une facette, soit une liste de facettes, renvoie soit le plan contenant la facette, soit la liste des plans contenant chacune des facettes de $L$. Un plan est une table du type \{A,u\} où $A$ est un point du plan et $u$ un vecteur normal au plan (donc deux points 3d).
    
    \item La fonction \textbf{reverse\_face\_orientation(F)} où $F$ et soit une facette, soit une liste de facette, soit un polyèdre, renvoie un résultat de même nature que $F$ mais dans lequel l'ordre sur les sommets de chaque facette a été inverser. Cela peut être utile lorsque l'orientation de l'espace à été modifiée.
    
\begin{demo}{Sphère inscrite dans un octaèdre avec projection du centre sur les faces}
\begin{luadraw}{name=sphere_octaedre}
require "luadraw_polyhedrons"
local g = graph3d:new{ window3d={-3,3,-3,3,-3,3}, size={10,10}}
local P = octahedron(Origin,M(0,0,3)) -- polyèdre défini dans le module luadraw_polyhedrons
P = rotate3d(P,-10,{Origin,vecK}) -- rotate3d sur un polyèdre renvoie un polyèdre
local V, H = g:Classifyfacet(P) -- V pour facettes visibles, H pour hidden
local S = map(function(p) return {proj3d(Origin,p),p[2]} end, facet2plane(V) )
-- S contient la liste de : {projeté, vecteur normal} (projetés de Origin sur les faces visibles)
local R = pt3d.abs(S[1][1]) -- rayon de la sphère
g:Dboxaxes3d({grid=true, gridcolor="gray", fillcolor="LightGray"})
g:Dfacet(H, {color="blue",opacity=0.9}) -- dessin des facettes non visibles
g:Dsphere(Origin,R,{mode=mBorder,color="orange"}) -- dessin de la sphère
g:Dballdots3d(Origin,"gray",0.75) -- centre de la sphère
for _,D in ipairs(S) do -- segments reliant l'origine aux projetés
    g:Dpolyline3d( {Origin,D[1]},"dashed,gray")
end
g:Dfacet(V,{opacity=0.4, color="LightBlue"}) -- facettes visibles de l'octaèdre
g:Dcrossdots3d(S,nil,0.75) -- dessin des projetés sur les faces
g:Dpolyline3d( {M(0,-3,3), M(0,0,3), M(-3,0,3)},"gray")
g:Show()            
\end{luadraw}
\end{demo}
\end{itemize}

\subsection{Découper un solide : cutpoly et cutfacet}

\begin{itemize}
    \item La fonction \textbf{cutpoly(P,plane,close)} permet de découper le polyèdre $P$ avec le plan \emph{plane} (table du type \{A,n\} où $A$ est un point du plan et $n$ un vecteur normal au plan). La fonction renvoie 3 choses : la partie située dans le demi-espace contenant le vecteur $n$ (sous forme d'un polyèdre), suivie de la partie située dans l'autre demi-espace (toujours sous forme d'un polyèdre), suivie de la section sous forme d'une facette orientée par $-n$. Lorsque l'argument facultatif \emph{close} vaut true, la section est ajoutée aux deux polyèdres résultants, ce qui a pour effet de les refermer (false par défaut).\par
    \textbf{Remarque} : lorsque le polyèdre $P$ n'est pas convexe, le résultat de la section n'est pas toujours correct.

\begin{demo}{Cube coupé par un plan (cutpoly), avec \emph{close}=false et avec \emph{close}=true}
\begin{luadraw}{name=cutpoly}
local g = graph3d:new{window3d={-3,3,-3,3,-3,3}, window={-4,4,-3,3},size={10,10}}
local P = parallelep(M(-1,-1,-1),2*vecI,2*vecJ,2*vecK)
local A, B, C = M(0,-1,1), M(0,1,1), M(1,-1,0)
local plane = {A, pt3d.prod(B-A,C-A)}
local P1 = cutpoly(P,plane)
local P2 = cutpoly(P,plane,true)
g:Lineoptions(nil,"Gold",8)
g:Dpoly( shift3d(P1,-2*vecJ), {color="Crimson",mode=mShadedHidden} )
g:Dpoly( shift3d(P2,2*vecJ), {color="Crimson",mode=mShadedHidden} )
g:Dlabel3d(
    "close=false", M(2,-2,-1), {dir={vecJ,vecK}},
    "close=true", M(2,2,-1), {}
    )
g:Show()            
\end{luadraw}
\end{demo}

     \item La fonction \textbf{cutfacet(F,plane,close)}, où $F$ est une facette, une liste de facettes, ou un polyèdre, fait la même chose que la fonction précédente sauf que cette fonction renvoie des listes de facettes et non pas des polyèdres. Cette fonction a été utilisée dans l'exemple des courbes de niveau à la figure \ref{courbeniv}.
\end{itemize}

\subsection{Clipper des facettes avec un polyèdre convexe : clip3d}

La fonction \textbf{clip3d(S,P,exterior)} clippe le solide $S$ (liste de facettes ou bien polyèdre) avec le solide convexe $P$ (liste de facettes ou bien polyèdre) et renvoie la liste de facettes qui en résulte. L'argument facultatif \emph{exterior} est un booléen qui vaut false par défaut, dans ce cas c'est la partie de $S$ qui est intérieure à $P$ qui est renvoyée, sinon c'est la partie de $S$ extérieure à $P$ qui est renvoyée.\par
\textbf{Remarque} : le résultat n'est pas toujours satisfaisant pour la partie extérieure.

\paragraph{Cas particulier} : clipper une liste de facettes $S$ (ou bien polyèdre) avec la fenêtre 3d courante peut se faire avec cette fonction de la manière suivante :

\begin{center}
\textbf{S = clip3d(S, g:Box3d())}
\end{center}

En effet, la méthode \textbf{g:Box3d()} renvoie la fenêtre 3d courante sous forme d'un parallélépipède.

\begin{demo}[clip3d]{Exemple avec clip3d : construction d'un dé à partir d'un cube et d'une sphère}
\begin{luadraw}{name=clip3d}
local g = graph3d:new{window={-3,3,-3,3},size={10,10}}
local S = sphere(Origin,3)
local C = parallelep(M(-2,-2,-2),4*vecI,4*vecJ,4*vecK)
local C1 = clip3d(S,C) -- sphère clippée par le cube
local C2 = clip3d(C,S) -- cube clippé par la sphère
local V = g:Classifyfacet(C2) -- facettes visibles de C2
g:Dfacet( concat(C1,C2), {color="Beige",mode=mShadedOnly,backcull=true} ) -- que les faces visibles
g:Dpolyline3d(V,true,"line width=0.8pt") -- contour des faces visibles de C2
local A, B, C, D = M(2,-2,-2), M(2,2,2), M(-2,2,-2), M(0,0,2) -- dessin des points noirs
g:Filloptions("full","black")
g:Dcircle3d( D,0.25,vecK); g:Dcircle3d( (2*A+B)/3,0.25,vecI)
g:Dcircle3d( (A+2*B)/3,0.25,vecI); g:Dcircle3d( (3*B+C)/4,0.25,vecJ)
g:Dcircle3d( (B+C)/2,0.25,vecJ); g:Dcircle3d( (B+3*C)/4,0.25,vecJ)
g:Show()            
\end{luadraw}
\end{demo}

\subsection{Clipper un plan avec un polyèdre convexe : clipplane}

La fonction \textbf{clipplane(plane,P)}, où l'argument \emph{plane} est une table de la forme \emph{\{A,n\}} représentant le plan passant par $A$ (point 3d) et de vecteur normal $n$ (point 3d non nul), et \emph{P} est un polyèdre convexe, renvoie la section du polyèdre par le plan, si elle existe, sous forme d'une facette (liste de points 3d) orientée par $n$.

\section{La méthode Dscene3d}

\subsection{Le principe, les limites}

Le défaut majeur des méthodes \textbf{g:Dpoly}, \textbf{g:Dfacet} et \textbf{g:Dmixfacet} est de ne pas gérer les intersections éventuelles entre facettes de différents solides, sans compter que parfois, même pour un polyèdre convexe simple, l'algorithme du peintre ne donne pas toujours le bon résultat (car le tri de facettes se fait uniquement sur leur centre de gravité). D'autre part, ces méthodes permettent de dessiner uniquement des facettes.

Le principe de la méthode \textbf{g:Dscene3d()} est de classer les objets 3d à dessiner (facettes, lignes polygonales, points, labels,...) dans un arbre (qui représente la scène). À chaque n{\oe}ud de l'arbre il y a un objet 3d, appelons-le $A$, et deux descendants, l'un des descendants va contenir les objets 3d qui sont devant l'objet A (c'est à dire plus près de l'observateur que $A$), et l'autre descendant va contenir les objets 3d qui sont derrière l'objet A (c'est à dire plus loin de l'observateur que $A$).

En particulier, pour classer une facette $B$ par rapport à une facette $A$ qui est déjà dans l'arbre, on procède ainsi : on découpe la facette $B$ avec le plan contenant la facette $A$, ce qui donne en général deux "demi" facettes, une qui sera devant $A$ (celle dans le demi-espace "contenant" l'observateur) , et l'autre qui sera donc derrière $A$.

Cette méthode est efficace mais comporte des limites car elle peut entraîner une explosion du nombre de facettes dans l'arbre augmentant ainsi sa taille de manière exponentielle, ce qui peut rendre rédhibitoire l'utilisation de cette méthode lorsqu'il y a beaucoup de facettes (temps de calcul long\footnote{Lua est un langage interprété donc l'exécution est en général plus longue qu'avec un langage compilé.}, taille trop importante du fichier tkz, temps de dessin par tikz trop long). Par contre, elle est très efficace lorsqu'il y a peu de facettes, et donc peu d'intersections de facettes (objets convexes avec peu de facettes). De plus, il est possible de dessiner sous la scène 3d et au-dessus, c'est à dire avant l'utilisation de la méthode \textbf{g:Dscene3d}, et après son utilisation.

Cette méthode doit donc être réservée à des scènes très simples. Pour des scènes 3d complexes le format vectoriel n'est pas adapté, mieux vaux se tourner alors vers des d'autres outils comme povray ou blender ou webgl ...

\subsection{Construction d'une scène 3d}

La méthode \textbf{g:Dscene3d(...)} permet cette construction. Elle prend en argument les objets 3d qui vont constituer cette scène les uns après les autres. Ces objets 3d sont eux-mêmes fabriqués à partir de méthodes dédiées qui vont être détaillées plus loin. Dans la version actuelle, ces objets 3d peuvent être :
\begin{itemize}
    \item des polyèdres, 
    \item des listes de facettes (avec point3d),
    \item des lignes polygonales 3d,
    \item des points 3d,
    \item des labels,
    \item des axes,
    \item des plans, des droites,
    \item des angles,
    \item des cercles, des arcs de cercle.
\end{itemize}

\begin{demo}[plans]{Premier exemple avec Dscene3d : intersection de deux plans}
\begin{luadraw}{name=intersection_plans}
local g = graph3d:new{viewdir={-10,60}, window={-5,5.5,-5.5,5.5},bg="gray", size={10,10}}
local P1 = planeEq(1,1,1,-2) -- plan d'équation x+y+z-2=0
local P2 = {Origin, vecK-vecJ} -- plan passant par O et normal à (1,1,1)
local D = interPP(P1,P2) -- droite d'intersection entre P1 et P2 (D = {A,u})
local posD = D[1]+1.85*D[2] -- pour placer le label
Hiddenlines = true; Hiddenlinestyle = "dotted" -- affichage des lignes cachées en pointillées
g:Dscene3d(
    g:addPlane(P1, {color="Crimson",edge=true,edgecolor="Pink",edgewidth=8}), -- ajout du plan P1
    g:addPlane(P2, {color="ForestGreen",edge=true,edgecolor="Pink",edgewidth=8}),  -- ajout du plan P2
    g:addLine(D, {color="Navy",edgewidth=12}),  -- ajout de la droite D
    g:addAxes(Origin, {arrows=1, color="Gold",width=8}),  -- ajout des axes fléchés
    g:addLabel( -- ajout de labels, ceux-ci auraient pu être ajoutés par dessus la scène
        "$D=P_1\\cap P_2$",posD,{color="Navy"},
        "$P_2$", M(3,0,0)+3.5*M(0,1,1),{color="white",dir={vecI,vecJ+vecK}},
        "$P_1$",M(2,0,0)+1.8*M(-1,-1,2), {dir={M(-1,1,0),M(-1,-1,2),1.125*M(1,-1,0)}}
        )
    )
g:Show()
\end{luadraw}
\end{demo}

\subsection{Méthodes pour ajouter un objet dans la scène 3d}

Ces méthodes sont à utiliser comme argument de la méthode \textbf{g:Dscene3d(...)} comme dans l'exemple ci-dessus.

\subsubsection{Ajouter des facettes : g:addFacet et g:addPoly}

La méthode \textbf{g:addFacet(list,options)} où \emph{list} est une facette ou bien une liste de facettes (avec points 3d), permet d'ajouter ces facettes à la scène. 

La méthode \textbf{g:addPoly(list,options)} permet d'ajouter le polyèdre $P$ à la scène. 

Dans les deux cas, l'argument facultatif \emph{options} est une table à 12 champs, ces options (avec leur valeur par défaut) sont :

    \begin{itemize}
        \item \opt{color="white"} : définit la couleur de remplissage des facettes, cette couleur sera nuancée en fonction de l'inclinaison de celles-ci. Par défaut, le bord des facettes n'est pas dessiné (seulement le remplissage).
        \item \opt{usepalette} (\emph{nil} par défaut), cette option permet de préciser une palette de couleurs pour peindre les facettes ainsi qu'un mode de calcul, la syntaxe est : \emph{usepalette = \{palette,mode\}}, où \emph{palette} désigne une table de couleurs qui sont elles-mêmes des tables de la forme \emph{\{r,g,b\}} où r, g et b sont des nombres entre $0$ et $1$, et \emph{mode} qui est une chaîne qui peut être soit \emph{"x"}, soit \emph{"y"}, ou soit \emph{"z"}. Dans le premier cas par exemple, les facettes au centre de gravité d'abscisse minimale ont la première couleur de la palette, les facettes au centre de gravité d'abscisse maximale ont la dernière couleur de la palette, pour les autres, la couleur est calculée en fonction de l'abscisse du centre de gravité par interpolation linéaire.        
        \item \opt{opacity=1} : nombre entre 0 et 1 pour définir l'opacité des facettes (1 signifie pas de transparence).
        \item \opt{backcull=false} : booléen qui indique si les facettes non visibles doivent être exclues de la scène. Par défaut elles sont présentes.
        \item \opt{clip=false} : booléen qui indique si les facettes doivent être clippées par la fenêtre 3d.
        \item \opt{contrast=1} : valeur numérique permettant d'accentuer ou diminuer de contraste de couleur entre les facettes. Avec la valeur 0 toutes les facettes ont la même couleur.
        \item \opt{twoside=true} : booléen qui indique si on distingue la face interne de la face externe des facettes. La couleur de la face interne est un peu plus claire que celle de l'externe.
        
        \item \opt{edge=false} : booléen qui indique si les arêtes doivent être ajoutées à la scène.
        \item \opt{edgecolor=} : indique la couleur des arêtes lorsqu'elles sont dessinées, c'est la couleur courante par défaut.
        \item \opt{edgewidth=}: indique l'épaisseur de trait (en dixième de point) des arêtes, c'est l'épaisseur courante par défaut.
        \item \opt{hidden=Hiddenlines} : booléen qui indique si les arêtes cachées doivent être représentées. \emph{Hiddenlines} est une variable globale qui vaut false par défaut.
        \item \opt{hiddenstyle=Hiddenlinestyle} : chaîne définissant le style de ligne des arêtes cachées. \emph{Hiddenlinestyle} est une variable globale qui vaut "dotted" par défaut.
        \item \opt{matrix=ID3d} : matrice 3d de transformation des facettes, par défaut celle-ci est la matrice 3d de l'identité, c'est à dire la table \{M(0,0,0),vecI,vecJ,vecK\}.
    \end{itemize}

\subsubsection{Ajouter un plan : g:addPlane et g:addPlaneEq}

La méthode \textbf{g:addPlane(P,options)} permet d'ajouter le plan $P$ à la scène 3d, ce plan est défini sous la forme d'une table \{A,u\} où $A$ est un point du plan (point 3d) et $u$ un vecteur normal au plan (point 3d non nul). Cette fonction détermine l'intersection entre ce plan et le parallélépipède donné par l'argument \emph{window3d} (lui-même défini à la création du graphe), ce qui donne une facette, c'est celle-ci qui est ajoutée à la scène. Cette méthode utilise \textbf{g:addFacet}.

La méthode \textbf{g:addPlaneEq(coef,options)} où \emph{coef} est une table constituée de quatre réels \{a,b,c,d\}, permet d'ajouter à la scène le plan d'équation $ax+by+cz+d=0$ (cette méthode utilise la précédente).

Dans les deux cas, l'argument facultatif \emph{options} est une table à 12 champs, ces options sont celles de la méthode \textbf{g:addFacet}, plus l'option \opt{scale=1} : ce nombre est un rapport d'homothétie, on applique à la facette l'homothétie de centre le centre de gravité de la facette et de rapport \emph{scale}. Cela permet de jouer sur la taille du plan dans sa représentation.
        

\subsubsection{Ajouter une ligne polygonale : g:addPolyline}

La méthode \textbf{g:addPolyline(L,options)} où $L$ est une liste de points 3d, ou une liste de listes de points 3d, permet d'ajouter $L$ à la scène. L'argument facultatif \emph{options} est une table à 10 champs, ces options (avec leur valeur par défaut) sont :
    \begin{itemize}
        \item \opt{style="solid"} : pour définir le style de ligne, c'est le style courant par défaut.
        \item \opt{color=} : couleur de la ligne, c'est la couleur courante par défaut.
        \item \opt{close=false} : indique si la ligne $L$ (ou chaque composante de $L$) doit être refermée.
        \item \opt{clip=false} : indique si la ligne $L$ (ou chaque composante de $L$) doit être clippée par la fenêtre 3d.
        \item \opt{width=} : épaisseur de la ligne en dixième de point, c'est l'épaisseur courante par défaut.
        \item \opt{opacity=1} : opacité du tracé de ligne (1 signifie pas de transparence).
        \item \opt{hidden=Hiddenlines} : booléen qui indique si les parties cachées de la ligne doivent être représentées. \emph{Hiddenlines} est une variable globale qui vaut false par défaut.
        \item \opt{hiddenstyle=Hiddenlinestyle} : chaîne définissant le style de ligne des parties cachées. \emph{Hiddenlinestyle} est une variable globale qui vaut "dotted" par défaut.
        \item \opt{arrows=0} : cette option peut valoir 0 (aucune flèche ajoutée à la ligne), 1 (une flèche ajoutée en fin de ligne), ou 2 (une flèche en début et en fin de ligne). Les flèches sont des petits cônes.
        \item \opt{arrowscale=1} : permet de réduire ou augmenter la taille des flèches.
        \item \opt{matrix=ID3d} : matrice 3d de transformation (de la ligne), par défaut celle-ci est la matrice 3d de l'identité, c'est à dire la table \{M(0,0,0),vecI,vecJ,vecK\}.
    \end{itemize}
    
\subsubsection{Ajouter des axes : g:addAxes}

La méthode \textbf{g:addAxes(O,options)} permet d'ajouter les axes : ($O$,\emph{vecI}), ($O$,\emph{vecJ}) et ($O$,\emph{vecK}) à la scène 3d, où l'argument $O$ est un point 3d. Les options sont celles de la méthode \textbf{g:addPolyline}, plus l'option \opt{legend=true} qui permet d'ajouter automatiquement le nom de chaque axe ($x$, $y$ et $z$) à l'extrémité. Ces axes ne sont pas gradués.
    
\subsubsection{Ajouter une droite : g:addLine}

La méthode \textbf{g:addLine(d,options)} permet d'ajouter la droite $d$ à la scène, cette droite $d$ est une table de la forme \{A,u\} où $A$ est un point de la droite (point 3d) et $u$ un vecteur directeur (point 3d non nul).  L'argument facultatif \emph{options} est une table à 10 champs, ces options sont celles de la méthode \textbf{g:addPolyline}, plus l'option \opt{scale=1} : ce nombre est un rapport d'homothétie, on applique à la facette l'homothétie de centre le milieu du segment représentant la droite, et de rapport \emph{scale}. Cela permet de jouer sur la taille du segment dans sa représentation, ce segment est la droite clippée par le polyèdre donné par l'argument \emph{window3d} (lui-même défini à la création du graphe), ce qui donne une segment (éventuellement vide).

\subsubsection{Ajouter un angle "droit" : g:addAngle}

La méthode \textbf{g:addAngle(B,A,C,r,options)} permet d'ajouter l'angle $\widehat{BAC}$ sous forme d'un parallélogramme de côté $r$ ($r$ vaut 0.25 par défaut), seuls deux côtés sont représentés. les arguments $B$, $A$ et $C$ sont des points 3d. Les options sont celles de la méthode \textbf{g:addPolyline}.

\subsubsection{Ajouter un arc de cercle : g:addArc}

La méthode \textbf{g:addArc(B,A,C,r,sens,normal,options)} permet d'ajouter l'arc de cercle de centre $A$ (point 3d), de rayon $r$, allant de $B$ vers $C$ (points 3d) dans le sens direct si \emph{sens} vaut 1 (indirect sinon). L'arc est tracé dans le plan passant par $A$ et orthogonal au vecteur\emph{normal} (point 3d non nul), c'est ce même vecteur qui oriente le plan. Les options sont celles de la méthode \textbf{g:addPolyline}.

\subsubsection{Ajouter un cercle : g:addCircle}

La méthode \textbf{g:addCircle(A,r,normal,options)} permet d'ajouter le cercle de centre $A$ (point 3d) et de rayon $r$ dans le plan passant par $A$ et orthogonal au vecteur\emph{normal} (point 3d non nul). Les options sont celles de la méthode \textbf{g:addPolyline}.

\begin{demo}{Cylindre plein plongé dans de l'eau}
\begin{luadraw}{name=cylindres_imbriques}
local g = graph3d:new{window={-5,5,-6,5}, viewdir={30,75},size={10,10},margin={0,0,0,0}}
Hiddenlines = false
local R, r, A, B = 3, 1.5
local C1 = cylinder(M(0,0,-5),5*vecK,R)  -- pour modéliser l'eau
local C2 = cylinder(Origin,2*vecK,R,35,true) -- partie du contenant au dessus de l'eau (cylindre ouvert)
local C3 = cylinder(M(0,0,-3),7*vecK,r) -- petit cylindre plongé dans l'eau
-- sous la scène 3d
g:Lineoptions(nil,"gray",12)
g:Dcylinder(M(0,0,-5),7*vecK,R,{hiddenstyle="noline"}) -- contour du contenant (grand cylindre)
-- scène 3d
g:Dscene3d(
        g:addPoly(C1,{contrast=0.125,color="cyan",opacity=0.5}), -- eau
        g:addPoly(C2,{contrast=0.125,color="WhiteSmoke", opacity=0.5}), -- partie du contenant au-dessus de l'eau
        g:addPoly(C3,{contrast=0.25,color="Salmon",backcull=true}), -- petit cylindre dans l'eau
        g:addCircle(M(0,0,2),R,vecK,{color="gray"}), -- bord supérieur du contenant
        g:addCircle(M(0,0,-5),R,vecK,{color="gray"}), -- bord inférieur du contenant        
        g:addCircle(Origin,R-0.025,vecK, {width=2,color="cyan"}) -- bord supérieur eau
        )
-- par dessus la scène 3d
g:Lineoptions(nil,"black",8); A = 4*vecK; B = A+r*g:ScreenX()
g:Dpolyline3d( {A,B}, "<->"); g:Dlabel3d("$3\\,$cm",(A+B)/2,{pos="N",dist=0.25})
A = Origin+(r+1)*g:ScreenX(); B = A-3*vecK
g:Dpolyline3d( {A,B}, "<->"); g:Dlabel3d("h",(A+B)/2,{pos="E"})
g:Lineoptions("dashed")
g:Dpolyline3d({{A,A-g:ScreenX()},{B,B-g:ScreenX()}})
A = Origin-(R+1)*g:ScreenX(); B = A-vecK
g:Dpolyline3d({{A,A+g:ScreenX()},{B,B+g:ScreenX()}})
g:Linestyle("solid")
g:Dpolyline3d( {A,B}, "<->"); g:Dlabel3d("$2$\\,cm",(A+B)/2,{pos="W"})
g:Show()
\end{luadraw}
\end{demo}

\paragraph{Remarques} : 
\begin{itemize}
    \item La méthode \textbf{g:ScreenX()} renvoie le vecteur de l'espace (point 3d) correspondant au vecteur d'affixe 1 dans le plan de l'écran, et la méthode \textbf{g:ScreenY()} renvoie le vecteur de l'espace (point 3d) correspondant au vecteur d'affixe i dans le plan de l'écran.
    \item Pour le petit cylindre (C3) on utilise l'option \opt{backcull=true} pour diminuer le nombre de facettes, par contre, on ne le fait pas pour les deux autres cylindres (C1 et C2) car ils sont transparents.
\end{itemize}

\subsubsection{Ajouter des points : g:addDots}

La méthode \textbf{g:addDots(dots,options)} permet d'ajouter des points 3d à la scène. L'argument \emph{dots} est soit un point 3d, soit une liste de points 3d. L'argument facultatif \opt{options} est une table à quatre champs, ces options sont :
\begin{itemize}
        \item \opt{style="ball"} : chaîne définissant le style de points, ce sont tous les styles de points 2d, plus le style "ball" (sphère) qui est le style par défaut. 
    \item \opt{color="black"} : chaîne définissant la couleur des points.
    \item \opt{scale=1} : nombre permettant de jouer sur la taille des points.
    \item \opt{matrix=ID3d} : matrice 3d de transformation, par défaut celle-ci est la matrice 3d de l'identité, c'est à dire la table \{M(0,0,0),vecI,vecJ,vecK\}.
\end{itemize}

\subsubsection{Ajouter des labels : g:addLabels}

La méthode \textbf{g:addLabel(text1,anchor1,options1, text2,anchor2,options2, ...)} permet d'ajouter les labels \emph{text1}, \emph{text2}, etc. Les arguments (obligatoires) \emph{anchor1}, \emph{anchor2}, etc, sont des points 3d représentant les points d'ancrage des labels. Les arguments (obligatoires) \emph{options1}, \emph{options2}, etc, sont des tables à 7 champs. Ces options sont :
\begin{itemize}
    \item \opt{color} : chaîne définissant la couleur du label, initialisée à la couleur en cours des labels.
    \item \opt{style} : chaîne définissant le style de label (comme en 2d : "N", "NW", "W", ...),  initialisée au style en cours des labels.
    \item \opt{dist=0} : exprime la distance entre le label et son point d'ancrage (dans le plan de l'écran).
    \item \opt{size} : chaîne définissant la taille du label,  initialisée à la taille en cours des labels.
    \item \opt{dir=\{\}} : table définissant le sens de l'écriture dans l'espace (sens usuel par défaut).
    En général, \emph{dir=\{dirX,dirY,dep\}}, et les 3 valeurs \emph{dirX}, \emph{dirY} et \emph{dep} sont trois points 3d représentant 3 vecteurs, les deux premiers indiquent le sens de l'écriture, le troisième un déplacement (translation) du label par rapport au point d'ancrage.
    \item \opt{showdot=false} : booléen qui indique si un point (2d) doit être dessiné au point d'ancrage.
    \item \opt{matrix=ID3d} : matrice 3d de transformation, par défaut celle-ci est la matrice 3d de l'identité, c'est à dire la table \{M(0,0,0),vecI,vecJ,vecK\}.
\end{itemize}

\begin{demo}{Construction d'un icosaèdre}
\begin{luadraw}{name=icosaedre}
local g = graph3d:new{window={-2.25,2.25,-2,2}, viewdir={40,60},bg="gray",size={10,10},margin={0,0,0,0}}
Hiddenlines = false
local phi = (1+math.sqrt(5))/2 -- nombre d'or
local A1, B1, C1, D1 = M(phi,-1,0), M(phi,1,0), M(-phi,1,0), M(-phi,-1,0) -- dans le plan z=0
local A2, B2, C2, D2 = M(0,phi,1), M(0,phi,-1), M(0,-phi,-1), M(0,-phi,1) -- dans le plan x=0
local A3, B3, C3, D3 = M(1,0,phi), M(-1,0,phi), M(-1,0,-phi), M(1,0,-phi) -- dans le plan y=0
local ico = {   {A1,B1,A3}, {B1,A1,D3}, {D1,C1,C3}, {C1,D1,B3},
                {B2,A2,B1}, {A2,B2,C1}, {D2,C2,A1}, {C2,D2,D1},
                {B3,A3,A2}, {A3,B3,D2}, {D3,C3,B2}, {C3,D3,C2},
                {A1,A3,D2}, {B1,A2,A3}, {A2,C1,B3}, {D1,D2,B3},
                {B2,B1,D3}, {A1,C2,D3}, {B2,C3,C1}, {C2,D1,C3}  }
g:Dscene3d(
    g:addFacet({A2,B2,C2,D2},{color="Navy",twoside=false,opacity=0.8}),
    g:addFacet({A1,B1,C1,D1},{color="Crimson",twoside=false,opacity=0.8}),
    g:addFacet({A3,B3,C3,D3},{color="Chocolate",twoside=false,opacity=0.8}),
    g:addPolyline(facetedges(ico), {color="Gold",width=12}), -- dessin des arêtes uniquement
    g:addDots({A1,B1,C1,D1,A2,B2,C2,D2,A3,B3,C3,D3}, {color="black",scale=1.2}),
    g:addLabel("A1",A1,{style="W",dist=0.1}, "B1",B1,{style="S"}, "C2",C2,{}, "C3",C3,{}, "A3",A3,{style="N"}, "D1",D1,{},  "A2",A2,{},  "D2",D2,{}, "B3",B3,{style="E"}, "C1",C1,{}, "B2",B2,{}, "D3",D3,{style="W"} )
)
g:Show()
\end{luadraw}
\end{demo}

\subsubsection{Ajouter des cloisons séparatrices : g:addWall}

Les cloisons séparatrices sont des objets 3d qui sont insérés en tout premier dans l'arbre représentant la scène. Ces objets ne sont pas dessinés (donc invisibles), leur rôle est de partitionner l'espace car une facette qui est d'un côté d'une cloison séparatrice ne peut pas être découpée par le plan d'une facette qui est de l'autre côté de la cloison. Cela permet dans certains cas de diminuer significativement le nombre de découpage de facettes (ou lignes polygonales) lors de la construction de la scène. Une cloison séparatrice peut être un plan entier (donc une table de deux points 3d la forme \{A,n\}, c'est à dire un point et un vecteur normal), ou bien seulement une facette.

La syntaxe est : \textbf{g:addWall(C,options)} où $C$ est soit un plan, soit une liste de plans, soit une facette, soit une liste de facettes. L'argument \emph{options} est une table. La seule option disponible est 
\begin{itemize}
    \item \opt{matrix=ID3d} : matrice 3d de transformation, par défaut celle-ci est la matrice 3d de l'identité, c'est à dire la table \{M(0,0,0),vecI,vecJ,vecK\}.
\end{itemize}

Dans l'exemple suivant les deux cloisons séparatrices ont été dessinées afin de les visualiser, mais normalement elles sont invisibles :
\begin{demo}{Exemple avec addWall (les deux facettes transparentes roses sont normalement invisibles)}
\begin{luadraw}{name=addWall}
local g = graph3d:new{size={10,10},window={-8,8,-4,8}, margin={0,0,0,0}}
local C = cylinder(M(0,0,-1),5*vecK,2)
g:Dscene3d(
    g:addWall( {{Origin,vecI}, {Origin,vecJ}}),
    g:addPlane({Origin,vecI}, {color="Pink",opacity=0.3,scale=1.125,edge=true}), -- to show the first wall
    g:addPlane({Origin,vecJ}, {color="Pink",opacity=0.3,scale=1.125,edge=true}), -- to show the second wall
    g:addPoly( shift3d(C,M(-3,-3,1)), {color="Cyan"} ),
    g:addPoly( shift3d(C,M(-3,3,0.5)), {color="ForestGreen"} ),
    g:addPoly( shift3d(C,M(3,-3,-0.5)), {color="Crimson"} )
)
g:Show()
\end{luadraw}
\end{demo}

\paragraph{Remarques sur cet exemple} : 
\begin{itemize}
    \item avec les deux cloisons séparatrices, il n'y a aucune facette découpée, et la scène en contient exactement 111 (37 par cylindre).
    \item sans les cloisons séparatrices, il y a 117 découpages (inutiles) de facettes, ce qui porte leur nombre à 228 dans la scène.
        \item avec les deux cloisons séparatrices, et l'option \opt{backcull=true} pour chaque cylindre, il n'y a aucune facette découpée, et la scène en contient 57 seulement.
\end{itemize}

Voici un autre exemple bien plus probant où l'utilisation de cloisons séparatrices est indispensable pour avoir un dessin de taille raisonnable. Il s'agit de l'obtention d'une lemniscate comme intersection d'un tore avec un certain plan. Le tore étant non convexe le nombre de découpage inutile de facettes peut être très important.

\begin{demo}{Tore et lemniscate}
\begin{luadraw}{name=torus}
local g = graph3d:new{size={10,10}, margin={0,0,0,0}}
local cos, sin, pi = math.cos, math.sin, math.pi
local R, r = 2.5, 1
local x0 = R-r
local f = function(t) return M(0,R+r*cos(t),r*sin(t)) end
local plan = {M(x0,0,0),-vecI} -- plan dont la section avec le tore donne la lemniscate
local C, wall = rotcurve(f,-pi,pi,{Origin,vecK},360,0,{grid={25,37},addwall=2})
local C1 = cutfacet(C,plan)  -- partie du tore dans le demi espace contenant -vecI
g:Dscene3d(
    g:addWall(plan), g:addWall(wall), -- ajout de cloisons séparatrices
    g:addFacet( C1, {color="Crimson", backcull=false}),
    g:addPlane(plan, {color="Pink",opacity=0.4,edge=true}), -- plan de coupe
    g:addAxes( Origin, {arrows=1})
)
-- équation  cartésienne du tore : (x^2+y^2+z^2+R^2-r^2)^2-4*R^2*(x^2+y^2) = 0
-- la lemniscate a donc pour équation (x0^2+y^2+z^2+R^2-r^2)^2-4*R^2*(x0^2+y^2)=0 (courbe implicite)
local h = function(y,z) return (x0^2+y^2+z^2+R^2-r^2)^2-4*R^2*(x0^2+y^2) end
local I = implicit(h,-4,4,-3,3,{50,50}) -- ligne polygonale 2d (liste de listes de complexes)
local lemniscate = map(function(z) return M(x0,z.re,z.im) end, I[1]) -- conversion en coordonnées spatiales
g:Dpolyline3d(lemniscate,"Navy,line width=1.2pt")
g:Show()
\end{luadraw}
\end{demo}
\paragraph{Remarques sur cet exemple} : 
\begin{itemize}
    \item Avec les cloisons séparatrices on a 30 facettes qui sont coupées et un fichier tkz de 140 Ko environ.
    \item Sans les cloisons séparatrices on a 2068 découpages de facettes (!) et un fichier tkz de 550 Ko environ.
    \item On aurait pu utiliser la section de coupe qui est renvoyée par la fonction \emph{cutfacet}, mais le résultat n'est pas très satisfaisant (cela vient du fait que le tore est non convexe).
    \item Si on n'avait pas voulu les axes traversant le tore et le plan de coupe, on aurait pu faire le dessin avec la méthode \textbf{g:Dfacet}, en remplaçant l'instruction \emph{g:Dscene3d(...)} par :
\begin{Luacode}
g:Dfacet(C1, {mode=mShadedOnly,color="Crimson"} )
g:Dfacet( g:Plane2facet(plan,0.75), {color="Pink",opacity=0.4}) 
\end{Luacode}
On obtient exactement la même chose mais sans les axes (et sans découpage de facettes bien sûr).
\end{itemize}

\paragraph{Pour conclure cette partie} : on utilise la méthode \textbf{g:Dscene3d()} lorsqu'il n'est pas possible de faire autrement, par exemple lorsqu'il y a des intersections (peu nombreuses) qui ne peuvent pas être traiter "à la main". Mais ce n'est pas le cas de toutes les intersections ! Dans l'exemple suivant, on représente une section de sphère par un plan mais sans passer par la méthode \textbf{g:Dscene3d()} car celle-ci obligerait à dessiner une sphère à facettes ce qui n'est pas très joli. L'astuce ici, consiste à dessiner la sphère avec la méthode \textbf{g:Dsphere()}, puis dessiner par dessus le plan sous forme d'une facette préalablement trouée, le trou correspondant au contour (chemin 3d) de la partie de la sphère située au-dessus du plan :
\begin{demo}{Section de sphère sans Dscene3d()}
\begin{luadraw}{name=section_sphere}
local g = graph3d:new{ window3d={-4,4,-4,4,-4,4}, window={-5.5,5.5,-4,5}, viewdir={30,75}, size={10,10}}
local O, R = Origin, 2.5 -- center et rayon
local S, P = sphere(O,R), {M(0,0,1.5),vecK+vecJ/2} -- la sphère et le plan de coupe
local w, n = pt3d.normalize(P[2]), g.Normal -- vecteurs unitaires normaux à P pour w et à l'écran pour n
local I, r = interPS(P,{O,R}) -- centre et rayon du petit cercle (intersection entre le plan et la sphère)
local C = g:Intersection3d(S,P) -- C est une liste d'arêtes
local N = I-O
local J = I+r*pt3d.normalize(vecJ-vecK/2) -- un point sur le petit cercle
local a = R/pt3d.abs(N)
local A, B = O+a*N, O-a*N -- points d'intersection de l'axe (O,I) avec la sphère
local c1, alpha = Orange, 0.4
local coul = {c1[1]*alpha, c1[2]*alpha,c1[3]*alpha} -- pour simuler la transparence
g:Dhline( g:Proj3d({B,-N})) -- demi-droite (le point B est non visible)
g:Dsphere(O,R,{mode=mBorder,color="orange"})
g:Dline3d(A,B,"dotted") -- droite (A,B) en pointillés
g:Dedges(C, {hidden=true,hiddenstyle="dashed"}) -- dessin de l'intersection
g:Dpolyline3d({I,J,O},"dashed") 
g:Dangle3d(O,I,J)  -- angle droit
g:Dcrossdots3d({{B,N},{I,N},{O,N}},rgb(coul),0.75) -- points dans la sphère
g:Dlabel3d("$O$", O, {pos="NW"})
local L = C.visible[1] -- partie visible de l'intersection (arc de cercle)
A1 = L[1]; A2 = L[#L] -- extrémités de L
local F = g:Plane2facet(P) -- plan converti en facette
-- plan troué sous forme de chemin 3d, le trou est le contour de la partie de la sphère au-dessus du plan
insert(F,{"l","cl",A1,"m",I,A2,r,-1,w,"ca",Origin,A1,R,-1,n,"ca"})
g:Dpath3d( F,"fill=Beige,fill opacity=0.6") -- dessin du plan troué
g:Dhline( g:Proj3d({A,N})) -- demi-droite, partie supérieure de l'axe (AB)
g:Dcrossdots3d({A,N},"black",0.75); g:Dballdots3d(J,"black",0.75)
g:Dlabel3d("$A$", A, {pos="NW"}, "$I$", I, {}, "$B$", B, {pos="E"}, "$J$", J, {pos="S"})
g:Show()            
\end{luadraw}
\end{demo}


\section{Constructions géométriques}

Dans cette section sont regroupées les fonctions construisant des figures géométriques sans méthode graphique dédiée.

\subsection{Cercle circonscrit, cercle inscrit : circumcircle3d(), incircle3d()}

\begin{itemize}
    \item La fonction \textbf{circumcircle3d(A,B,C)}, où $A$, $B$ et $C$ sont trois points 3d non alignés, renvoie le cercle circonscrit au triangle formé par ces trois points, sous la forme d'une séquence: $A,R,n$, où $A$ est le centre du cercle, $R$ son rayon, et $n$ un vecteur normal au plan du cercle.
    \item La fonction \textbf{incircle3d(A,B,C)}, où $A$, $B$ et $C$ sont trois points 3d non alignés, renvoie le cercle inscrit dans le triangle formé par ces trois points, sous la forme d'une séquence: $A,R,n$, où $A$ est le centre du cercle, $R$ son rayon, et $n$ un vecteur normal au plan du cercle.    
\end{itemize}

\subsection{Enveloppe convexe : cvx\_hull3d()}

La fonction \textbf{cvx\_hull3d(L)} où $L$ est une liste de points 3d \textbf{distincts}, calcule et renvoie l'enveloppe convexe de $L$ sous la forme d'une liste de facettes.

\begin{demo}{Utilisation de cvx\_hull3d()}
\begin{luadraw}{name=cvx_hull3d}
local g = graph3d:new{window={-2,4,-6,1},bbox=false,size={10,10}}
local L = {Origin, 4*vecI, M(4,4,0), 4*vecJ}
insert(L, shift3d(L,-3*vecK))
insert(L, {M(2,1,2), M(2,3,2)})
local V = cvx_hull3d(L)
local P = facet2poly(V)
g:Dpoly(P , {color="cyan",mode=mShadedHidden})
g:Show()
\end{luadraw}
\end{demo}

\subsection{Plans : plane(), planeEq(), orthoframe(), plane2ABC()}

Un plan de l'espace est une table de la forme $\{A,n\}$ où $A$ est un point du plan (point 3d) et $n$ un vecteur normal au plan (point 3d non nul).
\begin{itemize}
    \item La fonction \textbf{plane(A,B,C)} envoie le plan passant par les trois points 3d $A$, $B$ et $C$ (s'ils sont non alignés, sinon le résultat est \emph{nil}).
    \item La fonction \textbf{planeEq(a,b,c,d)} envoie le plan dont une équation cartésienne est $ax+by+cz+d=0$ (si les coefficients $a$, $b$ et $c$ ne sont pas tous nuls, sinon le résultat est\emph{nil}).
    \item La fonction \textbf{plane2ABC(P)} où $P=\{A,n\}$ désigne un plan, renvoie une séquence de trois points 3d $A,B,C$, appartenant au plan, et tels que $(A,\vec{AB},\vec{AC})$ soit un repère orthonormal direct de ce plan.
    \item La fonction \textbf{orthoframe(P)} où $P=\{A,n\}$ désigne un plan, renvoie une séquence de trois points 3d $A,u,v$, tels que $(A,u,v)$ soit un repère orthonormal direct de ce plan.
\end{itemize}

\begin{demo}{Faces d'un cube trouées avec un hexagone régulier}
\begin{luadraw}{name=plans}
local g = graph3d:new{window={-3,3,-3.25,3.25},margin={0,0,0,0},viewdir={20,60},bg="LightGray",size={10,10}}
Hiddenlines = true; Hiddenlinestyle = "dashed"
local p = polyreg(0,1,6)
local P = parallelep(M(-2,-2,-2),4*vecI,4*vecJ,4*vecK)
local V = g:Sortpolyfacet(P)
local list = {}
g:Filloptions("full","Crimson",1,true); -- true pour le mode evenodd
g:Lineoptions("solid","Gold",8)
for _, F in  ipairs(V) do
    local P1 = plane(isobar3d(F),F[1],F[2]) -- plan de la facette F
    local A, u, v = orthoframe(P1)  -- repère orthonormé sur la facette avec centre de gravité comme origine
    local p1 = map(function(z) return A+z.re*u+z.im*v end,p) -- hexagone reproduit sur la facette
    table.insert(p1,2,"m")
    local color = "Crimson"
    if not g:Isvisible(F) then  color = "Crimson!60!black" end
    g:Dpath3d( concat(F,{"l"},p1,{"l","cl"}),"fill="..color ) -- dessin de la facette "trouée" avec l'hexagone
end
g:Show()
\end{luadraw}
\end{demo}

\subsection{Sphère circonscrite, Sphère inscrite : circumsphere(), insphere()}

\begin{itemize}
    \item La fonction \textbf{circumsphere(A,B,C,D)}, où $A$, $B$, $C$ et $D$ sont quatre points 3d non coplanaires, renvoie la sphère circonscrite au tétraèdre formé par ces quatre points, sous la forme d'une séquence: $A,R$, où $A$ est le centre de la sphère, et $R$ son rayon.
    \item La fonction \textbf{insphere(A,B,C,D)}, où $A$, $B$, $C$ et $D$ sont quatre points 3d non coplanaires, renvoie la sphère inscrite dans le tétraèdre formé par ces quatre points, sous la forme d'une séquence: $A,R$, où $A$ est le centre de la sphère, et $R$ son rayon.
\end{itemize}

\subsection{Tétraèdre à longueurs fixées : tetra\_len()}

La fonction \textbf{tetra\_len(ab,ac,ad,bc,bd,cd)} calcule les sommets $A,B,C,D$ d'un tétraèdre dont les longueurs des arêtes sont données, c'est à dire tels que $AB=ab$, $AC=ac$, $AD=ad$, $BC=bc$, $BD=bd$ et $CD=cd$. La fonction renvoie la séquence de quatre points $A,B,C,D$. Le sommet $A$ est toujours le point $M(0,0,0)$ (\emph{Origin}) et le sommet $B$ est toujours le point \emph{ab*vecI} et le sommet $C$ dans le plan $xOy$. Le tétraèdre en tant que polyèdre peut ensuite être construit avec la fonction \textbf{tetra(A,B-A,C-A,D-A)}.

\begin{demo}{Un tétraèdre avec la longueur des arêtes fixée}
\begin{luadraw}{name=tetra_len}
local g = graph3d:new{window={-4,4,-4,4},margin={0,0,0,0},viewdir={25,65},size={10,10}}
Hiddenlines = true; Hiddenlinestyle = "dashed"
require 'luadraw_spherical'
local R = 4
local A,B,C,D = tetra_len(R,R,R,R,R,R)
local T = tetra(A,B-A,C-A,D-A)
g:Define_sphere({radius=R})
g:DSpolyline( facetedges(T), {color="DarkGreen"})
g:DSbigcircle( {B,C},{color="Blue"} )
g:DSbigcircle( {B,D},{color="Blue"} )
g:DSbigcircle( {C,D},{color="Blue"}  )
g:DSlabel("$R$",(2*A+C)/3,{pos="S"})
g:Dspherical()
g:Ddots3d({A,B,C,D})
g:Dlabel3d("$A$",A,{pos="S"},"$B$",B,{pos="SW"},"$C$",C,{},"$D$",D,{pos="N"} )
g:Show()
\end{luadraw}
\end{demo}

\subsection{Triangles : sss\_triangle3d(), sas\_triangle3d(), asa\_triangle3d()}

Ces fonctions sont la version 3d des fonctions  sss\_triangle(), sas\_triangle(), asa\_triangle() déjà décrites.
\begin{itemize}
    \item La fonction \textbf{sss\_triangle3d(ab,bc,ca)} où \emph{ab}, \emph{bc} et \emph{ca} sont trois longueurs, calcule et renvoie une liste de trois points 3d $\{A,B,C\}$ formant les sommets d'un triangle direct dans le plan $xOy$ dont les longueurs des côtés sont les arguments, c'est à dire $AB=ab$, $BC=bc$ et $CA=ca$, lorsque cela est possible. Le sommet $A$ est toujours le point $M(0,0,0)$ (\emph{Origin}) et le sommet $B$ est toujours le point \emph{ab*vecI}. Ce triangle peut être dessiné avec la méthode \textbf{g:Dpolyline3d}.
    \item La fonction \textbf{sas\_triangle3d(ab,alpha,ca)} où \emph{ab} et \emph{ca} sont deux longueurs, \emph{alpha} un angle en degrés, calcule et renvoie une liste de trois points 3d $\{A,B,C\}$ formant les sommets d'un triangle dans le plan $xOy$ tel que $AB=ab$, $CA=ca$, et tel que l'angle $(\vec{AB},\vec{AC})$ a pour mesure \emph{alpha}, lorsque cela est possible. Le sommet $A$ est toujours le point $M(0,0,0)$ (\emph{Origin}) et le sommet $B$ est toujours le point \emph{ab*vecI}. Ce triangle peut être dessiné avec la méthode \textbf{g:Dpolyline3d}.
    \item La fonction \textbf{asa\_triangle3d(alpha,ab,beta)} où \emph{ab} est une longueur, \emph{alpha} et \emph{beta} deux angles en degrés, calcule et renvoie une liste de trois points 3d $\{A,B,C\}$ formant les sommets d'un triangle dans le plan $xOy$ tel que $AB=ab$, tel que l'angle $(\vec{AB},\vec{AC})$ a pour mesure \emph{alpha}, et tel que l'angle $(\vec{BA},\vec{BC})$ a pour mesure \emph{beta}, lorsque cela est possible. Le sommet $A$ est toujours le point $M(0,0,0)$ (\emph{Origin}) et le sommet $B$ est toujours le point \emph{ab*vecI}. Ce triangle peut être dessiné avec la méthode \textbf{g:Dpolyline3d}.
\end{itemize}

\section{Transformations, calcul matriciel, et quelques fonctions mathématiques}

\subsection{Transformations 3d}

Dans les fonctions qui suivent :
\begin{itemize}
    \item l'argument \emph{L} est soit un point 3d, soit un polyèdre, soit une liste de points 3d (facette) soit une liste de listes de points 3d (liste de facettes),
    \item une droite \emph{d} est une liste de deux points 3d \{A,u\} : un point de la droite ($A$) et un vecteur directeur ($u$),
    \item un plan \emph{P} est une liste de deux points 3d \{A,n\} : un point du plan ($A$) et un vecteur normal au plan ($n$).
  \end{itemize}
Le résultat renvoyé est de même type que $L$.
  
\subsubsection{Appliquer une fonction de transformation : ftransform3d}

La fonction \textbf{ftransform3d(L,f)} renvoie l'image de \emph{L} par la fonction \emph{f}, celle-ci  doit être une fonction de $\mathbf R^3$ vers $\mathbf R^3$.

\subsubsection{Projections : proj3d, proj3dO, dproj3d}

\begin{itemize}
    \item La fonction \textbf{proj3d(L,P)} renvoie l'image de $L$ par la projection orthogonale sur le plan $P$.
    \item La fonction \textbf{proj3dO(L,P,v)} renvoie l'image de $L$ par la projection sur le plan $P$ parallèlement à la direction du vecteur $v$ (point 3d non nul).
    \item La fonction \textbf{dproj3d(L,d)} renvoie l'image de $L$ par la projection sur la droite $d$.
\end{itemize}

\subsubsection{Projections sur les axes ou les plans liés aux axes}

\begin{itemize}
    \item La fonction \textbf{pxy(L)} renvoie l'mage de $L$ par la projection orthogonale sur le plan $xOy$.
    \item La fonction \textbf{pyz(L)} renvoie l'mage de $L$ par la projection orthogonale sur le plan $yOz$.
    \item La fonction \textbf{pxz(L)} renvoie l'mage de $L$ par la projection orthogonale sur le plan $xOz$.
\item La fonction \textbf{px(L)} renvoie l'mage de $L$ par la projection orthogonale sur l'axe $Ox$.
\item La fonction \textbf{py(L)} renvoie l'mage de $L$ par la projection orthogonale sur l'axe $Oy$.
\item La fonction \textbf{pz(L)} renvoie l'mage de $L$ par la projection orthogonale sur l'axe $Oz$.
\end{itemize}

\subsubsection{Symétries : sym3d, sym3dO, dsym3d, psym3d}

\begin{itemize}
    \item La fonction \textbf{sym3d(L,P)} renvoie l'image de $L$ par la symétrie orthogonale par rapport au plan $P$.
    \item La fonction \textbf{sym3dO(L,P,v)} renvoie l'image de $L$ par la symétrie par rapport au plan $P$ et parallèlement à la direction du vecteur $v$ (point 3d non nul).
    \item La fonction \textbf{dsym3d(L,d)} renvoie l'image de $L$ par la symétrie orthogonale par rapport la droite $d$.
    \item La fonction \textbf{psym3d(L,point)} renvoie l'image de $L$ par la symétrie par rapport à \emph{point} (point 3d).
\end{itemize}

\subsubsection{Rotation : rotate3d, rotateaxe3d}

\begin{itemize}
    \item La fonction \textbf{rotate3d(L,angle,d)} renvoie l'image de $L$ par la rotation d'axe $d$ (orientée par le vecteur directeur qui est $d[2]$), et de \emph{angle} degrés.
    \item La fonction \textbf{rotateaxe3d(L,v1,v2,center)} renvoie l'image de $L$ par une rotation d'axe passant par le point 3d \emph{center} et qui transforme le vecteur \emph{v1} en le vecteur \emph{v2}, ces vecteurs sont normalisés par la fonction. L'argument \emph{center} est facultatif et par défaut c'est le point \emph{Origin}.
\end{itemize}


\subsubsection{Homothétie : scale3d}

La fonction \textbf{scale3d(L,k,center)} renvoie l'image de $L$ par l'homothétie de centre le point 3d \emph{center}, et de rapport \emph{k}. L'argument \emph{center} est facultatif et vaut $M(0,0,0)$ par défaut (origine).

\subsubsection{Inversion : inv3d}

La fonction \textbf{inv3d(L,radius,center)} renvoie l'image de $L$ par l'inversion par rapport à la sphère de centre  \emph{center}, et de rayon \emph{radius}.  L'argument \emph{center} est facultatif et vaut $M(0,0,0)$ par défaut (origine).

\subsubsection{Stéréographie : projstereo et inv\_projstereo}

Fonction \textbf{projstereo(L,S,N,h)}: l'argument \emph{L} désigne un point 3d ou une liste de points 3d ou une liste de listes de points 3d, appartenant tous à la sphère \emph{S}, où \emph{S=\{C,r\}} ($C$ est le centre de la sphère, et $r$ le rayon). L'argument \emph{N} désigne un point de la sphère qui sera le pôle de la projection. L'argument \emph{h} est un réel qui définit le plan de la projection, ce plan est perpendiculaire à l'axe $(CN)$, et passe par le point $I=C+h \frac{\vec{CN}}{CN}$ (avec $h=0$ c'est le plan équatorial, avec $h=-r$ c'est le plan tangent à la sphère au pôle opposé). La fonction renvoie l'image de $L$ par la projection stéréographique par rapport à la sphère $S$ avec $N$ comme pôle, et sur le plan \emph{\{I,N-C\}}.

Fonction inverse \textbf{inv\_projstereo(L,S,N)} : \emph{S=\{C,r\}} est la sphère de centre $C$ et de rayon $r$, \emph{N} est un point de la sphère $S$ (pôle), et \emph{L} est un point 3d ou une liste de points 3d ou une liste de listes de points 3d appartenant tous à un même plan orthogonal à l'axe $(CN)$. La fonction renvoie l'image de $L$ par l'inverse de la projection stéréographique par rapport à $S$ et de pôle $N$.


\subsubsection{Translation : shift3d}

La fonction \textbf{shift3d(L,v)} renvoie l'image de $L$ par la translation de vecteur $v$ (point 3d).

\subsection{Calcul matriciel}

Si $f$ est une application affine de l'espace $\mathbf R^3$, on appellera matrice de $f$ la liste (table) :
\begin{Luacode}
{ f(Origin), Lf(vecI), Lf(vecJ), Lf(vecK) }
\end{Luacode}
où $Lf$ désigne la partie linéaire de $f$ (on a \emph{Lf(vecI) = f(vecI)-f(Origin)}, etc). La matrice identité est notée \emph{ID3d} dans le paquet \emph{luadraw}, elle correspond simplement à la liste \mintinline{Lua}{ {Origin,vecI,vecJ,vecK} }.

\subsubsection{applymatrix3d et applyLmatrix3d}

\begin{itemize}
    \item La fonction \textbf{applymatrix3d(A,M)} applique la matrice $M$ au point 3d $A$ et renvoie le résultat (ce qui revient à calculer $f(A)$ si $M$ est la matrice de $f$). Si $A$ n'est pas un point 3d, la fonction renvoie $A$.
    
    \item La fonction \textbf{applyLmatrix3d(A,M)} applique la partie linéaire la matrice $M$ au point 3d $A$ et renvoie le résultat (ce qui revient à calculer $Lf(A)$ si $M$ est la matrice de $f$). Si $A$ n'est pas un point 3d, la fonction renvoie $A$.
\end{itemize}

\subsubsection{composematrix3d}
La fonction \textbf{composematrix3d(M1,M2)} effectue le produit matriciel $M1\times M2$ et renvoie le résultat.

\subsubsection{invmatrix3d}
La fonction \textbf{invmatrix3d(M)} calcule et renvoie l'inverse de la matrice $M$ lorsque cela est possible.

\subsubsection{matrix3dof}

La fonction \textbf{matrix3dof(f)} calcule et renvoie la matrice de $f$ (qui doit être une application affine de l'espace $\mathbf R^3$).


\subsubsection{mtransform3d et mLtransform3d}
\begin{itemize}
    \item La fonction \textbf{mtransform3d(L,M)} applique la matrice $M$ à la liste $L$ et renvoie le résultat. $L$ doit être une liste de points 3d (une facette) ou une liste de listes de points 3d (liste de facettes).
    \item La fonction \textbf{mLtransform3d(L,M)} applique la partie linéaire la matrice $M$ à la liste $L$ et renvoie le résultat. $L$ doit être une liste de points 3d (une facette) ou une liste de listes de points 3d (liste de facettes).
\end{itemize}

\subsection{Matrice associée au graphe 3d}

Lorsque l'on crée un graphe dans l'environnement \emph{luadraw}, par exemple :
\begin{Luacode}
local g = graph3d:new{size={10,10}}
\end{Luacode}
l'objet \emph{g} créé possède une matrice 3d de transformation qui est initialement l'identité. Toutes les méthodes graphiques appliquent automatiquement la matrice 3d de transformation du graphe. Une réserve cependant : les méthodes \emph{Dcylinder}, \emph{Dcone} et \emph{Dsphere} ne donnent le bon résultat qu'avec la matrice de transformation égale à l'identité. Pour manipuler cette matrice, on dispose des méthodes qui suivent.

\subsubsection{g:Composematrix3d()}
La méthode \textbf{g:Composematrix3d(M)} multiplie la matrice 3d du graphe \emph g par la matrice \emph{M} (avec \emph{M} à droite) et le résultat est affecté à la matrice 3d du graphe. L'argument \emph{M} doit donc être une matrice 3d.

\subsubsection{g:Det3d()}
La méthode \textbf{g:Det3d()} envoie $1$ lorsque la matrice 3d de transformation a un déterminant positif, et $-1$ dans le cas contraire. Cette information est utile lorsqu'on a besoin de savoir si l'orientation de l'espace a été changée ou non.

\subsubsection{g:IDmatrix3d()}
La méthode \textbf{g:IDmatrix3d()} réaffecte l'identité à la matrice 3d du graphe \emph g.

\subsubsection{g:Mtransform3d()}
La méthode \textbf{g:Mtransform3d(L)} applique la matrice du graphe 3d de \emph g à \emph{L} et renvoie le résultat, l'argument \emph L doit être une liste de points 3d (une facette) ou une liste de listes de points 3d (liste de facettes).

\subsubsection{g:MLtransform3d()}
La méthode \textbf{g:MLtransform3d(L)} applique la partie linéaire de la matrice 3d du graphe \emph g à \emph{L} et renvoie le résultat. L'argument \emph L doit être une liste de points 3d (une facette) ou une liste de listes de points 3d (liste de facettes).

\subsubsection{g:Rotate3d()}
La méthode \textbf{g:Rotate3d(angle,axe)} modifie la matrice 3d de transformation du graphe \emph g en la composant avec la matrice de la rotation d'angle \emph{angle} (en degrés) et d'axe \emph{axe}. 

\subsubsection{g:Scale3d()}
La méthode \textbf{g:Scale3d(factor, center)} modifie la matrice 3d de transformation du graphe \emph g en la composant avec la matrice de l'homothétie de rapport \emph{factor} et de centre \emph{center}. L'argument \emph{center} est un point 3d qui vaut \emph{Origin} par défaut.


\subsubsection{g:Setmatrix3d()}
La méthode \textbf{g:Setmatrix3d(M)} permet d'affecter la matrice \emph M à la matrice 3d de transformation du graphe \emph g.

\subsubsection{g:Shift3d()}
La méthode \textbf{g:Shift3d(v)} modifie la matrice 3d de transformation du graphe \emph g en la composant avec la matrice de la translation de vecteur \emph{v} qui doit être un point 3d.

\subsection{Fonctions mathématiques supplémentaires}

\subsubsection{clippolyline3d()}
La fonction \textbf{clippolyline3d(L, poly, exterior, close)} clippe la ligne polygonale 3d \emph{L} avec le polyèdre \textbf{convexe} \emph{poly}, si l'argument facultatif \emph{exterior} vaut true, alors c'est la partie extérieure au polyèdre qui est renvoyée (false par défaut), si l'argument facultatif \emph{close} vaut true, alors la ligne polygonale est refermée (false par défaut). \emph{L} est une liste de points 3d ou une liste de listes de points 3d.\par
\textbf{Remarque} : le résultat n'est pas toujours satisfaisant pour la partie extérieure.

\paragraph{Cas particulier} : clipper une ligne polygonale 3d $L$ avec la fenêtre 3d courante peut se faire avec cette fonction de la manière suivante :

\begin{center}
\textbf{L = clippolyline3d(L, g:Box3d())}
\end{center}

En effet, la méthode \textbf{g:Box3d()} renvoie la fenêtre 3d courante sous forme d'un parallélépipède.


\subsubsection{clipline3d()}
La fonction \textbf{clipline3d(line, poly)} clippe la droite \emph{line} avec le polyèdre \textbf{convexe} \emph{poly}, la fonction renvoie la partie de la droite intérieure au polyèdre. L'argument \emph{line} et une table de la forme \{A,u\} où $A$ est un point de la droite et $u$ un vecteur directeur (deux points 3d).

\paragraph{Cas particulier} : clipper une droite $d$ avec la fenêtre 3d courante peut se faire avec cette fonction de la manière suivante :

\begin{center}
\textbf{d = clipline3d(d, g:Box3d())}
\end{center}

En effet, la méthode \textbf{g:Box3d()} renvoie la fenêtre 3d courante sous forme d'un parallélépipède ($d$ devient alors un segment).

\subsubsection{cutpolyline3d()}
La fonction \textbf{cutpolyline3d(L,plane,close)} coupe la ligne polygonale 3d \emph{L} avec le plan \emph{plane}, si l'argument facultatif \emph{close} vaut true, alors la ligne est refermée (false par défaut).
\emph{L} est une liste de points 3d ou une liste de listes de points 3d, \emph{plane} est une table de la forme \{A,n\} où $A$ est un point du plan et $n$ un vecteur normal (deux points 3d).

Le fonction renvoie trois choses :
\begin{itemize}
    \item la partie de \emph{L} qui est dans le demi-espace contenant le vecteur $n$,
    \item suivie de la partie de \emph{L} qui est dans l'autre demi-espace,
    \item suivie de la liste des points d'intersection.
\end{itemize}

\subsubsection{getbounds3d()}
La fonction \textbf{getbounds3d(L)} renvoie les limites xmin,xmax,ymin,ymax,zmin,zmax de la ligne polygonale 3d \emph{L} (liste de points 3d ou une liste de listes de points 3d).

\subsubsection{interDP()}
La fonction \textbf{interDP(d,P)} calcule et renvoie (si elle existe) l'intersection entre la droite $d$ et le plan $P$.

\subsubsection{interPP()}
La fonction \textbf{interPP(P1,P2)} calcule et renvoie (si elle existe) l'intersection entre les plans $P_1$ et $P_2$.

\subsubsection{interDD()}
La fonction \textbf{interDD(D1,D2,epsilon)} calcule et renvoie (si elle existe) l'intersection entre les droites $D_1$ et $D_2$. L'argument \emph{epsilon} vaut $10^{-10}$ par défaut (sert à tester si un certain flottant est nul).

\subsubsection{interDS()}
La fonction \textbf{interDS(d,S)} calcule et renvoie (si elle existe) l'intersection entre la droite $d$ et la sphère $S$ où $S$ est une table $S=\{C,r\}$ avec $C$ le centre (point 3d) et $r$ le rayon. La fonction renvoie soit \emph{nil} (intersection vide), soit un seul point, soit deux points.

\subsubsection{interPS()}
La fonction \textbf{interPS(P,S)} calcule et renvoie (si elle existe) l'intersection entre le plan $P$ et la sphère $S$ où $S$ est une table $S=\{C,r\}$ avec $C$ le centre (point 3d) et $r$ le rayon. La fonction renvoie soit \emph{nil} (intersection vide), soit une séquence de la forme $I,r,n$, où I est un point 3d représentant le centre d'un cercle, $r$ son rayon et $n$ un vecteur normal au plan du cercle, ce cercle est l'intersection cherchée. 

\subsubsection{interSS()}
La fonction \textbf{interPS(S1,S2)} calcule et renvoie (si elle existe) l'intersection entre la sphère $S1=\{C1,r1\}$ et $S2=\{C2,r2\}$. La fonction renvoie soit \emph{nil} (intersection vide), ou bien une séquence de la forme $I,r,n$, où I est un point 3d représentant le centre d'un cercle, $r$ son rayon et $n$ un vecteur normal au plan du cercle, ce cercle est l'intersection cherchée. 

\subsubsection{merge3d()}
La fonction \textbf{merge3d(L)} recolle si c'est possible, les composantes connexes de \emph{L} qui doit être une liste de listes de points 3d, la fonction renvoie le résultat.

\subsubsection{split\_points\_by\_visibility()}
La fonction \textbf{split\_points\_by\_visibility(L, visible\_function)} où $L$ est une liste de points 3d, ou une liste de listes de points 3d, et où \emph{visible\_function} est une fonction telle que \emph{visible\_function(A)} retourne \emph{true} si le point 3d $A$ est visible, \emph{false} sinon, permet de trier les points de $L$ suivant qu'ils sont visibles ou non. La fonction renvoie une séquence de deux tables : \emph{visible\_points}, \emph{hidden\_points}.

\begin{demo}{Une courbe sur un cylindre}
\begin{luadraw}{name=curve_on_cylinder}
local g = graph3d:new{adjust2d=true,bbox=false,size={10,10}};
g:Labelsize("footnotesize")
Hiddenlines = true; Hiddenlinestyle = "dashed"

local curve_on_cylinder = function(curve,cylinder) 
-- curve is a 3d polyline on a cylinder, 
-- cylinder = {A,r,V,B}
    local  A,r,V,B = table.unpack(cylinder)
    if B == nil then B = V; V = B-A end
    local U = B-A
    local visible_function = function(N)
        local I = dproj3d(N,{A,U})
        return (pt3d.dot(N-I,g.Normal) >= 0)
    end
    return split_points_by_visibility(curve,visible_function)
end
-- test
local A, r, B = -5*vecJ, 4, 5*vecJ -- cylinder
local p = function(t) return Mc(r,t,t/3) end
local Curve = rotate3d( parametric3d(p,-4*math.pi,4*math.pi),90,{Origin,vecI})
local Vi, Hi = curve_on_cylinder(Curve,{A,r,B})
local curve_color = "DarkGreen"
g:Dboxaxes3d({grid=true,gridcolor="gray",fillcolor="LightGray"})
g:Dcylinder(A,r,B,{color="orange"})
g:Dpolyline3d(Vi,curve_color)
g:Dpolyline3d(Hi,curve_color..","..Hiddenlinestyle)
g:Show()
\end{luadraw}
\end{demo}

\section{Exemples plus poussés}

\subsection{La boîte de sucres}

Le problème\footnote{Problème posé dans un forum, l'objectif étant d'en faire des exercices de comptage pour des élèves.} est de dessiner des sucres dans une boîte. Il faut pouvoir positionner le nombre que l'on veut de morceaux, et où on veut dans la boite\footnote{Un morceau doit reposer soit sur le fond de la boîte, soit sur un autre morceau} sans avoir à réécrire tout le code. Autre contrainte : pour alléger au maximum la figure, seules les facettes réellement vues doivent être affichées. Dans le code proposé ci-dessous on garde les angles de vues par défaut, et :
\begin{itemize}
    \item les sucres sont des cubes de côté 1 (on modifie ensuite la matrice 3d du graphe pour les "allonger"),
    \item chaque morceau est repéré par les coordonnées $(x,y,z)$ du coin supérieur droit de la face avant, avec $x$ entier 1 et \emph{Lg}, $y$ entier entre 1 et \emph{lg} et $z$ entier entre 1 et \emph{ht}.
    \item pour mémoriser les positions des morceaux on utilise une matrice \emph{positions} à trois dimensions, une pour $x$, une pour $y$ et une pour $z$, avec la convention que \emph{positions[x][y][z]} vaut 1 s'il y a un sucre à la position $(x,y,z)$, et 0 sinon.
    \item pour chaque morceau il y a au plus trois faces visibles : celles du dessus, celle de droite et celle de devant\footnote{À condition de ne pas changer les angles de vue !}, mais on ne dessine la face du dessus que s'il n'y a pas un autre morceau de sucre au-dessus, on ne dessine la face du droite que s'il n'y a pas un autre morceau à droite, et on ne dessine la face de devant que s'il n'y a pas un autre morceau devant. On construit ainsi la liste des facettes réellement vues.
    \item Dans l'affichage de la scène, il faut \textbf{mettre la boîte en premier}, sinon les facettes de celle-ci vont être découpées par les plans des facettes des morceaux de sucre. Les facettes des morceaux de sucre ne peuvent pas être découpées par la boîte car ils sont tous dedans.
\end{itemize}

\begin{demo}{Boite de morceaux de sucre}
\begin{luadraw}{name=boite_sucres}
local g = graph3d:new{window={-9,8,-10,4},size={10,10}}
Hiddenlines = false
local Lg, lg, ht = 5, 4, 3 -- longueur, largeur, hauteur (taille de la boîte)
local positions = {} -- matrice de dimension 3 initialisée avec des 0
for L = 1, Lg do
    local X = {}
    for l = 1, lg do
        local Y = {}
        for h = 1, ht do table.insert(Y,0) end
        table.insert(X,Y)
    end
    table.insert(positions,X)
end
local facetList = function() -- renvoie la liste des facettes à dessiner (attention à l'orientation)
    local facet = {}
    for x = 1, Lg do -- parcours de la matrice positions
        for y = 1, lg do
            for z = 1, ht do
                if positions[x][y][z] == 1 then -- il y a un sucre en (x,y,z)
                    if (z == ht) or (positions[x][y][z+1] == 0) then -- pas de sucre au-dessus donc face du dessus visible
                        table.insert(facet, {M(x,y,z),M(x-1,y,z),M(x-1,y-1,z),M(x,y-1,z)}) -- insertion face du dessus
                    end
                    if (y == lg) or (positions[x][y+1][z] == 0) then -- pas de sucre à droite donc face de droite visible
                        table.insert(facet, {M(x,y,z),M(x,y,z-1),M(x-1,y,z-1),M(x-1,y,z)}) -- insertion face de droite
                    end
                    if (x == Lg) or (positions[x+1][y][z] == 0) then -- pas de sucre devant donc face de devant visible
                        table.insert(facet, {M(x,y,z),M(x,y-1,z),M(x,y-1,z-1),M(x,y,z-1)}) -- insertion face de devant
                    end
                end
            end
        end
    end
    return facet
end
-- création de la boîte (parallélépipède)
local O = Origin -0.1*M(1,1,1) -- pour ne pas que la boîte soit collée aux sucres
local boite = parallelep(O, (Lg+0.2)*vecI, (lg+0.2)*vecJ, (ht+0.5)*vecK)
table.remove(boite.facets,2) -- on retire le dessus de la boîte, c'est la facette numéro 2
-- on positionne des sucres
for y = 1, 4 do for z = 1, 3 do  positions[1][y][z] = 1 end end
for x = 2, 5 do for z = 1, 2 do positions[x][1][z] = 1 end end
for z = 1, 3 do positions[5][3][z] = 1 end
for z = 1, 2 do positions[4][4][z] = 1 end
for z = 1, 2 do positions[3][4][z] = 1 end
positions[5][1][3] = 1; positions[3][1][3] = 1; positions[5][4][1] = 1; positions[2][3][1] = 1
g:Setmatrix3d({Origin,3*vecI,2*vecJ,vecK}) -- dilatation sur Ox et Oy pour "allonger" les cubes ...
g:Dscene3d( -- dessin
    g:addPoly(boite,{color="brown",edge=true,opacity=0.9}),
    g:addFacet(facetList(), {backcull=true,contrast=0.25,edge=true})    )
g:Labelsize("huge"); g:Dlabel3d( "SUGAR", M(Lg/2+0.1,lg+0.1,ht/2+0.1), {dir={-vecI,vecK}})
g:Show()
\end{luadraw}
\end{demo}

\subsection{Empilement de cubes}

On peut modifier l'exemple précédent pour dessiner un empilement de cubes positionnés au hasard, avec 4 vues. On va positionner les cubes en en mettant un nombre aléatoire par colonne en commençant par le bas. On va faire 4 vues de l'empilement en ajoutant les axes pour se repérer entre ces différentes vues. Cela change un peu la recherche des facettes potentiellement visibles, il y a 5 cas par cube et non plus seulement 3 (devant, derrière, gauche, droite et dessus, on ne fait pas de vues de dessous). Pour plus de lisibilité de l'empilement, on utilise trois couleurs pour peindre les faces des cubes (deux faces opposées ont la même couleur).

\begin{demo}{Empilement de cubes}
\begin{luadraw}{name=cubes_empiles}
local g = graph3d:new{window3d={-6,6,-6,6,-6,6},size={10,10}}
Hiddenlines = false
local Lg, lg, ht, a = 5, 5, 5, 2 -- longueur, largeur, hauteur de l'espace à remplir, taille d'un cube
local positions = {} -- matrice de dimension 3 initialisée avec des 0
for L = 1, Lg do
    local X = {}
    for l = 1, lg do
        local Y = {}
        for h = 1, ht do table.insert(Y,0) end
        table.insert(X,Y)
    end
    table.insert(positions,X)
end
for x = 1, Lg do  -- positionnement aléatoire de cubes
    for y = 1, lg do
        local nb = math.random(0,ht) -- on met nb cubes dans la colonne (x,y,*) en partant du bas
        for z = 1, nb do positions[x][y][z] = 1 end
    end
end
local dessus,gauche,devant = {},{},{} -- pour mémoriser les facettes
for x = 1, Lg do -- parcours de la matrice positions pour déterminer les facettes à dessiner
    for y = 1, lg do
        for z = 1, ht do
            if positions[x][y][z] == 1 then -- il y a un cube en (x,y,z)
                if (z == ht) or (positions[x][y][z+1] == 0) then -- pas de cube au-dessus donc face visible
                    table.insert(dessus,{M(x,y,z),M(x-1,y,z),M(x-1,y-1,z),M(x,y-1,z)}) -- insertion face du dessus
                end
                if (y == lg) or (positions[x][y+1][z] == 0) then -- pas de cube à droite donc face  visible
                    table.insert(gauche,{M(x,y,z),M(x,y,z-1),M(x-1,y,z-1),M(x-1,y,z)}) -- insertion face droite
                end
                if (y == 1) or (positions[x][y-1][z] == 0) then -- pas de cube à gauche donc face visible
                    table.insert(gauche,{M(x,y-1,z),M(x-1,y-1,z),M(x-1,y-1,z-1),M(x,y-1,z-1)}) -- insertion face gauche
                end                    
                if (x == Lg) or (positions[x+1][y][z] == 0) then -- pas de cube devant donc face visible
                    table.insert(devant,{M(x,y,z),M(x,y-1,z),M(x,y-1,z-1),M(x,y,z-1)}) -- insertion face avant
                end
                if (x == 1) or (positions[x-1][y][z] == 0) then -- pas de cube derrière donc face de derrière visible
                    table.insert(devant,{M(x-1,y,z),M(x-1,y,z-1),M(x-1,y-1,z-1),M(x-1,y-1,z)}) -- insertion face arrière
                end
            end
        end
    end
end
g:Setmatrix3d({M(-a*Lg/2,-a*lg/2,-a*ht/2),a*vecI,a*vecJ,a*vecK}) -- pour centrer la figure et avoir des cubes de côté a
local dessin = function()
    g:Dscene3d(
        g:addFacet(dessus, {backcull=true,color="Crimson"}), g:addFacet(gauche, {backcull=true,color="DarkGreen"}),
        g:addFacet(devant, {backcull=true,color="SteelBlue"}),
        g:addPolyline(facetedges(concat(dessus,gauche,devant))), -- dessin des arêtes
        g:addAxes(Origin,{arrows=1}))
end
g:Saveattr(); g:Viewport(-5,0,0,5); g:Coordsystem(-11,11,-11,11); g:Setviewdir(45,60) -- en haut à gauche
 dessin(); g:Restoreattr()
g:Saveattr(); g:Viewport(0,5,0,5);g:Coordsystem(-11,11,-11,11); g:Setviewdir(-45,60) -- en haut à droite
dessin(); g:Restoreattr()
g:Saveattr(); g:Viewport(-5,0,-5,0);g:Coordsystem(-11,11,-11,11); g:Setviewdir(-135,60) -- en bas à gauche
dessin(); g:Restoreattr()
g:Saveattr(); g:Viewport(0,5,-5,0);g:Coordsystem(-11,11,-11,11); g:Setviewdir(135,60) -- en bas à droite
dessin(); g:Restoreattr()
g:Show()
\end{luadraw}
\end{demo}


\subsection{Illustration du théorème de Dandelin}

\begin{demo}{Illustration du théorème de Dandelin}
\begin{luadraw}{name=Dandelin}
local g = graph3d:new{window3d={-5,5,-5,5,-5,5}, window={-5,5,-5,6}, bg="lightgray",viewdir={-10,85}}
g:Linewidth(8)
local sqrt = math.sqrt
local sqr = function(x) return x*x end
local L, a = 4.5, 2
local R = (a+5)*L/sqrt(100+L^2) --grosse sphère centre=M(0,0,a) rayon=R
local S2 = sphere(M(0,0,a),R,45,45)
local k = 0.35 --rapport d'homothetie
local b, r = (a+5)*k-5, k*R -- petite sphère centre=M(0,0,b) rayon=r
local S1 = sphere(M(0,0,b),r,45,45)
local c = (b+k*a)/(1+k)  --deuxieme centre d'homothetie
local z = a+sqr(R)/(c-a) --image de c par l'inversion par rapport à la grosse sphère
local M1 = M(0,sqrt(sqr(R)-sqr(z-a)),z)--point de la grosse sphère et du plan tangent
local N = M1-M(0,0,a) -- vecteur normal au plan tangent
local plan = {M(0,0,c),-N} -- plan tangent
local z2 = a+sqr(R)/(-5-a) --image du sommet par l'inversion par rapport à la grosse sphère
local z1 = b+sqr(r)/(-5-b) -- image du sommet par l'inversion par rapport à la petite sphère
local P2 = M(sqrt(R^2-(z2-a)^2),0,z2)
local P1= M(sqrt(r^2-(z1-b)^2),0,z1)
local S = M(0,0,-5)
local P = interDP({P1,P2-P1},plan)
local C = cone(M(0,0,-5),10*vecK,L,45,true)
local ellips = g:Intersection3d(C,plan)
local plan1 = {M(0,0,z1),vecK}
local plan2 = {M(0,0,z2),vecK}
local L1, L2 = g:Intersection3d(S1,plan1), g:Intersection3d(S2,plan2)
local F1, F2 = proj3d(M(0,0,b), plan), proj3d(M(0,0,a), plan)  --foyers
local s1, s2 = g:Proj3d(M(0,0,a)), g:Proj3d(M(0,0,b))
local V, H = g:Classifyfacet(C) -- on sépare facettes visibles et les autres
local V1, V2 = cutfacet(V,plan)
local H1, H2 = cutfacet(H,plan)
-- Dessin
g:Dpolyline3d( border(H2),"left color=white, right color=DarkSeaGreen, draw=none" ) -- faces non visibles sous le plan, remplissage seulement
g:Dsphere( M(0,0,b), r, {mode=mBorder,color="Orange"}) -- petite sphère
g:Dpolyline3d( border(V2),"left color=white, right color=DarkSeaGreen, fill opacity=0.4" ) -- faces visibles sous le plan
g:Dpolyline3d({S,P})  -- segment [S,P] qui est sous le plan en partie
g:Dfacet( g:Plane2facet(plan,0.75), {color="Chocolate", opacity=0.8}) -- le plan
g:Dpolyline3d( border(H1),"left color=white, right color=DarkSeaGreen,draw=none,fill opacity=0.7" ) -- contour faces non visibles au dessus du plan, remplissage seulement
g:Dsphere( M(0,0,a),R, {mode=2,color="SteelBlue"}) -- grosse sphère
g:Dpolyline3d( border(V1),"left color=white, right color=DarkSeaGreen, fill opacity=0.6" ) -- contour faces visibles au dessus du plan
g:Dcircle3d(M(0,0,5),L,vecK) -- ouverture du cône
g:Dpolyline3d({{P,F1},{F2,P,P2}})
g:Dedges(L1,{hidden=true,color="FireBrick"})
g:Dedges(L2,{hidden=true,color="FireBrick"})
g:Dedges(ellips,{hidden=true, color="blue"})
g:Dballdots3d({F1,F2,S,P1,P,P2},nil,0.75)
g:Dlabel3d(
  "$F_1$",F1,{pos="N"}, "$F_2$",F2,{}, "$N_2$",P2,{},"$S$",S,{pos="S"}, "$N_1$",P1,{pos="SE"}, "$P$",P,{pos="SE"} )
g:Show()
\end{luadraw}
\end{demo}

On veut dessiner un cône avec une section par un plan et deux sphères à l'intérieur de ce cône (et tangentes au plan), mais sans dessiner de sphères ni de cônes à facettes. Le point de départ est néanmoins la création de ces solides à facettes, les sphères \emph{S1} et \emph{S2} (lignes 11 et 8 du listing) ainsi que le cône \emph{C} en ligne 23. Le principe du dessin est le suivant :
\begin{enumerate}
    \item On sépare les facettes du cône en deux catégories : les facettes visibles (tournées vers l'observateur) et les autres (variables \emph{V} et \emph{H} ligne 30), ce qui correspond en fait à l'avant du cône et l'arrière du cône.
    \item On découpe les deux listes de facettes avec le plan (lignes 31 et 32). Ainsi, \emph{V1} correspond aux facettes avant situées au-dessus du plan et \emph{V2} correspond aux facettes avant situées sous le plan (même chose avec \emph{H1} et \emph{H2} pour l'arrière).
    \item On dessine alors le contour de \emph{H2} avec un remplissage (seulement) en gradient (ligne 34).
    \item On dessine la petite sphère (en orange, ligne 35).
    \item On dessine le contour de \emph{V2} avec un remplissage en gradient et transparence pour voir la petite sphère  (ligne 36).
    \item On dessine le segment $[S,P]$ (ligne 37) puis le plan sous forme de facette transparente (ligne 38).
    \item On dessine le contour de \emph{H1} avec un remplissage en gradient (ligne 39). C'est la partie arrière au dessus du plan.
    \item On dessine la grande sphère (ligne 40).
    \item On dessine enfin le contour de \emph{V1} avec un remplissage en gradient (ligne 41) et transparence pour voir la sphère (c'est la partie avant du cône au dessus du plan), puis l'ouverture du cône (ligne 42).
    \item On dessine les intersections entre le cône et les sphères (lignes 44 et 45) ainsi qu'entre le cône et le plan (ligne 46).
\end{enumerate}

\subsection{Volume défini par une intégrale double}
\begin{demo}{Volume correspondant à $\int_{x_1}^{x_2}\int_{y_1}^{y_2}f(x,y)dxdy$}
\begin{luadraw}{name=volume_integrale}
local i, pi, sin, cos = cpx.I, math.pi, math.sin, math.cos
local g = graph3d:new{window3d={-4,4,-4,4,0,6},adjust2d=true,margin={0,0,0,0},size={10,10}}
local x1, x2, y1, y2 = -3,3,-3,3 -- bornes
local f = function(x,y) return cos(x)+sin(y)+5 end -- fonction à intégrer
local p = function(u,v) return M(u,v,f(u,v)) end -- paramétrage surface z=f(x,y)
local Fx1 = concat({pxy(p(x1,y2)), pxy(p(x1,y1))}, parametric3d(function(t) return p(x1,t) end,y1,y2,25,false,0)[1])
local Fx2 = concat({pxy(p(x2,y1)), pxy(p(x2,y2))}, parametric3d(function(t) return p(x2,t) end,y2,y1,25,false,0)[1])
local Fy1 = concat({pxy(p(x1,y1)), pxy(p(x2,y1))}, parametric3d(function(t) return p(t,y1) end,x2,x1,25,false,0)[1])
local Fy2 = concat({pxy(p(x2,y2)), pxy(p(x1,y2))}, parametric3d(function(t) return p(t,y2) end,x1,x2,25,false,0)[1])
g:Dboxaxes3d({grid=true, gridcolor="gray",fillcolor="LightGray",labels=false})
g:Filloptions("fdiag","black"); g:Dpolyline3d( {M(x1,y1,0),M(x1,y2,0),M(x2,y2,0),M(x2,y1,0)}) -- dessous
g:Dfacet( {Fx1,Fx2,Fy1,Fy2},{mode=mShaded,opacity=0.7,color="Crimson"} )
g:Dfacet(surface(p,x1,x2,y1,y2), {mode=mShadedOnly,color="cyan"})
g:Dlabel3d("$x_1$", M(x1,4.75,0),{}, "$x_2$", M(x2,4.75,0),{}, "$y_1$", M(4.75,y1,0),{}, "$y_2$", M(4.75,y2,0),{}, "$0$",M(4,-4.75,0),{})  
g:Show()  
\end{luadraw}
\end{demo}

Ici le solide représenté a des faces latérales (\emph{Fx1}, \emph{Fx2}, \emph{Fy1} et \emph{Fy2}) présentant un côté qui est une courbe paramétrée. On prend donc les points de cette courbe paramétrée (sa première composante connexe) et on lui ajoute les projetés des deux extrémités sur le plan $xOy$. Il faut faire attention au sens de parcours pour que les faces soient bien orientées (normale vers l'extérieur), cette normale étant calculée à partir des trois premiers points de la face, il vaut mieux commencer la face par les deux projetés sur le plan pour être sur de l'orientation.
On dessine en premier le dessous, puis les faces latérales, et on termine par la surface.

\subsection{Volume défini sur autre chose qu'un pavé}
\begin{demo}{Volume : $0\leqslant x\leqslant1;\ 0\leqslant y \leqslant x^2;\ 0\leqslant z\leqslant y^2$}
\begin{luadraw}{name=volume2}
local i = cpx.I
local g = graph3d:new{window3d={0,1,0,1,0,1}, margin={0,0,0,0},adjust2d=true,viewdir={170,40}, size={10,10}}
g:Labelsize("scriptsize")
local f = function(t) return M(t,t^2,0) end
local h = function(t) return M(1,t,t^2) end
local C = parametric3d(f,0,1,25,false,0)[1] -- courbe y=x^2 dans le plan z=0 (première composante connexe)
local D = parametric3d(h,1,0,25,false,0)[1] -- courbe z=y^2 dans le plan x=1, en sens inverse
local dessous = concat({M(1,0,0)},C) -- forme la face du dessous
local arriere = concat({M(1,1,0)},D) -- forme la face arrière
local  avant, dessus, A, B = {}, {}, nil, C[1]
for k = 2, #C do --on construit les faces avant et de dessus facette par facette, en partant des points de C
    A = B; B = C[k]
    table.insert(avant, {B,A,M(A.x,A.y,A.y^2),M(B.x,B.y,B.y^2)})
    table.insert(dessus, {M(B.x,B.y,B.y^2),M(A.x,A.y,A.y^2),M(1,A.y,A.y^2),M(1,B.y,B.y^2)})
end
g:Dboxaxes3d({grid=true, gridcolor="gray",fillcolor="LightGray", drawbox=false, 
    xyzstep=0.25, zlabelstyle="W",zlabelsep=0})
g:Lineoptions(nil,"Navy",8)  
g:Dpolyline3d(arriere,close,"fill=Crimson, fill opacity=0.6") -- face arrière (plane)
g:Filloptions("fdiag","black"); g:Dpolyline3d(dessous,close) -- dessous
g:Dmixfacet(avant,{color="Crimson",opacity=0.7,mode=mShadedOnly}, dessus,{color="cyan",opacity=1})
g:Filloptions("none"); g:Dpolyline3d(concat(border(avant),border(dessus)))
g:Show() 
\end{luadraw}
\end{demo}

Dans cet exemple, la surface a pour équation $z=y^2$ (cylindre parabolique), mais nous ne sommes plus sur un pavé. La face avant n'est pas plane, on construit celle-ci à la manière d'un cylindre (ligne 14) avec des facettes verticales qui s'appuient sur la courbe $C$ en bas, et sur la courbe $t\mapsto M(t,t^2,t^4)$ en haut.

De même, la face du dessus (la surface) est construite à la manière d'un cylindre horizontal qui s'appuie sur les courbes $D$ et $t\mapsto M(t,t^2,t^4)$.

On pourrait ne pas construire à la main la surface (appelée \emph{dessus} dans le code), et dessiner à la place la surface suivante (après la face avant) :
\begin{Luacode}
g:Dfacet( surface(function(u,v) return M(u,v*u^2,v^2*u^4) end, 0,1,0,1), {mode=mShadedOnly, color="cyan"})
\end{Luacode}
mais elle comporte bien plus de facettes (25*25) que la construction sous forme de cylindre (21 facettes), ce qui est moins intéressant.

\section{Extensions}

\subsection{Le module \emph{luadraw\_polyhedrons}}

Ce module est encore à l'état d'ébauche et est appelé à s'étoffer par la suite. Comme son nom l'indique, il contient la définition de polyèdres. Toutes les données numériques sont issues du site \href{https://dmccooey.com/polyhedra/}{Visual Polyhedra}.

Toutes les fonctions sont sur le même modèle : \textbf{<nom>(C,S,all)} où $C$ est le centre du polyèdre (point 3d) et $S$ un sommet du polyèdre (point 3d), lorsque $C$ ou $S$ ont la valeur \emph{nil}, c'est le polyèdre non transformé (de centre l'origine) qui est renvoyé . L'argument facultatif \emph{all} est un booléen, lorsqu'il a la valeur \emph{true} la fonction renvoie quatre choses : \emph{P, V, E, F} où :
    \begin{itemize}
        \item $P$ est le solide en tant que polyèdre,
        \item $V$ la liste (table) des sommets,
        \item $E$ la liste (table) des arêtes (avec points 3d),
        \item $F$ la liste des facettes (avec points 3d). Certains polyèdres ont plusieurs types de facettes, dans ce cas la résultat renvoyé est de la forme : \emph{P, V, E, F1, F2, ...}, où $F1$, $F2$ ..., sont des listes de facettes.Cela peut permettre de sles dessiner avec des couleurs différentes par exemple.
        \end{itemize}
L'argument \emph{all} la valeur \emph{false},qui est la valeur par défaut, la fonction ne renvoie que le polyèdre.

Voici les solides actuellement contenus dans ce module :

\begin{itemize}
    \item Les solides de Platon, ces solides n'ont qu'un type des faces :
        \begin{itemize}
            \item  la fonction \textbf{tetrahedron(C,S,all)} permet la construction d'un tétraèdre régulier de centre $C$ (point 3d) et dont un sommet est $S$ (point 3d).
            \item la fonction \textbf{octahedron(C,S,all)} permet la construction d'un octaèdre de centre $C$ (point 3d) et dont un sommet est $S$ (point 3d).
            \item la fonction \textbf{cube(C,S,all)} permet la construction d'un cube de centre $C$ (point 3d) et dont un sommet est $S$ (point 3d).
            \item la fonction \textbf{icosahedron(C,S,all)} permet la construction d'un icosaèdre de centre $C$ (point 3d) et dont un sommet est $S$ (point 3d).
            \item la fonction \textbf{dodecahedron(C,S,all)} permet la construction d'un dodécaèdre de centre $C$ (point 3d) et dont un sommet est $S$ (point 3d).
        \end{itemize}

    \item Les solides d'Archimède :
        \begin{itemize}
            \item la fonction \textbf{cuboctahedron(C,S,all)} permet la construction d'un cuboctaèdre de centre $C$ (point 3d) et dont un sommet est $S$ (point 3d). Ce solide a deux types de faces.
            \item la fonction \textbf{icosidodecahedron(C,S,all)} permet la construction d'un icosidodécaèdre de centre $C$ (point 3d) et dont un sommet est $S$ (point 3d). Ce solide a deux types de faces.
            \item la fonction \textbf{lsnubcube(C,S,all)} permet la construction d'un cube adouci (forme 1) de centre $C$ (point 3d) et dont un sommet est $S$ (point 3d). Ce solide a deux types de faces.
            \item la fonction \textbf{lsnubdodecahedron(C,S,all)} permet la construction d'un dodécaèdre adouci (forme 1) de centre $C$ (point 3d) et dont un sommet est $S$ (point 3d). Ce solide a deux types de faces.
            \item la fonction \textbf{rhombicosidodecahedron(C,S,all)} permet la construction d'un rhombicosidodécaèdre de centre $C$ (point 3d) et dont un sommet est $S$ (point 3d). Ce solide a trois types de faces.
            \item la fonction \textbf{rhombicuboctahedron(C,S,all)} permet la construction d'un rhombicuboctaèdre de centre $C$ (point 3d) et dont un sommet est $S$ (point 3d). Ce solide a deux types de faces.
            \item la fonction \textbf{rsnubcube(C,S,all)} permet la construction d'un cube adouci (forme 2) de centre $C$ (point 3d) et dont un sommet est $S$ (point 3d). Ce solide a deux types de faces.
            \item la fonction \textbf{rsnubdodecahedron(C,S,all)} permet la construction d'un dodécaèdre adouci (forme 2) de centre $C$ (point 3d) et dont un sommet est $S$ (point 3d). Ce solide a deux types de faces.
            \item la fonction \textbf{truncatedcube(C,S,all)} permet la construction d'un cube tronqué de centre $C$ (point 3d) et dont un sommet est $S$ (point 3d). Ce solide a deux types de faces.
            \item la fonction \textbf{truncatedcuboctahedron(C,S,all)} permet la construction d'un cuboctaèdre tronqué de centre $C$ (point 3d) et dont un sommet est $S$ (point 3d). Ce solide a trois types de faces.
            \item la fonction \textbf{truncateddodecahedron(C,S,all)} permet la construction d'un dodécaèdre tronqué de centre $C$ (point 3d) et dont un sommet est $S$ (point 3d). Ce solide a deux types de faces.
            \item la fonction \textbf{truncatedicosahedron(C,S,all)} permet la construction d'un icosaèdre tronqué de centre $C$ (point 3d) et dont un sommet est $S$ (point 3d). Ce solide a deux types de faces.
             \item la fonction \textbf{truncatedicosidodecahedron(C,S,all)} permet la construction d'un icosidodécaèdre tronqué de centre $C$ (point 3d) et dont un sommet est $S$ (point 3d). Ce solide a deux trois de faces.
            \item la fonction \textbf{truncatedoctahedron(C,S,all)} permet la construction d'un octaèdre tronqué de centre $C$ (point 3d) et dont un sommet est $S$ (point 3d). Ce solide a deux types de faces.             
            \item la fonction \textbf{truncatedtetrahedron(C,S,all)} permet la construction d'un tétraèdre tronqué de centre $C$ (point 3d) et dont un sommet est $S$ (point 3d). Ce solide a deux types de faces.
        \end{itemize}

    \item Autres solides :
    \begin{itemize}
        \item la fonction \textbf{octahemioctahedron(C,S,all)} permet la construction d'un octahémioctaèdre de centre $C$ (point 3d) et dont un sommet est $S$ (point 3d). Ce solide a deux types de faces.
        \item la fonction \textbf{small\_stellated\_dodecahedron(C,S,all)} permet la construction d'un petit dodécaèdre étoilé de centre $C$ (point 3d) et dont un sommet est $S$ (point 3d). Ce solide a un seul type de faces.
    \end{itemize}
\end{itemize}

\begin{demo}{Polyèdres du module \emph{luadraw\_polyhedrons}}
\begin{luadraw}{name=polyhedrons}
local i = cpx.I
require 'luadraw_polyhedrons' -- chargement du module
local g = graph3d:new{bg="LightGray", size={10,10}}
g:Labelsize("small"); Hiddenlines = false
-- en haut à gauche 
g:Saveattr(); g:Viewport(-5,0,0,5); g:Coordsystem(-5,5,-5,5,true)
local T,S,A,F = icosahedron(Origin,M(0,2,4.5),true) 
g:Dscene3d(
    g:addFacet(F, {color="Crimson",opacity=0.8}),
    g:addPolyline(A, {color="Pink", width=8}),
    g:addDots(S) )
g:Dlabel("Icosaèdre",5*i,{})
g:Restoreattr()
-- en haut à droite
g:Saveattr()
g:Viewport(0,5,0,5); g:Coordsystem(-5,5,-5,5,true)
local T,S,A,F1,F2 = truncatedtetrahedron(Origin,M(0,0,5),true) -- sortie complète, affichage dans une scène 3d
g:Dscene3d(
    g:addFacet(F1, {color="Crimson",opacity=0.8}),
    g:addFacet(F2, {color="Gold"}),
    g:addPolyline(A, {color="Pink", width=8}),
    g:addDots(S) )
g:Dlabel("Tétraèdre tronqué",5*i,{})
g:Restoreattr()
-- en bas à gauche
g:Saveattr(); g:Viewport(-5,0,-5,0); g:Coordsystem(-5,5,-5,5,true)
local T,S,A,F1,F2,F3 = rhombicosidodecahedron(Origin,M(0,0,4.5),true)
g:Dscene3d(
    g:addFacet(F1, {color="Crimson",opacity=0.8}),
    g:addFacet(F2, {color="Gold",opacity=0.8}), g:addFacet(F3, {color="ForestGreen"}),
    g:addPolyline(A, {color="Pink", width=8}), g:addDots(S) )
g:Dlabel("Rhombicosidodécaèdre",-5*i,{})
g:Restoreattr()
-- en bas à droite
g:Saveattr(); g:Viewport(0,5,-5,0); g:Coordsystem(-5,5,-5,5,true)
local T,S,A,F1 = small_stellated_dodecahedron(Origin,M(0,0,5),true)
g:Dscene3d(
    g:addFacet(F1, {color="Crimson",opacity=0.8}),
    g:addPolyline(A, {color="Pink", width=8}),
    g:addDots(S) )
g:Dlabel("Petit dodécaèdre étoilé",-5*i,{})
g:Restoreattr()
g:Show()
\end{luadraw}

\end{demo}

\subsection{Le module \emph{luadraw\_spherical}}

Ce module permet de dessiner un certain nombre d'objets sur une sphère (comme par exemple des cercles, des triangles sphériques,...) sans avoir à gérer à la main les parties visibles ou non visibles. Le dessin se fait en trois temps:
\begin{enumerate}
    \item On définit les caractéristiques de la sphère (centre, rayon, couleur,...)
    \item On définit les objets à ajouter dans la scène, grâce à des méthodes dédiées.
    \item On affiche le tout avec la méthode \textbf{g:Dspherical()}.
\end{enumerate}
Bien sûr, toutes les méthodes de dessin 2d et 3d restent utilisables.

\subsubsection{Variables et fonctions globales du module}

\begin{itemize}
    \item Variables avec leur valeur par défaut:
        \begin{itemize}
            \item \textbf{Insidelabelcolor} = "DarkGray": définit la couleur des labels dont le point d'ancrage est intérieur à la sphère.
            \item \textbf{arrowBstyle} = "->" : type de flèche en fin de ligne
            \item \textbf{arrowAstyle} = "<-" : type de flèche en début de ligne
            \item \textbf{arrowABstyle} = "<->": très peu utilisée car la plupart du temps les lignes tracées sur la sphère doivent être découpées.
        \end{itemize}
    \item Fonctions :
        \begin{itemize}
            \item \textbf{sM(x,y,z)}: renvoie un point de la sphère, c'est le point $I$ de la sphère tel que la demi-droite $[O,I)$ ($O$ étant le centre de la sphère) passe par le point $A$ de coordonnées cartésiennes $(x,y,z)$. Les nombres $x$, $y$ et $z$ ne doivent pas être nuls simultanément.
            \item \textbf{sM(theta,phi)}: où \emph{theta} et \emph{phi} sont des angles en degrés, renvoie un point de la sphère donc les coordonnées sphériques sont \emph{(R,theta,phi)} où $R$ est le rayon de la sphère.
            \item \textbf{toSphere(A)}: renvoie le même point de la sphère que \emph{Ms(A.x,A.y,A.z)}.
            \item \textbf{clear\_spherical()}: supprime les objets qui ont été ajoutés à la scène, et remet les valeurs par défaut.
        \end{itemize}
\end{itemize}

Si la variable globale \textbf{Hiddenlines} a la valeur \emph{true}, alors les parties cachées seront dessinées dans le style défini par la variable globale \textbf{Hiddenlinestyle}, cependant on peut modifier ce comportement l'option locale \emph{hidden=true/false} .

\subsubsection{Définition de la sphère}
Par défaut, la sphère est centrée à l'origine, de rayon $3$ et de couleur orange, mais ceci peut être modifié avec la méthode \textbf{g:Define\_sphere( options )} où \emph{options} est une table permettant d'ajuster chaque paramètres. Ceux-ci sont les suivants (avec leur valeur par défaut entre parenthèses):
\begin{itemize}
    \item \opt{center =} (Origin),
    \item \opt{radius =} (3),
    \item \opt{color =} ("Orange"),
    \item \opt{opacity =} (1),
    \item \opt{mode =} (\emph{mBorder}), mode d'affichage de la sphère (\emph{mWireframe} ou \emph{mGrid} ou \emph{mBorder}, voir \textbf{Dsphere}),
    \item \opt{edgecolor =} ("LightGray"),
    \item \opt{edgestyle =} ("solid"),
    \item \opt{hiddenstyle =} (Hiddenlinestyle),
    \item \opt{hiddencolor =} ("gray"),
    \item \opt{edgewidth =} (4),
    \item \opt{show =} (true), pour montrer ou non la sphère.
\end{itemize}

\subsubsection{Ajouter un cercle : g:DScircle}

La méthode \textbf{g:DScircle(P,options)} permet d'ajouter un cercle sur la sphère, l'argument \emph{P} est une table de la forme $\{A,n\}$ qui représente un plan (passant par $A$ et normal à $n$, deux points 3d). Le cercle est alors défini comme l'intersection de ce plan avec la sphère. L'argument \emph{options} est une table à 5 champs, qui sont :
    \begin{itemize}
        \item \opt{style =} (style courant de ligne), 
        \item \opt{color =} (couleur courante des lignes),
        \item \opt{width =} (épaisseur courante des lignes en dixième de point),
        \item \opt{opacity =} (opacité courante des lignes),
        \item \opt{hidden =} (valeur de \emph{Hiddenlines}),
        \item \opt{out =} (nil), si on affecte une variable de type liste à ce paramètre \emph{out}, alors la fonction ajoute à cette liste les deux points correspondant aux extrémités de l'arc caché, s'il y en a un, ce qi permet de les récupérer sans avoir à les calculer.
    \end{itemize}
    
\subsubsection{Ajouter un grand cercle : g:DSbigcircle}

La méthode \textbf{g:DSbigcircle(AB,options)} permet d'ajouter un grand cercle sur la sphère, l'argument \emph{AB} est une table de la forme $\{A,B\}$ où $A$ et $B$ sont deux points distincts de la sphère. Le grand cercle est alors le cercle de centre le centre de la sphère, et passant par $A$ et $B$. L'argument \emph{options} est une table à 5 champs, qui sont :
    \begin{itemize}
        \item \opt{style =} (style courant de ligne), 
        \item \opt{color =} (couleur courante des lignes),
        \item \opt{width =} (épaisseur courante des lignes en dixième de point),
        \item \opt{opacity =} (opacité courante des lignes),
        \item \opt{hidden =} (valeur de \emph{Hiddenlines}),
        \item \opt{out =} (nil), si on affecte une variable de type table à ce paramètre \emph{out}, alors la fonction ajoute à cette liste les deux points correspondant aux extrémités de l'arc caché, s'il y en a un, ce qi permet de les récupérer sans avoir à les calculer.
    \end{itemize}

\subsubsection{Ajouter un arc de grand cercle : g:DSarc}

La méthode \textbf{g:DSarc(AB,sens,options)} permet d'ajouter un arc de grand cercle sur la sphère, l'argument \emph{AB} est une table de la forme $\{A,B\}$ où $A$ et $B$ sont deux points distincts de la sphère, on trace alors l'arc de grand cercle allant de $A$ vers $B$. L'argument \emph{sens} vaut 1 ou -1 pour indiquer le sens de l'arc. Lorsque $A$ et $B$ ne sont pas diamétralement opposés, le plan $OAB$ (où $O$ est le centre de la sphère) est orienté avec $\vec{OA}\wedge\vec{OB}$.  L'argument \emph{options} est une table à 6 champs, qui sont :
    \begin{itemize}
        \item \opt{style =} (style courant de ligne), 
        \item \opt{color =} (couleur courante des lignes),
        \item \opt{width =} (épaisseur courante des lignes en dixième de point),
        \item \opt{opacity =} (opacité courante des lignes),
        \item \opt{hidden =} (valeur de \emph{Hiddenlines}),
        \item \opt{arrows =} (0), trois valeurs possibles : 0 (pas de flèche), 1 (une flèche en $B$), 2 (flèche en $A$ et en $B$).
        \item \opt{normal =} (nil), permet de préciser un vecteur normal au plan $OAB$ lorsque ces trois points sont alignés.
    \end{itemize}

\subsubsection{Ajouter un angle : g:DSangle}

La méthode \textbf{g:DSangle(B,A,C,r,sens,options)} où $A$, $B$ et $C$ sont trois points de la sphère, permet de dessiner un arc de grand cercle sur la sphère pour représenter l'angle $(\vec{AB},\vec{AC})$ avec un rayon de \emph{r}. L'argument \emph{sens} vaut 1 ou -1 pour indiquer le sens de l'arc, le plan $ABC$ est orienté avec $\vec{AB}\wedge\vec{AC}$.  L'argument \emph{options} est une table à 6 champs, qui sont :
    \begin{itemize}
        \item \opt{style =} (style courant de ligne), 
        \item \opt{color =} (couleur courante des lignes),
        \item \opt{width =} (épaisseur courante des lignes en dixième de point),
        \item \opt{opacity =} (opacité courante des lignes),
        \item \opt{hidden =} (valeur de \emph{Hiddenlines}),
        \item \opt{arrows =} (0), trois valeurs possibles : 0 (pas de flèche), 1 (une flèche en $B$), 2 (flèche en $A$ et en $B$).
        \item \opt{normal =} (nil), permet de préciser un vecteur normal au plan $ABC$ lorsque ces trois points sont "alignés" sur un même grand cercle.
    \end{itemize}
    
\subsubsection{Ajouter une facette sphérique : g:DSfacet}

La méthode \textbf{g:DSfacet(F,options)} où \emph{F} est une liste de points de la sphère, permet de dessiner la facette représentée par $F$, les arêtes étant des arcs de grands cercles. L'argument \emph{options} est une table à 6 champs, qui sont :
    \begin{itemize}
        \item \opt{style =} (style courant de ligne), 
        \item \opt{color =} (couleur courante des lignes),
        \item \opt{width =} (épaisseur courante des lignes en dixième de point),
        \item \opt{opacity =} (opacité courante des lignes),
        \item \opt{hidden =} (valeur de \emph{Hiddenlines}),
        \item \opt{fill =} (""), chaîne représentant la couleur de remplissage (aucune par défaut),
        \item \opt{fillopacity =} (0.3), opacité de la couleur de remplissage.
    \end{itemize}
    
\subsubsection{Ajouter une courbe sphérique : g:DScurve}

La méthode \textbf{g:DScurve(L,options)} où \emph{L} est une liste de points de la sphère, permet de dessiner la courbe représentée par $L$. L'argument \emph{options} est une table à 6 champs, qui sont :
    \begin{itemize}
        \item \opt{style =} (style courant de ligne), 
        \item \opt{color =} (couleur courante des lignes),
        \item \opt{width =} (épaisseur courante des lignes en dixième de point),
        \item \opt{opacity =} (opacité courante des lignes),
        \item \opt{hidden =} (valeur de \emph{Hiddenlines}),
        \item \opt{out =} (nil), si on affecte une variable de type table à ce paramètre \emph{out}, alors la fonction ajoute à cette liste les points correspondant aux extrémités des parties cachées.
    \end{itemize}
    
Nous allons maintenant traiter d'objets qui ne sont pas forcément sur la sphère, mais qui peuvent la traverser, ou être à l'intérieur, ou à l'extérieur.

\subsubsection{ Ajouter un segment : g:DSseg}

La méthode \textbf{g:DSseg(AB,options)} permet d'ajouter un segment, l'argument \emph{AB} est une table de la forme $\{A,B\}$ où $A$ et $B$ sont deux points de l'espace. La fonction traite les interactions avec la sphère. L'argument \emph{options} est une table à 5 champs, qui sont :
    \begin{itemize}
        \item \opt{style =} (style courant de ligne), 
        \item \opt{color =} (couleur courante des lignes),
        \item \opt{width =} (épaisseur courante des lignes en dixième de point),
        \item \opt{opacity =} (opacité courante des lignes),
        \item \opt{hidden =} (valeur de \emph{Hiddenlines}),
        \item \opt{arrows =} (0), trois valeurs possibles : 0 (pas de flèche), 1 (une flèche en $B$), 2 (flèche en $A$ et en $B$).
    \end{itemize}
    
\subsubsection{ Ajouter une droite : g:DSline}

La méthode \textbf{g:DSline(d,options)} permet d'ajouter une droite, l'argument \emph{d} est une table de la forme $\{A,u\}$ où $A$ et un point de la droite et $u$ un vecteur directeur (deux points 3d). La fonction traite les interactions avec la sphère.  Le segment tracé est obtenu en intersectant la droite avec la fenêtre 3d, il peut être vide si la fenêtre est trop étroite. L'argument \emph{options} est une table à 6 champs, qui sont :
    \begin{itemize}
        \item \opt{style =} (style courant de ligne), 
        \item \opt{color =} (couleur courante des lignes),
        \item \opt{width =} (épaisseur courante des lignes en dixième de point),
        \item \opt{opacity =} (opacité courante des lignes),
        \item \opt{hidden =} (valeur de \emph{Hiddenlines}),
        \item \opt{arrows =} (0), trois valeurs possibles : 0 (pas de flèche), 1 (une flèche en $B$), 2 (flèche en $A$ et en $B$),
        \item \opt{scale =} (1), permet de modifier la taille du segment tracé.
    \end{itemize}
    
\subsubsection{ Ajouter une ligne polygonale : g:DSpolyline}

La méthode \textbf{g:DSpolyline(L,options)} permet d'ajouter une ligne polygonale, l'argument \emph{L} est une liste de points de l'espace, ou une liste de listes de points de l'espace. La fonction traite les interactions avec la sphère. L'argument \emph{options} est une table à 6 champs, qui sont :
    \begin{itemize}
        \item \opt{style =} (style courant de ligne), 
        \item \opt{color =} (couleur courante des lignes),
        \item \opt{width =} (épaisseur courante des lignes en dixième de point),
        \item \opt{opacity =} (opacité courante des lignes),
        \item \opt{hidden =} (valeur de \emph{Hiddenlines}),
        \item \opt{arrows =} (0), trois valeurs possibles : 0 (pas de flèche), 1 (une flèche en $B$), 2 (flèche en $A$ et en $B$),
        \item \opt{close =} (false), indique si la ligne doit être refermée.
    \end{itemize}    

\subsubsection{Ajouter un plan : g:DSplane}

La méthode \textbf{g:DSplane(P,options)} permet d'ajouter le contour d'un plan, l'argument \emph{P} est une table de la forme \emph{\{A,n\}} où $A$ est un point du plan et $n$ un vecteur normal. La fonction dessine un parallélogramme représentant le plan $P$ en traitant les interactions avec la sphère. L'argument \emph{options} est une table à 7 champs, qui sont :
    \begin{itemize}
        \item \opt{style =} (style courant de ligne), 
        \item \opt{color =} (couleur courante des lignes),
        \item \opt{width =} (épaisseur courante des lignes en dixième de point),
        \item \opt{opacity =} (opacité courante des lignes),
        \item \opt{hidden =} (valeur de \emph{Hiddenlines}),
        \item \opt{scale =} (1), permet de changer la taille du parallélogramme,
        \item \opt{angle =} (0), angle en degrés, permet de faire pivoter le parallélogramme autour de la droite perpendiculaire passant par le centre de la sphère.
        \item \opt{trace =} (true), permet de dessiner ou non, l'intersection du plan avec la sphère lorsqu'elle n'est pas vide.
    \end{itemize}    

\subsubsection{Ajouter un label : g:DSlabel}

La méthode \textbf{g:DSlabel(text1,anchor1,options1, text2,anchor2,options2,...)} permet d'ajouter un ou plusieurs labels sur le même principe que la méthode \emph{g:Dlabel3d}, sauf qu'ici la fonction traite les cas où le point d'ancrage est à l'intérieur de la sphère, derrière la sphère ou devant la sphère. Dans le cas où il est à l'intérieur la couleur du label est donnée par la variable globale \textbf{Insidelabelcolor} qui vaut \emph{"DrakGray"} par défaut.

\subsubsection{Ajouter des points : g:DSdots et g:DSstars}

La méthode \textbf{g:DSdots(dots,options)} permet d'ajouter des points dans la scène, l'argument \emph{dots} est une liste de points 3d. La fonction dessine les points en gérant les interactions avec la sphère. L'argument \emph{options} est une table à 2 champs, qui sont :
    \begin{itemize}
        \item \opt{hidden =} (valeur de \emph{Hiddenlines}),
        \item \opt{mark\_options =} (""), chaîne qui sera passée directement à l'instruction \emph{\textbackslash draw}.
    \end{itemize}
Dans le cas où un point est à l'intérieur de la sphère, ou sur la face cachée, la couleur du point est donnée par la variable globale \textbf{Insidelabelcolor} qui vaut \emph{"DrakGray"} par défaut.

La méthode \textbf{g:DSstars(dots,options)} permet d'ajouter des points sur la sphère, l'argument \emph{dots} est une liste de points 3d qui seront projetés sur la sphère. La fonction dessine ces points en forme d'astérisque. L'argument \emph{options} est une table à 2 champs, qui sont :
   \begin{itemize}
        \item \opt{style =} (style courant de ligne), 
        \item \opt{color =} (couleur courante des lignes),
        \item \opt{width =} (épaisseur courante des lignes en dixième de point),
        \item \opt{opacity =} (opacité courante des lignes),
        \item \opt{hidden =} (valeur de \emph{Hiddenlines}),
        \item \opt{scale =} (1), permet de changer la taille du parallélogramme,
        \item \opt{circled =} (false), permet d'ajouter une cercle autour de l'étoile,
        \item \opt{fill =} (""), chaîne représentant une couleur, lorsqu'elle n'est pas vide, l'astérisque est remplacée par une facette hexagonale cerclée et remplie avec la couleur précise par cette option.
    \end{itemize}   
Les points qui sont sur la face cachée de la sphère ont la couleur donnée par la variable globale \textbf{Insidelabelcolor} qui vaut \emph{"DrakGray"} par défaut.

\subsubsection{Stéréographie inverse : g:DSinvstereo\_curve et g:DSinvstereo\_polyline}

La méthode \textbf{g:DSinvstereo\_curve(L,options)}, où \emph{L} est une ligne polygonale 3d représentant une courbe tracée sur un plan d'équation $z =$cte, dessine sur la sphère l'image de $L$ par stéréographie inverse, le pôle étant le point \emph{C+r*vecK}, où $C$ est le centre de la sphère et $r$ le rayon.

La méthode \textbf{g:DSinvstereo\_polyline(L,options)}, où \emph{L} est une ligne polygonale 3d tracée sur un plan d'équation $z =$cte, dessine sur la sphère l'image de $L$ par stéréographie inverse, le pôle étant le point \emph{C+r*vecK}, où $C$ est le centre de la sphère et $r$ le rayon.

Dans les deux cas, les \emph{options} sont les mêmes que pour la méthode \textbf{g:DScurve}.

\subsubsection{Exemples}

\begin{demo}{Cube dans une sphère}
\begin{luadraw}{name=cube_in_sphere}
local g = graph3d:new{window={-9,9,-4,5},viewdir={25,70},size={16,8}}
require 'luadraw_spherical'
arrowBstyle = "-stealth"
g:Linewidth(6); Hiddenlinestyle = "dashed"
local a = 4
local O = Origin
local cube = parallelep(O,a*vecI,a*vecJ,a*vecK)
local G = isobar3d(cube.vertices)
cube = shift3d(cube,-G) -- pour centrer le cube à l'origine
local R = pt3d.abs(cube.vertices[1])

local dessin = function()
    g:DSpolyline({{O,5*vecI},{O,5*vecJ},{O,5*vecK}},{arrows=1, width=8}) -- axes
    g:DSplane({a/2*vecK,vecK},{color="blue",scale=0.9,angle=20}); 
    g:DScircle({-a/2*vecK,vecK},{color="blue"})
    g:DSpolyline( facetedges(cube) ); g:DSlabel("$O$",O,{pos="W"})
    g:Dspherical()
end

g:Saveattr(); g:Viewport(-9,0,-4,5); g:Coordsystem(-5,5,-5,5)
Hiddenlines = true; g:Define_sphere({radius=R})
dessin()
g:Dlabel3d("$x$",5*vecI,{pos="SW"},"$y$",5*vecJ,{pos="E"},"$z$",5*vecK,{pos="N"})
g:Dlabel("Hiddenlines=true",0.5-4.5*cpx.I,{})
g:Restoreattr()

clear_spherical() -- supprime les objets précédemment créés

g:Saveattr(); g:Viewport(0,9,-4,5); g:Coordsystem(-5,5,-5,5)
Hiddenlines = false; g:Define_sphere({radius=R,opacity=0.7} )
dessin()
g:Dlabel3d("$x$",5*vecI,{pos="SW"},"$y$",5*vecJ,{pos="E"},"$z$",5*vecK,{pos="N"})
g:Dlabel("Hiddenlines=false, opacity=0.7",0.5-4.5*cpx.I,{})
g:Restoreattr()
g:Show()
\end{luadraw}
\end{demo}

\paragraph{Courbe sphérique}

\begin{demo}{Fenêtre de Viviani}
\begin{luadraw}{name=courbe_spherique}
local g = graph3d:new{window={-4.5,4.5,-4.5,4.5},viewdir={30,60},margin={0,0,0,0},size={10,10}}
require 'luadraw_spherical'
arrowBstyle = "-stealth"
g:Linewidth(6); Hiddenlinestyle = "dotted"
Hiddenlines = false; 
local C = cylinder(M(1.5,0,-3.5),1.5,M(1.5,0,3.5),35,true)
local L = parametric3d( function(t) return Ms(3,t-math.pi/2,t) end, -math.pi,math.pi) -- la courbe
g:DSpolyline(facetedges(C),{color="gray"}) -- affichage cylindre
g:DSpolyline({{-5*vecI,5*vecI},{-5*vecJ,5*vecJ},{-5*vecK,5*vecK}},{arrows=1}) --axes
Hiddenlines=true; g:DScurve(L,{width=12,color="blue"}) -- courbe avec partie cachée
g:Dspherical()
g:Show()
\end{luadraw}
\end{demo}

Pour ne pas nuire à la lisibilité du dessin, les parties cachées n'ont pas été affichées sauf celle de la courbe.

\paragraph{Un pavage sphérique}

\begin{demo}{Un pavage sphérique}
\begin{luadraw}{name=pavage_spherique}
local g = graph3d:new{window={-3,3,-3,3},viewdir={30,60},size={10,10}}
require 'luadraw_spherical'
require "luadraw_polyhedrons"
g:Linewidth(6); Hiddenlines = true; Hiddenlinestyle = "dotted"
local P = poly2facet( octahedron(Origin,sM(30,10)) )
local colors = {"Crimson","ForestGreen","Gold","SteelBlue","SlateGray","Brown","Orange","Navy"}
for k,F in ipairs(P) do
    g:DSfacet(F,{fill=colors[k],style="noline",fillopacity=0.7})  -- facettes sans les bords
end
for _, A in ipairs(facetedges(P)) do
    g:DSarc(A,1,{width=8}) -- chaque arête est un arc de grand cercle
end
g:Dspherical()
g:Show()
\end{luadraw}
\end{demo}

Pour ce pavage sphérique, on a choisi un octaèdre régulier de centre identique celui de la sphère et avec un sommet sur la sphère (et donc tous les sommets sont sur la sphère).

\paragraph{Tangentes à la sphère issues d'un point}

\begin{demo}{Tangentes à la sphère issues d'un point}
\begin{luadraw}{name=tangent_to_sphere}
local g = graph3d:new{window={-4,5.5,-4,4},viewdir={30,60},size={10,10}}
require 'luadraw_spherical'
Hiddenlines=true; g:Linewidth(6)
local O, I = Origin, M(0,6,0)
local S,S1 = {O, 3}, {(I+O)/2,pt3d.abs(I-O)/2}
-- the circle of tangency is the intersection between spheres S and S1
local C,r,n = interSS(S,S1) 
local L = circle3d(C,r,n)[1] -- list of 3d points on the circle
local dots, lines = {}, {}
-- draw
g:Define_sphere({opacity=1})
g:DScircle({C,n},{color="red"})
for k = 1, math.floor(#L/4) do
    local A = L[4*(k-1)+1]
    table.insert(dots,A)
    table.insert(lines,{I, 2*A-I})
end
g:DSpolyline(lines ,{color="gray"})
g:DSstars(dots) -- dessin de points sur la sphère
g:DSdots({O,I});  -- points dans la scène
g:DSlabel("$I$",I,{pos="S",node_options="red"},"$O$",O,{})
g:Dspherical()
g:Dseg3d({O,dots[1]},"gray,dashed"); g:Dangle3d(O,dots[1],I,0.2,"gray")
g:Show() 
\end{luadraw}
\end{demo}

\paragraph{Stéréographie inverse}

\begin{demo}{Méthodes \emph{DSinvstereo\_curve} et \emph{DSinvstereo\_polyline}}
\begin{luadraw}{name=stereographic_curve}
local g = graph3d:new{window3d={-5,5,-2,2,-2,2},window={-4.25,4.25,-2.5,2},size={10,10}, viewdir={40,70}}
Hiddenlines = true; Hiddenlinestyle="dashed"; g:Linewidth(6)
require 'luadraw_spherical'
local C, R = Origin, 1
local a = -R
local P = planeEq(0,0,1,-a)
local L = {M(2,0,a), M(2,2.5,a), M(-1,2,a)}
local L2 = circle3d(M(2.25,-1,a),0.5,vecK)[1]
local A, B = (L[2]+L[3])/2, L2[20]
local a,b = table.unpack( inv_projstereo({A,B},{C,R},C+R*vecK) )
g:Dplane(P,vecJ,6,6,15,"draw=none,fill=Beige")
g:Define_sphere( {center=C,radius=R, color="SlateGray!30", show=true} )
g:DSpolyline(L,{color="blue",close=true}); g:DSinvstereo_polyline(L,{color="red",width=8,close=true})
g:DSpolyline(L2,{color="Navy"}); g:DSinvstereo_curve(L2,{color="Brown",width=6})
g:DSplane(P,{scale=1.5})
g:DSpolyline({{C+R*vecK,A},{C+R*vecK,B}}, {color="ForestGreen",width=8})
g:DSpolyline({{-vecK,2*vecK}}, {arrows=1})
g:DSstars({C+R*vecK,a,b}, {scale=0.75})
g:Dspherical()
g:Dballdots3d({A,B},"ForestGreen",0.75)
g:Show()
\end{luadraw}
\end{demo}


\section{Le module \emph{luadraw\_palettes}}

Le module \emph{luadraw\_palettes}\footnote{Ce module est une contribution de \href{https://github.com/projetmbc/for-writing/tree/main/palcol}{Christphe BAL}.} définit $88$ palettes de couleurs portant chacune un nom. Une palette est une liste (table) de couleurs qui sont elles-mêmes des listes de trois valeurs numériques entre $0$ et $1$ (composantes rouge, verte et bleue). Les noms de ces palettes ainsi que leur rendu, peuvent être visualisés dans ce \href{luadraw_palettes_doc.pdf}{document}.

\section{Historique}

\subsection{Version 2.2}
Liste non exhaustive :
\begin{itemize}
    \item Ajout de l'option \emph{clip} pour les méthodes : \emph{Dfacet()}, \emph{Dmixfacet()}, \emph{addFacet()}, \emph{addPoly()} et \emph{addPolyline()}, ainsi que pour les méthodes de dessin de nuages de points, et les méthodes de dessin "au trait" comme \emph{Dpolyline3d()}, \emph{Dparametric3d()}, \emph{Dpath3d()}, etc.
    \item Ajout de l'option \emph{xyzstep} pour la méthode \emph{Dboxaxes3d()}, cette option définit un pas commun aux trois axes ($1$ par défaut).
    \item Ajout des méthodes \emph{DSdots()}, \emph{DSstars()}, \emph{DSinvstereo\_curve()} et \emph{DSinvstereo\_polyline()} dans le module \emph{luadraw\_spherical}.
    \item Ajout du module \emph{luadraw\_palettes}.
    \item Ajout de la fonction \emph{interDC()} (intersection entre une droite et un cercle en 2d) et de la fonction \emph{interCC()} (intersection entre 2 cercles en 2d).
    \item Ajout des fonctions \emph{curvilinear\_param()} et \emph{curvilinear\_param3d()} qui permettent d'obtenir une paramétrisation d'une liste de points (2d pour l'une, et 3d pour l'autre) avec une fonction d'une variable $t$ entre $0$ et $1$.
    \item Ajout de la fonction \emph{cvx\_hull2d()} qui renvoie l'enveloppe convexe (ligne polygonale) d'une liste de points en 2d, et de la fonction \emph{cvx\_hull3d()} qui renvoie l'enveloppe convexe (liste de facettes) d'une liste de points en 3d.
    \item Ajout des méthodes \emph{g:Beginclip(<chemin>)} et \emph{g:Endclip()} qui facilitent la mise en place d'un clipping par tikz.
    \item Ajout des fonctions \emph{normal()}, \emph{normalC()}, \emph{normalI()} qui renvoient la normale à une courbe 2d en un point donné. Les méthodes graphiques correspondantes ont également été ajoutées.
    \item Ajout de la fonction \emph{isobar()} qui renvoie l'isobarycentre d'une liste de complexes.
    \item Ajout de l'option \emph{usepalette=\{palette,mode\}} pour les méthodes \emph{Dpoly}, \emph{Dfacet}, \emph{Dmixfacet}, \emph{addFacet}.
    \item Ajout de la fonction \emph{clipplane()} qui permet de clipper un plan avec un polyèdre convexe, la fonction renvoie la section, si elle existe, sous forme d'une facette.
    \item Ajout des fonctions \emph{cartesian3d()} et \emph{cylindrical\_surface()} qui calculent et renvoient des surfaces avec la possibilité d'ajouter ou non des cloisons séparatrices pour la méthode \emph{Dscene3d()}.
    \item Ajout de la fonction \emph{evalf(f,...)} qui permet une évaluation protégée de $f(...)$, elle renvoie le résultat de l'évaluation s'il n'y a pas d'erreur d'exécution de la part de Lua, sinon, elle renvoie \emph{nil} mais sans provoquer la fin de l'exécution du script.
    \item Ajout de la fonction \emph{split\_points\_by\_visibility()} (3d) pour séparer une courbe en deux parties : partie visible, partie cachée.
    \item Dans les méthodes \emph{g:Dfacet}, \emph{g:Dmixfacet}, \emph{g:Dpoly}, \emph{g:Dedges}, \emph{g:addFacet}, \emph{g:addPolyline}, \emph{g:addPoly}, les valeurs par défaut des options de tracé de lignes (épaisseur, couleur et style), sont les valeurs courantes en cours.
    \item Correction de bug...    
\end{itemize}

\subsection{Version 2.1}
Liste non exhaustive :
\begin{itemize}
    \item Par défaut, les fichiers tikz sont sauvegardés dans un sous-dossier appelé \emph{\_luadraw}. La nouvelle option de package \emph{cachedir} permet d'en changer.
    \item L'option \emph{line join = round} est automatiquement ajoutée à l'environnement \emph{tikzpicture}.
    \item Deux options supplémentaires pour l'environnement \emph{luadraw} : \emph{bbox} et \emph{pictureoptions}.
    \item Un certain nombre de fonctions de constructions géométriques supplémentaires en 2d et 3d.
    \item Les axes gradués (2d, 3d) utilisent le package \emph{siunitx}  pour formater les labels lorsque la variable globale \emph{siunitx} a la valeur \emph{true}.
    \item Ajout des cônes tronqués droits ou penchés (\textbf{frustum} et \textbf{Dfrustum}).
    \item Ajout des pyramides régulières (\textbf{regular\_pyramid} et pyramides tronquées \textbf{truncated\_pyramid}).
    \item Les cylindres et les cônes ne sont plus forcément droits, ils peuvent désormais être penchés.
    \item Ajout de la fonction \textbf{cutpolyline(L,D,close)}.    
    \item Dessin (élémentaire) d'ensembles (fonction \emph{set}) et opérations sur les ensembles (\emph{cap}, \emph{cup}, \emph{setminus}).
    \item Modification de l'argument \emph{mode} de la méthode \textbf{g:Dplane}.
    \item Ajout de l'option \emph{close} pour la méthode \textbf{g:addPolyline}.
    \item Correction de bug...
\end{itemize}

\subsection{Version 2.0}

\begin{itemize}
    \item Introduction du module \emph{luadraw\_graph3d.lua} pour les dessins en 3d.
    \item Introduction de l'option \emph{dir} pour la méthode \textbf{g:Dlabel}.
    \item Menus changements dans la gestion des couleurs.
\end{itemize}

\subsection{Version 1.0}
Première version.
%

\end{document}
